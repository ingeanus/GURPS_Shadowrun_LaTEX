\section{Rules}

\begin{multicols*}{3}
	TODO: Change all ST, HT, and DX prices.
	
	Due to the nature of Shadowrun's setting, a variety of source books and optional rules are specifically used, alongside some suggested Home Rules (Which are incorporated into the statistics of this ruleset).
	
	\subsection{Books and Optional Rules}
	
	\paragraph{GURPS Basic Set}
	The core rules.
	\begin{itemize}
		\itemsep0em 
		\item Limited Enhancements (BS111): Used in creating complex advantages like Magic.
		\item Malfunction (BS279)
		\item Extra Effort in Combat (BS357): Both increases competency and survive-ability in combat.
		\item Dual-Weapon Attacks (BS417)
		\item Bleeding (BS420): Lasting threat from damage that requires stabilization.
		\item Random Hits: Gunfire should default to Random Hit Locations in order to reduce its general lethality.
		\item Critical Hit Tables
	\end{itemize}

	\paragraph{GURPS Action}
	While these books are often far too simplified and cinematic for a run, they still provide good inspiration for skill usage and modifiers alongside GM inspiration. Additionally, the rules for Chases can be found in GURPS Action 2: Exploits.
	
	\paragraph{GURPS Gun-Fu:}
	Gun-Fu is an integral part of much of the near-Wuxia styles of combat present in Shadowrun and its slice of Cyberpunk culture. Much of GURPS Gun-Fu however is \textbf{too} cinematic, portraying a John Wu film style. As a general rule, Tactical Shooting is the better source to go to, but Gun-Fu's Perks, Styles, and non-cinematic Skills, Techniques, and Advantages are all great sources for Gunbunnies. Seek GM approval for anything from this book however.
	
	\paragraph{GURPS High Tech}
	GURPS High-Tech provides much of the technology for the setting.
	\begin{itemize}
		\itemsep0em 
		\item Drawing Your Weapon (HT81): Provides bonuses for the many places one can hide their weapon.
		\item Stopping the Bleeding (HT162): Makes first-aid more important for certain hit locations.
		\item "You Shot Me, Mister!" (HT162): Usually only for goons and grunts.
		\item Explosions in Enclosed Spaces (HT181): Chunky Salsa returns.
		\item Side Effects of Explosions (HT181)
		\item Explosive Destruction of Materiel (HT182)
		\item Sheaths (HT198)
	\end{itemize}

	\paragraph{GURPS High Tech: Electricity and Electronics}
	Provides a wonderful overview on all things electricty and - most notably for this settings - electronics and computers. Provides computer options (p37) that are used throughout the book for many systems.
	
	\paragraph{GURPS Low Tech}
	GURPS Low Tech holds a variety of rules useful for all tech levels, such as tools, rules regarding armor and its effects, low tech armor at higher TLs, etc.
	
	\begin{itemize}
		\itemsep0em 
		\item Armor Fatigue (LT101)
		\item Chinks in Armor (LT101)
		\item Harsh Realism - Armor Gaps (LT101)
		\item Concealing Armor (LT102)
		\item Donning Armor (LT102)
		\item Blunt Trauma and Edged Weapons (LT102): This one is highly up to the GM. It makes cut much less effective against armored targets, but makes it very hard to make it completely uneffective due to blunt trauma. As well, it is fittingly realistic.
		\item Layered Armor (LT103)
	\end{itemize}
	
	\paragraph{GURPS Martial Arts}
	Martial Arts is the bread and butter for Street Samurai and Combat Adepts. It's more likely that something is used from this book than not.
	
	\begin{itemize}
		\itemsep0em 
		\item Expanded Combat Maneuvers (MA96)
		\item Limiting Multiple Dodges (MA123)
		\item Extreme Dismemberment (MA136)
		\item New Hit Locations (MA137)
		\item Pain in Close Combat
	\end{itemize}
	
	\paragraph{GURPS Tactical Shooting}
	GURPS Tactical Shooting in general is recommended for providing depth and flavour to a variety of gunfighter styles, alongside providing a multitude of interesting and useful skills and techniques.
	\begin{itemize}
		\itemsep0em 
		\item Using The Sights (TS13): Provides good definitions on what each maneuver implies when firing.
	\end{itemize}
	
	\paragraph{GURPS Social Engineering}
	GURPS Social Engineering is the bread and butter for Faces and Social Infiltrators. While the rules are not necessary to run even social focused characters, almost every non-cinematic rule is extremely fitting for such characters.
	\begin{itemize}
		\itemsep0em 
		\item Expanded Influence Rolls (SE31)
	\end{itemize}
	
	\paragraph{GURPS Ultra Tech}
	Ultra Tech provides much of the tech for the setting of Shadowrun. \textbf{However}, much of it is entirely out of place in the setting. See \hyperref[TL]{the Tech Level section} for more clarification on what technology available
	
	\paragraph{GURPS Pyramid \#3/21 - Cyberpunk}
	Used heavily for the Matrix's Decker and Technomancer hacking rules.
	
	\begin{itemize}
		\itemsep0em
		\item  Cowboy Console Matrix Rules
	\end{itemize}
	
	\paragraph{GURPS Pyramid \#3/34 - Alternate GURPS 1}
	
	\begin{itemize} 
		\itemsep0em 
		\item Do-or-Die Bullet Dodging
		\item He Who Hesitates
		\item Grazes
	\end{itemize}

	\paragraph{GURPS Pyramid \#3/51 - Tech and Toys 3}
	The \textit{Ultra-Tech Too} article provides tons of additional equipment and technologies for TL9, including power sources, hardware, explosive and warheads, melee weapon options, updates to armor, and more defensive options. As well, the \textit{Future Soldier} article has lots of good intuition about designing competent gunfighters, although it sometimes plays a little loose with rules!
	
	\paragraph{GURPS Pyramid \#3/55 - Military Sci-Fi}
	Provides a wide variety of equipment that is of plenty of interest to runners. Importantly, updates some of Ultratech's equipment to bring it in line with the standards of High-Tech and Tactical Shooting.
	
	\paragraph{GURPS Pyramid \#3/65 - Alternate GURPS 3}
	\begin{itemize}
		\itemsep0em 
		\item A Full Complement
	\end{itemize}

	\paragraph{GURPS Pyramid \#3/57 - Gunplay}\label{pyramid_357}
	The \textit{Modern Warfighter: Gear} provides a wide selection of military focused equipment that are very useful for any wannabe-gunbunny. There are some items that are relatively in conflict with things designed in this books, most notably their Improved Assault Armor.
	\begin{itemize}
		\itemsep 0pt
		\item Clothing treatment and design for battle dress uniforms
		\item Acoustic Detection System - A High Tech acoustic countersniper system
		\item Advanced Night Vision Goggles - NVGs that overcome many of the issues present in TL7/early TL8 models.
		\item Counter-IED EWAR Suite - Specialized jammers used to prevent improvised explosives from being remotely detonated.
	\end{itemize}
	
	\paragraph{GURPS Pyramid \#3/85 - Cutting Edge}
	Has a great set of perks for all matrix specialist to use! 
	
	The GM is recommended to follow the restriction of 1 perk per 10 points in related matrix skills, although can easily expand the default list to include skills such as Cryptography.
	
	\begin{itemize}
		\itemsep0em 
		\item The Perky L33t, with perks allowed for non H4xx0rs.
	\end{itemize}

	\paragraph{GURPS Pyramid \#3/91 - Thaumatology IV}
	Used as inspiration to incorporate the Technomancer advantage rules with the Pyramid \#3/21 - Cyberpunk.
	
	\begin{itemize}
		\itemsep 0pt
		\item Technomysticism
	\end{itemize}
	
	\paragraph{GURPS Power-Ups 3: Talents}
	This book provides many useful Talents that can be used by characters in order to fill their niche as a runner. Due to the attribute repricing (See \hyperref[PU9]{Power-Ups 9}), these are much more useful for specializing a character.
	
	\paragraph{GURPS Power-Ups 9: Alternate Attributes \label{PU9}}
	GURPS Power-Ups 9 details a system for pricing attributes accordingly to a campaign's setting and assumptions. I have gone through and implemented the rules here in order to provide better costs for the game. 
	
	Accordingly, all templates, lenses, equipment, etc. incorporate them; if you do not wish to include them into your games remember to re-calculate the points costs where necessary.
	
	For in-depth details on the choices for the prices, see \hyperref[behind_the_screen]{the Behind the Screen section}; in brief: Reducing ST costs to compete with small arms, Increasing DX and IQ costs to match and compete with Talents, Increasing HT costs to be balanced, and Increasing Will and FP costs to account for Magic and Extra Effort.
	
	\noindent\textbf{Strength}
	6 Points / Level
	
	\begin{itemize}
		\itemsep0em 
		\item Hit Points - 2 Points / Level
		\item Striking ST - 2 Points / Level
		\item Lifting ST - 2 Points / Level
		\item Arm ST - 2, 3, or 4 Points / Level
	\end{itemize}
	
	\noindent\textbf{Dexterity}
	20 Points / Level
	
	\begin{itemize}
		\itemsep0em 
		\item +0.25 Basic Speed - 5 Points / Level
		\item Arm DX - 9 or 12 point / Level
	\end{itemize}
	
	\noindent\textbf{Intelligence}
	13 Points / Level
	
	\noindent
	Intelligence no longer includes Willpower or Perception.
	
	\noindent\newline\textbf{Health}
	13 Points / Level
	
	\begin{itemize}
		\itemsep 0pt
		\item +0.25 Basic Speed - 5 Points / Level
		\item Fatigue Points - 3 Points / Level
	\end{itemize}
	
	\noindent\newline\textbf{Hit Points}
	2 Points / Level
	
	\noindent\newline\textbf{Willpower }
	7 Points / Level
	
	\noindent
	Willpower is now Independent of Intelligence.
	
	\noindent\newline\textbf{Perception}
	5 Points / Level
	
	\noindent
	Perception is now Independent of Intelligence.
	
	\noindent\newline\textbf{Fatigue Points}
	3 Points / Level
	
	\noindent\newline\textbf{Basic Move}
	5 Points / Level
	
	\noindent\newline\textbf{Basic Speed}
	5 Points / Level
	
	\paragraph{GURPS Social Engineering - Keeping Contact}
	
	This book is extensively used for Contact creation and rules. It is covered in more detail in \hyperref[Contacts]{the Contact Section.}
	
	\subsection{Home Rules}
	
	\textbf{Critical Hits \& Active Defenses:} When a critical hit is rolled on an attack, the opponent can still make active defense rolls at an additional -4.
	
	\textbf{Arm DX and ST:} These are priced by their absolute value (DX without Basic Speed and ST without HP).
	
	\textbf{Costs FP and HP:} The first level of these limitations costs double (-10\% and -20\% respectively).
	
	\textbf{Extra Effort in Combat:} An unmodified Will roll is required to gain the benefits of the Extra Effort. The FP is still spent on a failure.
	
	\subsection{Character Creation}
	
	A number of rules or guidelines are necessary to prevent characters from being built that do not in any way fit the setting. Like always, the GM should exercises their best judgement when banning psionic aliens from the table, but a list of helpful points are also included:
	
	\begin{itemize}
		\itemsep 0pt
		\item Characters should seek approval for attribute levels in excess of \(\pm\)30\% of Racial baselines, and levels in excess of \(\pm\)50\% should generally not be allowed.
		\item Characters should seek approval for wealth levels of Wealthy [20] and above. Very Wealth [30] should be highly scrutinized as it's extremely efficient for purchasing 'ware. If allowed, consider enforcing the 80\% settled lifestyle rules (B26) in order to prevent abuse.
		\item Independant Income should generally be banned, or at a minimum greatly limited.
		\item Multiple advantages or enhancements from differing sources should generally not stack. As an example, DR from Adept Powers and DR from Cyberware, should generally be highly scrutinized by the GM, in order to avoid cases that break the game's norms (Or at least, without an Unusual Background!). A good sanity test is to limit levels to be equal or less to a guiding attribute, such as HT or Will.
		\item Any enhancement for Affects Substantial/Insubstantial, or that which allows effects to cross planes, is strictly banned.
		\item Lowering Attributes does not count against your disadvantage limit\footnote{This one allows characters to better fine tune themselves, make back some of the higher point costs in this book, and focus more on the cool disadvantages instead.}.
		\item Kromm's home rule that further levels of Affliction cost 3 Points per level instead of 10.
	\end{itemize}
	
	\subsubsection{Talents}
	
	Talents are heavily recommended, especially the Alternative Talents from Power-Ups 3, which provide excellent flavor and utility. 
	
	Talent Levels are limited to 4, with the exception being Talents from special sources, including 'Ware, Magic, and Resonance, which are limited to 6 at character creation, but may be taken higher with permission by the GM.
	
	\subsubsection{High Basic Speed}\label{high_basic_speed}
	
	The extreme capabilities of many pieces of 'ware and adept powers unlock the ability to purchase a variety of advantages that are usually unavailable to characters (in the exact same way as with unusual backgrounds).
	
	Characters with a Basic Speed of 7.0 or above (whether through 'ware, magic, or natural ability) can purchase the Extra Attack - up to three times for Basic Speed 8.0 and 9.0 - as detailed below for Cyberware, or Bioware, or Mundanes respectively.
	
	\textit{\textcolor{OliveGreen}{Statistics: Extra Attack (Multi-Strike, +20\%) with Power (Technological), -35\%, Power (Bioware), -5\%; Magical, -10\%; or no Power modifier. These cost 22, 29, 28, and 30 points respectively.}}
	
	For individuals with switchable Basic Speed (Such as Wired Reflexes) that takes them above the necessary, they may add \textit{Accessibility, Basic Speed limited, -5\%\footnote{Priced according to assumption that you will use in ~90\% of combat situations.}}, costing 20, 28, 27, and 29 points respectively.
	
	Optionally, a GM may allow the purchase of Altered Time Rate for characters with Basic Speed of 8.0 or more, as detailed below:
	
	\textit{\textcolor{OliveGreen}{Statistics: Altered Time Rate 1 with Power (Technological), -35\%, Power (Bioware), -5\%; Magical, -10\%; or no Power modifier. These cost 65, 95, 90, and 100 points respectively.}}
	
	For individuals with switchable Basic Speed (Such as Wired Reflexes) that takes them above the necessary, they may add \textit{Accessibility, Basic Speed limited, -15\%\footnote{Priced according to assumption that you can benefit from ATR at all times, but will activate ~67\%}}, costing 50, 80, 75, and 85 points respectively. 
	
	It is not recommended to allow these Extra Attack and ATR together, as their effects are multiplicative and may prove destabilising even despite the prohibitive costs.
	
	\subsubsection{Rule of 16}
	
	A quick stroll through the \hyperref[spells]{Spell section} of Magic will show you that a lot of these abilities provide a Reliable bonus at higher Forces, increasing the Awakened's effective Spellcasting skill for that cast - and of course a similar case exists with Emerged. However, the \textit{Rule of 16} (B349) stipulates that resist powers have a maximum skill of 16 or the defender's resistence, which can greatly limit the ability to perform stunts at forces higher than Force 5.
	
	As such, the GM must make some decisions regarding this rule. Largely, this rule is put into place to prevent the possibility of someone taking extremely high points for a power's skill and effectively removing all defenses - which is only exacerbated by the fact that most defenses are \textit{more} expensive than respective increases to skill.
	
	At the same time, it greatly limits any capabilities to go nuclear on certain powers, which can make taking them at higher Magic/Resonance feel useless or poorly designed.
	
	A fair midpoint is to require Awakened or Emerged to take the \textit{Rule of 17}\footnote{Power-Ups 2 Perk, p20} for Spellcasting or Threading respectively. This perk raises the Rule of 16 by 1 per level for that respective skill. This allows for a small tax on the increased performance of high Force spells, without being cost prohibitive. A good heuristic is to tack two levels onto each level of Magic/Resonance 3 and above, although it's fair to simply look at an Awakened or Emerged's maximum skill should they use their maximum Force/Level and charge up to there. This will likely cost such characters 2-6 more points overall, with highly specialized characters going up to 12 more.
	
	Alternatively, the GM can simply ignore such a rule; Deckers, Samurai, Riggers, and so on have no such limitations - and many of their attacks \textit{are resisted}, such as the Decker using the Damage program to fry someone's brain through their Datajack. Perhaps, it's only fair to simply disregard the rule entirely. It's also entirely possible to ignore it, but only when taking into account the Reliable bonuses from magica abiltiies, meaning that you still cannot throw your base skill of 18 around trivially, however throwing a Force 8 Control Thoughts still works.
	
\end{multicols*}