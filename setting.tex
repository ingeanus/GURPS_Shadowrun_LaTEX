\begin{section_header}[purple]
	\textbf{\section{Setting}}
\end{section_header}

\begin{multicols}{3}
	
	Shadowrun is a Cyberpunk / Fantasy / Soft Science-Fiction setting, following the alternate history of the Earth after the takeover of political power by large corporations alongside the re-emergence of magic, creating what is now known as the Sixth World.
	
	This section covers the mechanical portions of the setting, from Tech Levels to Control Ratings, to character building limitations and suggestions.
	
	\subsection{Tech Level}\label{TL}
	
	Shadowrun is set in a TL9 Cyberpunk society (UT9). It has advanced Medical, Biotech, and Cyberware technology, occasionally reaching TL10. The setting also incorporates elements on the Nanotech Revolution (UT9), allowing the emergence of TL10 nanotechnology, largely focused on wet-nanotechnology.
	
	Attempting to compile an exhaustive list of included/excluded technology from Ultratech, Traveller, Transhuman Space, etc. would be a superhuman endeavor. It is recommended for GMs to familiarize themselves with the technologies present in Shadowrun's lore, their level of development, and their prevalence. A list of some examples is provided as a core reference below.
	
	\GURPS Ultratech includes many generic technological items (A prime example of this are the entirely standardized firearms); while there is no reason to disallow such things in the setting, it's generally recommended to stick to the equipment created to match the original Shadowrun equipment, especially because some of Ultratech is made for different genres of Sci-Fi. It is heavily recommended to not allow players to use \GURPS original trauma plates regardless; in \GURPS trauma plates are extremely effective for their weight (Covering the whole torso!), providing extreme defense for little downsides or mitigation. As such, it is generally recommended to stick to the trauma plates in this book.
	
	\subsubsection{Superscience Technologies}
	
	Shadowrun incorporates certain Superscience technologies. While the GM is advised to determine technologies themselves, a list is provided as reference.
	
	\begin{itemize}
		\item Monowire (UT103, 154, 163)
	\end{itemize}
	
	\subsubsection{TL 10 Technologies}
	
	TL 10 technologies in Shadowrun are relatively few and far between. Aside from any that the GM wishes to incorporate, here is a list of some of the more common ones. Keep in mind that some of these are easily replaced with TL9 technologies in the lore, so could easily be not included in a game (e.g. Fabricators vs Robofacs).
	
	Plenty of Monad Technology falls into this category (Anti-Grav, Nerve Guns, Railguns, Plasma Guns, Healing Goop, Brainwashing, etc), or even worse into TL 11.
	
	Because the Monad Super-Science is a mentally retarded addition to the setting that flagrantly ignores the consequences such technology would have on the setting even in small amounts (Anti-grav drives) alongside playing extremely loose (Even for Shadowrun) with the rules of reality in ways that fundamentally upend any scientific verisimilitude (Anti-grav drives again) - all while pushing the setting towards Science Fiction instead of Cyberpunk - I will not be including it. If you wish to include it as a GM, feel free to on your own accord.
	
	\begin{itemize}
		\itemsep 0pt
		\item Fusion Power (UT20)
		\item Robofacs (UT90)
		\item Wet Nanofabrication Systems (UT91)
		\item Gecko Gear (UT96)
		\item Paralysis Gas (UT160)
		\item Pheremone Spray (UT160)
		\item Nanoburn (UT161)
		\item Vibroblades (UT164)
		\item Various TL10 Cybernetics (UT212+)
	\end{itemize}
	
	\subsubsection{TL 9 Technologies}
	
	Most TL 9 technologies are by default included in the setting. While the GM should make judgement calls on any particular item, one of the important facets can be \textit{disallowing} certain technologies; The following is a list of example technologies that are generally not included in the setting (But might still make for interesting megacorp R\&D!).
	
	\begin{itemize}
		\itemsep 0pt
		\item Quantum Communicators\footnote{At least no mature Quantum Communications, due to the high levels of security being unsuitable for the Cowboy style of hacking in Shadowrun.} (UT47)
		\item Memory Augmentation\footnote{At least no mature Memory Augmentation due to how disruptive it could be to the setting. It fits the setting perfectly however.} (UT56)
		\item Virtual Tutors, AI Tutors (UT56, 59)
		\item Brainwiping (UT109)
		\item Electrolasers (UT119)
		\item ETC
		\item Liquid-Propellant Guns
	\end{itemize}
	
	\subsection{Control Rating}
	
	Control Rating specifies how difficult it is to get equipment in the setting. Shadowrun varies in its CR, but Seattle is a Control Rating 4, meaning that LC ratings are:
	
	\begin{itemize}
		\itemsep 0pt
		\item LC 5 - Anyone may carry it.
		\item LC 4 - Anyone but a criminal or SINless may carry it
		\item LC 3 - License is required. Licenses tend to cost 1dx10\% of the item.
		\item LC 2 - Prohibited to all but Military, Corporate Security, etc.
		\item LC 1 - Only permitted to Military, Spec Ops, etc.
		\item LC 0 - Usually banned for anyone or organization.
	\end{itemize}
	
	\subsection{Character Creation}
	
	Character Creation was already daunting in the original Shadowrun, and it can be even more so in a port like this. This section gives guidance, advice, and "new" rules or rulings on how to create characters for the settings.
	
	\subsubsection{Starting Points}
	
	Starting Points for a campaign can vary wildly depending on the campaign that the GM wants to run: 
	
	For most campaigns, this is 200 points, the default assumption being that the players are proven, but somewhat new individuals to the scene. They have passed the barrier of their initial runs without serious fuck-up or death, acquired a number of relevant skills to the field, and bring enough talent to outmatch the everyday competition for a Shadowrunner.
	
	Some campaigns may want to run at Street Level (Because god forbid that was impossible in normal Shadowrun), which is 100 points. These are individuals who have been forced into the Shadows. While they are still above average individuals, they don't necessarily have the expertise to outmatch average runner-level threats, meaning that they are more likely to be hired by non-professionals of the street. As well, they are likely to not be tested and may face their first real run in the beginning levels of the campaign.
	
	Although I do not prepare anything for it in this document, there is also the level of Prime Runners, usually around 300 points or more. These are individuals who have spent a good amount of time in the shadows, usually more than 3-5 years. These are the individuals who hit MCT Zero-Zones, make deals with dragons, and so on.
	
	\subsubsection{Advantages}
	
	There are plenty of  advantages that will give you an edge as a runner. Included here are some of them - alongside descriptions about how they might work in the Shadowrun setting!
	
	\subsubsection*{Alternate Identity}\label{alternate_identity}
	\begin{flushright}
		Basic Set 
	\end{flushright}
	
	An Alternate Identity can be a tempting advantage for many runners, due to its relative permanency compared to fake SINs purchased with money. Unlike normal fake SINs, this will never be burned by anything less than conclusive evidence - generally requiring a plot point, not too dissimilar to \hyperref[signature_gear]{Signature Gear, Fake SIN (See below)}, but without the threat of constant rolls and needing to cover up when it fails.
	
	
	\subsubsection*{Signature Gear}\label{signature_gear}
	\begin{flushright}
		Basic Set 
	\end{flushright}
	
	Signature Gear is useful for all of the normal reasons to take it, but it is somewhat notable for its combination with fake SINs, which must be covered here. Fake SINs usually deteriorate over time - or when being checked against, which can make the idea of giving them plot protection tempting.
	
	Plot protection does not prevent them from deteriorating normally - that is an important part of their value compared to the Alternative Identity advantage; instead, this advantage always gives you the opportunity to fix or rectify the burning. This can come in a variety of ways depending on the situation, but can range from the extreme of having to defraud a SIN registry's review of the SIN (perhaps with the help of a SIN Forger), convice a police officer that the errors popping up on scanner are a fact of life for you - not an indication that he should alert his superiors, or perhaps indication that someone or something is snooping around the ID and you might need to lay low.
	
	While many of these are actions you could \textit{normally} take to prevent your identity from being burned, the advantage here is that the GM is \textit{required} to provide you a suitable opportunity like this, and it should be tailored to your runner within reason, no Foundation Dives for the face, unless that's what he wants!
	
	\subsubsection*{Zeroed}\label{zeroed}
	\begin{flushright}
		Basic Set 100
	\end{flushright}
	
	Zeroed is something that will likely immediately appeal to most runners, given that the default state of Shadowrun has the characters as SINless. However, \textcolor{Blue}{\href{http://forums.sjgames.com/showpost.php?p=932347&postcount=35}{as noted here by Kromm}}, Zeroed is \textit{much more} than just that.
	
	Zeroed implies active maintenance of your utterly recordless existence. This can be a lot of different things, ranging from a SIN Forger or Decker who constantly scrubs the records (maybe even you scrub them!), a corporation or powerful individual who buries any evidence, the SIN holders and other individuals deleting what they believe to be erroneous data, individuals overlooking the strange holes as mistakes, or any combination of the lot! Alternatively, it could even be a dragon or spirit (or any suitable awakened) protecting you from divination magic!
	
	Do note, that you must select \textit{one or the other} when it comes to being Zeroed - High Tech records or Divination - assuming you do not have the Universal enhancement, and if you do you must justify it well or the GM is free to ban it! 
	
	Of course, as is mentioned in the Basic Set, your lack of records can still lead to investigation or detention if handled poorly, but you won't ever start accumulating a paper trail on its own!
	
	There is an additional limitation available for Zeroed however (As mentioned by Kromm above), that allows it to serve as a one-time Zeroing. The One-Use limitation means that you had some reason that you start the game without records - very often being SINless in Shadowrun - but your zeroing does not maintain itself! 
	
	If your face gets on the news, if you get arrested, if you show up on cameras, if your astral signature is recorded by the police, or whatever records you leave if your wake, they stay. Judicious use of \hyperref[fake_SINs]{Temporary Identities with fake SINs} is a good way to maintain this trait yourself (beyond just wearing a mask!), allowing your fake SIN to get burned, without necessarily exposing your SINless self to society.
	
	\subparagraph{Special Modifiers:\\}
	
	\textit{Universal:} You get both benefits of Zeroed. In a fantasy setting, records and evidence of you disappear; in high-tech worlds, supernatural abilities to divine your presence or true identity fizzle and do not work. +50\%. (\GURPS Pyramid \#3/97, p19). 
	
	\textit{One-Use:} Your Zeroed identity is as a result of some initial incident. Often, this is something such as a hidden or recordless birth, a false death, or spiriting away. Your lack of records do \textit{not} maintain themselves, meaning that paper trails, security footage, IDs, and more will stick around without any intervention. This can eventually lead to investigation - as with the original Zeroed advantage - however, it is just a matter of when, not if, with this limitation! x1/5.
	
	\subsubsection{Perks}
	
	There are a multitude of Perks that are well put to use by Shadowrunners. Great places to looks for these are Power-Ups 2 - Perks, Social Engineering, Tactical Shooting, and Martial Arts.
	
	\subsubsection*{Forgettable Face}\label{forgettable_face}
	\begin{flushright}
		Social Engineering 78
	\end{flushright}

	A classic quality of Shadowrun that lets one blend in and get away with their crimes easier, now appropriately priced!
	
	
	\subsubsection*{Passing Appearance}\label{passing_appearance}
	\begin{flushright}
		Social Engineering 78
	\end{flushright}

	Many of the poser qualities can function as this, allowing runners to get around without triggering the many biased groups of the Sixth World.
	
	TODO: More Perks
		
	\subsubsection{Disadvantages}
	
	\subsubsection*{Compulsive Behavior (Essence Loss)}
	\begin{flushright}
		-15/-20 Points
	\end{flushright}
	
	The character has a compulsive behavior to drain the essence of the living with the \hyperref[essence_drain]{Essence Drain power} or to have \textit{their} Eessence drained with it. Make a self control roll every day or whenever presented with an opportunity - such as spotting a passed out hobo in the Z-Zone or meeting an infected or someone you think is one!
	
	If you fail, you must indulge, usually meaning taking -10 points of essence loss. In this case, it's not necessarily bad roleplay to try avoid this! This addiction is often thrust upon the user, however it \textit{is} bad roleplay to avoid any consequences or instances when it comes up! It's perfectly fine to avoid going outside at night or to infected areas to avoid impulses - however if you see a hobo sleeping in an alley during your run, you had better roleplay the consequences!
	
	The character does not necessarily immediately give in of course, they are still able to rationally form plans for acquiring their hit! An example of this could be: After negotiations, you discretely meet with the Vampire Johnson to discuss a business opportunity and offer yourself to them once alone.
	
	The GM can allow those with this disadvantage to kick it using the rules for Addiction Withdrawal as if this was a Highly Addictive drug.
	
	In the case for individuals that are addicted \textit{to having their Essence} drained, increase the base cost to -20 points — those who are unable to restrain themselves from this are not long for this world.
	
	\subsubsection*{Social Stigma, Second-Class Citizen (SINless)}
	\begin{flushright}
		-5 Points
	\end{flushright}
	
	A large portion of the populous make up a class known as the SINless. These are the dregs of society that lack a System Identification Number and are relegated to a life outside the system. For most individuals, this is a society wide ailment that prevents them from doing almost any 'normal' task for your average citizen; they cannot buy from Stuffer Shack, apply for a corporate job, buy a house, and so on. 
	
	They must instead be supported by structures outside the corporate overlords, such as gangs, organized crime, support groups, charities, and so on. Notably, there's no easy way to achieve a real SIN if you are not born with one, they are usually issued by a specific company to children of people with SINs and there is no formal process for gaining one otherwise (I don't care what stupid things the later versions put in, it makes no sense to uplift your underclass in Dystopian Cyberpunk).
	
	Occasionally, this can include once SINners who have had their real SINs burned. This might be as simple as having a criminal SIN and opting to live in the shadows to prevent the consequences, but it can just as often be a result of higher powers taking a little too much notice of you or by your own actions in order to get off the grid (See: Shadowrun Hong-Kong).
	
	This is not the worst for Runners, who often have contacts that are able to get forged SINs that provide the benefits of a SINner's life without the drawbacks that come with being trackable.
	
	Notably, there is no Disadvantage for being a SINner (beyond other Social Stigmas like Criminal Record), due to the overall loss in benefits — even for someone who's job is committing crimes. There is a \textit{cost} for being protected from legal consequences for those crimes, in the form of an Alternative Identity, Signature Gear (Fake SIN), or Zeroed, which usually come out to net 0 or more points due to the fact that you now avoid the downsides of being a SINner \textit{and} many of the consequences for committing crimes!
	
	\subsubsection{Skills}
	
	Certain skills are very useful to a Shadowrunner - and some can mean the difference between life and death.
	
	\subsubsection*{Area Knowledge}
	
	Area Knowledge has an additional specialty that can be of use for deckers, or for an matrix inclined individuals.
	
	\textit{Cyberspace:} Covers information about how the matrix is entirely organized. Can be used for information on public hosts, grids, matrix architecture, people of important, etc. The area classes for Cyberspace are a bit different, but the GM can usually apply them similarly. Here are some example for inspiration (however the Basic Set rules are still incredibly useful regardless):
	
	\textit{Neighborhood:} Covers a single host or small collection of related hosts, alongside noteworthy people for it.
	
	\textit{Village or Town:} Covers a collections of hosts. Can often be the hosts representing a single town, but can also cover non-standard, but similarly sized areas like Arcologies, a single slice of a grid (e.g. the public sector of a city), etc. 
	
	\textit{City:} Covers all the hosts in a single city. Can also cover an amount of closely related hosts similar to this, such as the hosts for a A or small AA corporation.
	
	\textit{Barony, Count, Duchy, or Small Nation.:} Covers a regional area of multiple town, or a small nation. Can also cover an amount of closely related hosts similar to this, such as the hosts of a AA corporation. The GM \textit{may} allow this scope for AAA corporations that do not have/run huge matrix operations themselves. Generally only covers people and things of Status 5+.
	
	\textit{Large Nation:} Covers the hosts of a large nation, such as the UCAS. Can also cover an amount of closely related hosts similar to this, such as the hosts of a AAA corporation. Notably, huge matrix providers such as NeoNet are entirely within the GM's purvue to declare as in the Planet Class. Generally only covers people and things of Status 6+.
	
	\textit{Planet:} Covers the global matrix. Can also sometimes cover AAA matrix providers, such as NeoNet, at GMs discretion. Generally only covers people and things of Status 7+.
	
	\subsubsection*{Current Affairs}
	
	Current Affairs has some additional specialties that might be useful for runners.
	
	\textit{Cyberspace:} Covers news about the matrix, GOD, cybersecurity, and general cyberspace.
	
	\textit{Shadows:} Covers "news" or more accurately, word of mouth, about the shadows, jobs, people in the shadows, etc.
	
	\subsubsection*{Holdout}\label{holdout}
	Holdout is an invaluable skill for almost every Shadowrunner. It provides the ability to conceal weapons and equipment (B200) alongside armor and clothing (HT66). 
	
	Most weapons take a penalty equal to their bulk, while armor takes a penalty equal to its DR/3 if flexible, or DR if rigid, and equipment is assessed by the GM (Although many pieces already have it noted in their descriptions). All armor provides a +3 bonus to Holdout if it covers a majority of the body (Largely only full body or clothing as large as suits), providing a +4 bonus otherwise. This means that even Armor Clothing provides a net +1 to skill, while most others provide net +0 to -3, which can be difficult to keep hidden even with training.
	
	As well, most scanners provide a +4 to skill at TL9 (UT104), although they sometimes go lower (HT206) or higher (HT207), which can make getting equipment through them near impossible without obfuscation (hiding the equipment among other detectable objects using the Smuggling skill), distraction, or hacking - which makes the skill all the more invaluable.
	
	Some advantages can help with this, such as the \hyperref[skin_pocket]{Skin Pocket} cyberware or simple Payload advantage.	
	
	\subsubsection*{Professional Skill (Shadowrunner)}
	\begin{flushright}
		\textbf{IQ/Average}
	\end{flushright}
	\textit{\textcolor{NavyBlue}{Defaults: Streetwise-3}}
	
	This skill covers all of the practical job knowledge for running in the Shadows. 
	
	Successful rolls can provide information of subjects such as: How to safely arrange a meeting with a Johnson or Fixer, how to vet a mission or Johnson, what general actions might be dangerous for a Shadowrunner or their career (As it pertains to direct threats such as bullets or tracking, or more vague ones such as reputation and public exposure), \textcolor{Blue}{\href{https://docs.google.com/document/d/1ydfYWrtSEOtMoSOpVesdoS2iJw_fOHHDZA1RMCXlY8Y/edit?usp=drivesdk}{standard operating procedures}}, generalized or well known Shadowrunner tactics, etc.
	
	The GM should not let this skill replace other, more specialized skills, especially ones that are particularly close like Streetwise; it should represent general knowledge of the job and its practicalities, in contrast to Streetwise's ability to get along with people on the Street and in the Shadows.
	
	The GM can, in some ways, consider this the skill for telling the player's they're fucking up the obvious. While of course, the Common Sense advantage does cover this in all situations, it's quite common for your civilian goody-two-shoes players to not understand the finer complexities of running in the Shadows. Asking them to roll against this skill can allow you to give them advice that their characters would presumably know. This can be as helpful as you want, but should be aimed to be generally helpful information that a runner would know and not shortcut their decisions and the problem at hand. Remember, it's \textit{never} as obvious to them as it is to you.
	
	\subsubsection*{Magical Skills}
	
	These skills are only meaningful to Awakened, although technically anyone \textit{could} take them - representing knowledge of fundamentals and practicalities, without any meaningful experience.
	
	\paragraph{Assensing}\label{assensing_skill}
	\begin{flushright}
		\textbf{Per/H}
	\end{flushright}
	\textcolor{NavyBlue}{\textit{Defaults: None}}
	
	Assensing allows a user to interpret the auras of individuals seen with Astral Perception (or any other strange ability, such as Astral Rifts). You can roll Assensing for any aura that you can see.
	
	Success can provide any information Empathy (Sensitive) can (See: Social Engineering p36, 53, 56, 71), alongside magical information based on Margin of Success, with some examples ordered from easiest to most difficult: The subject's emotional state, the subject's general health, whether they are awakened or not, the class of magic, or whether you have seen this aura before, the presence and location of cyberware (with better grades being even harder to detect), the subject's essence and magic and force, diagnosis of maladies affecting the subject, the presence and location of bioware, and whether the subject is a technomancer. 
	
	Success also provides a +1 to all Detect Lies, Fortune-Telling, and Psychology rolls to analyze the subject.
	
	\paragraph{Binding}
	\begin{flushright}
		\textbf{10/H}
	\end{flushright}
	\textcolor{NavyBlue}{\textit{Defaults: None}}
	
	This is the skill of binding a summoned spirit in order to gain additional favors and act over a longer timeframe. Covered under the \hyperref[binding]{Binding advantage} section.
	
	\paragraph{Spellcasting}\label{spellcasting_skill}
	\begin{flushright}
		\textbf{10/H}
	\end{flushright}
	\textcolor{NavyBlue}{\textit{Defaults: None}}
	
	Spellcasting allows an Awakened to channel their magical abilities into certain short term effects. Spells have 5 categories, indicating their overall purpose, however there is often much overlap between the two.
	
	When casting a spell, you must select its Force, which can range from 1 up to 2\(\times\)Magic. The higher the force, the more powerful the spell, but the more Drain you must resist. Drain is the strain put on your body from casting spells. It costs 1 FP per Force, up to your Magic, after which is costs 1 HP per Force. 
	
	You can resist Drain, rolling against either (IQ + Will)/2 or (HT + Will)/2, depending on your magical tradition. You reduce the amount of FP or HP damage by your Margin of Success.
	
	\paragraph{Summoning}
	\begin{flushright}
		\textbf{10/H}
	\end{flushright}
	\textcolor{NavyBlue}{\textit{Defaults: None}}
	
	This is the skill of summoning magical spirits to perform tasks for you. Covered under the \hyperref[summoning]{Summoning advantage} section.
	
	\subsubsection*{Emerged Skills}
	
	\paragraph{Threading}
	\begin{flushright}
		\textbf{10/H}
	\end{flushright}
	\textcolor{NavyBlue}{\textit{Defaults: None}}
	
	This is the skill for using Complex Forms as an emerged. Covered under the \hyperref[threading]{Threading skill} section.
	
	\paragraph{Compiling}
	\begin{flushright}
		\textbf{10/H}
	\end{flushright}
	\textcolor{NavyBlue}{\textit{Defaults: None}}
	
	This is the skill for compiling an Emerged Sprite to perform tasks for you. Covered under the \hyperref[compiling]{Compiling advantage} section.
	
	\paragraph{Registering}
	\begin{flushright}
		\textbf{10/H}
	\end{flushright}
	\textcolor{NavyBlue}{\textit{Defaults: None}}
	
	This is the skill of registering compiled sprites in order to perform long and difficult tasks. Covered under the \hyperref[registering]{Registering advantage} section.
	
\end{multicols}

\par\rule{\textwidth}{0.5pt} 

\begin{multicols}{2}
	
	\subsection{Lenses}
	
	Lenses are templates that can be applied to characters in order to streamline the character creation process or to provide inspiration.
	
	\subsubsection{Shadowrunner}
	\begin{flushright}
		6 Points
	\end{flushright}
	
	This lense describes a somewhat experienced Shadowrunner, who has picked up a number of basic skills far better than the average joe to allow him to survive in the Shadows.
	
	Many Shadowrunners have lower Status and sometimes Wealth as well.
	
	
	\textbf{Advantages:}
	Zeroed (One-Use, x1/5) [2]
	
	\textbf{Disadvantages:}
	Social Stigma, Second Class Citizen (SINless) [-5]
	
	\textbf{Skills:}
	Professional Skill (Shadowrunner) (A) IQ+1 [4]; Streetwise (A) IQ [2]; Urban Survival (A) Per-1 [1]\\
	Choose 2 points from:\\
	Acting (A) IQ-1 [1]; Carousing (E) HT [1]; Fast-Talk (A) IQ-1 [1]; Intimidation (A) Will-1 [1]
	
	\subsubsection{Decker}
	\begin{flushright}
		95 Points, 33,000¥
	\end{flushright}
	
	A middlingly competent Decker, with the skills to be able to provide a wide variety of Matrix support and services. A decker of this caliber should be able to take on Hosts up to around Rating 7 to 9, while also being able to provide a general support of any matrix skill that would come up on most runs. As well, his Comfortable Wealth provides him enough funds to land a Cyberdeck of Complexity 6 with a good amount of programs, or a cherried-out Complexity 5 with every program he could want. He may have to place some aspects under settled lifestyle if he is not a wanderer, but that shouldn't be too difficult with his Status.
	
	This lens specializes into his field with the Talent (Born to Be Wired), which provides bonuses to all of his core skills, while reducing both the familiarity penalty for unknown systems and allows him to lower the \textit{No Equipment} penalty (B345) for not having software by rapidly improvising. 
	
	These benefits allow him to make use of most systems with no penalty - despite what strange design a corpo has set up for their host - and also gives him a hail-mary backup option if he finds that he needs specific software for a certain niche task, which can help for hacking, on-site analysis, technical skill use, etc.
	
	\textbf{Attributes:}
	IQ+1 [13]; Per +1 [5]
	
	\textbf{Advantages:}
	Born to Be Wired 2, Alt\footnote{Power-Ups 3 Talents p8} [12]; Comfortable [10]; Status 1 [5]
	
	\textbf{'Ware:}
	Datajack (Base Grade) [3, 6,000¥]; Cerebral Booster 1 (Cultured Bioware, Base Grade) [4, 27,000¥]
	
	\textbf{Essence Loss:}
	\textit{Choose [-4] points for any suitable disadvantages, such as:}
	
	Fearfulness 2 [-4], Noisy 2 [-4]	
	
	\textbf{Primary Skills:}
	Computer Operation (E) IQ+1+2 [2]-15; Computer Programming (H) IQ+1+2 [8]-15; Computer Hacking (VH) IQ+1+2 [12]-15; Cryptography/TL9 (Cryptanalysis) (A) IQ+0+2 [2]-14; Electronics Operation/TL9 (Security) (A) IQ [2]-12; Electronics Operation/TL9 (Surveillance) (A) IQ [2]-12; Electronics Repair/TL9 (Computers) (A) IQ+0+2 [2]-14; Expert Skills (Computer Security) (H) IQ+1+2  Research/TL9 (A) IQ+1 [4]-13
	
	\textbf{Secondary Skills:}
	Area Knowldge (Cyberspace) (A) IQ [2]-12; [8]-15; Mathematics/TL9 (Applied) (H) IQ-2 [1]-10; Mathematics/TL9 (Computer Science) (H) IQ-1 [2]-11;
	
	\textbf{Perks:}
	Console Monkey [1]
	
	\subsubsection{Face}
	\begin{flushright}
		98 points, 12,000¥
	\end{flushright}
	
	A middling competent and generalist Face, this individual has the skills to run a wide variety of cons, negotiate with the Johnson, detect and social threats to the group, and work the Shadows for services and info. Their appearance lets them gain a reasonable edge at the start of any interaction, and their multitude of social skills makes it possible to recover from many bad reactions and situations.
	
	They make use of the Alternative Benefit for Smooth Operator, letting them resist Influence skills better, however many players may opt for the better influence rolls of the original trait, which can greatly improve the moods of Johnsons, contacts, runners, and so on!
	
	While they will have enough money for basic gear, they will likely find themselves wanting if they want to perform complex social infiltration. As well, having lots of liquid cash is great for props - an important part of every con!
	
	Do note that any skill levels in this Lens \textit{do not} include the benefits of Tailored Pheremones, because it cannot be guaranteed if they will have their effect!
	
	\textbf{Attributes: }
	IQ +1 [13]; Per +1 [5]; Will +1 [7]
	
	\textbf{Advantages: }
	Appearance, Attractive [4]; Smooth Operator, Alt\footnote{Power-Ups 3 Talents p15} 1 [13]; 
	
	\textbf{'Ware:}
	Tailored Pheremones 2 (Cultured Bioware, Base Grade) [2, 12,000¥]
	
	\textbf{Essence Loss:}
	\textit{Choose [-1] points for any suitable disadvantages, such as:}
	
	Any Quirk really..
	
	\textbf{Skills: } 
	Acting (A) IQ+2+1 [8] - 14; Carousing (E) HT+1+1 [2] - 12; Current Affairs/TL9 (E) IQ [1] - 11; Detect Lies (H) Per+0+1 [4] - 12; Diplomacy (H) IQ+1+1 [8] - 13; Fast-Talk (A) IQ+2+1 [8] - 14; Intimidation (A) Will+1+1 [4]; Oberservation (A) Per+0 [2] - 11; Panhandling (E) Fast Talk-2+1 [0] - 13; Politics (A) IQ-1+1 [1]; Public Speaking (A) IQ+0+1 [2]; Savoire-Faire (Any) (E) IQ+0+1 [1] - 12; Sex Appeal (A) HT+1+1 [4] - 12\footnote{Bonuses For Appearance not included.}; Stealth (A) DX+1 [2] - 10; Streetwise (A) IQ+2+1 [8] - 14
	
	
	\subsubsection{Magician}
	\begin{flushright}
		184 Points
	\end{flushright}
	
	This Magician represent for a 200 point individual with middling competency. Their Magic 3 provides them with skill 13 in almost every major magical role, Spellcasting, Summoning, Binding, and Assensing. This allows them to acquire a selection of capabilities that can make them a good generalist, but lack in specialist capabilities. 
	
	Their spells will be powerful enough to supplant or replace many technological measures (Especially invisibility), while their summoning and binding allow them to have a collection of mid-tier spirits around them at any given time. None of these are amazingly stellar - their combat spells will often deal 3d-3 damage (As much as a handgun), only held up by their special abilities of incendiary or armor-bypassing; their spirits will often amount to 75 points, able to specialize into some aspect of their powerful abilities, or to take somewhat generalist approaches. However, their abilitiy to bring things to the table that no one else can, especially Astral Projection, 3 spirits, Invisibility, Heal, and so on, make up for it.
	
	They will likely be lacking in many other skills without taking more disadvantages, which can limit the magician's usefulness in other aspects of a run. The player should make sure to acquire the minimum necesities for being in the Shadows regardless!
	
	Some players may compare this to the original Shadowrun and find it baffling that they can only take this much as a Magician, compared to their old awakened with 10 spells, multitudes of spirits, and Magic 6. This is specifially addressed in the \hyperref[gurps_awakened]{\GURPS Awakened section.}
	
	\textbf{Attributes: } IQ +1 [13]; Will +3 [21]
	
	\textbf{Advantages: } Astral Perception [12]; Astral Projection [20]; Binding 2 [6]; Magician [5]; Magic 3 [24]; Spark [1]; Summoning [5]
	
	\textbf{Disadvantages: } \textit{Choose [-15] points from appropriate mental disadvantages such as:}
	
	Disciplines of Faith (Mysticism or Ritualism) [-5]
	
	\textbf{Skills: } Assensing (H) Per [4]; Binding (H) MAG [4]; Spellcasting (H) MAG [4]; Summoning (H) MAG [4]
	
	\textbf{Spells: } Manabolt (Magic 3) [9]; Fireball (Magic 3) [7]; Heal (Magic 3) [52]; Invisibility (Magic 3) [8]
	
	
	\subsubsection{Street Samurai}
	
	This Street Samurai represents a classic version of the archetype for a 200 point individual. They can do some side roles of stealthing and climbing about or intimidating mooks, but their main focus is in direct melee competency, which when combined with TL9 equipment should be able to overcome most opposition short of professional security forces.
	
	They are still very generic, but keep to the overall archetype of a melee user with a strong code of some kind. This allows them to either branch into other roles such as Infiltrator or Face.. or lets them double down on their combat capabilities in order to deal with any threat short of an HTR team.
	
	Their cybware is relatively powerful despite its short list, focusing on core buffs to their attributes and capabilities that helps their overall capabilities. Players and GMs are heavily advised to edit this list to their own likings (The Punk is important in Cyberpunk after all!). adding or removing as they see fit - this selection just works for almost every idea for a generic Street Samurai. They have enough money to account for their ware if settled or wandering, however it can be prudent to lower it if playing a wanderer.
	
	Don't forget to include any SM modifiers to the ST and Cyberware in this Lens should you choose a larger metatype.
	
	\begin{flushright}
		100 Points; 13,000¥
	\end{flushright}
	
	\textbf{Attributes:} ST +2 [12]; DX+1 [20]; Will +1 [7]
	
	\textbf{Advantages:} Combat Reflexes [15]; Wealthy [20]
	
	\textbf{'Ware:}	Muscle Replacement 3 (Invasive, Base Grade) [15, 9,000¥]; Wired Reflexes 1 (Invasive, Base Grade) [7, 3,000¥]
	
	\textbf{Essence Loss:} \textit{Choose [-10] points for any suitable disadvantages, such as:}
	
	Berserk [-10*]; Bloodlist [-10*]; Callous [-5]; One the Edge (SC 15) [-7]
	
	\textbf{Disadvantages:}	\textit{Choose [-25] points from appropriate disadvantages such as:}
	
	Code of Honor (Professional) [-5]; Code of Honor (Bushido) [-15]; Code of Honor (Chivalry) [-15]; Disciplines of Faith [-5/-10/-15]
	
	\textbf{Primary Skills:}
	Acrobatics (H) DX+0 [4] - 12; Melee Weapon Skill (A) DX+2 [8] - 14; Stealth (A) DX+0 [2] - 12
	
	\textit{Choose one of:}
	Brawling (E) DX+2 [4] - 14; Boxing (A) DX+1 [4] - 13; Judo (H) DX+0 [4] - 12; Karate (H) DX+0 [4] - 12; Sumo Wrestling (A) DX+1 [4] - 13; Wrestling (A) DX+1 [4] - 13
		
	\textit{Choose [5] points from approriate Techniques, such as:}
	Combinations (MA80); Feint (H); Targeted Attack (Neck) (H); Targeted Attack (Skull) (H); Targeted Attack (Vitals) (H); Targeted Attack (Chinks) (H)
		
	\textbf{Secondary Skills:}
	Climbing (A) DX+0 [2] - 12; Fast-Draw (Knife/Sword/Two-Handed Sword) (E) DX+0+2 [1] - 14; Intimidation (A) Will+0 [2] - 11; Tactics (H) IQ+0 [4]
	
	\textit{Choose one of:}
	Saviore-Faire (Dojo) (E) IQ+1 [2] - 11; Savoire-Faire (Military) (E) IQ+1 [2] - 11; Savoire-Faire (High Society) (E) IQ+1 [2] - 11; Soldier (A) IQ+0 [2] - 10; or Streetwise (A) IQ+0 [2] - 10
	
	\subsection{Contacts}\label{Contacts}
	
	Contacts are a common tool for Shadowrunners to gain vital information and services for their jobs, however they can be somewhat deceptive in their usefulness in \GURPS, alongside requiring some rules from \GURPS Social Engineering - Keeping Contact (SE:KC) in order to function as expected in the setting.
	
	The source book also provides many modifiers for contacts that are extremely useful in fleshing out their abilities in order to better fit a player's ideas.
	
	Social Engineering also provides some useful resources, notably under \textit{The Benefits of Status} (SE59). As well, Pyramid \#3/47 Who's Gonna Buy This? covers very useful information for realistic fencing.
	
	\subsubsection{Contacts!}
	
	One of the more useful tools when creating contacts that is allowed is the new Contact! advantage. This lets a contact have a wide breadth of skills and resource (e.g. Business skills for a CEO) to a level similar to a Contact Group, allows access to some things limited to Contact Groups normally, but lacks the bonuses related to multiple questions or survivability. 
	
	It also provides a small number of \textit{Contact Points}, which allow a character to automatically succeed on FoA rolls, assist in related tasks, or so on. This should be generally kept on a somewhat tight leash, to avoid the advantage feeling too cinematic, even for the shadows.
	
	\subsubsection{Fixers}
	
	Shadowrunner groups often times share Fixers, who set the entire group up for their jobs. In cases like these, the GM should use the \textit{Sharing the Load} rules on SE:KC6.
	
	\subsubsection{Using Contacts}
	
	As noted in \textit{The Benefits of Contacts}, a Contact's skill is an abstraction and not necessarily the hard limits on their abilities. They should be freely sought after to provide Secret Information related to their profession (Very useful for organized crime members, corporation employees, infobrokers, etc.), Convenient Information (Often used with Infobrokers specifically), or Information Synthesis (Most useful for contacts skilled in technical fields). As well, don't pass up on Organized Knowledge for Group Contacts.
	
	Not all of these often fall under a strict skill. Instead, the skill should influence what type of information might be supplied by the contact.
	
	When players are asking for favors, it's highly recommended that the GM makes use of the \textit{Alternative to Rejection} (SU:KC13) optional rules, which allow for rolls to be made at penalties (or sometimes bonuses) based on the favor. This allows for penlties to skill or reliability to be applied instead of favors outright being denied, which lets the Contacts perform favors of a level of difficulty that would seem reasonable in Shadowrun, but would be unavailable in \GURPS.
	
	These penalties should be able to be counteracted through the \textit{Boosting Skill} (SE:KC18) section, allowing for bribes, extra time, or their own skills to assist with difficult favors. As well, \textit{Managing Risk} (SE:KC20) can provide similar benefits, allowing runners to trade favors in return for lowering penalties - a very common tactic for building a reliable network in the shadows.
	
	As well, players should remember to take advantage of the \textit{Common Skills} (SE:KC23) section, which describes a number of skills that contacts are considered to be able to make use of at all times, many of which can be very useful for a runner.
	
	\subsubsection{Example Contacts}
	
	A selection of Shadowrun focused contacts are provided here, however the \textit{Contact Categories} (SE:KC23) section provides a great list of inspiration for any characters making their associates. It also provides a number of rulings for specific categories that can enhance their capabilities (e.g. the Criminal/Street category allowing double value for monetary favors if the character is okay with being complicit with it being illegal).\\
	
	\textbf{Arms Dealer:} This Contact represents an Arms Dealer that is able to provide information regarding the legal and more often, less than legal weapons community alongside providing favors in regards for acquiring weapons. He is less helpful than a contact with Streetwise otherwise, only being able to use his capabilities in regards to the Arms Dealing community.
	
	\textit{\textcolor{OliveGreen}{Statistics: Contact, Streetwise (Skill 15; FoA 12; Somewhat Reliable; Less Helpful, Single Category, Arms Dealing, -60\% [2]}}
	
	\textbf{Fixer:} A well connected individual who sells a groups talents to Johnsons, acting as the trustworthy middleman between the two. When selected, the GM should specify whether the Professional Skill (Shadowrunner) or (Fixer) are necessary for the job. Many Fixers vary in the Frequency of Appearance, but they usually are not less than 12 and rarely less than 9, due to their job consistent of lots of networking. Many runners only use their Fixers for favors setting up jobs, but they can also be a valuable source of information on working in the shadows, especially when it comes to things like networking, etiquette, people of note, general advice, and - seeing as many Fixers were once runners themselves - general tactics.
	
	\textit{\textcolor{OliveGreen}{Statistics: Contact, Professional Skill (Shadowrunner) or (Fixer) (Skill 15; FoA 12; Usually Reliable) [8]}}
	
	\textbf{Infobroker:} A classic infobroker, this Contact! provides their skill for anything that might be related to general information gathering. This can be overly vague, so the GM is entirely within their purvue to provide more vague information than normal for other Contact!s.
	
	\textit{\textcolor{OliveGreen}{Statistics: Contact!, Information (Skill 18; FoA 9; Somewhat Reliable) [9]}}
	
	\textbf{Detective:} A small time detective, able to provide Forensics analysis, tamper with evidence, answer questions the runners may have about information they found, etc. He is limited in scope to portions of the city for which is police company has contracts, as such a Knight Errant would not be able to tamper with evidence gathered by Lone Star. For a relationship where the detective will willingly provide his assistance for tasks outside of his company's purvue, remove Limited Scope.
	
	\textit{\textcolor{OliveGreen}{Statistics: Contact, Forensics (Skill 15; FoA 9; Usually Reliable; Limited Scope, -50\%) [2]}}
	
	\textbf{Knight Errant:} You have some inns with the officers and detectives of Knight Errant (Or any other big time police company in the city), allowing you to call upon many of their members for information and favors. They are able to provide any information that a beat cop or detective might be able to about the organization, its general goals, investigations, etc. They can also provide information and services that any police group could, such as holding off patrols and responses, arresting certain individuals, looking the other way, tampering with evidence, etc. They are limited in effect to the areas on their contracts, unable to affect places that are under the control of other companies, such as Lone Star. For a relationship in which the group will step out of their own domain to help you, remove Limited Scope.
	
	\textit{\textcolor{OliveGreen}{Statistics: Contact Group (Organized), Police Skills (Skill 15; FoA 12; Somewhat Reliable; Limited Scope, -50\%) [10]}}
	
	\textbf{AA CEO:} This represents a big player, such as a CEO, CTO, etc. for a AA Corporation. They're able to call upon anything that their company reasonably could, from research, to supplies, to information, to jobs, to whatever - assuming you can get in contact with them of course, seeing as they're booked for the next week. They're notably capable of certain feats that a \textit{unusual} for their capabilities, whether this be access to powerful or large numbers of wagemages, secret R\&D technology, magical or matrix artefacts, or more.
	
	\textit{\textcolor{OliveGreen}{Statistics: Contact!, Business Skills (Skill 21; FoA 6; Somewhat Reliable; Unusual Connections) [8]}}
	
	\textbf{AAA CEO:} This contact is about as big as they come. They are some form of Chief Officer for one of the big ones, able to muster the expertise and resources of an entire megacorp, assuming they ever respond to your calls of course. They're even harder to get in touch with than their Frequency of Appearance would suggest, as all attempts to contact them are also two steps less convenient than normal (SE:KC19), meaning that their average convenience is \textit{Seriously Inconvenient}. However, when they come through, they come through. Like any AAA CEO, they're capable of certain feats that are \textit{unusual} for their capabilities, whether this be access to powerful or large numbers of wagemages, secret R\&D technology, magical or matrix artefacts, or more.
	
	\textit{\textcolor{OliveGreen}{Statistics: Contact!, Business Skills (Skill 24; FoA 6; Somewhat Reliable; Unusual Connections; Inaccessible 2, -20\%) [8]}}
	
	\textbf{Best Buds with Damian:} For the players that want to have Damian's personal number (Or any other key player in the setting), able to call upon them like any other contact, this is the place for you. Exactly as above, they're capable of certain feats that are \textit{unusual} for their capabilities, whether this be access to powerful or large numbers of wagemages, secret R\&D technology, magical or matrix artefacts, or more.
	
	\textit{\textcolor{OliveGreen}{Statistics: Contact!, Business Skills (Skill 24; FoA 12; Usually Reliable; Unusual Connections) [72]}}
	
	\subsection{Lifestyles}
	
	Lifestyles represent the costs associated with living. The rules for how to run this are already covered in \GURPS B265, but this section provides some context for it in the Shadowrun setting.
	
	\begin{center}
		\begin{tabularx}{0.49\textwidth}{|c|X|c|}
			\hline
			Status & Lifestyle & Cost of Living \\
			\hline
			\hline
			Status 4 & Luxury & 60,000¥ \\
			Status 3 & Luxury & 12,000¥ \\
			Status 2 & High & 6,000¥ \\
			Status 1 & Medium/High & 1,200¥ \\
			Status 0 & Medium & 600¥ \\
			Status -1 & Low/Squatter & 300¥ \\
			Status -2 & Squatter/Street & 100¥ \\
			\hline
		\end{tabularx}
	\end{center}
	
	When creating a character, the player must select between one of \textcolor{Blue}{\href{http://forums.sjgames.com/showpost.php?p=1792209&postcount=7}{two options}}:
	
	\textbf{Settled:} You have 20\% of your starting money for hand-picked personal gear that you'll use on runs. You also start with \textit{What Cost of Living Gets You} (B266) for your Status – even if 80\% of your starting money couldn't possibly cover that – because that abstracts a lifetime of accumulation.
	
	\textbf{Wanderer:} You have all of your starting money for hand-picked gear that you'll use on adventures. You do not get \textit{What Cost of Living Gets You} (B266). Moreover, spending cost of living each month does not automagically feed, clothe, and shelter you... you have to buy food, clothing, and lodging explicitly, out of whatever money you earn on your adventures, see Temporary Accommodations section on B266.
	
	GMs are highly recommended to enforce Settled Lifestyles for individuals with levels of Wealth of Wealthy [20] and above. 
	
	As well, it is possible for settled characters to have equipment that falls into both categories (e.g. a Rigger's vehicle, a Decker's microframe, and some reagents for a Magician are all covered under \textit{What Cost of Living Gets You} alongside counting as Adventuring Gear). In such cases, it's recommended for the GM to decide what aspects are adventurous and charge for those aspects only (e.g. Charge for the upgrades to a Rigger's vehicle or a Decker's microframe).
	
	A character with higher levels of Status and/or Wealth can talk to their GM about paying for some 'ware with their settled income. In such cases, the GM is highly advised to limit them to a reasonably sized selection of fully legal or licensed 'ware that is capped in its grade, that is not extremely expensive (up to 15/20\% starting wealth) which is explicitly useful for the character in a mundane lifestyle. 
	
	For most characters Status 0 or 1, this takes the form of things like Standard or Used Grade Datajacks, Sleep Regulators, perhaps some Cyber-replacement parts without many upgrades.
	
	If a character acquired 'ware through their settled income, they must still may the CP values as normal, but the nuyen prices are already abstracted out and payed for by their settled income.
	
	Often times, a player will talk to the GM about whether certain expensive equipment may fall under a settled income, and the GM might find it possible, but perhaps too valuable. A good example of this may be a rigger - especially one who say, had previous police work as a rigger - who wants to include a Control Rig in their settled income. The GM might consider the high price tag too much to reasonably provide, but should consider making comprimises. They can for instance, allow it at a lower grade than normal - if that would bring it low enough to be reasonable. As well, they can add additional disadvantages to compensate (That reduce point totals, not provide points), although they are cautioned against doing this too much. Most importantly, they can also remove other benefits from the settled income; the rigger gets his Control Rig, but in return his Status 1 house only has the security of a Status 0 house, or anything else the GM and player deem reasonable.
	
	While this can be difficult to balance for a specific character, the most important thing for a GM to keep in mind is to mainly balance among the party. While it may be a bit powerful to allow a Control Rig Rating 2 in return for downgrading the security of a lifestyle, it is nothing compared to allowing that while denying similar benefits to other players - \textbf{especially magical ones}, who should be given ample opportunities to include things like foci and reagents. If done so for Foci, these work similarly to Cybernetics, where one must still pay any CP cost, but can waive nuyen costs for their Foci. The most important part of any purely GM driven character design is to limit favoritism as much as possible (especially unconscious favoritism). GMs may find it valuable to attempt to enumerate the value gained from the equipment compared to the value lost from changing the lifestyle (or whatever method they use), and attempt to equal those out.
	
	When in doubt, remember that despite the fact that TL8 Wealth gives 20,000\$, most TL8 individuals own cars that cost around 20,000\$ (Perhaps used), mortgage houses, have decent support for their hobbies, and more. It's not unreasonable for settled individuals to have gross wealth that well exceeds the starting wealth shown in the Basic Set.
	
	Some cybernetics and equipment may be taken with or require the Maintenance trait. Character with suitably high Status and Cost of Living (usually at least Status 1)can include some or all of these costs in their lifestyle, otherwise the GM should reference B484 for the possibility of breakdown, continuing costs of repairs/maintenance, and so on.
	
	\begin{coloredbox}
	
	\subsubsection*{What Cost of Living Gets You in the Sixth World}
	
	Here are a couple examples of what you can expect to have at a given Status and corresponding Settled lifestyle (With Seattle security zoning included for convenience):
	
	\textbf{Status 4:} The character will have a large spread of houses and locations, with a common setup being a mid-sized mansion with grounds, a holiday home or getaway location, and a reasonable city apartment where they can stay for work or convenience. Security zones range from A to AAA, with their more permanent residences being more secure. 
	
	They will have more amenities than they could wish for, likely causing some absurd ones to pop up (Awakened Creature breeding anyone?). The best Matrix services, any membership or service they care for, any public technology that's not absurdly expensive, any private recreational service, plenty of very expensive hobbies such as fine-dining, yachts, luxury cars, and so on. They will likely have multiple teams of servants to manage their estates, which can cause problems in vetting them all as a Runner - they might need to hire someone to vet their staff! 
	
	They will have a wide variety of expensive vehicles, including a yacht, luxury cars, and their own private light aircraft. If they give up a decent bit they may even have a private/leased suborbital! Or it they want to get fancy they could reasonbly acquire things like helicopters instead (Don't forget none of this means the police will allow them to start flying in city limits, there's a limit to how much you can pay someone to overlook your stupidity).
	
	They can be expected to have any hobby. Hell at this point there's literally an in setting hobby that is faking being a Shadowrunner, so it's not even out of the question to buy adventuring gear with this (Although the GM is well within their rights to restrict it to overpriced and inefficient novelty items as opposed to actual Runner equipment, it's a hobby like Combat Sport (Karate) is to Karate in \GURPS). See Status 3 for some descriptions of practical equipment they might have. 
	
	They will of course, have a multitude of nominally illegal equipment from any category they want - as long as there's an option to pay off, they likely can acquire anything they have major interest in. They can expect one perfect SIN (Raying 6 Lifestyle) and a handful of throwaways (Rating 2-3 Weekly), or mulitple great ones (Rating 5 Lifestyle).
	
	Their 'ware and magical options don't agressively increase in capabilities. While it's reasonbly to allow them to go further, the GM should start \textit{aggressively} considering the implications in capabilities gained from this compared to the cost of Status, Wealth, and maintaining these two.\\
	
	\textbf{Status 3:} The character will likely have a small mansion to themselves, or alternatively a prime location midcity suite, alongside some more small properties that are used for convenience, holiday getaways, or more. Security will range from A to AA, perhaps even AAA in some cases, with their permanent grounds being more secure.
	
	They will have practically any amenity they wish for, even some of the absurd, including any fast Matrix services, every service or membership within reason, novel and gimmicky technologies, private recreational services and areas, multiple recreational things that people usually consider large financial decisions such as boats and cars, and so on. They will likely have a reasonable group of servants that tend to their needs, such as scheduling, maintenance, cooking, housekeeping, and so on - although for runners they will likely have less due to the added expense of vetting, bribes, increased wages, and so on to ensure close-lipped servants.
	
	They will have a number of expensive vehicles, ranging from luxury cars to yachts, alongside a handful of other transportation options. It's not out the question to have a private or leased jet, although it would have to be at the expense of other options. If their main hobbies involve these vehicles, they can easily support some more exotic upgrades, allowing them to ignore the "non-adventuring" rule to a certain degree.
	
	They can be expected to have any resources for any hobby that isn't straight up boogie nonsense, which can often overlap with runner specialties. The area itself will be extremely secure - although there tend to be decreasing returns as one gets more expensive without hiring straight up personal security - and usually includes private security for the community that includes astral patrols and ritual spellcasting security, security for their most important houses, services to clean up the neighborhood, technological suites that \textit{might} include nonsense like turrets, primetime response by officers, and so on.
	
	They can expect high quality equipment if they want - including Good Quality and Fine Quality, although even the rich don't get Fine one every little thing, only what they see as pretty important. This will include well stocked First Aid Kits and Supploes, any drug they'd reasonably want, any home defence weapons they wish for, plenty of emergency supplies (or just enough general supplies they double as emergency), expenmsive commlinks for each aspect of their life, a Microframe with many upgrades or a Mainframe with somewhat less upgrades. 
	
	Their equipment will almost certainly incorporate illegal items, because at this point fines can largely work as licensing fees to them. They can expect one great SIN (Rating 5 Lifestyle) and a handful of garbage ones (Rating 1-2 Weekly), or multiple good ones (Rating 5 Weekly or Rating 4 Lifestyle).
	
	They will have a great selection of 'ware, which might even include a bit of Betaware for certain hobbyists. Generally it includes mostly Alphaware with some Standard and Used thrown in alongside. At the GM's discretion they can include some Used Grade or possibly Standard Grade 'ware that ignores the restrictions on not being adventuring gear.
	
	Awakened will have any regeants they want that aren't of high quality alongside a well made Magical Lodge or multiple average ones in their residences. Alternatively, they might have some powerful Foci (Such as a Force 3 one), but will have to sacrifice in power if they want any that are more adventuring focused (i.e. Combat Spellcasting Foci). They might be able to mix these two options, allowing for some decent reagents, lodges, and petty Foci. \\
	
	\textbf{Status 2:} Depending on where the character lives, their residence can range from a large house with grounds to a prime location apartment to multiple small residences, likely all within A zones or better.
	
	They will have any amenity within reason, such as multiple fast Matrix services, every service or membership within reason, most any piece of novel technology - and some gimmick ones too, recreational services that require physical areas (such as a pool), perhaps some recreational things people usually consider large financial decisions such as recreational boats and cars, etc. Most people of this status have a small selection of servants that help with maintenance, cooking, scheduling and so on - however, for runners this number might be \textit{very} small or nonexistant due to the added expenses of vetting, bribing, better wages, and so on all to ensure their servants remain close-lipped.
	
	They have one expensive vehicle, which can either be simply Good Quality or have a number of legal modifications, alongside a small number of other decent vehicles. 
	
	They can be expected to have any resources to fund their hobbies that they want, which can often overlap with runner specialties. The area itself will be quite secure, likely including private security for the community that usually includes emergency astral response and possibly some ritual spellcasting security, security services for the house itself, technology suites to protect their property, primetime response by officers, and some services that are meant to "clean up" their neighborhood and prevent undesirables from accumulating. 
	
	They can expect quality equipment - most of which can be Good Quality and a select few can even be Fine Quality - including but not limited to: well stocked First Aid Kits and Supplies, most any drug they could want - legal or not, more food than could last them, any legal home defence weapon(s) they want, plenty of emergency supplies, an expensive commlink (Small Computer with Fast and more) alongside some other commlinks, a Microframe with many upgrades or even a Mainframe with a few minor upgrades. 
	
	The equipment can very easily incorporate illegal items, such as drugs, firearms, and especially fake SINs and licenses, for which the character would like have either one good SIN (Rating 4 Lifestyle) or multiple worse SINs (Rating 4 Weekly or Rating 3 Lifestyle).
	
	They will almost certainly have a good selection of 'ware, which often includes some Alphaware Grade, or could have a fairly expansive set of Standard and Used Grade 'ware. At the GM's discretion they might be able to have a small number of Used Grade 'ware that does not follow the rules aligned before in some way (Which often means it is a piece of 'ware that is not explicitly mundane).
	
	Awakened individuals could have fully functional lodges, including a fairly large amount of reagents or a selection of higher purity reagents. They may also include some equipment for things such as Alchemy or Enchantments. Alternatively, they may be able to include a decent powerful Foci (Force 2, perhaps 3), or multiple very weak Foci (Force 1), with a focus on some not-purely adventuring aspect of Magic. It may be possible to include stronger Foci such as\\
	
	\textbf{Status 1:} The character has a comfortable house or condo in an A or B zone, with any reasonable amenity they could want, including fast - or multiple - Matrix connections, plenty of services and memberships, a good selection of technology to make things easier - such as a labor drone, and so on. 
	
	They have a nice vehicle - which may have a few legal modifications - or perhaps multiple older ones. They can be expected to have expendable resources to support their hobbies well, which could overlap with some runner specialties. 
	
	The area itself will be fairly secure, through bribes to police and/or gangs, perhaps gated community, and their house itself might have a small selection of above average security resources. 
	
	They can expect a great selection of handy equipment - some of which may be taken as Good Quality - including but not limited to First Aid Kit(s), a good selection of drugs - legal and not, plenty of food of good quality, a good quality commlink (Small Computer, perhaps with Fast or High Capacity) and any specialized software they need, supplies for emergencies, a reasonable home defence weapon, they can easily afford a Microframe with some cheap upgrades or a possibly cheap Mainframe. 
	
	Some of the equipment can easily be illegal, such as some drugs, a firearm or two, and likely a fake SIN, which is likely of good quality (Rating 3/4 Lifestyle).
	
	They will likely be able to afford some standard grade 'ware, such as Wireless Datajacks, Sleep Regulators, etc or a somewhat larger selection of used grade 'ware.
	
	Awakened will also be able to fund some magical aspects, such as a decent selection of Foci and a small Magical Lodge built into their residence. Alternatively, they may have a weak Foci (Force 1-2), focused on some not-purely adventuring aspect of magic. \\
	
	\textbf{Status 0:} The character mortgages or rents a nice little house, condo, or apartment with a good selection of comforts, including common services such as Matrix connections, matrix services and physical memberships, average technology like autocookers, and so on. 
	
	Usually they will have a vehicle of their own, often a car or truck, in decent condition. 
	
	The area itself can be expected to be average security, not lacking in gangs, but well kept enough by police to be considered "safe". 
	
	As well, they can expect to have a good selection of equipment that may come in handy, including but not limited to: First Aid supplies, plenty of clothing and even some quality clothes, food of all kinds - sometimes even natural food, a commlink (Small Computer) and sometimes some specialized software, perhaps a small home defence weapon, possibly a Personal Computer or Microframe, general repair equipment for many items, hobby equipment, etc. 
	
	Some of the equipment will likely be illegal, such as some drugs, maybe a firearm, or they might have a decent fake SIN (Rating 3 Lifestyle) or multiple worse ones (Rating 2 Default).
	
	They may even have some Standard or Used Grade 'ware, such as a Datajack or Sleep Regulator, to make their lives more convenient.
	
	Awakened might be able to offset a small cost of reagents, just enough for a splurge of spells, or alternatively might have a shabby little Magical Lodge for their house. Alternatively to those, they might also have a very weak Foci (Force 1), focused on some everyday capabilities (strictly no adventuring ones, such as a Combat Spellcasting Focus).\\
	
	\textbf{Status -1:} The character is likely squatting in some place ranging from a small or shared apartment to a derelict house. 
	
	They can expect little in the way of amenities; eating nutrisoy, power and water during rationing periods, limited Matrix connection, etc. 
	
	The area itself is not explicitly dangerous, as long as the door is bolted and regular bribes are given. 
	
	They likely have no vehicle, but if they do it is poorly maintained or if not, stolen. 
	
	They may have some equipment of note: Improvised First Aid supplies, poorly kept clothing and perhaps one good pair, limited food and water, a poor quality commlink (Small Computer w/ Slow or similar qualities), etc. 
	
	While some of the equipment can be illegal, it's likely limited to drugs or maybe a firearm. If they're a bit lucky they might have a low rating SIN (Rating 1/2 Lifestyle) available for purchasing from Stuffer Shack's.
	
	The GM is free to assign conditions or penalties to characters living in such conditions, some examples being: Missing FP from missed meals, water, or sleep, penalties or issues arising when healing such as infection, burglers, etc. 
	
	If they're extremely lucky, they might have a Used Grade piece of 'ware.
	
	If they're extremely lucky, an awakened might have a couple reagents.\\
	
	\textbf{Status -2:} The character has - at most - a room in a flophouse or shelter, and is much more likely to have a tent in a D or Z Zone. 
	
	In terms of amenities, there are no amenities; food and water are whatever can be bought, scavenged, or stolen, protection from the elements is rudimentary at best, matrix access is nonexistant, etc. 
	
	The character only has themselves for security - and if they don't make enough to pay off the gangs will have to contend with those issues as well. 
	
	They will likely have the utter minimum in terms of equipment, including base toiletries and urban survival equipment, a very poor quality commlink (Small Computer w/ everything the GM wants to throw at it), limited resources, etc. 
	
	Much of their limited equipment is likely illegal in that it was stolen or scrounged since they both won't make enough, nor have a SIN (Unless they are a SINner) to pay for anything.
	
	The GM is recommended to assigns conditions and penalties as appropriate, with examples being: Missed FP from missed meals, water, and sleep, health issues from the environment, environmental effects such as pollution and acid rain, thieves and police harassing them, etc. \\
	
\end{coloredbox}
	\newpage % TODO: Make less arbitrary somehow
	\subsubsection{Magic and Technology}
	
	A lot of magic struggles to affect high tech equipment and individuals, causing great difficulty for magicians to do things such as Heal their injured Street Samurai. Most often, this is incorporated into the spells themselves, providing a penalty respective to their target's lack of essence. Optionally, individuals with low essence can be allowed to purchase Magic Resistance, allowing for them to have further generalized resistance. The GM should set the limit themselves, but about 1 level for every 20-30 points of cyberware is a decent heuristic.
	
	\subsection{Equipment}\label{setting_equipment}
	
	\subsubsection{\GURPS Pyramid \#3/55 - Military Sci-Fi}\label{3/55}
	
	The article \textit{Tactical Shooting: Tomorrow} updates many of the Ultratech firearm accessory systems to the High-Tech and Tactical Shooting standards. In general, all of it should be \textit{available} by default, however accessories are not immediately included on firearms by default, and should be decided upon by the GM. As usual, a list of items that should be included for the setting is provided:
	
	\textit{A Better Gun} covers some of the advances that TL9 firearms have over their predecessors, including electrical ignition, reduced moving parts (which lower volume), caseless cartridges, higher density magazines, ETC guns, Liquid-Propellant guns, and taggants. All of these are fitting for Shadowrun, except for ETC, Liquid-Propellant, and taggants.
	
	\textit{Malfunctions and Other Issues} covers the improved reliability of firearms.
	
	\textit{Camouflage} covers Chameleon Coating for weapons.
	
	\textit{Smartchokes} details how multiple-projectile guns (like shotguns) are fitted with auto-adjustable chokes.
	
	\textit{Lockouts and Tags} are sometimes in use, largely only for Military and Security firearms - and only ever through additional expenses. These options are generally available to be added to particularly paranoid runner's weapons. One part of note, is that due to the profitability of black ops for corporations, most forms of forensic aiding taggants are not included in their public designs.
	
	\textit{Handgrips and Stocks, Accessory Rails} both cover conventional changes for weapon ergonomics and accessories, and are easily included.
	
	\textit{Diagnostic Computers} explains the many benefits provided by the Diagnostic Computers that can be found in Smartgun Systems.
	
	\textit{Easy Hitting} is a critical overhaul to many of the Ultratech accessories that had fallen behind the norm of development and is a must to include. Most notably, this includes the updating of the HUD Link to provide +1 Guns within LoS (5,000 yards), negate up to -3 in darkness penalties when shooting, and provide both the benefits of sighted and unsighted shooting simultaneously. 
	
	It also details the rule specifics for using a HUD Link to fire around corners, details the effects of back glow on non DNI HUDs, details the HUD Link's devaluation of Masked Shooting (TS44), and covers how it stacks with the Targeting Software of Smartgun systems.
	
	\textit{Iron Sights} gives the option to save money by removing iron sights.
	
	\textit{Laser Sights} covers how TL9 technology combines with laser sights to provide increased capabilities such as Rangefinding, detection, lighting, and dazzling.
	
	\textit{IFF Interrogators} covers the bonuses to Situation Awareness (TS11) from IFF and TacNets.
	
	\textit{Targeting Scope} covers the specific usage of the Compact Targeting Scope (UT149).
	
	\subsection{World Design}
	
	\subsubsection{Security Devices}
	
	When designing security, it's important to have a good grasp of the tools at their disposal, how they are implemented, and how they are circumvented. GMs should look into Security and Surveillance (HT202) and Covert Ops and Security (UT93) for examples.
	
	\paragraph{Locks}
	
	Locks (HT203) are still a standard facet of security are are a topic in Pyramid \#3/47 Safes and How to Open Them. Whether it comes to their use as deterrents, slowing down, or simply keeping track of who has access to what, they are here to stay. 
	
	The average lock is Standard Construction and Basic Quality, providing locksmiths relative ease to bypassing it. Security focused installations (less common that you would think!) might make use of Good Quality locks with Tough Construction for those that are hard to monitor or catch people trying to break off. Fine Quality locks are very uncommon, simply because electronic methods tend to do it better - however sites that are paranoid of deckers may still implement these, or better yet mix them in! Sufficiently important locks are likely to trigger additional security devices, such as alarms, relockers (PY47:33), and so on. As well, some high Quality or large locks can have larger timeframes (PY47:34), such as an hour.
	
	The obvious way to circumvent these are to lockpick them using the Lockpicking (B206) skill, usually requiring a set of Lockpicks (HT213) or their modern counterpart the Electronic Lockpick (UT95). For a basic lock, a Lockpick Gun (HT213) is invaluable for saving time! As well, many locks are old, simply due to cost or the fact that they can just be left on. This can lead to Tech Level bonuses (HT203) or lowered HT (see below). Additional tools are found in Pyramid \#3/47 p35.
	
	For particularly important locks, it can be important to research their working (PY47:34), often by determining its make and model with Sense rolls, Observation, social engineering, Expert Skill (Locks and Safes), or Mechanic (Locks and Safes). Research can generally be used as well, although it will likely require a successful Sense or Observation roll to find info that is noteworthy to research. This can also turn up \textit{drill points} (PY47:35), secret locations intentionally or unintentionally left weakened that can be exploited to access the mechanisms, which are a great assist to lockpicking.
	
	For combination locks, brute forcing and guessing(PY47:34) is an option (given there is no penalty for trying), and can be assisted by Research (to determine likely candidates), Search (to find records of the code), or just by luck. Certain machines can automatically iterate through combinations, but often have to be built for a specific lock using Mechanic (Locks and Safes).
	
	Sensors can provide valuable assistance, such as fiber optic scopes or X-ray machines to image the mechanism (PY47:45).
	
	Brute force to the mechanism is a favourite option as well. Locks themselves tend to have low HP and reasonable DR if left outside accessible outside a reinforced barrier (such as a safe (HT203, PY47:33) or armored door (HT202, UT101)). If you wish to attack the lock itself, you should have Forced Entry (B196) and use specialized equipment, such as rams (HT29), frangible rounds (HT103, HT167, TS78), or at least a boot. Door Breaching from Tactical Shooting p24 covers how to do this without ruining the lock! Explosives work perfectly well too, making use of the Explosives (Demolition) (B194) skill; see Demolition (B415) for rules on calculating damage, Demolitions (UT88) and Explosive (PY51:3) for Ultra Tech explosive options (Including TL9 Thermite!), and the Explosives and Incendiary (HT181) section. Make sure to use explosives designed for the job! 
	
	\paragraph{Electronic Locks}
	
	Electronic Locks or Maglocks (HT204, UT102) are relatively standard locking systems that rely on digital authentication rather than and physical key. This is very beneficial for ease of use and maintenance, however it can also open them up to other vulnerabilities.
	
	These locks come in the same grades as Locks, with the addition of Simple and Complex locks from Ultratech (The GM should choose which he cares to use). They tend to be more expensive themselves, but their saving in maintenance and administration tend to make up for that, leaving locks that are not as old or outdated alongside being easier to secure.
	
	The straightforward way to bypass an electronic lock is to tamper with the circuits controlling it. This requires some method of gaining access to them, usually popping open a case, however that might not be easy for every model. With access, an individual can use Electronic Repair Tools (HT23, UT82). an Electronic Lockpicking Kit (HT213) or its modern brother the Electronic Lockpick (UT95) to perform an Electronics Operation (Security) (B189) to bypass the circuit. Particularly valuable circuits will have anti-tamper systems, which can provide effects such as penalties to checks (HT), requiring a second roll to avoid setting off the alarm (HT206) requiring being disabled first, or simply triggering systems on failures.
	
	Indirect methods work extremely well here, depending on the type of lock. Keycard locks can be fooled by cloning the Keycards of individuals. This often requires knowledge of the type of keycard, sometimes obtainable through Observation, Expert Skill (Computer Security), social engineering, Electronic Operation (EW or Security) (To simply detect the format yourself), or suitable Research rolls from bits of information. Then, a tool can be set up to clone keycards within a small range using Electronic Operation (EW or Security) (For store-bought tools), Engineer (Electronics) (to put one together), requiring suitable Stealth, Savoire-Faire, and so on (to get close enough), etc. Skimmers are another route, which are used to cover the actual Keycard Scanner and scan the keycards alongside them, requiring suitable Engineer (Electronics) (to design and make it) alongside a suitable skill for hiding the appearance of the Skimmer.
	
	Keypads are easier. One can simply watch an individual put in their PIN and write it down. Some keycards have shields blocking sight to them. You can also wipe the screen clean, wait for someone to enter, and then dust for fingerprints (A Forensics roll at +4 or more) in order to determine what numbers were pressed - and try to work it out from there. Skimmers can be put over the keypad as well, reading the button presses while transferring them to the real system below, requiring suitable Engineer (Electronics) (to design and make it) alongside a suitable skill for hiding the appearance of the Skimmer.
	
	Biometric locks (HT205, UT104) are the most difficult, requiring things like fingerprints, retina or face scans, or sometimes even more invasive tests. Biometric Cracker Tools (UT95) are invaluable here, providing sensors and decoders that can be used to scan an individual's biometrics and implement them in ways to spoof the lock. For Fingerprint and Retina biometrics (HT205), an Electronic Thumb (UT96) is a perfect way to bypass, although more mundane ways work on cheap versions (Breathing on fingerprint scanners to scan the last fingerprint for instance.
	
	Secure locks will require multiple of these systems, making accessing the circuits a much more viable option. The best electronic locks will make use of two or even three of the following categories: Something you know (A PIN), Something you have (A keycard, phone, or fob), and Something you are (Your body).
	
	Failure to bypass electronic locks can have more dangerous consequences. Their digital nature makes it much easier to sound the alarm, often through hard-wired control systems - because wireless ones would be very vulnerable to deckers, although they are still extremely convenient and ergo still used. They can also record logs of who opened them and when, which can prove very suspicious to anyone watching them. For extremely secure facilities, just opening a door might be reason to investigate. At the same time, being interconnected with the facility can make them easy prey for Deckers who have hacked a host, allowing control, spoofing, disabling, and more of any locks that are even hard-wired to the host's systems.
	
	Lastly, all the same brute force methods for normal locks apply here! In addition, a common method is to cut the power! Most locks are necessarily "fail-safe" locks, meaning that they open with a lack of power, due to emergencies like fires. Some extremely secure doors, such as prison doors, are "fail-secure" and will stay shut when power is lost, which is very dangerous!
	
	\paragraph{Doors}
	
	Doors are as important a consideration to security as locks are. Having an extraordinary and fancy maglock means nothing if the opposition simply knocks out the hinges on your door. Additionally, armored doors (HT202, UT101) or safes (HT203) can be used to greatly improve the security of the door itself.
	
	Doors are homogenous objects with varying DR and HP based on their size (B558). Their common materials are plastics, woods, and metals, for increasing levels of durability and cost. 
	
	Hinges should be responsibly placed on the interior to prevent removing them (Requiring only a small metal stick and hammer-like object) or shooting them (TS24), although it's still commonplace to have them anywhere. Some doors that are required to have hinges externally (usually for space considerations) may include covers for them, weld or secure their pins inside, or provide other security to protect them. As well, even if they are inside, if the door is not made of a strong material, one can still simply shoot through to door to hit the hinges (With blind-fire penalties of course).
	
	Many doors are sliding instead, removing the issue of hinges, but inserting one of both power and control. The system that controls the door can be hacked (if wireless) or spoofed as normal. One of the most common system is an IR Motion Sensor that automatically activates the door, often only placed on the secure side. If there are any opening, an intruder can wave tools through or spray smoke behind the door to trigger such a sensor. 
	
	By far, the easiest way to bypass doors is through social engineering. Because they are a constantly used public tool, piggybacking behind people with access is a common technique, requiring a simple Savoire-Faire roll. Obstructions can be placed so that they prevent the door from shutting, although secure doors can have warnings that trigger if they fail to shut entirely.
	
	An indirect method is to simply ignore the door and go for the wall. If your intention is to destroy the door to enter, make sure that the wall is just as strong as it, because a metal door won't be near as strong as a brick wall to its side.
	
	\paragraph{Fences and Walls}
	
	Security Fences (HT204)
	Wires and Fences (UT102)
	
	\paragraph{Portal Scanners}
	
	Instrusion Detection Devices (HT205)
	Screening Systems (HT206)
	Surveillance Sensors (UT104)
	Portal Scanners (UT104)
	
	\paragraph{Scanners}
	
	Instrusion Detection Devices (HT205)
	Surveillance Sensors (UT104)
	
	
	\paragraph{Traps}
	
	Traps are an extremely uncommon form of security, for the simple reason that they don't discriminate. Most any place of important has people that are working there are creating the things of import, meaning that leaving claymores, buzz saws, automatic turrets, and so on can be extremely dangerous because \textit{somebody.. eventually.. \textbf{will} be stupid or unlucky enough to set them off}.
	
	They should only be used in cases where the risk of intrusion \textit{heavily} outweighs the threat to workers (MCT Zero Zones), the area where they are set up does not see traffic (Faux sites, fake entrances, honeypots - however these are still risky as people wander!), or are extremely good at discriminating (Keycard and facial recognition based turrets that are fail-safe).
	
	The only real professional option are traps that are manually triggered in emergency situations. Some common examples of these are raising bollards or nets to stop cars, Electromagnetic Car Stoppers (HT203), closing security doors, etc.
	
	\subsubsection{NPCs \& Power Levels}
	
	Shadowrun and \GURPS both leave the fine parts of balancing up to the GM, giving general guidance over something as complex and useless and Challenge Ratings. In general, the most useful metric is simply how competent and well paid the bad guys are, described by a Professional Rating from 1 to 6. 
	
	Below is a list of examples, however keep in mind that most NPCs \textit{will not} be as focused on Shadowrunning or Security as the player characters are - even those in the business! Make sure to not put all of their points towards being an obstacle to the players, as they have lives, mundane 'ware, hobbies, wealth, social advantages, etc that are less present in the SINless. If you find it so helpful, I have included Pyramid \#3/77's Combat Effectiveness Rating, although note that I find it highly variable and largely flawed.
	
	While one might looks at the \textit{Power Level} section (B487) in order to either compare or create their own ranges, there should be some additional concerns to keep in mind! TL9, even in a cyberpunk dystopia, will naturally have higher point ranges than one's intuition of TL8. 
	
	In the same way that a TL8 character receives better education, nourishment, opportunities, and so on than a TL7 character, the same is (partially) true for TL9, especially when one includes cyberware and bioware which can each cost as much as a small advantage. All of these qualities cost points, and that naturally raises the average point level. This is combined with the relatively cinematic world of Cyberpunk and Shadowrun, which naturally raises the points as well.
	
	For a hyperbolic example, a TL0 hunter gatherer might have some decent ST, HT, and perhaps even DX, from training, but will have a very limited selection of skills, training, wealth, so on. This means that the hunter gatherer will have a much lower point count.	
	 
	 Of course, all of these are before this document raised most attribute costs, which also should increase the Power Levels. All of this together is why each PR category seems to be around 1 to 2 categories \textit{above} their respective Power Level categories (e.g. Beat Cops being ~150 Points here and 50-100 points in Power Levels).
	
	\begin{itemize}
		\item PR 1 (~50 Points): Street Trash; This consists of muggers, wageslaves, and so on. Runners should be able to take on large numbers of these individuals in competition.
		\item PR 2 (~100 Points): Mooks; These are poorly trained, but at least experienced individuals when it comes to things. These might be Script Kiddies, Corpos or College Students, Gang Enforcers, Street-level Runners and so on. They can be dangerous in larger numbers, but are overall speed bumps.
		\item PR 3 (~150 Points): Professionals; These are individuals that have either received large amounts of training, large financial backings, or a mild mix. This can be Beat Cops, Organized Crime Members, Security Guards, Social Defender Staff, Scientists, many Contacts, and so. While these people aren't major threats individually, they are often well organized and supported by security layers and backup, making them genuine threats if not approached carefully.
		\item PR 4 (~200 Points): Experts; These are individuals who have either tons of expert training and experience, lots of financial backing and resources, or more often some milder mix of the two. This is corporate private security, Shadowrunners, leading scientists, diplomats and leads, DemiGODs, etc. These people can pose genuine threats to runners in an individual situation, however the GM should note that - while they are the same point levels - these NPCs should not be as competent as runners (e.g. make sure to set aside a portion of points for their non-security lives!) without security to back them up.
		\item PR 5 (~250 Points): Specialists; These are the individuals that can match or exceed runners. These people have tons of expert training and experience \textit{and} the financial backings of powerful players. Some Special Operators, some Corporate Black-Ops, leading R\&D scientists and engineers, high ups on the food chains, GOD, many infected, so on. When combined with security they can present an extreme challenge.
		\item PR 6 (~300+ Points): Pinnacles; These are individuals that are meant to challenge part of - if the the whole - runner team. Prime Runners, named individuals or organization such as the Red Samurai, the high ups higher ups, tech leads for GOD, and even worse.. like drakes and free spirits. These individuals are centerpoints of security - or are flags to tell your players they're in over their heads!
	\end{itemize}
	
	\subsubsection{Example NPCs}
	
	\subsubsection*{Special Operator, PR 5}
	\begin{flushright}
		235 points, 141,000¥, CER: 130.5 (85/45.5)
	\end{flushright}
	
	\textbf{Attributes: }
	ST +1(+2) [6]; DX +2(+1) [40]; IQ +2 [26]; HT +1 [13]; Per +2 [10]; Will +2 [14]; HP +1 [2]; FP +1 [3]
	
	\textbf{\\Advantages: }
	Combat Reflexes [15]; Fit [5]; Military Rank 2 [10]; Patron (Megacorp, FoA 6) [30]; Status 1 [5]; Comfortable [10];
	
	\textbf{\\'Ware: }\footnote{CP:¥ Ratios may vary, as I didn't consider heavily the GM dependant features such as invasiveness.} 14 TODO: Ware Reprices
	Cybereyes (Deltaware; Protected Vision; Nictating Membrane 4; Hyperspectral Vision; Acute Vision 1) [2, 24,000¥]; Dermal Plating 2 (Betaware; Steel) [1, 15,000¥]; Muscle Replacement (Deltaware, Invasive) [6, 42,000¥]; Wired Reflexes 2 (Deltaware) [3, 36,000¥]; Wireless Datajack (Deltaware) [2, 24,000¥];
	
	\textbf{\\Disadvantages: }
	-50 points chosen from the following:\\
	Bad Temper []; Bloodlust [-10*]; Code of Honor (Soldier's) []; Code of Honor (Officer's) [-10]; Code of Honor (Professional's) []; Fanaticism []; Honesty []; Sense of Duty [];
	
	-7 points from Essence Loss Disadvantages:\\
	Bad Temper [-10*]; Bloodlust [-10*]; Fearfulness [-2e]
	
	\textbf{\\Perks: }\footnote{Pretty much all of these perks are from Tactical Shooting.} 3 
	Style Perk (Assaulter) [1]; Battle Drills [1]; Barricade Tactics (Rifle) [1]; Cool Under Fire [1]\\
	
	\textbf{Primary Skills: }
	Armoury (Small Arms) (A) IQ-1 [1] - 11; Brawling (E) DX+1 [2] - 14;Climbing (A) DX+1 [4] - 14; Explosives (Demolition) (A) IQ+1 [4] - 13; Explosives (EOD) (A) IQ-1 [1] - 11; Fast Draw (Pistol) (E) DX+1 [1] - 14; Fast Draw (Ammo) (E) DX+1 [1] - 14; Forced Entry (E) DX [1] - 13; Guns/TL9 (Pistol) (E) DX+1 [2] - 14; Guns/TL9 (Rifle) (E) DX+3 [8] - 16; 8 points chosen from other Guns and Gunner specialties; Shield (E) DX [1] - 13; Soldier (A) IQ+1 [4] - 14; Stealth (A) DX+1 [4] - 14; Tactics (H) IQ+1 [8] - 13; Throwing (A) DX-1 [1] - 12; Traps (A) IQ [2] - 12; \\
	
	\textbf{Secondary Skills: }
	Acting (A) IQ-1 [1] - 11; Body Language (A) Per-1 [1] - 11;  Camouflague (A) IQ [2] - 12; Driving (Automobile) (A) DX-1 [1] - 12; Electronics Operation/TL9 (Communications) (A) IQ [2] - 12; Electronics Operation/TL9 (Security) (A) IQ [2] - 12; First Aid/TL9 (E) IQ [1] - 12; Hiking (A) HT [2] - 11; Lockpicking (A) IQ-1 [1] - 11; Navigation (Land) (A) IQ [2] - 12; Savoire-Fair (Military) (E) IQ+1 [2] - 13; Savoire-Faire (Corporations) (E) IQ+1 [2] - 13; Search (A) Per-1 [1] - 11; Streetwise (A) IQ-1 [1] - 11; Urban Survival (A) Per [2] - 12;\\
	
	\textbf{Techniques: } 17
	Close Quarters Battle (Guns (Pistol)) (A) Guns (Pistol)+1 [1] - 15; Close Quarters Battle (Guns (Rifle)) (A) Guns (Rifle)+4 [4] - 20; Immediate Action (Armoury (Small Arms)) (A) Armoury (Small Arms)-3 [1] - 8; Quick Shot (Guns (Rifle)) (A) Guns (Rifle)+0 [6]-16; Targeted Attack (Skull) (H) Guns (Rifle)-3 [5]-13\\
	
	\textbf{Equipment: }
	Full Body Armour w/ Plates; Ares Alpha
	
	\subsection{Not My Cyberpunk}
	
	One of the likely possibilities you have when reading this document is a gut reaction along the lines of: Why does this have more crunch than Shadowrun itself? Or perhaps: why does this aspect of the game not play like the original? Maybe even: Why does it seem so hard to go full Pink Mohawk on these drek-heads? \GURPS is a toolkit, not necessarily a system nor setting in its own right, which means that each decision and rule generally has to be included by choice in order to create the campaign that is desired. While some of these are very simple (e.g. include the Monowire superscience tech, because monowire is in the setting), others are not quite so much (e.g. do you include Precision Shooting from Tactical Shooting?).
	
	Some of these grey cases are simply a limitation of porting over a system, which cannot always be done in a one to one fashion without.. well just playing Shadowrun instead of \GURPS Shadowrun. This is definitely the case with many core loops for archetypes; Deckers and Technomancers have a more realistic hacking system that requires program management instead of MARKs, riggers have to manage command lengths and Tactics rolls instead of PAN weirdness, Awakened have lower power caps and some very different spell and ability effects, and cyberware requires a separate system of managing ratios of points to money. The goal with such design is to maintain the core identity of the rulesets, while trying to recreate the best rules in the \GURPS system - which is not always an easy process. Instead of blindly bringing over everything from Shadowrun, this document is an attempt to take its meaning, the parts that gave it style and fun to play, alongside incorporating the lore and ideas of the setting and genre.
	
	Additionally, some of these rule choices shoot for specific sub-genres or playstyles of Cyberpunk. In terms of the scale of Black-Trenchcoating to Pink Mohawk, I fall quite far on the Black-Trenchcoat side (Surprising I know). My rule choices reflect a setting with high consequences alongside rewarding detailed planning and execution. This generally translates to the use of supplements such as Tactical Shooting and Pyramid rules such as Do-or-Die Bullet Dodging. Those rules have given this adaptation a feeling very reminiscient to some earlier editions of Shadowrun - or even more so the cyberpunk novels of the Sprawl Trilogy (or Neuromancer if you've heard that one before) - which has a much gloomier and pessimistic feel with the struggle to resist and find oneself amongst the technological woes of a society a difficult problem. A good portion of people do not want that, they want Pink Mohawk - a sub-genre full of bombastic revolution and little consequences, focusing on action and feeling more so than what version of software the corp's server is running. These ideas can conflict with the rules I've set out here, but that's not actually an issue with \GURPS itself, it just takes ignoring some rules and including some others.
	
	So, what happens if that's Not Your Cyberpunk? Of course, the easy answer is to simply do what you want, keep the good, and toss the bad - but that's a pretty daunting task in many ways, so I have a collection of sections here on how to begin changing this conversion to your likings.
	
	\subsubsection{Out with the Black, in with the Pink!}
	
	An often cited, but mostly incorrect, idea when looking at \GURPS is that it is only able to support simulations, high consequence, and/or crunchy settings and play - which of course lends itself heavily to Black Trenchcoat style of play. However, by incorporating rulesets from other \GURPS material and ignoring the parts that aren't useful. While I can't and won't make an entire separate ruleset, I will at least provide some guidance for such a playstyle:
	
	\textbf{Rules to possibly \textit{not} include:}
	This section includes the rules that I've recommended are not particularly conducive to Pink Mohawk gameplay. Of note, this does not really include many crunchy or detailed rules, such as Advanced NVGs or Expanded Influence Rolls. The core of Pink Mohawk is to have a setting with low enough consequences to empower players to make fun, dynamic, and risky choices; crunch is in now way antithetical to that, but many GM should also consider removing those rules regardless if it suits them.
	\begin{itemize}
		\itemsep 0pt
		\item Bleeding (B420)
		\item Stopping the Bleeding (HT162)
		\item You Shot Me, Mister! (HT162)
		\item Armor Fatigue (LT101)
		\item Harsh Realism - Armor Gaps (LT101)
		\item Extreme Dismemberment (MA1369)
		\item Almost the entire Tactical Shooting book.
		\item Do-or-Die Bullet Dodging
		\item Certain Cinematic rules:
		\begin{itemize}
			\itemsep 0pt
			\item Many skills available through Trained by a Master/Weapon Master/Gunslinger could be considered \textit{too} cinematic (e.g. Breaking Blow). If disallowing such skills, consider lowering the cost of these advantages, especially Trained By a Master.
			\item Enthrallment skills are generally far too cinematic for the setting.
			\item Musical Influence skill is generally too cinematic for the setting.
		\end{itemize}
	\end{itemize}
	
	\textbf{Rules to possibly include:}
	\begin{itemize}
		\itemsep 0pt
		\item The Cinematic Campaign section and rules of Basic Set and Action
		\begin{itemize}
			\itemsep 0pt
			\item Cannon Fodder
			\item Cinematic Knockback
			\item Flesh Wound
			\item Mook Markmanship
			\item Super-Silencers
		\end{itemize}
		\item \GURPS Action
		\begin{itemize}
			\itemsep 0pt
			\item BAD (A:2 4) (I don't particularly like this system, but it definitely lowers crunch.)
			\item Getting the Ball Rolling (A:2 6+) (A series of quick rules for accomplishing many classic goals.)
			\item Simplified Falling Damage from Action 2 p19
			\item Shooting Made Easy section from Action 2 p36.
			\item It's Better to Be Lucky from Action 2 p41.
			\item Ten Rules to Use Sparingly from Action 2 p44.
		\end{itemize}
		\item Consider allowing cinematic advantages for:
		\begin{itemize}
			\itemsep 0pt
			\item Gizmo
			\item Gunslinger (Don't forget the benefits in Gun-Fu!)
			\item Weapon Master
			\item Social Chameleon	
			\item Trained by a Master (Don't forget the benefits in Martial Arts!)
			\item All Enhanced Defenses (Including Power ones)
			\item Normal Gadgeteer (Especially H4xx0r), and \textit{perhaps} Quick Gadgeteer
			\item Wildcard Skills (If running a more rules-light setting, although it can easily destroy archetypes and specializations.)
		\end{itemize}
		Abstract Wealth from Pyramid \#3/44
	\end{itemize}
	
	\textbf{Possible changes to this document:}
	\begin{itemize}
		\itemsep 0pt
		\item Raise point levels to at least 250 or 300 (Don't forget to re-assess Spirit costs, although you can eyeball it without much issue). 
		\item Raise maximums for attributes to 50\% at any time and 70\% with GM approval, alongside 80\% and 100\% for ST.
		\item Raise maximums for Special Attributes like Resonance and Magic to 8.
		\item Be very lenient with Rigger command durations, generally Free Action for most cases or a single Ready maneuver otherwise.
		\item Use the Action 2 p13 rules for hacking as opposed to the Cyberpunk article.
		\item Use the Ultratech Trauma Plates instead of the ones designed here.
		\item Allow the use of ETC (Especially if you find Weapon Master makes melee too powerful).
	\end{itemize}
	
	\subsubsection{\GURPS Awakened versus Shadowrun Awakened}\label{gurps_awakened}
	
	One of the things that many players will quickly come to notice is that their Awakened characters are \textit{much} less competent (or at least have less options and lower numbers) than in the original Shadowrun game. This is by design. 
	
	There was a reason that Shadowrun was called Mage-Run; Awakened had access to superior options to most other characters at the table: spirits were extraordinarily open to abuse - even without being built into - spells provided wholly unique capabilities which were sometimes literally impossible to counter (High Force mind control for instance), magic as a whole has very little mundane counters beyond Gray Mana, line of sight blocking, or a bullet, and so on.
	
	Because I have tried to maintain as close as I can to the original game in this port, only comprimising where it either did not fit the \GURPS system or would make things feel overall better - Awawkened are \textbf{expensive.} The capability to summon near endless minions with Summoning, easily scout far away and secure locations with Astral Projection, perform budget mind reading with Astral Perception (which also countered many forms of stealth), create spell effects ranging from humongous area death bombs, to near unresistable mind control or debilitation, to near unstoppable effects... those are all things that could define a \textit{single} character in \GURPS.
	
	This is additionally exacerbated by \GURPS' dice system, meaning that have a default of Magic 6 would likely mean a skill of 16, which is extremely high in combination with the already expensive systems. From my limited idea, it seems that Magic 3 or 4 in this port is the equivalent in terms of intra-party competency to Magic 6 in Shadowrun.
	
	Lastly, a lot of things are simply much more potent in \GURPS due to the game system, largely due its emphasis on realism. In Shadowrun, it's not difficult to make a character that survives a handgun - or a handgun level Combat Spell. In \GURPS, handguns are an existential threat, even to people with armor. Non Combat Spells also have much more powerful assumptions and capabilities - Invisibility provides +9 to stealth, Detection spells explicitly give more information, and so on. While these are capabilities that were possible in Shadowrun, they often did not have these explicit and extremely powerful modifiers and effects layed out for the player to use and abuse.
	
	All around, players should expect to either make much more tamed generalists or much more specialized Awakened compared to Shadowrun. As well, don't forget that players can take magical advantages at a level lower than their Magic, potentially allowing for large savings on points.
	
	As an aside, this perhaps helps with one of the stranger questions in Shadowrun, being the oddity of tons of highly trained Awakened individuals, demographically amounting to under 1 in 10 million people, always making their ways into the shadows in large numbers for some reason. Instead, runners would now keep to the above average Magic levels of 3 and 4, with some savants and specialists mixed in. They are still competent enough and mandatory on many runs, but they aren't the equivalent of PhD Professors or Olympic Athletes turning to a life of crime in staggering numbers.
	
	Nevertheless, if you \textit{really} loved your Mage-Run, there are still some ways you can probably make it back. The first warning is that you should probably not mess with the points values in the Awakened sections by their lonesome. They're all based off of genuine \GURPS Point calculations, so you'd only be giving an imbalanced edge to Awakened - which is ill advised unless you want an All Magesmen Party or simply think that the point costs are wrong, in which case go ahead! 
	
	The easiest fix is of course, to simply increase point totals. This brings everyone up higher and mages do tend to disproportionately benefit from higher point levels. Alternatively, you can go down the route of \GURPS Sorcery and consider Magical Advantages as Alternative Abilities to even more things, perhaps Magic or perhaps each other, whichever way you can reasonably justify to lower point costs more. You can also allow additional limitations on Magery and spells in order to eek out those additional points, In a similar vein you may allow for the bundling of tradition-specific advantages and disadvantages into a meta-trait for that tradition, which can often reduce the overall amount of disadvantages and allow for more to be taken before the cap!
	
\end{multicols}