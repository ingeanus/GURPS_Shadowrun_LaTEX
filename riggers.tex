\section{Riggers}\label{riggers}
\begin{multicols*}{3}
	
	\subsection{Rigging and You}\label{rigger_rules}
	
	Riggers make use of \hyperref[control_rig]{Control Rigs} in order to jump into drones. This provides them with a wide variety of "bodies" to use for runs, while also putting a layer of separation between them and the real world.
	
	To jump in he must win a quick contest of IQ versus the systems Will. Most system's have a Will of Complexity\(\times\)2. If this roll fails, he is unable to attempt to jump into this system for 24 hours, indicating either incompatibilities, errors, or being locked out by security. If the system knows it is under attack or otherwise on guard, it adds +5 to its defense. Systems owned by the Rigger may choose (or be directed) to not resist.
	
	When jumped in, his body is unconscious - while his mind has complete control of the vehicle he has jumped into. The Rigger uses the physical statistics of the vehicle, while maintaining his mental attributes and skills alongside the relative skill level of his physical skills.
	
	While jumped in, he does not have any special access to the system's memory, although his prior access to the system does allow him that.
	
	Because of the close link between the rigger and the system, damage to the system can cause lethal biofeedback to the rigger. Whenever the system takes damage, the rigger takes an equal amount of burning damage. He can resist this by making an HT check, lowering it by his Margin of Success to a minimum of 0, adding a +3 bonus if his RCC or Control Rig are hardened (Depending on whichever he routes his traffic through). As well, if the drone "dies" from damage, the rigger must resist the higher of 3d burning damage or the damage the drone took, resisting with HT as usual. 
	
	While jumped in, the rigger benefits from a Talent level equal to his Control Rig rating×2. This provides a bonus to all rolls to "rig" well, which include control, piloting, sensor, and mounted weaponry rolls, among others.
	
	\subsection{Drones}
	
	Drones are the tool of choice for Riggers. They can be found in the \hyperref[drones]{Drone section of Equipment.}
	
	\subsubsection{To Ally or Not?}
	
	Because drones are not singular advantages, the tactics used for purchasing 'Ware by converting the CP cost to Nuyen would be extremely expensive and impractical, easily ending up in north of 100,000¥. As such, one of the obvious solutions that might come to mind is buying a drone as an Ally with Minion, +50\%. However, drones are common and standardized equipment that anyone can purchase (or at least, anyone with a license) - meaning that they should cost straight Nuyen, not be built as an advantage or bought with CP. As an example, would you require all characters to buy their vehicles as allies? It is somewhat different for AI, since their bodies can only be drones or computers, but for a Rigger they fall much closer to normal characters, and should purchase drones with nuyen.
	
	The question then is, how much should they cost? One might assume that, similarly to many other parts in this book, creating the drone as an advantage (notably as an ally) and then converting that to nuyen may produce good results. However, given that most drones with their pilot programs fall into the 25\% CP Total for allies, almost every drone would be 5 Points, ergo 15,000¥. This is much higher than normal in Shadowrun, and also homogenizes most of the drone's values.
	
	Even expanding the ally costs, as noted under GURPS Social Engineering p42, does not help this very much, especially considering that bodies and pilot programs should be bought separately in the first place! As such, the recommendation is to simply prices via fiat, using Shadowrun's prices as a gold standard compared to GURPS' starting wealth. These are what the prices provided are designed from.
	
	\subsection{RCCs}
	
	Rigger Control Consoles are computers that are specialized for managing and coordinating large numbers of drones at the same time. While they are not mandatory for riggers like a control rig is, they are still an extremely common sight, as they allow for better control of the drones that a rigger is not currently jumped in to.
	
	RCCs are very similar in design to Cyberdecks, in that they are computers that have a built in program for x1.25 cost - that program being a TacNet (UT149, PY55:31). This necessarily limits Cyberdecks to complexity 5 and above. TacNets provide the usual bonus of +1/+2 to tactics and allows for the rigger to issues commands to his drones as any commander would to his troops. Notably, the riggers is not restricted to only drones in his Tacnet, and can include his teammates who may also benefit from it.
	
	Pyramid \#3/55 also notes some additional benefits and options for TacNets. Notably, there is the inclusion of a TacNet Server program (p17) for a Complexity 7 computer, which provides a +3 to tactics and may be a useful option for some riggers. As well, it provides advanced rules for TacNets (p31) that GMs should consider whether to use or not:
	
	\begin{itemize}
		\itemsep 0pt
		\item Allowing TacNets to be used for complimentary bonuses to Area Knowledge, Camouflage, Expert Skill (Military Science), Intelligence Analysis, and Strategy with a skill level of 12 (Complexity 5), 14 (Complexity 6), or 16 (Complexity 7).
		\item Adding an equipment bonus to Situational Awareness equal to its Complexity.
	\end{itemize}
	
	The first option should generally be a good inclusion, while the latter (If you intend to use Situation Awareness rules from Tactical Shooting) can short circuit any reason to include Situational Awareness in the first place due to high bonuses, and should be scrutinized.
	
	\subsubsection{Using an RCC}
	
	While you are connected to your gaggle of drones, you can issue commands to them via interface, DNI, voice command, text command, etc. 
	
	Depending on the complexity of the command alongside your choice of medium, the time required can vary, usually ranging from a Free Action (e.g. Voice Command to "Watch that building and shoot anyone who doesn't respond to IFF") to multiple seconds of Concentration (e.g. Typing out a command to patrol a given location with random timing intervals and in an randomly determined alternating route).
	
	Commands do no necessarily have to be sent in bulk, but it does make issuing them a lot simpler. You can issue commands to any individual drone simply enough, alongside any group, as well you can often issue multiple commands at the same time, but if this takes more than a sentence this will often require a Concentrate maneuver or skill roll.
	
	When in doubt, see B363 for guidance on the Talk free action - with some extra leniency for DNI mediums. However, if there is still doubt, you can always ask for a skill roll (such as Tactics), in which case Situational Awareness (TS11) provides good inspiration for modifiers. As well, see Typing (B228) to determine the amount of time necessary to type a command, although a quick guesstimate of 0.5 words per second for an untrained individual and 1 word per second for a trained individual works well. Powers: Enhanced Senses p23 covers How Fast You Can Read, alongside how ETS affects this (p30).
	
	Keep in mind that most pilot programs above Rating 1 are fairly intelligent for these purposes, able to quickly understand languages, understand or interpret vague commands (or ask for help otherwise), react to an unknown situation with their best judgement, etc. A drone that is given a vague order shouldn't act like a literal minded retard, it should query its operator if it's outside the norm for a drone ("Confirm?: Shoot the unarmed, restrained individual.") or too vague ("Elaborate: What does, "Go show them pieces of drek who's the real boss in town" mean?). 
	
	This does not mean that they are capable of anything distinctly sapient, they will still fail to understand if you don't explain to them in generic terms and concepts related to their role that they are pre-programmed to process and understand. Nor does it give them great capabilities to generalize; in unknown situations that they are familiar with, they will try to determine their owner's / manufacturer's best response and follow that, while in situations that they are unfamiliar with they may retreat, ask for elaboration, take whatever action they think is best (which is likely strange or erroneous), or sometimes straight up spazz out.
	
	In situation where the GM is uncertain how a drone will interperet a command, they can call for the Pilot Program to make an IQ or Tactics roll, usually for more generic commands or commands more related to their original programming, respectively. Since these commands are often very simple, this is usually done with a bonus, ranging from +0 to +1 for complicated commands or quite strange situations, +2 to +3 for slightly complicated commands or strange situation, +4 or +5 for straightorward commands or situations that are just outside a drone's normal programming, or +6 or +7 for very simple commands or situations that a drone could even generalize their information to. As well, if the drone is connected to the RCC it can benefit from the TacNet bonus and the Rigger can use their Tactics as a complementary skill.
	
	\textcolor{OliveGreen}{\textit{Example: Joe the Rigger is clearing a building with his Rotodrone, Doberman, and Steel Lynx.}}
	
	\textcolor{OliveGreen}{\textit{On his first turn, he uses a voice command over his RCC: "Rotodrone, shoot anyone we don't know leaving the building.", as a Free Action, while he boots up his other two drones.}}
	
	\textcolor{OliveGreen}{\textit{He types into his RCC "Doberman, Steel Lynx, boot up and guard me.". The GM guesses this as about 10 words, so he says it takes 10 seconds, while in the meantime the Rotodrone moves to overwatch.}}
	
	\textcolor{OliveGreen}{\textit{Once they are booted up, Joe and his two drones head inside, as Joe uses DNI to command: "Lynx go right, us two will go left.". While it does command multiple drones to do different things, it's handedly within one sentence, so the GM agrees it's a free action.}}
	
	\textcolor{OliveGreen}{\textit{In the left room, Joe and his Doberman turn the corner to spot a Nosferatu, out for blood. He quickly commands over DNI: "Doberman, open fire! Lynx get back here!". While it is two sentence, because they are both short and over DNI, the GM asks for a Tactics roll at -4, which Joe barely makes due to his TacNet, making it a Free Action.}}
	
	\textcolor{OliveGreen}{\textit{The Nosferatu casts a physical illusion, making the abandoned building looks as if it were a field of flowers. Joe verbally says: "Shit, Doberman open fire at 1 oclock!". The Doberman's pilot is confused about the order, unsure of how it has teleported, where its owner is, and what it is shooting at.}}
	
	\textcolor{OliveGreen}{\textit{The GM calls for the Pilot Program to make a Tactics roll against the Doberman's Tactics 5. He gives it a +2 for a "strange situation", alongside the TacNet's +2. Joe succeeds on his complementary Tactics, providing another +1. The Doberman rolls against Tactics 10, barely failing, and confusedly asks for confirmation and context from Joe. Joe uses a Concentration maneuver to confirm, hoping his Steel Lynx gets here quick."}}
	
	\subsubsection{RCC Examples}
	
	RCCs are easily creatable and cutstomizeable using the Ultratech computer rules and a number of example RCCs are included here as well. Do remember that riggers will have to purchase their own ICE \& Firewalls among any other desireable programs for their RCCs, which can be found in the \hyperref[software_packages]{Software Packages section of the Matrix.}
	
	A GM can optionally allow for a Complexity 4 TacNet program that provides no skill bonuses, but allows for controlling drones; this allows a poor rigger to have lower-grade RCCs.
	
	\textbf{CompuForce Taskmaster}
	
	A standard RCC, built into a compact form to allow for portability. It has the minimum complexity necessary for running a TacNet and is well suited to controlling a suite of drones in the field.
	
	\textcolor{OliveGreen}{\textit{Statistics: Small Computer, Complexity 5 (RCC, \(\times\)1.25¥; Fast, \(\times\)20¥) 2,500¥ 0.5 lbs; Datapad 10¥, 0.05 lbs; TacNet, Complexity 5, 1,000¥}}
	
	\textbf{Essy Motors DroneMaster}
	
	One of the most standard RCCs on the market, it comes in a laptop sized format and meets the minimum complexity to run a TacNet. It meets the middle ground of portability and price.
	
	\textcolor{OliveGreen}{\textit{Statistics: Personal Computer, Complexity 5 (RCC, \(\times\)1.25¥) 1,250¥ 5 lbs; Portable Terminal 50¥, 0.5 lbs; TacNet, Complexity 5, 1,000¥}}
	
	\textbf{Maersk Spider}
	
	A bulky and hardened RCC, able to withstand all sorts of digital and physical abuse. It comes in a very bulky laptop form factor and is able to run the minimum complexity for a standard TacNet. Its size can make it less useful with boots on the ground.
	
	\textcolor{OliveGreen}{\textit{Statistics: Personal Computer, Complexity 5 (RCC, \(\times\)1.25¥; Hardened, \(\times\)2¥, \(\times\)2 lbs) 2,500¥ 10 lbs; Portable Terminal 50¥, 0.5 lbs; TacNet, Complexity 5, 1,000¥}}
	
	\textbf{Vulcan Liegelord}
	
	A high quality model used by many a Shadowrunner and corporate rigger in the field. It provides a quality form factor alongside having powerful Complexity 6 hardware that can give it a solid edge in the field.
	
	\textcolor{OliveGreen}{\textit{Statistics: Personal Computer, Complexity 6 (RCC, \(\times\)1.25¥; Fast, \(\times\)20¥) 25,000¥ 5 lbs; Portable Terminal 50¥, 0.5 lbs; TacNet, Complexity 6, 3,000¥}}
	
	\textbf{Lone Star Remote Commander}
	
	A desktop sized RCC commonly used for remote operations or secured in command vehicles on the scene. It trades all portability (By anyone smaller than a troll) in return for better hardware, allowing for Complexity 6.
	
	\textcolor{OliveGreen}{\textit{Statistics: Microframe, Complexity 6 (RCC, \(\times\)1.25¥) 12,500¥ 40 lbs; Workstation Terminal 500¥, 5 lbs; TacNet, Complexity 6, 3,000¥}}
	
	\textbf{MCT Drone Web}
	
	A control console sized RCC, used in permanent installations by well-funded Riggers - or even teams of riggers. Its powerful hardware allows for Complexity 7, which has the capabilities to manage multiple squads of drones and operators.
	
	\textcolor{OliveGreen}{\textit{Statistics: Microframe, Complexity 7 (RCC, \(\times\)1.25¥; Fast \(\times\)20¥) 250,000¥ 40 lbs; Workstation Terminal 500¥, 5 lbs; TacNet, Complexity 7, 10,000¥}}
	
	\subsubsection{Autosofts}
	Autosofts are pieces of Software that can be run on Drones or RCCs to allow them to make use of various skills. Their cost and skill usage are covered under their spot in the \hyperref[drones]{Drone section of Equipment}. however this section covers their usage for Riggers and RCCs.
	
	Drones are relatively lacking in computer Complexity, with most of them capping out at Complexity 4. This only lets them use up to 4 points in skills, which can be a deal breaker for extremely important ones (Such as the Guns skill).
	
	However, the Rigger is able to network his RCC with his drones using his TacNet, which also can allow him to share his programs among them. This is also possible without a TacNet, however the rigger must run an individual program for each drone, instead of making use of the TacNet's better software to distribute commands between them all! This allows the rigger to both save on money and run higher Complexity programs for his drones, although it is still a good idea to keep mission critical programs on the drone itself, in case a Decker jams your connection!
	
	
\end{multicols*}