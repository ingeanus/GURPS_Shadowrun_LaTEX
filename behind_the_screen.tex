\section{Behind the Screen} \label{behind_the_screen}

\begin{multicols*}{2}
	
	This section is purely dedicated to detailing the reasoning and design behind the many facets of this document. It's purpose is twofold, for helping GMs modify the rules to their liking and for myself to keep tracking of how these systems were made in the first place.
	
	\subsection{Rules}
	
	\subsubsection{Home Rules}
	
	\textbf{Critical Hits \& Active Defenses:} This was chosen for pretty much one singular purpose, being the asymmetry between NPCs and PCs in the number of rolls. Because the story naturally follows the PCs at all times, they will often be making an order of magnitude more rolls that the GM, even in combat. 
	
	As well, in order to make good use of time, sane GMs will not be rolling every single relevant action for NPCs, instead simply having them perform the actions. This means that, while NPCs just have to get lucky for a single roll in combat, PCs have to be continually lucky throughout all encounters. 
	
	Usually, this is just normal game design, compensated heavily by the fact that PCs are highly competent at what they do compared to NPCs. However, this can fall apart in Shadowrun (Especially GURPS) due to both the lethality of combat and the hyper-specialization of characters.
	
	In combat, 7d pi damage to a character is deadly almost no matter what, and even more so when it comes in a 15 bullet care package. A critical hit already means that a character has to deal with critical effects and 6 or more bullets, which is incredibly lethal even with Active Defense.
	
	As well, for both extreme ends of the specialization spectrum, Deckers are additionally frail in such combat situations where they lose any capability of Active Defense and Street Samurai are additionally punished for creating dodge focused characters (A staple of Shadowrun) as opposed to armor tanks.
	
	Overall, this was included to remedy those minor disparities and in general increase the lifespan of characters and campaigns.
	
	\textbf{Arm DX and ST:} Original Arm DX and ST are priced without taking into account their lack of HP and Basic Speed. They are accordingly repriced in order to make these already mediocre traits more competitive.
	
	\textbf{Costs FP and HP:} Costs FP and HP are somewhat notorious in GURPS for the contention regarding their price. The value between an ability that can be used at will, with no opportunity cost, and repetitively in short bursts, is much different compared to an ability that can be used nearly at will, with a small pool of regenerating resources, and only in quite small bursts.
	
	Compared to other -5\% and -10\% limitations such as Nuisance Effects and Reduced Range, the impact that FP has on the trait is quite large.
	
	\textcolor{Blue}{\href{http://www.mygurps.com/index.php?n=Main.GURPSHouseRules}{Some Home Rules}} suggest doubling the value, but I find that this creates an even worse over valuation of Costs FP and HP. Costs FP 1 and Costs FP 2 are extremely similar in effect, not doubled: They cost a medium-small proportion of resources, with a similar continuing cost over time as applicable. 
	
	As well, as you increase in value, it becomes strange to consider options such as: Costs FP 4, -40\% compared to Limited Use, 1 per day, -40\%. Arguments about the validity of Limited Use's pricing aside, an ability that leaves you somewhat winded for 0.5 hours, while still being usable twice in an encounter and multiple times in a day compared to an ability with no maleffects being usable only once per day, is a strange equality. It seems plain to me that the semi-mild FP cost is much less impactful (Especially in campaigns without Extra Effort or with long timescales).
	
	That is why the compromise of double value for the first cost and normal value for all further costs was chosen. 
	
	\textbf{Extra Effort in Combat:} Extra Effort has always been focused on the idea of providing FP costs for combat to represent over exertion, adrenaline, etc. However, it ignores the interesting interaction high Will provides with Extra Effort out of combat. As well, many of the traits are extremely valuable for their opportunity cost (Heroic Charge, I am looking at you), so this serves to lower their value somewhat.
	
	\subsubsection{Character Creation}
	
	\textbf{Attribute Limits: } One of the goals in the Attribute cost rebalancing was to make Talents more viable and overall promote Specialist characters as opposed to extremely competent generalists. 
	
	This is most difficult around attribute levels of 14 and 15, where defaulting can be better than some people's skills alongside singular point investments be insurmountable to lower attribute companions. Due to how high the point totals for Shadowrunners can be, it can be pretty easy for the Fridge Decker to simply take IQ 15 and easily school the mage on thaumatological skills or for the DX 15 Street Samurai to be extremely competitive with a Rigger.
	
	Of course, there is a point to having character both be rewarded for their point investment alongside for playing their niche, whether it be a braniac or 'wared up badass, which is why the costs are just balanced to incentivize Specialists, not ban generalists.
	
	\textbf{Wealth Levels and Independant Income: } Due to how the system for 'Ware works, in that you can convert the CP cost to Nuyen, these traits can make it extremely easy to just buy Beta/Deltaware everything in an attempt to create a disproportionately powerful character.
	
	As well, high wealth characters often don't fit the genre for Shadowrun, being that the default character is assumed to literally be a rightless non-citizen committing crimes against the elite for money.
	
	Wealthy [20] provides 150,000¥, which normally converts to 50 points of 'Ware, or a 40 point increase from the baseline. If the 80\% lifestyle rule is enforced, this becomes 10 points of 'Ware, or an 8 point increase from the baseline. This provides money for all sorts of equipment, while still providing the other benefits from Wealth (Improved job pay, higher status, etc).
	
	Very Wealthy [30] provides 600,000¥, which converts to 200 points of 'Ware, a 190 point increase from the baseline. And if the 80\% lifestyle rule is enforced this becomes 40 points, a 38 point increase - or at best a 30 point increase if you \textit{only enforce the lifestyle rule for characters with higher Wealth.} This is why it is advised to watch it very close, because it can be extremely profitable for collecting Cyberware. 
	
	If you want to allow it (Especially since its values approach Resources A from Shadowrun), it is advised to not enforce the 80\% rule for other characters as strictly (Lowering the proportions) or perhaps not enforcing at all.
	
	Independant Income only exacerbates the problem, providing regular CPs per month, for a small upfront cost. Given that many Shadowrun campaigns will have periods of downtime in between missions, this can easily become extremely cost effective.
	
	\subsection{Magic}
	
	\subsubsection{Spells}
	
	\textbf{Drain FP/HP:}
	See \textcolor{Blue}{\href{https://www.ravensnpennies.com/gurps101-fp-cost-limited-by-margin/}{Christopher Rice's work}} creating a resistible Costs FP. This system uses his second recommendation in Picking Over The Bones.
	
	In essence, it means that each \textit{Force X} ability has a limitation \textit{Costs FP X} and \textit{NOT Costs FP X (Limited by Margin, X*-5\%)}, or double this for Costs HP as detailed below at \textit{Spell Force} (\ref{Spell Force}).
	
	\begin{center}\label{drain_mods}
		\begin{tabular}{|c|c|c|c|}
			\hline
			Level & Value & Level & Value \\
			\hline
			\hline
			1 & -0.5\% & 9 & -22.5\% \\
			2 & -1.5\% & 10 & -27.5\% \\
			3 & -3.0\% & 11 & -33.0\% \\
			4 & -5.0\% & 12 & -39.0\% \\
			5 & -7.5\% & 13 & -45.5\% \\
			6 & -10.5\% & 14 & -52.5\% \\
			7 & -14.0\% & 15 & -60.0\% \\
			8 & -18.0\% & 16 & -68.0\% \\
			\hline
		\end{tabular}
	\end{center}
	
	Each level after subtracts 0.04\%, as -0.05\% + 0.01\%.
	\\\\
	For Costs HP these values are as follows:
	\begin{center}
		\begin{tabular}{|c|c|c|c|}
			\hline
			Level & Value & Level & Value \\
			\hline
			\hline
			1 & -1\% & 9 & -45\% \\
			2 & -3\% & 10 & -55\% \\
			3 & -6\% & 11 & -66\% \\
			4 & -10\% & 12 & -78\% \\
			5 & -15\% & 13 & -91\% \\
			6 & -21\% & 14 & -105\% \\
			7 & -28\% & 15 & -120\% \\
			8 & -36\% & 16 & -136\% \\
			\hline
		\end{tabular}
	\end{center}
	
	Each level after subtracts 0.08\%, as -0.10\% + 0.02\%.\\
	
	\textbf{Spell Force\label{Spell Force}}
	
	When casting a spell, the user selects a Force, which determines the level on an ability, alongside the FP/HP Drain. This is built according to \textcolor{Blue}{\href{http://forums.sjgames.com/showpost.php?p=1399558&postcount=27}{this thread on Variable Costs FP}}, because the official Variable FP ignores the -80\% cap.
	
	So: For each Magic, we take one of two types: FP Levels and HP Levels.
	
	FP Levels are a summary of levels up to Magic (I.e. there are 6 FP levels for Magic 6). Each level's cost is determined by taking:
	
	Advantage N (FP Drain N) - Advantage N-1 (FP Drain N).
	
	Where N is the levelled trait for levels 1..Magic.
	
	HP Levels are the same, but we START at N=Magic+1, and go up to Magicx2.
	
	As an example: We take a basic Corrosive Innate Attack [10] and Magic 2
	
	(IAC is Innate Attack, Corrosive while FPD is FP Drain and HPD is HP Drain)
	Level 1: IAC 1 (FPD 1, -0.5\%) [9.995]
	Level 2: IAC 2 (FPD 2, -1.5\%) [19.7] - IAC 1 (FPD 2, -1.5\%) [9.85] = 9.85
	
	Level 3: IAC 3 (HPD 3, -6\%) [28.2] - IAC 4 (HPD 3, -6\%) [18.8] = 9.4
	Level 4: IAC 4 (HPD 4, -10\%) [36] - IAC 4 (HPD 4, -10\%) [27] = 9
	
	Summing them all up gives 38.245, for a total [39] points.
	Break this out gives up:
	1 Level 1 attack with 1 FP Drain.
	1 Level 2 attack with FP 2 Drain that removes the cost for its lower Level 1 attack.
	1 Level 3 attack with HP 3 Drain that removes the cost for its lower Level 2 attack.
	1 Level 4 attack with HP 4 Drain that removes the cost for its lower Level 3 attack.
	
	All in all, with variable you can now switch between the 4 levels, and each one has a different Drain FP level alongside giving that amount of dice. Of course, you definitely do not end up with something silly like 1 + 2 FP Drain and 3 + 4 HP Drain, any more than 3 Levels with Costs FP 1 would give you 1 + 1 + 1 FP Cost on a normal ability! These are applied to the overall total!
	
	While this is definitely more complicated, I can utterly assure you it is the best system for leveled advantages! The alternative list of things I have gone through:
	
	Alternative Abilities: Alt Abilities between the HP and FP levels. Creates a curve where higher levels end up cheaper than earlier ones. Sometimes it waffles between that! Makes for high prices at its cap, but never really accounts for the massive capability a F16 Spell can have!
	
	Counter-Limitations: Create 1 level with FP Drain and 1 level with HP Drain and Counter FP Drain. This one is better than Alt Abilities as it does not decrease the cost at higher levels NEARLY as much. But it still happens. As well, higher levels abilities tend to increase cost by singular points or less, leading to a similar issue from before.
	
	This completely ignores ideas that just don't work to start, such as Either/Or Limitations. Some of these don't technically ALLOW you to create a spell that functions like in Shadowrun, meaning that you end up with some weird garbage that is really close to a fair price most of the time, but can be 10+ off at other times, all while NOT EVEN TECHNICALLY FUNCTIONING AS YOU WANT.
	
	For AoE that increases by Force:
	
	72.1348 log(x); where x is the radius in meters.
	
	
	\% at each m radius
	25
	50.00003324
	79.24817772
	100.0000665
	116.0964819
	129.248211
	140.3678394
	150.0000997
	158.4963554
	166.0965152
	172.9716959
	179.2482442
	185.0221089
	190.3678727
	195.3446596
	200.000133
	
	\textbf{Combat Spells}:
	
	Every combat spell as a selection of modifiers applied: \textit{Increased Range, LOS, +40\%; Magical, -10\%; Requires (Spellcasting) Roll, -20\%\footnote{Is Requires Attribute (10) Roll, -20\%, Requires Skill Roll (Spellcasting), -0\%}; Variable, +5\%}. This is combined into: \textit{Combat Spell, +15\%}.
	
	\paragraph{Acid Stream}: Levels: 9.8085 19.2735 28.1835 36.3135 43.4385 49.3335 53.7735 56.5335
	
	\paragraph{Toxic Wave}: Levels: 12.8085 25.2735 37.1835 48.3135 58.4385 67.3335 74.7735 80.5335
	
	\paragraph{Punch}: Levels: 12.80175 24.85925 35.71425 44.87925 51.86675 56.18925 59.69925 63.46925
	
	\paragraph{Clout}: Levels: 13.3475 26.1225 37.9725 48.5225 57.3975 64.2225 68.6225 71.5225
	
	\paragraph{Blast}: 	Levels: 12.84325 25.28575 37.08075 47.96575 57.67825 65.95575 72.53575 77.15575
	
	\paragraph{Death Touch}: Levels: 11.078 21.698 31.578 40.418 47.918 53.778 57.698 60.018
	
	\paragraph{Manabolt}: Levels: 14.4746 28.6286 42.2646 55.1726 67.1426 77.9646 87.4286 95.3246
	
	\paragraph{Manaball}:Levels: 9.8712 19.5592 28.9512 37.9272 46.3672 54.1512 61.1592 67.2712
	
	\paragraph{Flamethrower}: Levels: 12.3475 24.1225 34.9725 44.5225 52.3975 58.2225 61.6225 64.5225
	
	\paragraph{Fireball}: Levels: 12.14325 23.88575 34.98075 45.16575 54.17825 61.75575 67.63575 71.55575
	
	\paragraph{Lightning Bolt}: Levels: 12.3475 24.1225 34.9725 44.5225 52.3975 58.2225 61.6225 64.5225
	
	\paragraph{Ball Lightning}: Levels: 12.14325 23.88575 34.98075 45.16575 54.17825 61.75575 67.63575 71.55575
	
	\paragraph{Shatter}:
	\paragraph{Powerbolt}:
	\paragraph{Powerball}:
	\paragraph{Knockout}:
	\paragraph{Stunbolt}:
	\paragraph{Stunball}:
	
	\textbf{Detection Spells:\\}
	
	Detection Spells are straight forward in their basic design: Create an Affliction with the Detection Spell modifier and add the Advantage for the spell effect to it.
	
	The Detect Spell Modifier is basically a combo of modifiers that lets you give the spell effect to some via touch via Spellcasting and let it last for 1 day while you maintain concentration.
	
	When doing higher levels using things like Reliable +5\% or Acute Sense [2] e.g. +20\%, you actually end up cancelling basically everything out and just adding the cost of the enhancement! So each level gives +1.5 or +2 points, respectively.
	
	Example using Analyze Device:\\
	Aff (+155\%, +150\%, -0.05\%) [40.45]\\
	Aff (+155\%, +170\%, -1.5\%) [42.35] - Aff (140\%, +150\%, -1.5\%) [40.35] = [2]\\
	
	The issue here, is that, it IGNORES the effects of Drains FP/HP. This is because the Drain only decreases the cost of the base! As such, there's no interaction with the enhancement.
	
	As such, we have to include the +20\% modified by Costs FP/HP. This doesn't cause multiple stacking of FP/HP Loss as a meta-trait (i.e. using the whole affliction causes it). This may not be 100\% correct, but it feels way more accurate.
	
	\paragraph{Analyze Device}: Levels: Levels: 40.98 43.85925 46.50675 48.81675 50.67675 51.97425 52.76175 53.57175
	
	\paragraph{Analyze Magic}: Levels: 32.49 33.44975 34.33225 35.10225 35.72225 36.15475 36.41725 36.68725
	
		
	\paragraph{Clairvoyance and Clairaudience: } Levels: 69.5638 73.90224 76.24079 77.70544 78.63809 79.195065 79.50884 79.80294
	
	These were made by using Logistic Regression to determine the percentage cost for increased range as it \textit{specifically} applied to the 10 yard range of Clairsentience. You get: 12.9228 * ln(0.220941 * yd range). Then, multiply yd range by x5 to get steps of +5 yards and you'll have the enhancement value at each level.
	
	Enhancement Value (\% per 5 yards, starting at Force 2): 10.24 15.48 19.2 22.09 24.44 26.43 28.16 29.68 31.04 32.27 33.4 34.43 35.39 36.28 37.12
	
	Since Force 1 is Reduced Range x1/2, you can see it mostly match increased range's costs, so 10\% at Force 2 (10 yards), 19\% (20\%) at Force 4, (20 yards), 31\% (30\%) at Force 10 (50 yards).
	
	\paragraph{Combat Sense}: Levels: 42.99 43.94975 44.83225 45.60225 46.22225 46.65475 46.91725 47.18725
	This marks a departure in a subtle different design. Analyze and Detect Spells want Reliable on the Detect Advatage. Some of these later spells want it on the Affliction!
	
	\paragraph{Detect Enemies}: Levels: 32.2425 32.722375 33.163625 33.548625 33.858625 34.074875 34.206125 34.341125
	
	\paragraph{Detect Enemies, Extended}: Levels: 33.5425 34.022375 34.463625 34.848625 35.158625 35.374875 35.506125 35.641125
	
	\paragraph{Detect Individual}:
	
	\textbf{Health Spells:\\}
	
	\paragraph{Heal}: Levels: 44.955 50.7135 56.0085 60.6285 64.3485 66.9435 68.5185 70.1385
	
	This is unfortunately my worst creation here. There's not much to say in the Heal does not like using HP to heal HP, perhaps with good reason. Much of it is good, but there are two parts that are basically just BS Made up:
	
	Injurious Magic. This is the limitation that says: Hey, whenever you spend FP you must spend HP instead. This was chosen for two reasons over Empathic, firstly you can't take Capped with Empathic (For what reason, I don't fucking know). Secondly, it turns it into a 1:1 ratio instead of 1:2, which fucks with any scaling for high forces. As such, this was based on the -30\% limitation, with an accessibility to limit its use to top half of the Magic. This feels mostly right, since it is be default worth less than the -50\% for Empathic's 1:1 ratio, but provides the 2:1 ratio.
	
	Resistable Drain, +5\% per level. There's no getting around that this is completely and utterly arbitrary. It's based on the assumption that Margin-Based was -5\% per level, so we're just applying it to Healing flat and saying it works for its base FP. One might say: Hold on retard, why not take Reduced FP (Margin-Based) to lower the cost? That's because it's retardely expensive and makes no sense in a fair game. For a Force 6 spell that is: Reduced FP (Margin-Based 6, -30\%), +84\%. EIGHTY FOUR PERCENT. For, what will basically mean you reduce 6 Drain to 3 most of the time. Eight Four Percent is almost 2 cheating cosmics. It's more than XenoHealing, All Life. It's stupid is what it is.
	
	\textbf{Illusion Spells}:
	
	\paragraph{Agony}: 
	
	\paragraph{Massy Agony}: Levels: 30.965 36.7355 40.2065 42.6285 44.2805 45.3295 45.9395 46.5495
	
	This is the first AoE Spell I made. SO. Here's comes that.
	
	I want AoE to be linear radius. That means we need a logistic regression of the cost of AoE. This is 72.138 * ln (radius in yards)\%. We can take this and add it as a decreasing enhancement to powers in essence. I'll put the table of it above.
	
	\paragraph{Confusion}: Levels: 24.49 25.44975 26.33225 27.10225 27.72225 28.15475 28.41725 28.68725
	
	\paragraph{Mass Confusion}: Levels: 29.465 35.2355 38.7065 41.1285 42.7805 43.8295 44.4395 45.0495
	
	\paragraph{Chaos}: Levels: 29.49 30.44975 31.33225 32.10225 32.72225 33.15475 33.41725 33.68725
	
	\paragraph{Chaotic World}: Levels: 34.465 40.2355 43.7065 46.1285 47.7805 48.8295 49.4395 50.0495
	
	\paragraph{Invisibility}:Levels: 55.485 58.3745 61.0545 63.3845 65.3145 66.6295 67.4795 68.2195
	
	Invisibility works largely as expected. It uses \href{http://forums.sjgames.com/showpost.php?p=669736&postcount=2}{Kromm and PK's ruling on Quick Contests} in order to make is a contest between Will/Per and Spellcasting. The only slightly weird bit, is that Glamour does not go above HT-5, -5\%. This is because HT-6 is basically considered moot, fairly enough. This means that, while we'll keep stacking penalties ourselves, there's actually no increase in cost past Magic 6, since they're effectively the exact same in price.
	
	Building this one might be a bit confusing, but it's quite similar to  a Detection Spell's design. First, you make the Invisibility advantage and level it like a generic advantage. Then you add that to an affliction, and create levels for the affliction for each corresponding level of Invisibility. Then, apply drain and everything to the affliction advantage.
	
	\paragraph{Phantasm}
	Levels: 34.17 47.156625 54.510375 59.410375 62.610375 64.584125 65.715375 66.835375
	Levels: 36.67 49.656625 57.010375 61.910375 65.110375 67.084125 68.215375 69.335375
	Levels: 40.42 53.406625 60.760375 65.660375 68.860375 70.834125 71.965375 73.085375
	
	\paragraph{Trid Phantasm}
	Levels: 36.67 49.656625 57.010375 61.910375 65.110375 67.084125 68.215375 69.335375
	Levels: 39.17 52.156625 59.510375 64.410375 67.610375 69.584125 70.715375 71.835375
	Levels: 42.92 55.906625 63.260375 68.160375 71.360375 73.334125 74.465375 75.585375
	
	\paragraph{Manipulation Spells:}
	
	Manipulation spells don't usually have distinct levels to them just like Detection Spells. Most of the time, they will gain +1 level of Reliable, +5\%, unless there's something better for that spell.
	
	\paragraph{Levitate}
	This is built out of two TKs. One is a Margin-Based Move Only, meaning that the margin of victory determines move. The other is a lift only one that scales with force. Add them together in stacked levels for the whole power.
	
	\paragraph{Physical Barrier}
	
	Calculating the amount of weight is done via \href{https://forums.sjgames.com/showpost.php?p=1936968&postcount=10}{PK's method here.}  A drop will do about HP/20 dice, or HP/5.7 damage. For 2 DR this is  === to 11.4 HP. Convert that to unliving HP
	
	\subsection{Spirits}
	
	Spirits are actually kind of straight forward. There are a number of abilities that represent the powers. Spirits gain a combination of traits and powers that puts them close to 0 CP. Every Force allows a user to add 35.5 free CP - 10.5 for Spirit Force - Automatic Trait CP.  Then, as each Force and type costs around the same amount, create brackets for ally point values \& that becomes the necessary points to modular ability w/.
	
	Of note, we specifically allow for allies above the normal limit.. because while Force 14 Spirits could very easily be Patrons (Being 325 points for a 200 points campaign...), they are more often one time deals with extremely powerful beings... a common theme in Shadowrun.
	
	\subsubsection{Spirit Math}\label{spirit_math}
	
	Because calculating the cost for the Summoning and Binding advantages are campaign dependant, this section goes over how it is generally done, for those playing outside the general 200 and 100 point games.
	
	Looking at the \hyperref[spirit_ally_cost]{Spirit Ally Cost Table} in Magic tells you the Percentage of Starting Points that an Ally with Appears Constantly, Special Abilities, Summonable, and Favor. You first need to determine what Force you can summon at a given percentage.
	
	For example, for a 100 point game, a Force 1 spirit averages around 0 points, with each Force adding 25.5 points. This means that a Force 1 is equivalent to 5\%, while each increase in Force happens at multiples of 25\%. For a game of 300 points, Force 1 is still 5\%, but each increase in Force happens at multiples of 1/10\% (10\%), meaning that Force 2 matches 10\%, Force 3 matches 20\%, Force 4 matches 25\%, and so on. Note down the point values for each Force; you will want to determine this up to double your highest Magic stat in the game, I recommend for Magic 8.
	
	Next, you need to price your Modular Abilities for \hyperref[summoning]{Summoning} and \hyperref[binding]{Binding}. Their sections detail the base cost and modifiers - 4+4/lvl with +20\% modifiers on Summoning and 4+2/lvl with +20\% modifiers on Binding - however, they still need respective FP Drain and HP Drain modifiers, which can be found \hyperref[drain_mods]{here.}
	
	Each Force of Spirit is made by taking the Modular Ability with individual levels to summon each Force (As calculated at the beginning) and then applying the Drains FP limitation at a level equal to each given the Force for Forces that are equal to or less than the Magic Level, and the Drains HP limitation for those that are above. 
	
	As an example: For a 200 point campaign, Magic 4 Summoning is made up of a Force 1, 2, 3, and 4 level with Drains FP and a 5, 6, 7, and 8 level with Costs HP. 
	
	
	
	TODO: FIX: Force 1 allies requires 1 point as seen on the \hyperref[spirit_ally_cost]{table}, meaning you need 4 (slot) + 4 (1 point) * (1 + 0.2 - 0.005) = 9.56 points. Force 2 requires 2 points, however you only need to pay for any increase from Force 1, which is a single points - as such it costs 4 * (1 + 0.2 - 0.015) = +4.74 points. Force 3 also costs 2 points, so it does not have any increased cost. Force 4 is 3 points, costing 4 * (1 + 0.2 - 0.05) = +4.6 points. 
	
	Force 5 now uses Drains HP and also requires 4 points, costing 4 * (1 + 0.2 - 0.15) = +4.2 points. Force 6 costs 5 points, so it is 4 * (1 + 0.2 - 0.21) = +3.96 points. Force 7 is 6 points, costing 4 * (1 + 0.2 - 0.28) = +3.68 points. Force 8 is 8 points, meaning it costs 4 * 2 * (1 + 0.2 - 0.36) = +6.72 points.
	
	Add up all of these values, 9.56 + 4.74 + 0 + 4.6 + 4.2 + 3.96 + 3.68 + 6.72 = 37.46 points, which rounds up to a final cost of 38 points for Magic 4 Summoning.
	
	Now do this for every Magic. :\}
	
	I recommend a spreadsheet.
	
	\subsection{Resonance}
	
	Resonance functions very similarly to Magic. Each Complex Form has a Level that causes a certain amount of Drain. This drain is FP or HP based on your Resonance. Almost none of the Complex Forms are anything resembling an attack, so instead using the progression for Manipulation spells and so on.
	
	These also all include the FP limitations on the level up enhancements (e.g. Reliable w/ (Drain FP), etc). Yeah it's not 100\% legal, but it feels right.
	
	\paragraph{Editor:} 
	Levels 1 and 2 are technically 4 and 6, because the default limitations are at 15\%, so the +5\% for the first Reliable is ignored.	
	
	Levels: 4.99 6.9095 8.6745 10.2145 11.4545 12.3195 12.8445 13.3845
	Levels: 12.475 17.27375 21.68625 25.53625 28.63625 30.79875 32.11125 33.46125
	
	\paragraph{Pulse Storm}
	
	Levels: 32.358 39.46015 45.99065 51.68865 56.27665 59.47715 61.41965 63.41765
	
	\paragraph{Puppeteer}
	Levels: 27.45 32.24875 36.66125 40.51125 43.61125 45.77375 47.08625 48.43625
	
	\paragraph{Resonance Spike}
	
	Levels: 12.3475 24.1225 34.9725 44.5225 52.3975 58.2225 61.6225 64.5225
	
	\paragraph{Static Bomb}
	This one's a bit weird, because -10 is the max penalty. So Forces higher than 10 don't provide anything normally, but I'm allowing higher penalties.
	
	Levels: 8.139 16.049 23.589 30.609 36.959 42.489 47.049 50.489
	
	\paragraph{Wiretap}
	Levels: 20.97 23.84925 26.49675 28.80675 30.66675 31.96425 32.75175 33.56175
	
	\subsubsection{Designing Resonance Powers}
	
	Because Resonance Powers in many ways need to be analogues to real world contemporaries, they can sometimes seem a little disconnected from realistic matrix work, especially when compared to Deckers.
	
	I've settled on a group of assumptions that influence their design, which are:
	
	\begin{itemize}
		\item If devices are on the same network, they're considered to be practically touching. This is supported by Technomysticism's ruling that networks count as touching.
		\item If devices have the Melee Attack trait, they need to either be holding/containing their target (i.e. having a target file in memory), or need to be able to engage it in cybercombat.
		\item As discussed in the Complex Form section, software and hardware are given analogous HT and Will scores.
	\end{itemize}
	
	\subsubsection{Sprites}
	
	The Sprite meta-trait is kind of weird. They're a digital construct that doesn't necessarily require a device to exist on the internet, or at least can exist on the wider matrix without a dedicated device. That's why they lack a lot of traits that might be obvious options for them, like Doesn't Breath, because they're physical traits and/or utterly inconsequential.
	
	\subsubsection{GURPS Pyramid \#3/91 - Thaumatology IV}
	
	The Technomysticism article proved invaluable in merging the Decker and Technomancer systems while still allowing for magical-esque powers for the Emerged. However, it's core component is the Netrunning advantage, otherwise known as Possession (Digital), which NOT how technomancer's work.
	
	\paragraph{Netrunning}
	
	Netrunning has a number of issues that make it not align well for Emerged.. in no particular order: 
	
	Firstly, failure makes the system permanently immune to you, which can of course be ignored with the Reduced Immunity - but that costs a literal fortune. 
	
	Secondly, the Technomancer's brain literally shows up on the matrix as a device, working as any computer would beyond some minor losses (no PANS, no storage, no programs, etc); this means that Netrunning, which has the user temporarily mentally live inside a computer, doesn't work well. 
	
	Thirdly, a Technomancer doesn't live inside the device they're interacting with, nor is it constrained by it; this is to say that a Technomancer's Living Persona determines their ASDF traits, not whether they're possessing a desktop or commlink (Although the could do that in the normal rules as an alternative). They also are not inside the computers: if their internet is shut off they return the the meat world, not get trapped inside the host.
	
	Fourthly, they don't gain admin control of the system after entering it. This note depends on if you're looking at the Pyramid article or Psionic Powers' Netrunning. The latter is more realistic to the Possession trait, especially since it's costs 100 fucking points! Possession lets you control a human body in any way it normally works, so it's stupid to assume you only gain user access to a computer after a successful possession.
	
	Fifthly, the power needs to work over the matrix alone, so requiring touch or allowing ranged sight is an obvious no no.
	
	\paragraph{Telecommunication}
	
	All of of this points towards Telecommunication, Radio serving as the superior version. In general, where Netrunning allowed for something to work, Telecommunication can usually fit in exactly the same, but it does require some work and analysis to perform correctly. 
	
	The Technomancer still needs to function as a computer with just their mind, but without possessing any computer to get their stats. This is done with the Digital Mind trait and using the Complexity rules from Thaumatology IV; this trait must still be limited to only work in the Matrix, as the Technomancer would otherwise be immune to Mind Control spells at all times!
	
	Lastly, the Technomancer still needs some way to run the Cyberpunk programs, which is an adventure in and of itself:
	
	\paragraph{Resonance Programs}
	
	Interacting with the Decker system requires at least some capability to use the Cyberpunk programs. However, while they're easy for a decker to grab, not so much for a Technomancer. What's worse is that they provide a number of basic functions for a computer - in essence a mix of programs like ICE, Listen, Trigger, Alter, and Control would make up a normal Operating System for the computer!
	
	Ostensibly, Telecommunication + Digital Mind should cover the ability to act like a normal computer, so many of the functionality for an everyday commlink should work with those, but the question is how to allow the other programs?
	
	Thaumatology IV as a pretty decent idea, being that Resonance Advantages could be used as programs. However, their analysis is pretty flawed when implemented: Their system allows those who purchase the advantages to instead roll against the programs as if they had them, adding Talent to make up for Higher Complexity. This is flawed in two large ways, however.
	
	Firstly, the advantages are \textit{waaaay} more expensive that the functionality of a single fucking program. Mind Control (Cybernetic Only) is [25], which is insane to only function for a couple of programs! It takes less than [30] to have a good deck and every hacking program through money! As well, Technomancers can hack normally mixed with Complex Forms, and as such need to be able to switch between the two. As such, I consider it that the Advantages automatically incorporate the ability to use the programs as with the Resonance Program advantage, which I'll get into later.
	
	Secondly, Technomancers are going to have a really rough time once Decker's realize they can buy Talent too. Thaumatology IV says that Technomancers can compete with high complexity systems through their Power Talent, but that falls apart when a Decker simply takes the Born to Be Wired Talent, gaining the bonuses of High Complexity \textbf{and} Talent. As such, the Resonance Program advantage is designed to allow the Technomancer to run the programs at Complexities up to their Living Persona's complexity (3+Resonance), which bridges that gap easily.
	
	The Resonance Program itself is built around modified Accessory perks. As \textcolor{NavyBlue}{\href{http://forums.sjgames.com/showpost.php?p=561052&postcount=8}{pointed out by Kromm}}, it's possible to modify Accessories into creating entire powers from them. Because programs are easily purchasable for characters and can be internally incorporated for characters with the Digital Mind advantage, they should be prime candidate for that. As such, I used these ideas to modify the perk and create an Accessory that could be used as a power.	
	
	\subsection{Matrix}
	
	\subsubsection{Host Ratings} \label{Host Ratings BTS}
	
	\begin{flushleft}
		\begin{adjustwidth}{-4mm}{}
			\scalebox{1.0}{
				\begin{tabularx}{0.53\textwidth}{|c|c|c|c|c|}
					\hline
					Description & IQ & RSL & Complexity \footnote{ICE considered at Complexity-1 and Complexity, providing bonus to skill.} & Skill\\
					\hline
					\hline
					Home LAN & 10 & IQ-6 - IQ-3 & 4 (+0/+1) & 4-8 \\
					Personal Site & 10 & IQ-3 - IQ-2 & 4 (+0/+1) & 7-9 \\
					Mom \& Pop & 10 & IQ-3 - IQ-2 & 5 (+1/+2) & 8-10 \\
					Small Business & 11 & IQ-1 - IQ & 5 (+1/+2) & 11-12 \\
					School & 11 & IQ-1 - IQ & 6 (+2/+3) & 12-14 \\
					Local Police & 11-12 & IQ-1 - IQ & 6 (+2/+3) & 12-15 \\
					University & 11-13 & IQ-1 - IQ+1 & 6 (+2/+3) & 12-17 \\
					Low Gov. & 11-13 & IQ - IQ+1 & 7 (+3/+4) & 14-18 \\
					Maj Gov. & 13-14 & IQ - IQ+1 & 7 (+3/+4) & 16-19 \\
					Secure Site & 14-15 & IQ - IQ+1 & 8 (+4/+5) & 18-21 \\
					Military & 15-16 & IQ - IQ+1 & 8 (+4/+5) & 19-22 \\
					Megacorp & 16-17 & IQ - IQ+2 & 8-9 (+4 to +6) & 20-25\\
					\hline
				\end{tabularx}
			}
		\end{adjustwidth}
	\end{flushleft}
	
	\subsection{'Ware}
	
	Here's the old text on the meta-trait limitations. I don't really like how squished they are. PCs can't reaallly plan around this so I don't think it's great as an accessibility.
	
	When taking the Meta-Trait, it must be limited to whatever parts of the body that are cybered up. This must use \textcolor{Blue}{\href{http://forums.sjgames.com/showpost.php?p=623207&postcount=1}{Kromm's Post on Partial DR for Hit Location}}. Here are some pre-made limitations for ease-of-use:
	
	\begin{itemize}
		\itemsep0em 
		\item Skull: -40\%
		\item Face: -40\%
		\item Eyes: -45\%
		\item Ears: -45\%
		\item One Limb: -35\%
		\item Two Limbs: -30\%
		\item Three Limbs: -25\%
		\item Four Limbs: -20\%
		\item Torso: -25\%
		\item Torso and Four Limbs: -5\%
	\end{itemize}
	
	\label{partial}
	
	I instead went with some good old \href{https://www.wolframalpha.com/input?i=logarithmic+regression&assumption=%7B%22C%22%2C+%22logarithmic+regression%22%7D+-%3E+%7B%22Calculator%22%7D&assumption=%7B%22F%22%2C+%22LogFitCalculator%22%2C+%22data2%22%7D+-%3E%22%7B%7B0.463%2C9%7D%2C%7B1.85%2C7%7D%2C%7B2.78%2C5%7D%2C%7B11.57%2C4%7D%2C%7B9.72%2C4%7D%2C%7B11.57%2C4%7D%2C%7B16.66%2C4%7D%2C%7B36.57%2C1%7D%2C%7B21.29%2C2%7D%2C%7B28.23%2C2%7D%7D%22}{logarithmic regression.} This simply does regression to find the hit location penalty given a random hit location chance, e.g. Give it 1.85\% for skull and get around 7 in penalties out.
	
	The following formula takes in the percentage of body covered and outputs a limitation value based off the partial rules.
	16.8243 log(0.0102412 x)
	
	Hit locations are presented below. Sublocations (e.g. Jaw/Nose/Ears/Cheek/Eyes) add up to the total for their main location (e.g. Face). As such, to use this, first add up the percentage for all locations you want covered, then plug it in as the value of \textit{x} in the formula above, and it will spit out the limitation value. Round down to the nearest 5\% for simplicity (lol). Use any values for Partial over this! Any negative values round up to 0. Values for left/right arms and legs were averaged before processing.
	
	Example: You're making some weirdo with a cybered up Jaw, Skull, Right Forearm, and Torso (Chest only) but nothing else. Add up 0.463 + 4.86 + 1.85 + 24.07 = 31.243. Plug this in like: 16.8243 log(0.0102412 * 31.243) = -19.172\%, which rounds to a -20\% limitation. Tadja!
	
	\begin{verbatim}
	Location |	% Coverage
	Skull		1.85 Total
	Face		2.78 Total
	> Jaw		0.463
	> Nose		0.463
	> Ears		0.463
	> Cheek		0.927
	> Eyes		0.463
	Right Leg	11.57 Total (Avg for legs 14.115)
	> Shins		5.785
	> Knees		1.9283
	> Thighs		1.9283
	> Thighs (Veins)	1.9283
	Right Arm	9.72 Total (Avg for arms 10.645)
	> Forearm		4.86
	> Elbows		1.62
	> Upper Arms	1.62
	> Shoulder(Veins) 1.62
	Torso		24.07 Total
	> Vitals		4.0117
	> Chest		20.0583
	Groin/Abdomen	12.5 Total
	> Vitals		2.083
	> Digestive Tract 6.25
	> Pelvis		2.083
	> Groin		2.083
	Left Arm	11.57 Total
	> Forearm		5.785
	> Elbows		1.9283
	> Upper Arms	1.9283
	> Shoulder(Veins) 1.9283
	Left Leg	16.66 Total
	> Shins		8.33
	> Knees		2.777
	> Thighs		2.777
	> Thighs (Veins)	2.777
	Right Hand	2.315 Total
	> Joint		0.38583
	> Extremity	1.92917
	Left Hand	2.315 Total
	> Joint		0.38583
	> Extremity	1.92917
	Right Foot	1.39 Total
	> Joint		0.2317
	> Extremity	1.1583
	Left Foot	1.39 Total
	> Joint		0.2317
	> Extremity	1.1583
	Neck		1.85 Total
	> Vein/Spine	0.3083
	> Neck		1.5417
	\end{verbatim}
	
	
	\subsubsection{Control Rig}\label{bts_riggers}
	
	One of the defining components of the Control Rig is the fact that you take damage whenever your drones do. This is very similar the Ally disadvantage, however it also needs to have a second layer of resistability. Given that the most common AR does average 21 damage, alongside most drones that will get shot having 8 DR, I'm judging this as needing 13 MoS to ignore.
	
	Thus, if we start with -25\% being: Death of one party reduces you to 0 HP, we can vaguely equate this to: Death of 1 party does 11 irresistible damage, given that 11 is a good estimate for average HP. So we can start with: Death gives 3d damage.
	
	As for how to do duplicated damage... This one is really hard. I could think that the possibility of: Costs HP 13 w/ Accessibility, Only when Hit, -80\% at -28\% is a decent option. But that's incredibly bodgy. I think it's honestly better to just wing it, since I believe that another -25\% makes sense here and comes close to that value anyways.
	
	Then, we toss on Margin-Based. -55\% for 11 and -65\% for 13. Yes I do go above its max 10, bite me. This gives -11.25\% -8.75\%, summed to -20\%.
	
	\subsection{Equipment}
	
	\subsubsection{Fake SINs}\label{bts_fake_SIN}
	
	I'm not a fan of the Basic Set Temporary Identity. There's firstly no disparity in rating, but especially the weekly roll against 8 makes it incredibly useless for many runners, as you'll be buying a new SIN every mission and likely needed to jump ship constantly on your housing. It's not the worst, and perhaps fits the grungier Neuromancer style of cyberpunk, but I don't think it's great for Shadowrun, which characterizes the SIN registries as labyrinthine, corrupt, and almost useless outside of a given jurisdiction.
	
	These are a custom case of Alternative Identity, Illegal [15]. Firstly, I make use of the One-Use, x1/5 multiplier, commonly seen for things like Favor on contact. I modify it by Unreliable, to create an ability to fails via the unreliable skill (i.e. fail on a <=11, -20\%). This modifies the x1/5 and I also flip the dice so that it's roll to succeed not roll to fail, and also extended it a bit, giving:
	
	\begin{itemize}
		\itemsep 0pt
		\item One-Use 19, x1.0
		\item One-Use 16, x0.84
		\item One-Use 13, x0.52
		\item One-Use 10, x0.36
		\item One-Use 7, x0.28
		\item One-Use 4, x0.24
		\item One-Use 1, x0.20
	\end{itemize}
	
	Then, I have to decide on a monthly interval check, as we see with the Temporary Identity equipment. I initially considered Accessibility, declaring a "base" time and using this to calculate it, but it doesn't really produce meaningful results. Instead, I simply settled on -80\% being weekly, and every +20\% incrementing a category, giving:
	
	\begin{itemize}
		\itemsep 0pt
		\item Weekly, -80\%
		\item Biweekly, -60\%
		\item Monthly, -40\%
		\item 6 Months, -20\%
		\item Yearly, +0\%
	\end{itemize}
	
	As well, I allowed for partial points, because I'm using the 3000¥ === 1 point system, which gave me some results for really shitty SINs. These results get pretty close to the Cyberpunk p20 recommended prices, which I consider a success.
	
	\subsubsection{Armor}
	
	All armor is made using Pyramid \#3/85. I've \textcolor{Blue}{\href{https://github.com/ingeanus/GURPS_Eidetic_Memory}{got a program out there to make it much faster,}} if you want to make some armor yourself. One of the important things is that I had to remove the groin section so that we get full torso sans groin.
	
	Actioneer Business Clothing: This one's going to actually be fairly unique. Since Shadowrun is early TL9, Basic Nanoweave is obviously banned, and we could use STF (Also known as Reflex)... but Arachnoweave is perfect for a fancy ass business suit... However, that's too expensive for its style so Reflex it is... I'll save it for the Run \& Gun Suits. It also has an undercover holster, with cost and weight included.
	
	Armor Clothing: Due to its description as being T-Shirt like, I made it on the high end of light clothing. It's lacking groin coverage.
	
	Armor Jacket: Obviously bulkier than something like the Actioneer, given that it's often described as a hoodie of sorts. It's lacking groin coverage and also only has 1/2 Skull for rear only.
	
	Armor Vest: Chest only, but covers groin and is in between armor jacket and Actioneer.
	
	Chameleon: Firstly, this includes a Thero-Optic Cameleon Surface, adding its cost and weight. It's nominally the same DR as a vest, but it also covers the entire body! Therefore I've made it much lower (Because it's supposed to be a stealthy piece of equipment, so it should impact equip load as little as possible).
	
	Full Body Armor: This is pretty simply DR 18 to the full Torso, Limbs, and Neck. It lacks boots and gloves because those are often just separate. It's also the same DR as the jacket, since covering more spots should make up for it.
	
	Trauma plates however.. those are difficult. The average Trauma plate is about \(0.806 ft^2\). If you have a good grasp on area, that is a good bit less than the front half of your upper chest. These things are designed to protect your organs. Not your stomach. Not your groin. Not your side. Not your entire torso (Excluding when combined with additional plates.. which weight much more).
	
	We'll be using these sizes... from games I've played in, plates can be \textit{really good} for their weight, making characters near unstoppable to non-AP rounds to the torso. Random hits improves this a fair amount, but will never put someone down in one hit. To make the change a little bit less lethal, I'm using a surface area of 2.5 instead of 2.75 (1/2 chest) for the plates. The larger plates are torso sans groin.
	
	The helmet is a lot easier; Solid Titanium over skull for 3lb dr 18. Visor is a bit of a cheat, Polymer Nanocomposite but I allowed it to be Transparent.
	
	TODO: Re-visit with respect to cost in mind.
	
	\subsubsection{The Spreadsheet}
	
	\textcolor{Blue}{\href{http://forums.sjgames.com/showpost.php?p=2124462&postcount=37}{Used to calculate TL 9 bullets.}}
	
	One of the sticking point with Douglass Cole's spreadsheet, is the fact that its 1/2D range are both more realistic and also entirely don't match the GURPS 1/2D. For long arms, they should be doubled - while for pistols they should be halved! I have opted for realism (Not that it even matters), but if the inconsistency bothers you, feel free to fix it.
	
	\subsubsection{Equipment Weight}
	
	One of the major staples of Shadowrun's equipment is how much it is just "better tody guns". Most of the firearms are heavily inspired by real life versions, just with things like Smartlinks and Caseless ammunition. Most of the improvements naturally come to weight, through the use of more advanced materials and techniques.
	
	\textcolor{NavyBlue}{\href{http://futurewarstories.blogspot.com/2012/10/fws-armory-caseless-ammunition.html}{Casless}} \textcolor{Blue}{\href{http://www.projectrho.com/public_html/rocket/sidearmslug.php}{Ammunition}} is bullets that have the actual projectile embedded in a solid propellant with no brass casing. They're most useful for having no spent casings to leave as evidence and saving immensely on weight. As shown in the links, the G11 had a 5.1:1 for 7.62mm and 2.08:1 ratio of bullets for 5.56. Another real example is the LSAT, which currently sees 44/43\% weight reductions in weapon weight and 40\% less ammo weight for Cased Telescopic Ammo (1.65:1).
	
	Ultratech varies on the improvements. 10mmCLP to 10mm auto is 1:7 (Which, admittedly the CLP is a garbage designed round by weight). 5.57CL to 5.56 NATO is 1:2.08. 7.62x39 to 7mmCL is 1.33:1, while 7.62 NATO is 2.07:1. 18.5mm vs 12G is 1.95,1.4,1.19,1.08:1, depending on your 12G shell. While the 4.73 was a smaller round, the Ultratech numbers cut real close to what was already accomplished in the 1900s with the G11.
	
	 I'm assuming that the technology has progressed reasonably, maturing for general use and slightly improving since then. Barring outliers like the 5.1:1 for the G11 and 1:7 for 10mm CLP, it hangs around a ratio just under 2:1. This makes general sense, as estimates seem to place it saving 40-60\% weight. We can take the easy route and use a 2:1 ratio, i.e. halving all bullet weights.
	 
	 For weapons, it's probably likely that we'll see some level of improvements, but not anything amazingly drastic. Given that the LSAT saved 44\% weight compared to its bretheren, it's possible we could also shoot for 50\% weapon weight reduction. However, this doesn't make sense for a lot of firearms; weight is a major factor in felt recoil, so while it's great to lower a heavy LMG's weight, a pistol is not so much. As well, strength correlates extremely closely with weight of a material, meaning that weight reductions tend to reduce material strengths, while we're also increasing the chamber pressure of the rounds too! As such, it could help heavy weapons a good bit, but less so for others.
	 
	 However, the Ultratech ones are pretty palid in this respect. The prime comparison is the Light Support Weapon, weighing 15 lbs compared to the LSAT's 9.8/9.9 lbs. That's closer to the SAW's 17lb empty. We see this a bit across the board: Most TL8 Rifles hit 7-9 lbs, Ultratech's are 7-8 lbs. TL8 Pistols are just under 2-3, TL9 is 2-2.5/3. Overall there's \textit{very marginal} improvements, when the LSAT demonstrates that likelihood for great savings. I don't think this is so much a factor of CT Ammo though,  as the SAW is an early TL8 weapon, while the LSAT is a a late TL8 / early TL 9 weapon, so there are lots of improvements in the interim, such as simpler actions and improved materials. As well, based on early edition weights, they were closer to UT level's of improvement.  I'm going to shoot for 30-50\% reductions for LMG size, 10-20\% for Rifle, and 5-15\% for Pistol size.
	 
	 \subsubsection{Equipment Cost}
	 
	 This is not as simple as, TL9 character has 1.5x the money, so 1.5x TL8 gun and ammo cost. Weapons don't just miraculously become more expensive. Additionally, most Shadowrun guns lack components considered standard in Ultratech's TL 9 guns! In fact, prices are pretty similar across the board, with perhaps a small markup overall. This seems to range from 0-10\%, so I'll markup guns and magazines (All are high density alloys anyways) by 5\%. Rifle magazine costs are 1/3 cost, because otherwise they're like 3x as much. I'm doing 1/2 for other magazines too, since they're costing a bit too much in comparison.
	 
	 Overall, ammo is much cheaper. This ranges from 0\% to around 60\%, with the majority being 40-50\% discounts, and higher discounts on smaller calibers. I'm going to go with 60\% pistols, 50\% rifles, 40\% anything bigger. Ofc, we'll add cost mulitpliers for weird stuff like MF as normal.
	 
	 \subsubsection{Ammunition}
	 Overall, the story is simply: Halve the weight of the ammo. Some ammo isn't in HT though. 
	 
	 10mm MF is based on 10x25mm auto.
	 
	 28G is 0.07  WPS. and 0.4\$ base.
	 
	 7.62mm CL is based on the 7.62x39mm.
	 
	 10mm CL is an entirely new one. It's supposed to be a somewhat more realistic 10mm round at 288 grains. We're assumign the full cartiridge is double the weight. We'll price it like 10x68mm Mauser at 1.5\$.
	 
	 7mm CL is also entirely new. It's made as 189 grains, also assumed to be doubled weight in full. I'll put it at 0.7\$ Base.
	
	\subsubsection{Assault Cannons}
	
	Assault Cannons are kind of weird, in that they don't have a very grounded real world contemporary and also have some truly hysterical claims at times. They are ostensibly man-portable cannons that make use of "rounds used in some light tanks". That's pretty vague and that's also a vague category, which could reasonably range from say 20mm to 100mm or so, mostly around 70-75mm or below, which is up to 60d with 5d [4d-1] follow-up for tank designs - obviously much less if hand held, if we try lowering barrel length and such on douglas' sheet we can get a high of around 40-50d. On the low end we have 20mm autocannons dealing 18d with 2d[1d] follow-up in a semi-semi-semi-portable fashion, so it could be 16d if made more portable. So a range of 16-50d before TL9ing them seems reasonablish. TL9ing gives abooout 14\% more damage, so we get a final range of 18d+1 to 57d. 
	
	Ares Thunderstruck: Low/Very low end of the range. Not quite as low as the Portable Railgun, but slightly above. See References for the basis, but I love GURB's gauss rifles. +25\% weight on magazines and +1 C cell for good measure. Add in that the TL 9 railguns are very similar but with +1 ST and we're including that. But then, we have the free recoil; the railgun is 188.83 Ns where as a comparative .50 BMG (ST 13M) is 38.72 Ns, all while weighing similarly to a Barret .50 cal. Of course that's a really small picture, e.g. the Browning 30-06 firing 8Ns rounds but being ST 16 due to its weight, mountings, ammo capacity and full-auto. There's not great guesstimate, but I'll say that since it's 4 times the free recoil, I'll add the sqrt of that to the Barrett's Recoil for 15, then +1 for TL 9's higher ST. Upped to 40 lbs to be somewhat comparative with these 20mm freaks.
	
	Krime Cannon: Middle of the range, based loosely off the Vulcan. I used the Blaster Design to guesstimate ST and weight multipliers for riflizing, giving ST 20 and 150 lbs. I'm gonna lower that a bit more due to the overhead of the Vulcan for mounting and so on to 19 and 120 lb. Magazine is (6*0.29*1.1) = 1.914. Cost is (6*5*0.29+33) + 5*6 = 72.
	
	Panther XXL:  High end of the range. Ammo is placed as 40mm since it's just under of light tank but could also be.. "portable". So that's something. Given that it's around 28d for 37mm and 60d for 75mm, we'll go in between with 30d, upped to 35d with TL9. Most any cannon of 50mm design is around 350 lb with 24+ ST (if recoilless) or 38 or so if not.. Using the same heuristics as above, we'll lower that to 64/125 that, so 180 lb and say 21 ST. Magazines are (15*1*1.1) = 16.5=17. Cost is  (15*1*5+33) + 15*10 = 258 = 260
	
	\subsubsection{Holdout Pistols}
	
	Fichetti Needler: Flechette rounds are always a bit of a hassle on small caliber weapons. We'll assume it's the minimum caliber (10mm), This gives NP of (10mm/2mm)\^3 / 40 = 3.125, giving NS 0.57. 10mmCLP is consistently 3d pi+, 3d*NS=5.88~==1d+2. For it's weight, I'm going in between the other two palm pistols for 1.2 lbs unloaded. Ammo is 4*0.021*1.1 = 0.0924. Dedicated Flechette guns will be somewhat less expensive, and their inability to switch off flechette rounds will improve that somewhat. I'll make it 120¥. MF is x4 cost. Magazine is (5*4*.24+33)*1.05 = 39.69/2 + .96*4 = 23.685. Programmable camoufalge is 1000¥ and 2lb, but for SM -7, we need to scale it down to 1/200. 5¥ and .01lb
	
	Streetline Special: Baby Browning. .25 ACP with TL9 improvements. This is one of the few cases I'll use the High-Tech ranges over Cole's, because 850/2,600 seems.. a bit off? 0.6*.9. 6*0.006*1.1 =.04. Magazine is (5*6*.04+33)*1.05 = 35.91/2 + ammo cost 6*.04 = 20.355¥. I'm just gonna round up to 175 for the composite materials.
	
	\textit{\textcolor{OliveGreen}{Cole Spreadsheet Statistics: 25000 6.4 15.6 6.4 53.6 180 1.8 3 6.4 180}}
	
	Walter Palm Pistol: Remington Model 95. Because of its high damage, we'll consider it in .45 ACP, like the Bond Arms Derringer. That weights 18.5 oz so weight medium optimization it's (1.156)*.9 = 1.04. Ammo is .0235*2 = .047. 
	
	\textit{\textcolor{OliveGreen}{Cole Spreadsheet Statistics: 21875 11.43 23 11.43 76 230 1.5 5.5 11.43 230}}
	
	\subsubsection{Light Machine Guns}
	
	Ingram Valiant: This one is a bit difficult. There's no official art or description I could find, so all there is to go on is its 5.56 equivalent damage and clip size. Nothing stick, making me think it's an original. All in all, I think I'll base it off an M249, because its same caliber, relatively close clip size, and unrepresented elsewhere for how iconic it is. I'm using LSAT weights. Ammo is 100*0.0135*1.6 (Plastic Drum) = 2.16. Cost is (100*0.0135*5+33) + 100*.25 = 64.75 = 65
	
	\textit{\textcolor{OliveGreen}{Cole Spreadsheet Statistics: 68750 5.7 45 9.29 521 62 4.1 11.1 5.7 62}}
	
	GE Vindicator: While this is classified as an LMG minigun, it uses Gunner skill due to its use case. Because it has lower damage, it probably doesn't use 7.62 like its inspiration M134. We'll say it instead uses 5.56 like the XM214. The XM214 had pretty variable RoF from 7! to 100!, with some versions having 16! and 66!, and official in 16! and 100!. For rule of cool and nothing else I'll match the GAU-2B/A (HT 135) in 50! and 100!. ST is difficult: It's obviously much less than the 20M of the original and 17M of other MMGs which are 50lbs. It itself is clocking in with an empty weight of 22 (Before TL9 materials). It would also be above 22lb/11ST  or 30lbs/12STLMGs, due to lots of ammo, battery weight, ammo, and feeding system. I'm settling on a middle ground of 14ST for now. This also, really makes me question the clip sizes.. LMGs being 1/2 size is one thing... 1/5 - 1/10 for most miniguns? That's really low. This thing only gets 2-4 seconds of shooting.
	
	Weight is difficult here. We'll use the XM214's weight of 22 lb, with a medium optimization of 30\% for 15.4 lb. Add on 5 lb for integral D cell for 20.4 lb. Ammo is 200*0.0135*1.6 (Drum magazine as seen in art) + 200*0.0135 = 7.02 lb. 100lb one is 100*0.0135*1.6 + 100*0.0135 = 3.51. Cost for 200 is 96.5, round up to 100. 100 is 64.75, round to 65.
	
	\textit{\textcolor{OliveGreen}{Cole Spreadsheet Statistics: 68750 5.7 45 9.29 533 62 4.1 11.1 5.7 62}}
	
	\subsubsection{Medium Machine Guns}
	
	Stoner-Ares M202:
	
	\subsubsection{Heavy Machine Guns}
	
	RPK HMG: The original RPK was an LMG in 7.62, so it will obviously not fit the bill here. However, the NSV (Or the Kord) both serve as a similar enough example that matches well in all aspects. I'll use the NSV Weight with heavy optimization for 40\% (55-16.9)*.6 = 22.86. Ammo is  100*0.155*1.1 (Estimate for disintegrating belt or similar method) = 17.05.
	
	\textit{\textcolor{OliveGreen}{Cole Spreadsheet Statistics: 65000 12.98 108 12.98 1560 745 4.1 52 12.98 745}}
	
	\subsubsection{Light Pistols}
	
	Ares Light Fire 70/75: Both of these are originals as far as I can tell. They're basically identical, so I'm using the same statline for them too. They're low damage light pistols, so I based it off .380ACP with a short barrel length. Weight is based off old edition weights.  +0.4 for 75's silencer with +0.1 for better sealing and design to justify no increase of bulk. As for the magazine, it's 16*0.0105*1.1=0.19. Since the light fire's supposed to be super cheap, the 70 I'll make 580¥. The 75 is operator's, so I'll bump it to 600¥. Magazine is (5*16*.08+33)*1.05 = 41.37/2 + 16*.08 = 21.965. Supressor is 600¥ and .5 lb (Already included above).
	
	\textit{\textcolor{OliveGreen}{Cole Spreadsheet Statistics: 26875 9 17.3 9 64 95 1.85 6.7 9 95}}
	
	Beretta 201T: Based off the M9. High end weight optimization, so (2.8-0.5)*.85 = 1.95. Bullets are 21*0.013*1.1 = 0.3. Magazine is (5*21*0.12 + 33)*1.05 = 47.88/2 + 21*.12 = 26.46. Shoulder stock is 100, and I'll knock weight to 0.4lb
	
	\textit{\textcolor{OliveGreen}{Cole Spreadsheet Statistics: 41125 9 19 9 125 124 1.85 8 9 124}}
	
	Colt America L36: Based off something like the Defender. Normal weights 24oz and edium weight optimization, so (1.5*0.9)=1.35. 11*0.0235*1.1 = 0.28. Defender is 1000\$. Magazine (5*11*.2+33)*1.05 = 46.2/2 + 11*.2 = 25.3
	
	Fichetti Executive Action: This one is likely an original, but it does bear a little resemblance to glocks. Either way, its stats align with 9mm, so we'll be using that. I'll make it's weight the Glock 18 with low optimization (Even though the creusader is as well shut up), so (2.6-1.1)*.95 = 1.425. Ammo is 18*0.013*1.1 + 18*.013 = 0.49 lb. I'll make this 650¥ since I'll lower its Acc and RoF. Magazine (5*18*.013+33) + .12*18 = 36.33 = 37
	
	Fichetti Security 600: This one is definitely an original. I'm going to base it off the glock 19 since the executive was a glock too with high optimization. (1.8-.5)*.85 = 1.105. Loads are .013*30*1.1 + .013*30 = 0.819 = .82. Cost is 840 like glock. Load is (30*5*.013 + 33) + .12*30 = 38.55 = 38.6
	
	\textit{\textcolor{OliveGreen}{Cole Spreadsheet Statistics: 41125 9 19 9 115 124 1.85 8 9 124}}
	
	\subsubsection{Heavy Pistols}
	
	Ares Predator V: Seems like it was based off the Deagle. Light Optimization, since weight is an important feel for a Deagle gun, (4.6-0.6)*0.95 = 3.8. 15*0.0335*1.1 = 0.55. Magzine (5*15*.4+33)*1.05 = 66.15/2 + 15*.4 = 39.075. 550 for smartgun.
	
	\textit{\textcolor{OliveGreen}{Cole Spreadsheet Statistics: 45000 12.7 33 12.7 152.4 300 1.6 9.625 12.7 300}}
	
	
	TODO: Lower caliber.
	Ares Viper Silvergun: Once again, flechette guns are original. This one is intended to be more deadly than the Needler, so it'll have to be higher than 10mm. We'll go with 28 gauge (13.97mm) even though it's a mild bit comically large, because it's common use, and will give a noticeable improvement from the minimum 10mm. NP = (13.97/2.13mm)\^3/40 = 7.1, NS = 0.38. Unfortunately, Cole's sheet won't do 28G well, and Ultratech shotguns are closer to a downgrade from High Tech ones (Shorter Range, same damage, similar shots and RoF, same pellet count, etc). I'll give it 2.8 lb unloaded, inbetween most other ST 12 and 10 guns. Ammo is 30*0.035*1.1 = 1.155. I'll make it 1,150¥. Magazine is (5*30*.2+33)*1.05 = 66.15/2 + 30*.2 = 39.075
	
	As such, we'll have to eyeball. 12G slugs are 4d+4, functionally equivalent to 5d. A similar round to that is 5.56, which when TL9'ed usually becomes 6d. 6d/4 = 1d+2, which would imply +1 damage to buckshots. Because 28G would likely be 1 step worse than 20G, which is usually 1d pi-, this would imply 28G is 1d-1 pi- and becomes 1d pi- for TL9. This is 4d pi+, which becomes 1d+2 pi- flechettes. Since 20G is around 9†, if we lower to 7-8, then x1.5 we get ST 11 or 12. We'll go with 11 since it's considered a common weapon.
	
	\textit{\textcolor{OliveGreen}{Cole Spreadsheet Statistics: None, see above.}}
	
	Browning Ultra Power: Obviously based off the Browning Hi-Power. .40S\&W given the TL9 treatment. Old, so low optimization, (2.4-0.5)*0.95 = 1.81. 10*0.0175*1.1 = 0.19. Magazine (5*10*.12+33)*1.5 = 40.95/2 + 10*.12 = 21.675. 75 for laser.
	
	\textit{\textcolor{OliveGreen}{Cole Spreadsheet Statistics: 42468.75 10 21 10 119 180 1.8 3.4 10 180}}
	
	Colt Government: Its description literally says it. Given it the TL9 treatment. Medium optimization as middleground between 3 TL difference and needing to maintain the old style, (2.8-0.5)*0.9 = 2.07. 14*0.0235*1.1 = 0.36. Magazine (5*14*.2+33)*1.05 = 49.35/2 + 14*.2 = 27.475.
	
	\textit{\textcolor{OliveGreen}{Cole Spreadsheet Statistics: 21875 11.43 23 11.43 127 230 1.5 5.5 11.43 230}}
	
	Remington Roomsweeaper: Likely an original. Due to its DP it's likely not 12 gauge, 20 gauge would be around 13ST, so 28 gauge it is. To differentiate between the Silvergun, we'll go with 2..57mm flechettes. (13.97/2.57)\^3/40 = 4 = NP, so NS = 0.5. Since the Silvergun decided TL9 28G is 4d slugs, we get 2d pi-. While similar in weight to the Silvergun, I'm gonna assume it's a bit more for its versatility. I'll make it 2.9 unloaded. 8*0.035*1.1 = .308. Supposed to be pretty cheap, so 400¥. Magazine is (5*8*.2+33)*1.05 = 43.05/2 + .2*8 = 23.125
	
	\textit{\textcolor{OliveGreen}{Cole Spreadsheet Statistics: None}}
	
	Ruger Super Warhawk: Obviously based on the Ruger Super Redhawk. The issue is that, the .454 Casull is MUCH more powerful on Cole's sheet, pushing 6d pi+. This is largely because it seems to overtune the velocity (Reaching 580 m/s instead of 490ish, which is still 5d+1). This seems to be a discrepancy just between Cole's and Hurst's work, and I'm in favor of Cole's more, since other less powerful rounds do 5d. So we'll TL9 ify from Cole's numbers, fudging them a bit to get the more correct velocity.
	
	Medium optimization, so (3.6-0.4)*0.9 = 2.88. 6*0.33 = 0.2. 
	
	\textit{\textcolor{OliveGreen}{Cole Spreadsheet Statistics: 81250 11.43 35 11.43 330 335 1.5 1 11.43 335}}
	
	Taurus: Taurus makes a lot of revolvers. I'm choosing the Taurus Tracker due to their similar descriptions (Focus on ruggedness) and its ability to be chambered in both 357 and 38 Special +P. The original weighs 40 oz unloaded, and I think light optimization is best based on its description, so (2.5)*.95 = 2.38. For .357 it's 6*0.0175 = 0.105. For 38 special it's 0.0165*6 = 0.1. Tracker is 600\$. 75¥ for laser sight.
	
	\textit{\textcolor{OliveGreen}{Cole Spreadsheet Statistics: 21875 9.1 29.3 9.1 165 125 1.5 6 9.1 125}}
	\textit{\textcolor{OliveGreen}{Cole Spreadsheet Statistics: 43125 9 33 9 152 125 1.7 2 9 125}}
	
	\subsubsection{Machine Pistols}
	
	Ares Crusader II: This one has no real equivalent. The best we can say is that it's high damage, high accuracy, but with large clip size. Guess I'll go with glock 18. We'll go with medium optimization, so (2.6-1.1)*.9 = 1.35. For ammo, 40*0.013*1.1 = 0.572. We'll cost it as 1,000¥. Magazine is (5*40*.12+33)*1.05 = 59.85/2 +40*.12 = 34.725. 88 for compensator with .05lb
	
	\textit{\textcolor{OliveGreen}{Cole Spreadsheet Statistics: 41125 9 19 9 114 124 1.85 8 9 124}}
	
	Black Scorpion: CZ Scorpion. Although I vaguely think High Optimization could work so such an old gun, I'm going to stick with Medium to be safe, (3.7-0.9)*.9 = 2.52. 35*0.009*1.1 = 0.347. Magazine (35*5*.04+33)*1.05 = 42/2 + 35*.04 = 22.4. 100 and 0.4 lb for stock.
	
	\textit{\textcolor{OliveGreen}{Cole Spreadsheet Statistics: 25625 8 17.3 8 115 73 1.35 9 8 73}}
	
	Steyr TMP: The.. uh the Steyr TMP. Medium Optimization (3.8-1)*.9 = 2.61. 30*.013*1.1 = .429. Magazine (5*30*.12+33)*1.05 = 53.55/2 + 30*.12 = 31.275. 75 for laser
	
	\textit{\textcolor{OliveGreen}{Cole Spreadsheet Statistics: 41125 9 19 9 130 124 1.85 8 9 124}}
	
	\subsubsection{Rifles}
	
	AK-97: Based on the AK-47 of course. It's assumed to have a round similar to the 7.62x39mm, just with the increased performance that is seen at TL9. This was done by estimating the qualities of the 7mmCL, and then applying those improvements to the 7.62x39mm. The gun is considered rugged and heavy, so low Optimization (10\% for rifles), so (11.3-1.8)*.9 = 8.55. Ammo is 38*0.018*1.1 = 0.752. Magazine (5*38*.3+33)*1.05 = 94.5/3 + 38*.3 = 42.9
	
	\textit{\textcolor{OliveGreen}{Cole Spreadsheet Statistics: 66000 psi, 7.36mm barrel bore, 39mm case length, 7.36mm chamber bore, 420mm barrel length, 122 grain bullet, 3 Aspect Ratio, 32mm burn length, 7.36mm caliber, 122 grain accelerated mass.}}
	
	Ares Alpha: This one is much simpler, as it seems to be a Shadowrun original, as such it's likely just a sane version of the Storm Carbine. I worked back from a semi-accurate 10mm round to 10mmCL, then modified it a bit to match High-Tech rules, such as being pi+. The grenade launcher is the Ultratech Grenade Launcher, losing 0.5 lbs for losing RoF 3, and gaining .25 + .15 lbs for 1 more grenade. I'm basing its weight off of the SCAR-H, high optimization (8.8-1.6)*.8 = 5.76. 42*0.0415*1.1 = 1.92. Magazine (5*40*.25+33)*1.05 = 87.15 + 40*.25 = 97.15. I like its book price, but we'll lower for grenade launcher and such 2350. Magazine (5*42*.75+33)*1.05 = 200.025/3 + 42*.75 = 97.175. +550+88 for smartgun and compensator. + 200 for GL
	
	\textit{\textcolor{OliveGreen}{Cole Spreadsheet Statistics: 35000 psi, 10.6mm barrel bore, 72.4mm case length, 10.6mm chamber bore, 400mm barrel length, 280 grain bullet, 3 Aspect Ratio, 60mm burn length, 10.6mm caliber, 280 grain accelerated mass}}
	
	Colt M23: Obvisouly based of the Colt series of rifles, likely the AR-15, but perhaps the M16. The round is based off the 5.56 with a proportional boosting similar to the 7.36 round. Medium optimization, (7.2-0.7)*0.85 = 5.525. Ammo is 40*0.0135*1.1 = .594. Magazine (5*40*.25+33)*1.05 = 87.15/3 + 40*.25 = 39.05
	
	\textit{\textcolor{OliveGreen}{Cole Spreadsheet Statistics: 68750 psi, 5.7mm barrel bore, 45mm case length, 9.29mm chamber bore, 508mm barrel length, 62 grain bullet, 4.1 aspect ratio, 12mm burn length, 5.7mm caliber, 62 grain accelerated mass.}}
	
	FN HAR: is in a bit of a weird position. On the one hand, it has a lower damage than guns like the Ares Alpha, but is billed as a Heavy Assault Rifle, which is supported by its lower clip size and ostensibly being on the FN FAL. It may be based on the FAMAS, as seen by its video game depiction however. I've decided that lower damage alongside its video game depiction lean towards the FAMAS. With medium optimization it's 7.96*.85 = 6.77. 35*0.027*1.1 = 1.04.
	
	\textit{\textcolor{OliveGreen}{Cole Spreadsheet Statistics:  56000 psi, 7mm barrel bore, 43mm case length, 9mm chamber bore, 488mm barrel length, 189 grain bullet, 3 aspect ratio, 32mm burn length, 7mm caliber, 189 accelerated mass.}}
	
	Yamaha Raiden: Exact same as Ares Alpha, but with a longer barrel. I'm just increasing its weight by 0.5, 5.76 + 0.5 = 6.26. Ammo 60*0.0415*1.1 = 2.739. Bit more than Alpha, so 2,450¥. Magazine (5*60*.75+33)*1.05 = 270.9 / 3 + 60*.75 = 135.3. +550+88+600 for smartgun, compensator, and silencer.
	
	\subsubsection{Sniper RIfles}
	
	Sniper universally ignore Douglas' ranges from the Spreadsheet. They all simply use the High Tech values - and will perhaps receive a slight boost due to better ballistic performance... but that will come later.
	
	X Ares Desert Strike: Most snipers aren't heavily based on real-life equivalents, so we'll have to make do with what we can. Given what I've set the others as, we've got to get an AI AW in here somewhere and a lot of people draw this one similarly. Given that it's a mid value sniper optimized for extreme environments: low optimization, (15-1)*0.9 = 12.6. I'm increasing ammo capacity for the 7.62 to 20, because the Shadowrun version has high damage with 14 shots, so that's the .300 chambered one, and they halve shots with the AWM-F.. but I'm going safe on the low side and matching ammo capacity of guns like the FAL. So 7.62 is (20*0.028*1.1) = 0.616 lb. .300 is (14*0.0375*1.1) = 0.5775 lb. Cost is (20*0.028*5+33) + (20*0.4) = 43.8¥ and (14*0.0375*5+33) + (14*.75) = 46.125¥. Cost mods are simply 750 for scope. 4700 and 5800 for the rifles.
	
	7.62 version matches Remington down below.
	For .300:
	\textit{\textcolor{OliveGreen}{Cole Spreadsheet Statistics: 67500 7.62 66.5 12.42 609.6 180 4 30 7.62 180}}
	
	Barrett Model 122. Obviously based on the Barrett 50 cal. High optimization (32.7-4.4)*0.8 = 22.64. Ammo is  14+1, so  14*0.125*1.1, however the Barrett magazine seems about 15\% heavier, so we'll increase to 4.5 lbs. Cost is (0.125*5*14+33)*1.05 = 41.75. 14800 +550 smartgun, +600 silencer.
	
	\textit{\textcolor{OliveGreen}{Cole Spreadsheet Statistics: 68750 12.7 99 19.2 1070 660 3.8 14.7 12.7 660}}
	
	A+B+C+D
	Cavalier Arms Crockett EBR: There's no real world equivalent for this, so it's just an original. Its damage is similar to Ares Alpha / Raiden, so it's around 7d. Since it's better AP I'm going with 7.62 again, but with a longer barrel to get it up to 8d; lower wounding mod, but better damage. Weight is hard to decide,  It looks like a stockier body, but shorter overall, so I'll place it at 10 lbs. Magazine is (20*0.028*1.1) = 0.616. Cost is (5*0.028*20+33) + (20*0.4) = 43.8. Since it's cheaper in Shadowrun, we'll make it 3,500. + 750 for scope.
	
	\textit{\textcolor{OliveGreen}{Cole Spreadsheet Statistics: 62500 7.62 51 11.53 650 150.5 4.265 23.6177777777778 7.62 150.5}}
	
	Ranger Arms SM-5: We'll go with the CheyTac for this one due to its high accuracy, caliber, and semi-auto. Medium optimization because while it is probably well optimized, I'm making it lower due to being easily disassembled (34.5-1.5)*0.85 = 28.05 + 1lb scope. (15*.07*1.1)  = 1.155 lb. (5*15*.07+33)*1.05 = 40.1625. 11500\$ + 600 silencer +  1000 scope.
	
	Remington 950: Based off Model 700.  Low Optimization so (7.8-0.3)*0.9 = 6.75 +1lb scope.   Unlike its original, it has a 5 round magazine, so (5*0.028*1.1) = 0.154. (5*5*0.028+33)*1.05 = 35.39. 450*1.75 fine + 750 scope 
	
	\textit{\textcolor{OliveGreen}{Cole Spreadsheet Statistics: 62500 7.62 51 11.53 609.6 150.5 4.265 23.6177777777778 7.62 150.5}}
	
	A+B+C+D
	Ruger 100: Based off the Ruger Mini-14. I know it's not bolt action and ruger's got tons of those I could base off.. but I don't care that much at this point it's the last gun and it sucks and most of those are like 22s and would do dinky damage not the 11P it should be. Low optimization so weight is (7.5-0.9)*0.9 = 5.94. Magazine is (8*.013*1.1) = 1.1144 lb. Cost is (8*5*.013+33) + (8*.25) = 35.52. Cost is 750 scope + 655 gun -28 old magazine.
	
	\textit{\textcolor{OliveGreen}{Cole Spreadsheet Statistics: 68750 5.7 45 6 559 36 3 30 5.7 36}}
	
	A+B+C+D
	Terracotta ARMS AM-47:  We're going with 14.5 for the caliber because fuck you that's why. Also just because it's a round used in modern AM Rifles and is larger caliber than .50 for that higher damage. I'm vaguely basing off the Snipex Alligator because it fits that bill, so we'll go with a weight of 55 with low optimization for 55*0.9 = 49.5. The only 14.5 I'm aware of in GURPS is the Zid KPV, which is 104 lb and ST 23.  Since it's about double the weight and fully automatic I'll assume that we can divide ST by sqrt(2)  for ST 16B. Magazines are (18*.22*1.1) =  4.356. Cost is (5*.22*18+33) + (3.4*18) = 114. Zid costs 18k,  and the shadowrun version is more than the Barretta, so we'll go with a mild lowering for not full auto to 16k + 550 smartgun + 750 scope + 250 for night vision, priced like the E-OP Surveillance Camera.
	
	\textit{\textcolor{OliveGreen}{Cole Spreadsheet Statistics: 65000 14.5 114 14.5 1200 926 4 90 14.5 926}}
	
	
	\subsubsection{Shotguns}
	
	Shotguns are a bit troublesome overall. Cole's sheet doesn't easily handle them, and his works dealing with multi-projectile... are not to my taste (Which is to say, I don't think that his method of grouping pellets into bigger meta-pellets is a great idea, nor do I think that things like 00 Buck should be 1d+3 pi with RoF 3x8.. I don't think a single 00 Buck pellet matches a pistol in terms of penetration). 
	
	As such, I'll have to be winging it, because (As I've mentioned in some other places), TL9 shotguns are garbage; they have worse range, worse accuracy, same damage, similar RoF, similar clip sizes, and similar weights. In total, they're just slightly downgraded TL8 shotguns. 
	
	I'll be TL9-ifying a shotgun by considering the improvement in damage a similar round got from TL9 and applying that to a slug. Then, I'll work backwards from the slug to get the improved shot damage. 
	
	12G: 4d+4 slugs are effectively 5d. 5.56 does 5d originally and 6d at TL9, which would give 1d+2 pi- for TL9 12G Shot.
	20G: 4d slugs. This damage is really uncommon unfortunately, but for similar ish damage weapons like some SMGs, we can guess at 5d slugs, which gives 1d+1 pi- TL 20G shot.
	28G: I've done 28G like twice before for pistols; it's 1d pi- shot, 4d slug.
	10G: 2d-1 shot.. At this point it seems obvious that it ends up 2d. And since I don't need range numbers, I'll take that!
	
	For ranges, I'm just tacking on 1/8 to both (The TL8 Ranges ofc).
	
	Defiance T-250: This one is a bit tricky. Official art looks very much like the Remington 870, but Shadowrun already Remington 990. Even more so, the 990 doesn't particularly look like the 870, instead looking like a Benelli or Remington 887. As such, I'll let the Defiance be based off the 870... not that there's much difference in shotguns anyways. I'll use medium optimization for rifles, so (7.6-0.6)*.85 = 5.95. 5*0.065 = .325.
	
	Enfield: This one is also a bit tricky, but it's likely the USAS-12 based off its official art. Because of its much higher damage, I'll load it with 10G (Dear god). Medium Optimization, (14.2-2.1)*.85 = 10.285. The USAS Clip is extremely heavier than others, likely because it's SHELLS. Since it's twice as much as expected, I'll double this one too, for 10*0.075*1.1*2 = 1.65. Magazine (5*.35*10+33)*1.05 = 53.025/3 + 10*.35 = 21.175. +75 for laser.
	
	PJSS: Another weird one. It's by description a double barreled shotgun, but doesn't look like that in the art... Oh well, double barreled it is. We'll base weight of the LeFever with heavy optimization, so (10.3-0.3)*.8 = 8. Ammo is  2*0.065 = .13.
	
	\subsubsection{SMGs}
	
	Colt Cobra TZ-120: The TZ is somewhat difficult because it doesn't have a clear description, drawing, or real world counterpart. The Cobra series are all revolvers and its folding stock makes it seem closer to a machine pistol than a straight up SMG. Its name is somewhat reminiscient of the CZ Scorpion too, but that matches with the Black Scorpion better, so we'll push closer to the CZ Evo 3. That one is 6.10 LOADED. I could calculate GURPS magazine weight, but some decent sites say around 5.62 lbs unloaded. Medium optimization, using pistol ones, so (5.62)*.9 = 5.058. Ammo is 32*.0.013*1.1 = .458. There's lots of mixed, but I'm going with a 1,450\$ price tag base. Magazine (5*32*.12+33)*1.05 = 54.81/3 + 32*.12 = 22.11 +100 for stock, +88 compensator.
	
	\textit{\textcolor{OliveGreen}{Cole Spreadsheet Statistics: 41125 9 19 9 115 124 1.85 8 9 124 }}
	
	FN P93 Praetor: Obviously based off the P90. As such, it got the upgraded 5.7 treatment. One of the noteworthy parts is that Cole's sheet has the 5.7x28mm's range much lower than the book. High optimization is possible, but the P90 is already a fair bit overoptimized, so I'm going with Medium for (6.6-1) = 5.04. Ammo is 50*0.0065*1.1 = .358. Magazine (5*50*.16+33)*1.05 = 76.65/3 + 50*.16 = 33.55 +88 compensator
	
	\textit{\textcolor{OliveGreen}{Cole Spreadsheet Statistics:  62547.5, 5.7 28 6.2 264 31 2.25 4.2 5.7 31}}
	
	HK-227: This is likely either based off the popular HK MP/5/7 or the HK UMP. Given its lower damage, accuracy, and clip size, I'm going with the UMP. Medium optimization, (6.5-1.6) = 3.6. Ammo is 28*.0235*1.1 = .724. Magazine (5*28*.2+33)*1.05 = 64.05 + 28*.2 = 26.95. +100+550+600 for stock smartgun and silencer.
	
	\textit{\textcolor{OliveGreen}{Cole Spreadsheet Statistics: 21875 11.43 23 11.43 200 230 1.5 5.5 11.43 230}}
	
	Ingram Smartgun X: Based off the Mac-10. Given the general upgrade treatment. Given it's considered a quality gun, we'll go with medium optimization over low, for (7.5-1.3)*.9 = 5.58. Ammo is 32*0.013*1.1 = 0.458. Magazine (5*32*.12+33)*1.05 = 54.81/3 + 32*.12 = 22.11. +100+75+550+88.
	
	\textit{\textcolor{OliveGreen}{Cole Spreadsheet Statistics: 41125 9 19 9 146 124 1.85 8 9 124}}
	
	SCK Model 100: This is a weird one. The \textcolor{Blue}{\href{https://en.wikipedia.org/wiki/New_Nambu_M66}{SCK-65}} was a Japanese test firearm that never made it production. However, its official arts looks strikingly like \textcolor{Blue}{\href{https://en.wikipedia.org/wiki/FB_PM-63}{the FB PM-63}}. I'm going to mostly take inspiration from the later, although it will be pretty modernized. We know it's mass was 3.53 lb, which I'll take as unloaded and add low optimization, since it seems to be on the low end already, for (3.53)*.9 = 3.177. Ammo is 30*0.013*1.1 = 0.429. Supposedly on higher end, so 1,200. Magazine is same as colt cobra.. +100+550 stock and smartgun
	
	\textit{\textcolor{OliveGreen}{Cole Spreadsheet Statistics: 41125 9 19 11 164 124 1.85 8 9 124}}
	
	UZI IV: Easily just the Uzi at TL9. Due to the fact that it has lower accuracy and damage than its comparables, I'm opting that it has a much shorter barrel than the original. Medium optimzation, for (8.8-1.1) = 6.93. Ammo is 24*0.013*1.1 = 0.343. Magazine (5*24*.12+33)*1.05 = 49.77/3 + 24*.12 = 19.47. +100+75 stock and laser.
	
	\textit{\textcolor{OliveGreen}{Cole Spreadsheet Statistics: 41125 9 19 9 120 124 1.85 8 9 124}}
	
	\subsubsection{Tasers}
	
	These are a lot easier. The Defiance can just be a stronger normal taser, while the Pulsar I simply doubled range to account for it being wireless. Other than that it's standard changes like shots and RoF. Given that their powers are all stronger than a normal taser, I'll keep their weights. Cartridge costs are just the 16\$ for Tasertron and Taser's 20\$ one.
	
	\subsubsection{Accessories}
	
	Laser Rangefinder is priced at 100 bucks, being effectively a laser sight + detector. Its weight is also just slightly increased from the sight.
	Laser sights are priced at 1/10 High Tech Cost.
	
	
	\subsubsection{Imaging Scopes}
	
	So, Ultratech has a pretty good selection of passive sensors and scopes.. but what it doesn't have is any way to generalize those or mix and match options. For instance, there are 0 scopes that use Night Vision. There are 0 scopes with x16 magnification. So, this is an attempt to generally work out a cost per Magnification and poundage from the Passive Sensors section and Scope Section (Also High Tech's) in order to create some contemporary options here.
	
	Let's start with Infrared in order to compare with the CTS.
	
	Mag/Acc		lbs		Cost		Cost/Acc	Notes
	64 (+6)		50		40000		6666.67		No peripheral, no terminal
	16 (+4)		3		2500		833.33		No peripheral, HUD, Laser Rangefinder, digital Camera
	4 (+2)		0.6		250			125			No peripheral, no terminal
	2 (+1)		0.6		500			500			Includes HUD, digital Camera
	
	Well. That's a little bit useful. We can very broadly say that the Laser Rangefinder, Digital Cameras, HUDS, and Terminals account for some of the increased Costs/Acc here, but not reallly how much that is. Additionally, those should be a one time cost while Acc should be a scaling cost, so we really want an (C-X)/A formula for the cost of an Infrared Accuracy instead, where X is all sums and minuses for notes.
	
	\href{https://www.wolframalpha.com/input?i=2X+%2B+P+%3D+250+%3D+X+%2B+A+when+A+%3E%3D+0%2C+P+%3C%3D+0%2C+X+%3D+250}{We have one solution here for the last three,} which is:
	Cost/Acc = 250
	Cost no Peripheral = -250
	Cost HUD + Camera + Terminal = 0
	Rangefinder = 1750
	
	This is fine I think. HOWEVER, there's a big difference between something that provides image magnification and something that provides scope bonuses. The latter requires tons of design for precision and its specific usage that should increase the cost. Also, compared to the CTS, this is quite cheap. If we double cost for Infrared, then we get:
	Cost/Acc = 500
	Cost no Peripheral = -250
	Cost HUD + Camera + Terminal = 0
	Rangefinder = 1750
	
	This will match the CTS. Hyperspectral from UT61 seems to be x4 cost, which would make it 2k per acc and weight is different, being 0.11726 e\^(0.81055x), so 0.1, 0.25, 0.6, 1.3, 3, 6.7, 15.2 (idc about the big sensor weight tbh); but that doesn't match the ETS, so we'll go more in depth with hyperspectral too
	
	The immediate issue that stands out is the lack of similar magnification structures. Hypersepctral Goggles/Visors are double to cost.. for the same magnification. So it's ostensibly 1000\$ for a HUD and camera which is dumb, so that's a typo fuck you. If we simply go for x4 cost, then we get 6000\$ for a x8 scope, which is 2000 less than the ETS. Given its description and weight, I'm going to assume it's got a rangefinder on it too. for about 1750.
	
	\subsubsection{Drugs}
	
	Made using B429 \textit{Ultra-Tech Drugs}. For positive ones, treat the negative side effects as negative points not absolute value. For <= 0 point cost, divide cost by 1+|cost|.
	
	\textbf{Cereprax: 18}
	IQ +2 [30]; Per +2 [10]; Eidetic Memory [5]; Lose FP 2d6 (7) (Aftermath, -50\%) [-14]; Chronic Pain, Severe (4 hours, $\times$1.5; FoA 15, $\times$2; Aftermath, -50\%) [-15]; HT-4 to Overdosing [-4]
	Medium-Term (MoF Hours)
	Potency -3
	960¥
	
	\textbf{Cram: -7}
	+0.5 Basic Speed (No Basic Move, -25\%) [8]; Impulsiveness (SC 12) [-10]; Odious Personal Habit, -1 (Fidgeting and jittery) [-5]
	Medium-Term (MoF Hours)
	Potency -3
	10¥/20¥ for Aerosol
	
	\textbf{Deepweed}
	
	
	
	\textbf{Jazz: 3}
	+1.0 Basic Speed (No Basic Move, -25\%) [15]; Odious Personal Habit, -1 (Hyperactive) [-5]; Chronic Depression (SC 12; Aftermath, -50\%) [-7]
	Short (MoF*5 Minutes)
	Potency -3
	Aerosol
	96¥
	
	\textbf{Kamikaze: 2}
	DX +1 (No Basic Speed, -5) [20];  Basic Speed +1.0 (No Basic Move, -5)[15]; ST +4 (No HP, -2) [20]; Resistant, Pain, +3 [3]; Berserk (SC 12) [-10]; Megalomania [-10]; On the Edge (SC 12) [-15]; Lose FP 2d6+1 (8) (Aftermath, -50\%) [-16]; Chronic Pain, Mild (2 hours, $\times$1; FoA 15, $\times$2; Aftermath, -50\%) [-5];
	Short-Term (MoF * 5 Minutes)
	Potency -4
	Aerosol
	120¥
	Rounded down to 120 for hallucinations at high doses.
	
	\textbf{K-10: 9}
	DX +2 (No Basic Speed, -5) [40]; Basic Speed +1.25 (No Basic Move, -5)[20]; ST +8 (No HP, -2) [40]; Will +1 [7] High Pain Tolerance [10]; Berserk (No SC, *2.5) [-25]; Berserk (No SC, *2.5; Side Effect\footnote{Basically only if you fail by 5}, x1/5; Permanent, Medical Miracles, +300\%) [-20]; Lose FP 6d6 (-21) (Aftermath, -50\%) [-42];
	Short-Term (MoF * 5 Minutes)
	Potency -6
	Injection
	1,152¥, rounded down
	
	\textbf{Laes: 15}
	Innate Attack, Fatigue 1 (Cyclic, 1 sec, 12 cycles, +550\%; Symptom, Selective Amnesia, 1/2 HP, +10\%; Resistible, HT-5, -5\%; Self Only, -20\%; Blood Agent, -40\%; Melee Attack, C, -30\%) [-15]
	Short-Term
	Potency -5
	Ingestion, Injection, Inhalation
	960¥
	
	\textbf{Long Haul: }
	Doesn't Sleep [20]; Lose FP 1+4*4 = 17 (Aftermath, -50\%; Sleep, +50\%) [-68]; Extra Sleep 20 (Aftermath, -50\%) [-20]
	Very-Long Term (1/2 because 4 days)
	Potency -4
	Injection
	58¥
	
	\textbf{Nitro: 0}
	+2 Strength (No HP, -2 HP) [10]; +1 Will [7]; +2 Acute Vision and Hearing [8]]; High Pain Tolerance [10]; Lose FP 3d6 (10) (Aftermath, -50\%) [-20]; Short Attention Span (SC 12) [-10]; Odious Personal Habit, -1 [-5]
	Short (5*mof minutes)
	Potency -4
	64¥
	Aerosol
	
	\textbf{Novacoke: -12}
	Charisma +2 [10]; Will -2 (Aftermath, -50\%) [-7]; -30 in social disadvantages (Aftermath, -50\%)[-15]
	Medium duration (MoF Hours)
	Potency -4
	Aerosol
	12¥/24¥
	
	\textbf{Psyche: 0}
	+1 IQ [15]; Attentive [-1]; Perfectionist [-1]; Methodical [-1] Obsession (SC 9) (Short Term) [-7]; Odious Personal Habit, -1 (Obsession) [-5]. Sustaining bonus is simply -1 IQ, Per, DX (When sustaining only, -50\%) [-20] with +20 points for sustaining.
	Medium Duration (MoF Hours)
	Potency	-4
	160¥
	
	\textbf{Zen: }
	Just LSD
	
	Addiction Rating is simply : Shadowrun Addiction Rating  / 2, round down -1 if AT 3, +1 if AT 2.
	
	
	Poisons for the most part are really hard to get functional, so some of them are skipped here and just assigned stats.
	
	\subsection{References}
	
	This is a section to cover any references I used when designing this that did not make their way into the paragraphs themselves.
	
	\href{https://www.ravensnpennies.com/gurps101-how-to-build-a-technopath/}{How to build a technopath.}
	
	\href{https://gurps-sr.obsidianportal.com}{An Obsidian Portal wiki with solid group of pre-made items, drones, etc. They're not 100\% to my taste, but they serve as a good inspiration.}
	
	Stable Diffusion, for creating the "Art" here, because I am not an artist.
	
	\href{http://www.ci-n.com/~jcampbel/rpgs/shadowrun/weapons.php?altskills}{Has some good firearm weights, pretty sure it's based on the original edition's weights.}
	
	\href{http://www.ambient.ca/cpunk/shadowguns/subguns.html}{Used for a variety of gun ideas and art designs for better determining what to base the weapons off of.}
	
	\href{http://forums.sjgames.com/showthread.php?t=152300}{Inspiration for Essence Drain.}
	
	\href{http://forums.sjgames.com/showpost.php?p=507199&postcount=8}{Much of Kelley's work ended up similar to mine and I also took some inspiration and some differing opinions. This is the one that I will blatantly trust in their work the most in, so I'm definitely putting it here.}
	
	\href{http://gurb3d6.blogspot.com/2017/01/ultra-tech-quickie-more-blaster-and.html}{Lots of good ideas for estimating firearm qualities.}
	
	\href{http://gurb3d6.blogspot.com/2017/05/gauss-weapons-reloaded.html}{Some nice and sane fixes for gauss weapons for the Thunderstruck.}
	
	\href{http://gurb3d6.blogspot.com/2016/09/have-gauss-will-travel.html}{And his glorious Heavy Anti-Material Rifle is exactly what I wanted for a better Thunderstruck.}
	
\end{multicols*}