\begin{section_header}[blue]
	\textbf{\section{Magic}}
\end{section_header}

\begin{multicols}{3}
	
	Magic is broadly broken up into two main categories of Awakened individuals: Magicians and Adepts. Magicians channel their magic to create spells and similar outwardly effects. Adepts channel their magic to empower their bodies, which is why they are often called Physical Adepts. Mystic Adepts are a rare combination of the two.
	
	Magicians and Mystic Adepts can further break down their capabilities into Spellcasting, Summoning; with Magicians also able to do Alchemy and Enchanting alongside Astral Projection. Magicians, Mystic Adepts, and some Adepts also have Astral Perception.
	
	When creating an Awakened, you have a variety of Unusual Backgrounds you may purchase. Some of these build upon each other, while some are mutually exclusive, as mentioned in their descriptions. 
	
	\subsection{Awakened Types}
	
	Each of these advantages represents a type of Unusual Background that allows you to purchase a certain selection of Magical Abilities. The lowest of these is \textit{Spark}, but represents the minimum amount of magical capabilities and is a pre-requisite for most Awakened Type Advantages. The others are usually mutually exclusive, designating your type of awakened, such as Magician or Adept, and they each describe what type of skills and advantages you are allowed to take.
	
	\paragraph{Spark}\label{spark}
	\begin{flushright}
		1 Point
	\end{flushright}
	
	You have extremely limited magical abilities, allowing the purchase of Astral Perception (Which would classify the character as an 'Aware'). Otherwise, this counts as Magic 0.
	
	\paragraph{Magician}
	\textit{Prerequisite: Spark, Astral Perception, Astral Projection}
	\begin{flushright}
		5 Points
	\end{flushright}
	
	This advantage allows you to take the Sorcery, Conjuring, Enchanting, and Alchemy skills alongside their respective advantages (Such as spells for Sorcery). It is mutually exclusive with Adept and Mystic Adept.
	
	\paragraph{Adept}
	\textit{Prerequisite: Spark}
	\begin{flushright}
		2 Point
	\end{flushright}
	
	This advantage allows you to take the Adept Powers. It is mutually exclusive with Magician and Mystic Adept.
	
	\paragraph{Mystic Adept}
	\textit{Prerequisite: Spark}
	\begin{flushright}
		6 Points
	\end{flushright}    
	
	This advantage allows you to take the Sorcery and Conjuring skills alongside their respective advantages (Such as spells for Sorcery). They can purchase Astral Perception through the Adept Power. It is mutually exclusive with Adept and Magician.
	
	\paragraph{Aspected Magician}
	\textit{Prerequisite: Spark, Astral Perception}
	\begin{flushright}
		5 Points before limitation
	\end{flushright}
	
	There are a wide variety of Aspected Magicians, serving as highly restricted magicians. Choose from the list below and apply their noted limitations to this advantage and the Magic advantage as noted.
	
	\begin{itemize}
		\itemsep0em 
		\item Aspected Sorceror: Can only take Sorcery, -20\%; Apply to magic: Can only cast spells, -10\%.
		\item Aspected Conjuror: Can only take Conjuring, -40\%; Apply to magic: Can only summon, -20\%.
		\item Enchanter: Can only take Enchanting, -60\%; Apply to magic: Can only enchant, -30\%.
		\item Explorer: Can only take Astral projection, -60\%; Do not take Magic.
		\item Apprentice: Can only use one category of spells and one category spirits, -40\%; Apply to magic: Can only cast and summon one category, -30\%
	\end{itemize}
	
	\paragraph{Awakened Critter}\label{awakened_critter}
	\textit{Prerequisite: Spark}
	\begin{flushright}
		5 Points
	\end{flushright}
	
	This indicates a creature as an Awakened critter, which allows them to purchase Awakened Critter Powers. Usually these are not available to PCs without particularly good reason - which usually means playing a Metasapient or Infected character - and are always limited to certain ranges of powers.
	
	\paragraph{Spirit}
	\textit{Prerequisite: Spark}
	\begin{flushright}
		5 Points
	\end{flushright}
	
	Indicates a creature as a magical spirit, allowing the purchase of special abilities outside the norm. Functionally equivalent to Awakened Critter above.
	
	\subsubsection{Magical Traditions}
	
	An Awakened's magical tradition defines their approach to magic. It's an extremely personal one that can range from ethical and moral ideas, spiritual experiences, religious beliefs, personality types, and so on. They can largely be broken into two extremely broad categories however, being Hermetic and Shamanic traditions.
	
	Hermetic traditions overall tend towards intellectual experiences with Magic. This can manifest as simply as a scientific view of the Sixth World and its peculiarities, but can have much more complexity elsewhere. Religions such as Zoroastrianism, which focuses on strict and stark moral ideas and active participation in life, falls under this banner. As well, Chaos Mages fall under this banner, mixing and matching styles and parts of traditions as they see fit. Hermetic mages resist drain with (IQ + Will) / 2.
	
	Shamanic traditions tend towards empathetic or natural focused experiences with Magic. A common example of this are those focused on Magic's interaction with the environment and nature, classical shamans. Religions fall under this regularly as well, such as Christianity. Many darker ideals count as this as well, such as Black Magic - a hedonistic ideation that often treads the line between healthy indulgences and fulfillment of wants and darker spirals of abuse and danger. Shamanic traditions resist drain with (HT + Will) / 2.
	
	Certain traditions place more or less emphasis on the IQ/HT versus Willpower aspects of their drain resistance, with scientific versus philosophical hermetics appealing to IQ and Will respectively and natural versus religious shamanic appealing to HT and Will respectively as well.
	
	These are by no means hard and fast rules, as the GM and player should work out where their tradition falls on the spectrum; however, it's ill-advised to start including things like Per and the Charisma advantage into the mix when resisting Drain. IQ + Will and HT + Will are worth the same amount of points, making the two types equal, while a (Per + Will)/2 or ( (10 + Charisma) + Will) / 2 drain would cost 10 points less! 
	
	If a player is dead set on such a case, the GM can charge an additional Unusual Background to their Per or Charisma, increasing its cost to be relatively on par with IQ or HT (although keep in mind that IQ and HT are valuable in and of themselves, while Unusual Background is not). A good heuristic is an additional +5 to +7 points; e.g. making Charisma (Affects Drain, +7) [12] and Per (Affects Drain, +7) [12]. It's still overall cheaper than the IQ and HT variants, but the mage misses out on the larger benefits that IQ and HT bring.
	
	\subsection{Magic Advantages}
	
	These are the advantages that make up an Awakened's abilities and are usually restricted to certain Awakened Types (See above).
	
	\paragraph{Magic}\label{magic}
	\textit{Prerequisite: Spark}
	\begin{flushright}
		7.5 Points / Level
	\end{flushright}
	
	Magic serves as the core of an Awakened character's capabilities. Add your Magic to all rolls to use Magical skills well (Just like a talent or Magery); this means that all Magical skills, activation rolls for powers, and so on. 
	
	As well (and \textit{especially} for Adepts), the GM should look for indirect ways for Magic to apply to passive abilities (as is the case with many Adept Powers), as described in \GURPS Supers p22; this can take the form of adding to Power Blocks, to crippling rolls for Unbreakable Bones, to Stealth for invisibility, etc. 
	
	Magic also serves as a cap for various magical advantages: For Sorcery, Enchanting, Alchemy, and Conjuring, can only buy spells, formulae, and the conjuring advantage up to a level equal to your Magic. For Adepts, you can only buy levels up to your Magic as well.
	
	When purchasing magical advantages such as spells, Adept Powers, and so on, you must only pay full cost for your most expensive spell, formula, and adept power; for all other ones you pay only 1/5 cost. However, you can only sustain one spell or power at a given time; if you want to sustain multiple different effects, you must pay full price for each of your next most expensive spell or ability for the amount of additional effects you wish to sustain. Some abilities are possible to cast multiple times (such as those based on Afflictions), which can allow you to sustain \textit{the same spell} multiple times. Instances of these abilities are noted in their description.
	
	Additionally, anything that affects or disables your abilities affects them as a whole. As such, anti-magic effects, critically failed rolls, magical abilities that are recharging or being sustained, etc. can all prevent your from using \textit{any magical effects at all} while they are happening.
	
	When casting spells or activating powers, you must usually make a single Concentrate Maneuver. However, when switching between classes of spells (Combat to Detection, etc), you must make a separate Concentrate Maneuver before casting!
	
	An Awakened's relationship with their magic is always very complex and personal, often delving into religion, philosophy, ethos, and psyche. As such, it's heavily recommended that Awakened add enhancements and limitations as found in Basic Set pg 66, Thaumatology, Magic, and more, in order to better reflect their character's approach to Magic.
	
	The GM should use their best judgement when determining whether a given enhancement or limitation works properly with the system and setting, although most - in not all - should work fine. If you do this, do not miss the built in limitation, as described in the Statistics below.
	
	Optionally, the GM may allow you to buy the ability to sustain multiple spells with a sustaining penalty. This is done by purchasing a penalty to an attribute with no secondary attributes (e.g. With no Basic Speed if choosing DX) with the limitation: \textit{Accessibility, Only while Sustaining, -50\%} alongside paying an \textit{equal} amount of points towards paying off your next most expensive spell. Any positive or negative point discrepancies count as either a meta-advantage or disadvantage respectively. This allows the awakened to sustain an additional spell, at the cost of incurring said attribute penalty whenever they are doing so. Such strain affects individuals and traditions in differing ways, incurring things such as IQ penalties for headaches, DX penalties for cramps, and so on; the awakened should work with their GM to decide what attributes are available. 
	
	In general, the GM should enforce penalties being spread over multiple attributes instead of staking a high penalty on one attribute alone. Additionally, the GM is advised to not allow players to wantonly apply the limitation from before to anything they want to simulate such sustaining effects, it is somewhat unfairly priced in the player's advantage to allow for this specific situation\footnote{Most limitations don't apply symmetrically to disadvantages. Accessibility is somewhat of an exception to this, but look into Counter Advantages/Disadvantages in Power-Ups 8 for a great explanation.}- they are also entirely within their rights to increase the limitation value in order to more fairly represent this.
	
	\textit{\textcolor{OliveGreen}{Statistics: Magery (Astral, -15\%) [7.5]}}
	
	\paragraph{Astral Perception}\label{astral_perception}
	\textit{Prerequisite: Spark}
	\begin{flushright}
		12 Points
	\end{flushright}
	
	With a Ready maneuver, you can open your mind to the Astral Plane, allowing you to "see" auras. All living beings and magical effects have auras, while nonliving - especially technological - objects and creatures appear muted, opaque, and dark gray. This makes navigating while Astrally Perceiving a difficult task, providing a -4 penalty to all Vision rolls while perceiving (Except as it pertains to auras). 
	
	While Astrally Perceiving an aura, you can analyze it using \hyperref[assensing_skill]{the Assensing Skill,} which can see an individual's aura, showing their emotional state, magical effects, and more.
	
	While Astrally Perceiving you are dual-natured, meaning you exist on both the Physical and Astral plane at the same time. This means that creatures on either plane can affect you. You can affect the Astral Plane with any Mana spells as normal, however Physical spells do not affect anything purely on the Astral Plane.
	
	\textit{\textcolor{OliveGreen}{Statistics: See Invisible, Astral (Partially Exclusive, -20\%; Nuisance Effect, Dual-Natured, -10\%; Magical, -10\% [9] and Empathy, Sensitive (Astral Only, -30\%; Magical, -10\%) [3]}}
	
	\paragraph{Astral Projection}\label{astral_projection}
	\textit{Prerequisite: Spark, Magic 1+, Astral Perception}
	\begin{flushright}
		20 Points
	\end{flushright}
	
	You can project your consciousness onto the Astral Plane, allowing you to move about the plane at the speed of thought. To do so, you must concentrate for 1 minute, spend 1 FP, and make an IQ roll. On a failure, you go nowhere. On a critical failure, you arrive at the wrong destination. If there is no "safe" corresponding location on the Astral Plane, your projection will fail and you will know why. You always arrive naked and without any equipment, although your astral form may "clothe" itself, this provides no protection.
	
	You may also "hitch-a-ride" or follow other projectors, allowing you to either be taken to specific parts of the the astral plane, or - as is often the case in initiations - be taken to the metaplanes! If you have not bought off \textit{Limited Acces (Astral Plane)} during such a case, the GM can allow you to spend unspent CP to buy it off immediately.
	
	The Astral Plane works similarly the Astral Perception. all living beings and magical effects have auras that can be assensed as with Astral Perception (See above). Unliving creatures or objects area a muted, opaque, dark gray, providing a -4 penalty to all Vision rolls while projecting (Except as it pertains to auras). You can affect the Astral Plane with any Mana spells as normal, however Physical spells (that is: spells that create physical effects as opposed to affecting a target's aura directly via Malediction) do not work. As well, you cannot target or attack creatures purely on the Physical Plane. However, you may also not be affected by anything purely on the Physical Plane.
	
	You can traverse the Astral Plane at the speed of thought, phase through matter, and fly. You cannot pass through auras however, meaning that living beings (or clumps of large living beings), and certain spells such as mana barriers can impede your movement. The Earth also has an aura, preventing you from traversing very far into the ground. Lastly, you cannot fly out of the atmosphere (or, more specifically, the Earth's Gaiasphere).
	
	While you are on the Astral Plane, your body lays unconscious where you left it. You must navigate back to it yourself in order to stop Astrally Projecting (Meaning that hucksters moving your body can create quite a dangerous situation for you).
	
	You can only stay on the Astral Plane for a number of hours equal to your Magic. After that, your Essence slowly disconnects from your body. For every hour you remain past that, lower your Magic by one. If your Magic drops to 0, you die. Lost Magic returns at a rate of 1 per hour, after you have re-entered your body.
	
	\textit{\textcolor{NavyBlue}{Modifiers: +6 for reliable, +1 per Level of Magic, -1 per 6 seconds less concentration.}}
	
	Particularly skilled Magicians can buy off \textit{Limited Access (Astral Plane), -20\%}, replacing it with \textit{Limited Access(Astral Plane, 1 Metaplane), -15\%} for 5 points. This allows the Magician to travel to the Metaplanes for initiation purposes - assuming they can make it past The Watcher of course. Travel to the Metaplanes directly may cost up to 10 FP as opposed to the standard 1 FP. 
	
	Further experience may increase the amount of Metaplanes they can visit, eventually removing the limitation entirely for a grand total of 20 points. Sometimes it is replaced with the limitations \textit{Cannot Escort, -10\% [-10] and/or Cannot Follow, -20\% [-20].}
	
	\textit{\textcolor{OliveGreen}{Statistics: Jumper, Astral (Improved, +10\%; Reliable 6, +30\%; Immediate Preparation Required (1 minute), -30\%; Maximum Duration, (Magic) Hours, -5\%; Naked, -30\%; Nuisance Effect, Die after (Magic) additional hours, -5\%; Projection, Physical -25\%;  Limited Access (Astral Plane), -20\%; Magical, -10\%) [20]}}
	
	\paragraph{Counterspell}\label{counterspell}
	\textit{Prerequisite: Spark, Magic 1+}
	\begin{flushright}
		
	\end{flushright}

	% TODO: This
	
	

	
	\paragraph{Spell Defense}\label{spell_defense}
	\textit{Prerequisite: Spark, Magic 1+}
	\begin{flushright}
		5+(Magic) Points / Level
	\end{flushright}
	
	By disrupting the effects of other magician's spells, an Awakened can provide magical defense to themselves and their teammates.
	
	To do so, they make an Active Defense of (Counterspelling + Magic) / 2 + 3. For each successive attempt in a turn, this is made at a -4. If successful, the Awakened and a number of individuals up to his Magic all gain Magic Resistance at the level of his Spell Defense against the effect. This does not interfere with allied spellcasting.
	
	\textit{\textcolor{OliveGreen}{Statistics: Magic Resistance 1 (Affects Others (Magic), +50×(Magic-1)\%; Improved, +150\%; Increased Range, LOS, +40\%; Ranged, +40\%; Active Defense, -40\%) further levels increase Magic Resistance, while higher Magic increases the level of Affect Others by one.}}
	
	\paragraph{Summoning}\label{summoning}
	\textit{Prerequisite: Spark, Magic 1+}
	\begin{flushright}
		Campaign Dependant points
	\end{flushright}
	
	An Awakened can summon spirits from the astral Plane with promise of small gifts in return for their services. As for what these are, metahumanity can only guess, but the value that spirits provide cannot be underestimated.
	
	By Concentrating for 1 second for every force, the user can attempt to summon a spirit from their tradition. This requires a Quick Contest between the Awakened's Summoning and the spirit's Force+5\footnote{Determined by average starting Will for a Spirit being 6.}, with the Awakened gaining favors equal to their Margin of Success. 
	
	The spirit's time in the Physical Plane is limited however. It can only stay until the next sunrise or sunset, upon which it automatically departs regardless of remaining favors.
	
	Attempting a summoning requires the Awakened to resist FP loss equal to the spirits Force - or HP loss if the Force is great than their Magic. They can summon up to double their Magic in Force.
	
	The summoner can use their favors for any discrete task, such as: scout this location, help in combat, use this specific power. However, the summoner's treatment of the spirit does not go unnoticed. Extremely long tasks, forced services, tasks that go against the spirit's nature, cruel treatment, and anything else the GM decides can be taken note of by the spirit world. A character with a bad reputation will find it harder to summon and bind spirits, whether for less respondants, harsher FP costs, steeper requests, or harsh resistance. If bad enough, spirits may even attack the user.
	
	A character may only ever have one slot for Summoning, and ergo one Spirit summoned at a time.
	
	Because campaigns do not all start at the same amount, it's not possible to provide a definitive table for summoning costs based on Magic. The table here provides costs for 200 point and 100 points campaigns, using \hyperref[spirit_ally_cost]{the table for spirit ally costs.} If you want a sufficiently different campaign level, you will have to recalculate the cost. For instructions on how to do so see the \hyperref[spirit_math]{Spirit Math Section.} 
	
	Note that it is more expensive for lower point characters because higher Force Spirits will be a higher percentage of your points and ergo more impactful.
	
	\begin{center}
		\begin{tabular}{|c|c|c|}
			\hline
			Magic & 200 Points & 100 Points \\
			\hline
			\hline
			Magic 1 & 3 & 3 \\
			Magic 2 & 4 & 6 \\
			Magic 3 & 5 & 13 \\
			Magic 4 & 8 & 21 \\
			Magic 5 & 12 & 29 \\
			Magic 6 & 16 & 37 \\
			Magic 7 & 20 & 45 \\
			Magic 8 & 24 & 53 \\
			\hline
		\end{tabular}
	\end{center}
	
	\textit{\textcolor{OliveGreen}{Statistics: Modular Ability, 4 per slot, 4 per point\footnote{External Influence only.} (Reduced Time 1\footnote{Because campaign points would affect a 1 sec / point change, we'll apply this and take a middle ground of 1 second per Force. If you wish, you can calculate a time to summon per Ally point cost.}, +20\%; Social Only, +0\%; Magical, -10\%; Nuisance Effect, Ends on Sunrise or Sunset, -5\%; Requires Summoning Roll, -35\%\footnote{This modifier is made up of: Requires Attribute (10) Roll, -20\%; Requires Skill Roll (Summoning), -0\%; Quick Contest, -15\%}; Trait-Limited, Only Allies with Summonable, Special Abilities, and Favor, -50\%), each Force has it appropriate levels of points alongside FP and HP drain.}}
	
	\paragraph{Binding}\label{binding_spirits}
	\textit{Prerequisite: Spark, Magic 1+}
	\begin{flushright}
		Campaign Dependant points
	\end{flushright}
		
	A summoner can attempt to bind an already summoned spirit in or to both keep it around semi-permanently and to keep a larger number of spirits on call.
	
	Binding a spirit takes 1 hour and 1500 nuyen per Force. Afterwards, the Awakened must succeed on a Quick Contest between his Binding and the spirit's Force+8, gaining additional favors equal to the Margin of Success.
	
	Attempting a binding requires resisting FP loss equal to the spirits Force - or HP loss if the Force is higher than the Awakened's Magic. 
	
	After binding, the spirit stays until all of the favors are used up, although keeping it around too long will also incur negative responses from the spirit world. Otherwise, the services can be spent exactly as in Summoning, but with the added benefit of time. At the GMs discretion they may also be able to provide services such as sustaining spells for the Awakened, physically and magically aid their spellcasting, and so on.
	
	A character may have as many slots as the GM wishes to permit. Any spirit that becomes bound takes up a bound slot and no longer takes up the summon slot.
	
	As mentioned in Summoning, these costs are dependant on the Campaign Starting Points level. Costs for 200 points and 100 points are provided here.
	
	\begin{center}
		\begin{tabular}{|c|c|c|}
			\hline
			Magic & 200 Points & 100 Points \\
			\hline
			\hline
			Magic 1 & 2 & 2 \\
			Magic 2 & 2 & 4 \\
			Magic 3 & 3 & 7 \\
			Magic 4 & 4 & 11 \\
			Magic 5 & 5 & 15 \\
			Magic 6 & 6 & 19 \\
			Magic 7 & 8 & 23 \\
			Magic 8 & 10 & 27 \\
			\hline
		\end{tabular}
	\end{center}
	
	\textit{\textcolor{OliveGreen}{Statistics: Modular Ability, 4 per slot, 2 per point\footnote{External Influence, Expensive, and Long Time.} (Reduced Time 1, +20\%; Social Only, +0\%; Hard to Use, -5\%\footnote{Applied as bonus to resist.}; Magical, -10\%; Requires Binding Roll, -35\%\footnote{This modifier is made up of: Requires Attribute (10) Roll, -20\%; Requires Skill Roll (Binding), -0\%; Quick Contest, -15\%};; Trait-Limited, Only Allies with and Summonable, Special Abilities, and Favor -50\%), each Force has it appropriate levels of points alongside FP and HP drain.}}
	
	\subsection{Mentor Spirits}
	
	A curious type of spirit that is commonly seen in the Sixth World, mentor spirits provide guidance and assistance for Awakened that traverse the paths according to their specific styles. This is sometimes believed to be a sort of self-fulfilling idea, where the spirit aims to support ideas, motifs, and feelings that create and represent its identity, however this is far from universally accepted as just as many Awakened believe such spirits simply axiomatically value these ideas and do their best to support Awakened that aid such things.
	
	Mentor Spirits themselves are able to communicate with Awakened that follow their path and even physically interact with them — however that \textit{are not} the type of spirit that you summon and banish, they are often magnitudes more powerful and far more interested in their cryptic goals and ideals. They do still provide plenty of benefits however; the standard Mentor Spirit is bought as a Patron that will provide you advice, information, and sometimes magical boons as well. All of these are dependent on how closely your character follows the goals and ideals for the Mentor Spirit — stricter adherence should be rewarded with greater boons, while going against the ideals should cause loss of powers or worse. Notably, you \textit{do not} have to be Awakened for a Mentor Spirit to take notice of you, although it is extremely rare for such an occurrence.
	
	All Mentor Spirits include the following advantage. Additionally, depending on the Mentor Spirit, the user will gain a number of benefits (Advantages, skills, etc.) alongside drawbacks (limitations on existing Advantages, Disadvantages, etc). When choosing this advantage, you must \textit{also} choose the specific Mentor Spirit down below. These types notate a limitation value for their \textit{Pact}, which must be applied to the \textit{Mentor Spirit} advantage you are buying (Alongside elsewhere notated in that specific Mentor Spirits section); costs for this trait \textit{are already} included in the specific Mentor Spirit sections, do not buy the \textit{Mentor Spirit} advantage twice!
	
	The GM is recommended to take the FOA as a guideline as is expressed in the Basic Set. If it makes sense for a Mentor Spirit to show up, they should. If it would be disruptive or not make sense, they should not. One can always consider the FOA similar to guidelines on how often a trait should come up and to simply have it appear as appropriate to the story within those bounds (e.g. FOA 6 has ~9\% chance of success, meaning it should occur ever 10-11 sessions, whenever you deem appropriate).
	
	\paragraph{Mentor Spirit}\label{mentor_spirit}
	\begin{flushright}
		15 Points base
	\end{flushright}
	
	You have taken a certain magical spirit up as your Mentor Spirit. As long as you abide by its codes and ideals alongside maintaining a friendly relationship with it, the Mentor Spirit will provide you with a wide variety of assistance. 
	
	Your Mentor Spirit will generally be forthecoming with any assistance that is critical to either its goals or your safety, although beyond that its intervention may depend on its whim, your actions, and your requests; in such cases, roll against its Frequency of Appearance to access it (Which can be done mentally at any time or place); success meaning that the spirit either deigns to interdict or to answer your calls — however this does not necessarily agree that it will assist you! 
	
	After contacting it, make a Reaction Roll for your Mentor Spirit, with positive modifiers for your character's behaviour (adhering to its code, fulfilling requests for it, and so on) alongside its stance on the subject (e.g. penalties for helping outside its domain or larger penalties for doing something opposed to it). On a Neutral reaction or better, your Mentor Spirit will render assistance in the way \textit{it} thinks best for the situation, with higher rolls allowing for more control and input on the Awakened's part. This might not always mean them being explicitly helpful, even if they do perform what you wanted!
	
	In regards to its general benefits and capabilities, your Mentor Spirit foremost provides information and guidance — generally focused on the magical side of things, especially knowledge under its domain. It is able to physically interact with the world and you, although this is often much rarer, especially in physically risky situations. It can also provide boons for awakened that closely follow its ideals while revoking powers or providing banes for those that do not. The spirit may also be able to call upon those in its service to provide assistance (which may sometimes include the character, although the GM should not abuse this without the Duty disadvantage), allowing it to provide assistance indirectly. Lastly, simply being a follower of the spirit can carry weight itself; depending on the spirit, its followers may be well known enough to warrant respect, fear, or otherwise.
	
	\textit{\textcolor{OliveGreen}{Statistics: Patron (2$\times$+ points, 15; FAO 9, $\times$1; Highly Accesssible, +50\%, Minimal Intervention, -50\%; Pact, -X\%) [15 base]}}
	
	Optionally, the GM can allow Awakened to take a Meta-Trait that combine the \textit{Mentor Spirit} advantage with a Duty to their Mentor Spirit, allowing for a discount to the Advantage's point cost. These are most commonly: Duty (FOA 9) [-5], although depending on the Mentor Spirit this could have lower FOA, Extremely Hazardous, Nonhazardous, or sometimes even Involuntary! This indicates that the Mentor Spirit regularly calls upon the character to render service and aid to its goals or other followers.
	
	Each of the Mentor Spirits's can be chosen by all types of Awakened, although some are better for certain types; they are usually split into some advantage or a talent that can include a certain magical skill in it. Generally, you are not allowed to take further levels in these talents, they are purpose made to be somewhat more appealing than standard ones. 
	
	The GM is free to alter these as they wish, especially if the Awakened already has some of the traits provided! These new traits can be anything that fits the Mentor Spirit's style, such as higher levels in the provided traits, fitting talents, or so on. In all cases, simply apply the Pact limitation to the newly determined traits. GMs are also free to change the traits that make up the Pacts, just remember to change the costs of all the abilities they are applied to!
	
	As well, as the Awakened's relationship with their mentor flourishes, they may end up improving the traits granted to them by their mentor. This can also be compensated by increasing the drawbacks of their pact! This is one of the few ways that it \textit{\textbf{may}} be acceptable to increase the Talent levels provided by these traits, although the GM should scrutinize this highly.
	
	\subsubsection{Arcana}
	\begin{flushright}
		24 or 30 Points
	\end{flushright}
	Pact: -5\%
	
	The Arcana is an interesting and non-traditional case where the followers/practicioners describe its influence as more of an intimacy with the Tarot. In contrast to literal beings providing information and assistance, the Arcana provides fortunes and cryptic messages for those that place their destiny in the hands of the cards. Mechanically, this can end up giving advice and information similarly to how a normal Mentor Spirit would, and can even move other practitioners to assist or give physical aid through the use of their fortunes as well.
	
	Followers are extremely diverse, more interested in their relationship with destiny, tarot, and magic itself than any more monolithic identity or ideal. As such, the only thing that seems to run constant between them is their constant practice and faith in such things.
	
	\textbf{Mentor Bonuses:} 
	
	Serendipity (Aspected, Related to a prior Tarot Reading result, -20\%; Magical, -10\%; Pact, ) [10]
	
	\textit{Choose One:}
	\begin{itemize}
		\itemsep 0pt
		\item Talent, Arcana (Spellcasting (Detection) or Fast-Talk, Body Language, Fortune-Telling (\textit{Choose one}), Occultism, Sleight of Hand, Thaumatology) (Magical, -10\%; Pact, -5\%) [4.25]
		\begin{itemize}
			\itemsep 0pt
			\item \textit{Reaction Bonus:} +1 to Reaction Rolls from fortune-tellers and those whom you give fortunes to.
			\item \textit{Alt Benefit:} The Awakened's fortunes can sometimes predict the future similarly to the Precognition advantage. Usually, this occurs by chance during a normal Tarot divination, requiring a successful IQ-3 roll and providing information as per Precognition; rarely this can be forced, requiring 10 minutes a successful IQ-11 roll alongside costing 2 FP — you may want to take Extra Time for this!
		\end{itemize}
		\item Danger Sense (Aspected, Related to a prior Tarot Reading result, -20\%; Magical, -10\%, Pact, -5\%) [10]
	\end{itemize}
	
	\textit{Pact Traits:} Disciplines of Faith, Ritualism [-5]
	
	\subsubsection{Adversary}
	\begin{flushright}
		7 or 14 Points
	\end{flushright}
	Pact: -15\%
	
	The Adversary is the cosmological rebel, the one that spurns order, that resists authority, that yearns for ultimate freedom. Throughout many traditions it has taken many forms, from trickster figures of many North American and Asian Shamanic traditions to that of Satan among Christian followers. There is nothing inherently evil about them, however their very nature compels conflict with power structures in ways that can just as easily be horrible as they are heroic.
	
	Followers of the Adversary tend to be highly independent and idealistic. They generally react extremely poorly to authority figures, usually being people with Status 2+, Rank 2+, or similar. Beyond that they will usually lash out at those trying to prevent them from doing as they wish, acting as if they had either Selfish or Stubbornness. In extreme cases they are instead straight Fanatics for Independence or Freedom, and such candles tend to burn the brightest and the shortest.
	
	The Adversary tends to have a bad reputation as a corrupted or near-corrupted Mentor Spirit, meaning that it can often be taken with other negative social traits to reflect this for the character.
	
	\textbf{Mentor Bonuses:} 
	
	Talent, Adversary (Counterspelling or Fast-Talk, Explosives, Philosophy (\textit{Choose one}), Streetwise, Leadership, Sociology) (Magical, -10\%; Pact, -15\%) [4] %TODO: This
	\begin{itemize}
		\itemsep 0pt
		\item \textit{Reaction Bonus:} +1 to Reaction Rolls from anarchist, rebels, and insurrectionists.
		\item \textit{Alt Benefit:} +1 to resist Influence skills when used to oppress, control, or otherwise prevent you from following your goals. The GM should be \textit{strict} about what counts for this, it is not simply someone opposing you, this has an ideological nature.
	\end{itemize}
		
	\textit{Choose One:}
	\begin{itemize}
		\itemsep 0pt
		\item Fearlessness 3 (Magical, -10\%, Pact, -10\%) [5]
		\item Daredevil (Magical, -10\%, Pact, -10\%) [12]
	\end{itemize}
	
	\textit{Pact Traits:} Intolerance (Authority Figures) [-5], Selfish, SC 12 [-5], and Stubbornness [-5], \textit{or}
	
	Fanaticism (Independence or Freedom) [-15]
	
	\subsubsection{Bear}
	\begin{flushright}
		18 Points
	\end{flushright}
	Pact: -10\%
	
	The Bear symbolizes strength and protection. They are a healer and a fighter, protecting those who are under its care ferociously and never turning down those who seek genuine aid. 
	
	Its followers are much the same, often going berserk if themselves or their companions are harmed — or sometimes as healers or rescuers that cannot refuse to assist others.
	
	\textbf{Mentor Bonuses:} 
	
	Damage Resistance 1 (Tough Skin, -40\%; Magical, -10\%; Pact, -10\%) [3]
	
	\textit{Choose One:}
	\begin{itemize}
		\itemsep 0pt
		\item Talent, Bear (Spellcasting (Health) or First-Aid, 2, 3, 4, 5, 6?) (Magical, -10\%; Pact, -10\%) [4] %TODO: This
		\begin{itemize}
			\itemsep 0pt
			\item \textit{Reaction Bonus:} +1 to Reaction Rolls from individuals under your care.
			\item \textit{Alt Benefit:} +1 to HT rolls made by a patient to recover from one complaint if you treat them full-time and exclusively.
		\end{itemize}
		\item Rapid Healing (Magical, -10\%, Pact, -10\%) [4]
	\end{itemize}
	
	\textit{Pact Traits:} Berserk, SC 12 [-10] \textit{or} 
	
	Charitable, SC 15 [-7], Compulsive Generosity, SC 15 [-2], and Extremely Limited Disadvantage, Bad Temper (Only toward those who mess with those under your care who cannot defend themselves) [-1]
	
	\subsubsection{Berserker}
	\begin{flushright}
		13 or 17 Points
	\end{flushright}
	Pact: -10\%
	
	The Berserker loves fighting just for the sake of fighting. It is very common for the Pacts for this to be -15\% as well, representing stricter Codes of Honor or a lower SC.
	
	Its followers often fall into one of two extremes: Those that are eager for altercation at the drop of a hat and those that have extremely idealistic views about fighting.
	
	\textbf{Mentor Bonuses:} 
	
	Fearlessness 3 (Magical, -10\%, Pact, -10\%) [5]
	
	\textit{Choose One:}
	\begin{itemize}
		\itemsep 0pt
		\item Talent, Berserker (Spellcasting (Combat) or \textit{One combat skill of choice}, Savoir-Faire (Military), Soldier, Tactics) (Magical, -10\%; Pact, -10\%) [4]
		\begin{itemize}
			\itemsep 0pt
			\item \textit{Reaction Bonus:} +1 to Reaction Rolls from individuals under your care.
			\item \textit{Alt Benefit:} +1 to HT rolls made by a patient to recover from one complaint if you treat them full-time and exclusively.
		\end{itemize}
		\item High-Pain Threshold (Magical, -10\%, Pact, -10\%) [8]
	\end{itemize}
	
	\textit{Pact Traits:} Bad Temper, SC 12 [-10] \textit{or} 
	
	Code of Honor [-10]
		
	\subsubsection{Cat}
	\begin{flushright}
		8 Points
	\end{flushright}
	Pact: -15\%
	
	The Cat is the mystic keeper of all secrets, proud and foolish the trickster is always in search of more knowledge for which is will not share. 
	
	Its followers tend to be of like mind, specializing in trades that involve subterfuge, trickery, and both the keeping and discovering of secrets. In similar fashion, they tend to have a sense of pride that leads to them performs tricks and pranks on dangerous foes — sometimes to their own downfall.
	
	\textbf{Mentor Bonuses:} 
	
	Talent, Cat (Spellcasting (Illusion) or Acrobatics, Climbing, Cryptography, Hidden Lore(\textit{Choose one}), Stealth) (Magical, -10\%; Pact, -15\%) [3.75]
	\begin{itemize}
		\itemsep 0pt
		\item \textit{Reaction Bonus:} +1 to Reaction Rolls from those who deal in subterfuge and secrecy, such as thieves, detectives, and spies.
		\item \textit{Alt Benefit:} +1 to Contests involving Filch, Forgery, Holdout, Pickpocket, and Sleight of Hand.
	\end{itemize}

	\textit{Choose One:}
	\begin{itemize}
		\itemsep 0pt
		\item Catfall (Magical, -10\%, Pact, -15\%) [7.5]
		\item Super Jump (Magical, -10\%, Pact, -15\%) [7.5]
	\end{itemize}
	
	
	\textit{Pact Traits:} Trickster, SC 12 [-15]
	
	\subsubsection{Chaos}
	\begin{flushright}
		15 or 12 Points
	\end{flushright}
	Pact: -15\%
	
	The embodiment of unpredictability, Chaos encourages all things random — although what this is varies wildly by the follower. This could be constant arguments, trickery, amorality, and so on.
	
	Its followers are only alike in how unalike they are, they have a wide variety of strange traits that they must adhere to for their powers — and even these traits sometimes change! The GM and player should feel free to switch up the disadvantages making up the Pact with new ones as they see appropriate (Just don't let players get around their restrictions by doing this). 
	
	Adherents do also typically align to one of two broader ideas: Randomness, which includes adherents who strive for surprise, unpredictability, and entropy, and then an eschewing of order, which tends to attract anarchist, amoral individuals, and so on.
	
	\textbf{Mentor Bonuses:} 
	
	Talent, Chaos (Spellcasting (Illusion) or \textit{One Skill of Choice, with GM approval}, Acting, Detect-Lies, Digsuise, Fast-Talk) (Magical, -10\%; Pact, -15\%) [3.75]
	\begin{itemize}
		\itemsep 0pt
		\item \textit{Reaction Bonus:} +1 to Reaction Rolls from those whom appeal to your current ideals (e.g. if you have Callous, mercenaries, murderers, and so on).
		\item \textit{Alt Benefit:} +1 to Contests against skills this talent covers.
	\end{itemize}
	
	\textit{Choose One:}
	\begin{itemize}
		\itemsep 0pt
		\item Wild Talent (Magical, -10\%, Pact, -15\%) [15]
		\item Indomitable (Magical, -10\%, Pact, -15\%) [11.25]
	\end{itemize}
	
	
	\textit{Pact Traits:} Choose -15 points from practically any mental disadvantages. Common ones include: Trickery [-15*], Callous [-5], Bad Temper [-10*], Bersesrk [-10*], Stubbornness [-5], Code of Honor [Varies], Disciplines of Faith [Varies], and so on.
	
	\subsubsection{Dark King}
	\begin{flushright}
		6 Points
	\end{flushright}
	Pact: -10\%
	
	Hades, Osiris, the keeper of the dead, King of the Underworld. The Dark King could go by many names, but as a keeper of secrets embodying the phrase to 'take it to the grave', one can never be certain. He embodies death, pain, dying in more ways than just their physical effects; at the same time his domain covers secrets, mysteries, and contracts most of all.
	
	His followers are all those interested in secrets and contracts. As such, he tends to attract many business suits, blackmailers, and inside traders, all of which are often looking to make contracts for hidden knowledge.
	
	Although his domain covers many nasty things, it's not believed that the Dark King is a corrupted or twisted spirit. His focus is generally believed to be more theological and philosophical than it is one of twisted human emotion and suffering. Some associate him with the Plane of Black.
	
	\textbf{Mentor Bonuses:} 
	
	Low-Paint Threshold\footnote{This is not a pact limitation, however the awakened's connection with the Dark King hurries them to his side, enhancing their sense of pain. Because it is dependent on the connection, it does not function without mana or if the Dark King has revoked other powers.} (Magical, -10\%; Pact, -10\%) [-8] 
	
	Talent, Dark King (Ritual Spellcasting (Contracts\footnote{This affects any spells that are used in service of creating a contract. This could be spells to Detect Lies or Emotions when drawing up terms, for Analyzing items before making an offer, and so on. The GM should be very generous with this so that it's comparable to other specializations.}) \textit{or} Spellcasting (Contracts) or Detect-Lies, Diplomacy, Expert Skill (Thanatology) Law(\textit{Choose one}), Intimidation, Merchant) (Magical, -10\%; Pact, -10\%) [4]
	\begin{itemize}
		\itemsep 0pt
		\item \textit{Reaction Bonus:} +1 to Reaction Rolls from those you are making deals with.
		\item \textit{Alt Benefit:} +1 to Contests against skills this talent covers.
	\end{itemize}
	
	\textit{Choose One:}
	\begin{itemize}
		\itemsep 0pt
		\item Charisma 2 (Magical, -10\%; Pact, -10\%) [8]
		\item Voice (Magical, -10\%; Pact, -10\%) [8]
	\end{itemize}
	
	\textit{Pact Traits:} Curious, SC 9 [-10]
	
	\subsubsection{Death}
	\begin{flushright}
		8 or 4 Points
	\end{flushright}
	Pact: -15\%
	
	Death has shown up in almost every culture in recorded history, with interactions ranging entirely from worship to fear. Death, the Mentor Spirit, pushes for the patient end to all things. This does not necessarily mean it is evil, more so that it is rational in its nihilism — that the end comes for everything at some point, and such a momentous and universal occasion should be celebrated and shared. Sometimes this means the spirit embraces killing of all kinds (it is not generally concerned with mortality in its quest) as does it sometimes embrace release for those in suffering.
	
	Followers tend to have a relatively alien mindset to many modern individuals. They worship the idea of death and endings while pushing for the recognition of life and existence as a temporary spark. Many of its followers turn to spreading death themselves, whether that be as serial killers, mercenaries, or also as morticians and physicians who simply help those along to their final resting place. They also tend to reject providing aid that would prevent death, although to what degree depends on the individual — most could be convinced if there is great importance to either render or receive enough aid as is necessary.
	
	\textbf{Mentor Bonuses:} 
	
	Talent, X (Spellcasting (Death Spells)\footnote{Any offensive spell that is designed to kill. So, this includes most Combat Spells, but not say Punch. It also includes non-combat skills like Turn to Goo.} or Thaumatology, Expert Skill (Thanatology, Undead, Necromancy), Philosophy (Nihilism, Thanatology), Physiology, Occultism, Theology) (Magical, -10\%; Pact, -15\%) [3.75]
	\begin{itemize}
		\itemsep 0pt
		\item \textit{Reaction Bonus:} +1 to Reaction Rolls from those who revere or acknowledge death: mercenaries, nihilists, killers, some medical practitioners, etc.
		\item \textit{Alt Benefit:} +1 to Fright Checks caused by undead and by ordinary dead bodies, however foul and decayed.
	\end{itemize}
	
	\textit{Choose One:}
	\begin{itemize}
		\itemsep 0pt
		\item High-Pain Threshold (Magical, -10\%; Pact, -15\%) [7.5]
		\item Unfazeable (Familiar Horrors, -50\%; Magical, -10\%; Pact, -15\%) [3.75]
	\end{itemize}
	
	\textit{Pact Traits:} Code of Honor (Death)\footnote{Revere death for its finality. Accept the temporary nature of life and existence. Give death to freely to those that want or deserve it. Do not halt death to those approaching it without great reason. Abhor that which lives forever, but embrace the unliving that have been touched by Death.} [-15]
	
	\subsubsection{Disease}
	\begin{flushright}
		6 Points
	\end{flushright}
	Pact: -15\%
	
	Disease is a \textbf{Toxic} Mentor Spirit whose domain covers all things pestilence, parasitics, and pervasiveness. The spirit aims to spread pestilence of any form, not just physical disease, throughout metahumanity, making using of tons of discrete operators and followers in order to do so.
	
	Its followers are like cogs in a machine, generally cowardly they tend to only work openly in large numbers and with the upper hand — otherwise scheming a manipulating to cause cascades of events that spread their toxic effects and diseases.
	
	As a Toxic Mentor Spirit, the player usually has a number of mental issues beyond their pact with the spirit itself. Additionally, they are commonly accompanied with negative reputations and other social effects that can offset the cost of the Mentor Spirit.
	
	\textbf{Mentor Bonuses:} 
	
	Resistant, Disease, +3 (Magical, -10\%; Pact, -15\%) [2.25]
	Resistant, Poison, +3 (Magical, -10\%; Pact, -15\%) [3.75]
	
	\textit{Choose One:}
	\begin{itemize}
		\itemsep 0pt
		\item Talent, Disease (Summoning (Plague Spirits) or Stealth, Biology, Demolitions (Dirty Bombs), Expert Skill (Epidemiology), Streetwise, Urban Survival) (Magical, -10\%; Pact, -15\%) [3.75]
		\begin{itemize}
			\itemsep 0pt
			\item \textit{Reaction Bonus:} +1 to Reaction Rolls from those who study or revere disease or similar things, such as Epidemiologists, Memeticists, Bio-Terrorists, and so on.
			\item \textit{Alt Benefit:} +1 to HT rolls to resist your own diseases and weapons.
		\end{itemize}
		\item Resistant, Disease +8 [1.5] and Resistant, Poison +8 [1.875]
	\end{itemize}
	
	\textit{Pact Traits:} Cowardice, SC 12 [-10], Chummy [-5]
	
	\subsubsection{Dog}
	\begin{flushright}
		18 or 26 Points
	\end{flushright}
	Pact: -10\%
	
	
	
	\textbf{Mentor Bonuses:} 
	
	Discriminatory Smell (Magical, -10\%; Pact, -5\%) [12]
	
	\textit{Choose One:}
	\begin{itemize}
		\itemsep 0pt
		\item Talent, Dog (Spellcasting (Detection) or 1, Tracking, 3, 4, 5, 6?) (Magical, -10\%; Pact, -10\%) [4]
		\begin{itemize}
			\itemsep 0pt
			\item \textit{Reaction Bonus:} +1 to Reaction Rolls from those who you can call friends or companions.
			\item \textit{Alt Benefit:} +1 to .
		\end{itemize}
		\item Discriminatory Hearing (Magical, -10\%; Pact, -10\%) [12]
	\end{itemize}
	
	\textit{Pact Traits:} Sense of Duty (Companions) [-5]; Stubbornness [-5]
	
	\subsubsection{Doom}
	\begin{flushright}
		8 or 4 Points
	\end{flushright}
	Pact: -15\%
	
	
	As a Toxic Mentor Spirit, the player usually has a number of mental issues beyond their pact with the spirit itself. Additionally, they are commonly accompanied with negative reputations and other social effects that can offset the cost of the Mentor Spirit.
	
	\textbf{Mentor Bonuses:} 
	
	Resistance, Disease, +3 (Magical, -10\%; Pact, )
	
	\textit{Choose One:}
	\begin{itemize}
		\itemsep 0pt
		\item Talent,  (Summoning (Plague Spirits) or Stealth, Expert Skill (Epidemiology), 3, 4, 5, 6?) (Magical, -10\%; Pact, -15\%) [3.75]
		\begin{itemize}
			\itemsep 0pt
			\item \textit{Reaction Bonus:} +1 to Reaction Rolls from those who study or revere disease or similar things, such as Epidemiologists, Memeticists, Bio-Terrorists, and so on.
			\item \textit{Alt Benefit:} +1 to HT rolls to resist your own diseases and weapons.
		\end{itemize}
		\item Unfazeable (Familiar Horrors, -50\%; Magical, -10\%; Pact, -15\%) [3.75]
	\end{itemize}
	
	\textit{Pact Traits:} Cowardice, SC 12 [-10], Chummy [-5]
	
	\subsubsection{Dove}
	
	\subsubsection{Dragonslayer}
	
	\subsubsection{Eagle}
	
	\subsubsection{Fire-Bringer}
	
	\subsubsection{Green Man}
	
	\subsubsection{Planar Entity}
	
	\subsubsection{Moon}
	
	\subsubsection{Mountain}
	
	\subsubsection{Mutation}
	
	\subsubsection{Pollution}
	
	\subsubsection{Rat}
	
	\subsubsection{Raven}
	
	\subsubsection{Sea}
	
	\subsubsection{Seducer}
	
	\subsubsection{Shark}
	
	\subsubsection{Snake}
	
	\subsubsection{Sun}
	
	\subsubsection{Thunderbird}
	
	\subsubsection{War}
	
	\subsubsection{Wild Hunt}
	
	\subsubsection{Wise Warrior}
	
	\subsubsection{Wolf}


	
	\subsection{Initiation}
	
	As an awakened grows in power, they will often run up against the limits of their abilities in fundamental ways. In order to overcome them, awakened must reshape their understanding and relationship with magic in order to increase their connection to the Awakened world. The process is fundamentally personal and relates heavily to the awakened's tradition and personality, but the one constant is that Initiation is a hard process; even one's first Initiation can take months of effort and is comparable in time and effort to a Master's Thesis, with further Initiations going beyond that. The GM should work with their player in order to create a suitable goal, but some guidelines are as follows:
	
	\begin{itemize}
		\itemsep 0pt
		\item All Initiations should contain some sort of ordeal or event. This need not necessarily be something that a player and GM sit down and play out every detail in a solo session (Although in my experience that tends to be a great time), however this event should contain some milestone for the character.
		\item All ordeals should be difficult and new to the Awakened. The specifics depend on the individual, but this should not be a task that is run of the mill for them; it should strain them mentally and/or physically.
		\item All aspects of Initiation should be heavily influenced by the tradition of the awakened. Some lean more spiritual, some more academic, some more fraternal, but there is plenty to work with for any tradition.
	\end{itemize}

	As some food for thought, here are some commonly used templates for Initiation ordeals that can be followed:
	
	\subsubsection*{Metaplanar Quest}
	
	A perfect choice for an awakaned who can project's first Initiation, due to them usually gaining access to the Metaplanes with said Initiation. These involve going to the metaplanes and undergoing a metaphorical and literal ordeal surrounding the character. This often strikes at their fears and anxieties, but can also serve as a vessel for story hints, a way to build up hubris before the fall, as a difficult test of capabilities (especially astral ones), and more. The perfect first case is the encounter with \textcolor{Blue}{\href{https://en.wikipedia.org/wiki/Guardian_of_the_Threshold}{The Dweller on the Threshold}}. First found in the novel Zanoni, a neat summary for it \textcolor{Blue}{\href{https://web.archive.org/web/20160402204637/http://www.andras-nagy.com/dweller.html}{be found here.}} The Dweller shouldn't simply be a dramatic attempt to scare the initiate with horrible scenes from their past, it should test them entirely to see if they are worthy of magic and the metaplanes; it should not only see if they are able to overcome their baggage but also to see if they are able to grow and cast off their physical attachments - and to that end failure should be in many ways the expected outcome, taking multiple tries to get across such growth to the character. Of course, the Dweller is different for each person and each tradition, which should play a core role is these trials.
	
	\subsubsection*{Nine Paths to Enlightenment}
	
	Very similar to the Buddhist Eightfold Path and a common choice for many adepts, this ordeal follows nine smaller ordeals that deal with controlling and overcoming physical and mental weaknesses of the individual. The GM should devise a series of ordeals for each of the nine paths, with some good examples already present on Street Grimoire p141.
	
	\subsubsection*{Sacrifice/Geas}
	
	The awakened must give up something of permanent value from them, in return for release and enlightenment; this can either be physical or mental, in the form of Negative Disadvantages gained, or sometimes Positive Advantages lost. The GM should feel free to allow those points to be spent towards Initiation advantages, but it's not required - after all the point of this \textit{is} to learn to accept loss and seek enlightenment from petty physical wants.
	
	\subsubsection*{Thesis/Masterpiece}
	
	A popular option for the more braniac types, this involves creating a piece of media - whether it be artworks, a ThD, a story, or so on - so powerful and imapctful to enlighten and push forward the awakened's understanding of his tradition and magic - alongside anyone else who partakes in it. Hermetics are well known for writing ThDs on all sorts of magical phenomena and the like, which is the most common application of this, but don't discount options such as art, music, plays, and so on - think of the stories about pieces of media so impactful the revolutionize an individual's life, such as the Painting of Dorian Gray or the play for The King in Yellow, these are extreme examples, but just as poignant. These should always be entirely new information and art for their field and tradition - no recreating something that's already been done at all, unless you have something poignant to add to it. Often, this should require lots of time and effort and - given the nature of magic and awkaned phenomena - danger in collecting information, materials, or inspiration.
	
	\subsubsection*{Completing Initiation}
	
	After completing their ordeal, if the GM decides the awakened has sufficiently progressed enough in their understanding of magic, they can successfully Initiate, increasing their grade by one. Many Initiate Metamagics reference the Awakened's Initiate Grade, and as such whenever it is increased all corresponding abilities should also be bought at higher levels - which \textit{does} make higher grades progressively more expensive! As an additional note, success in the ordeal is not always necessary to Initiate - all that is important is growth and understanding on the character's part. Each grade increases the maximum Magic that an awakened can have by 1 (from whatever limit the GM set beforehand). Additionally, for Initiate Grade 1, the awakened gains access to the Metaplanes and must buy off \textit{Limited Access (Astral Plane), -20\%} limitation for Astral Projection, replacing it with \textit{Limited Access(Astral Plane, 1 Metaplane), -15\%} for 5 points \textit{at a minimum}, but can lower it further or remove it entirely as they wish. If they do not remove it entirely, they can choose to reduce it further at later Initiations or if taught the information about accessing more planes. Additionally, they are allowed to purchase an additional Metamagic after each Initiation.
	
	\subsection{Metamagics}
	
	The most important benefit of Initiations are Metamagics. These are powerful scaling abilities that allow for unorthodox improvements to an awakened's abilities. Only one can be taken at each Initiate Grade, although they often scale in power with the awakened's Initiate Grade as well.
	
	\subsubsection{Centering}
	\begin{flushright}
		2.8 Points per Initiate Grade.
	\end{flushright}
	
	An Initiate's enlightened understanding of their magic allows them to engage in mundane actions related to their tradition, allowing them to better process drain. Whenever the Initiate uses an ability that causes drain, they can engage in some obvious physical technique related to their tradition; this can be thing like arcane gestures, magical chants, playing a musical instrument,  praying loudly to your gods, and so on. 
	
	Whenever you are performing Centering, add +1 to your Will for the purposes of determining your resistance roll.
	
	This \textit{can} be interrupted in any normal way, usually via grappling or gagging, denying you your \textit{channeling bonus}, but not the spell - magic requires none of this to work, it simply allows you to work better. Certain techniques might be worth more for being more easily interruptable, in which case the cost for this can be adjusted accordingly.
	
	\textcolor{OliveGreen}{\textit{Statistics: Will +1 (Accessibility, Only for Drain, -40\%\footnote{A subsection of the Will resistance for the attribute, so it's a small portion of a portion.]; Magical, -10\%; Requires Gesture/Magic Words, -10\%}) [2.8 per level] }}
	
	\subsubsection{Masking}
	\begin{flushright}
		7.6 Points per Initiate Grade
	\end{flushright}

	Masking allows a character to make their astral signature harder to detect or interpret. The metamagic can be activated, deactivated, or changed as a Ready Action. Individuals trying to assense or notice the Initiate's astral signature take a -2 penalty per Initiate Grade to their roll (To a maximum Grade of 5,  at which point it is impossible to see through). The player may have this either be a penalty to notice their aura at all or alternatively masking it as different aura.

	\textcolor{OliveGreen}{\textit{Statistics: Additionally, add Obscure 2, Astral Perception (Defensive, +50\%; Stealthy, +100\%; No AOE, -50\%; Magical, -10\%) [7.6] }}
	
	\subsubsection{Adept Powers}
	
	The Initiate can take additional powers, past the amount usually allowed by the GM. A usual limit is an extra 10-20 points per Initiate Grade.
	
	\subsubsection{Shielding}
	\begin{flushright}
		5 Points per two Initiate Grades
	\end{flushright}
	
	The adept is particularly skilled at protecting themselves and others via Counterspelling. For every two Initiate Grades (round up), add a +1 to all Power Defense rolls using the Counterspelling skill.
	
	\textcolor{OliveGreen}{\textit{Statistics: Enhanced Power Defense (Counterspelling) [5] }}
	
	\subsubsection{Psychometry}
	\begin{flushright}
		32 Points
	\end{flushright}
	
	A powerful and unique metamagic steeped in occult and mysticism of pre-awakened magic. It allows the Initiate to read the emotions imbued in objects and possibly even experience flashbacks pertaining to its point of view and events. The power is relatively uncontrollable, with early Initiate often swarmed with emotions and visions for every object before they learn to put up mental shields. 
	
	To use it, they must lower their shield and touch the object for 8 seconds and make an IQ roll. The GM can choose suitable events, which can include any uneventful event alongside emotionally charged ones (although it leans towards emotionally charged ones). Roll at no penalty for an event that occured in the same day, -1 for up to 10 days ago, -2 for up to 100 days ago, -3 for 3 years, -4 for 30 years, -5 for 300 years, etc. Success gives the general sense of the emotions and event, which can cause Fright Checks or other suitable effects! Additionally, on a Margin of Success of 3 or more or a Critical Success, you have \textit{genuine flashbacks}, experiencing the events from the object's point of view. The Initiate must stick around throughout the entirety of the vision or effect, no cutting things off early even if they want to!
	
	It's not uncommon to have things slip through the cracks in your shield. You might automatically be able to notice noteworthy or strong effects on an IQ-4 roll, even if you are not concentrating! However, during times of fear or stress you must also make a Will roll (14 and above automatically fails) or have your ability played as if a hostile or impish entity.
	
	On a failure, you receive no impressions at all and cannot attempt again for that object or place 24 hours.
	
	\textcolor{OliveGreen}{\textit{Statistics: Psychometry (Immsersive, (Limited by Margin, 3 -15\%) +85\%; Sensitive, +30\%; Magical, -10\%; Nuisance Effect, Must take entire time, -5\%; Takes Extra Time, 8 seconds, -30\%; Uncontrollable, -10\%) [32] }}
	
	\subsubsection{Danger Sense}
	\begin{flushright}
		14 Points
	\end{flushright}
	
	The Initiate's connection to their mystical abilities is used to sharpen their senses, subtly alerting them to the most subtle of threats. They gain the Danger Sense advantage as per the Danger Sense Adept Power.
	
	\textcolor{OliveGreen}{\textit{Statistics: Danger Sense (Magical, -10\%) [14] }}
	
	\subsubsection{Exorcism}
	\begin{flushright}
		10 Points + 2.5 Points per Initiate Grade past 1
	\end{flushright}

	The Initiate is able to perform a unique form of astral combat that is specialized towards forcing spirits out of their vessels. 
	
	To perform an Exorcism, the Initiate must make contact with the aura of a currently possessed vessel. This is usually done via an Attack maneuver to touch them, after which they must roll a Quick Contest of Will + Magic versus the spirit's Will + Magic. 
	
	\textit{Winning} forces the spirit out and prevents them from using their Possession, Inhabitation, Chanelling, or similar ability for a number of minutes equal to 3 times the Margin of Failure, while tying or failure has no effect. Critical failure on the Will roll cripples the ability for 1d hours.
	
	Exorcism can only be used on objects/bodies that Spirits are actively Possessing, Inhabiting, Channeling, or so on.
	
	At higher Initiate Grades, the Initiate is more capable of driving out Spirits. Add a +1 to their Quick Contest for every Initiate Grade past 1.
	
	\textcolor{OliveGreen}{\textit{Statistics: Neutralize (Extended Duration, $\times$3, +20\%; Accessibility, Currently active, -10\%; One Ability, Possession, -80\%; Magical, -10\%) [10] further levels add Reliable, +x }}
	
	\subsubsection{Extended Masking}
	\begin{flushright}
		2 Points per Initiate Grade
	\end{flushright}

	The Initiate is able to Extend their Masking to cover spells, effects, foci, and so on in their vicinity. While an object is within 2 yards of the Initiate, it gains the same penalty as for Masking.
	
	\textcolor{OliveGreen}{\textit{Statistics: Remove No AOE, -50\% on Masking [2] }}
	
	\subsubsection{Paradigm Shift/Spirit Expansions}
	\begin{flushright}
		5 Points
	\end{flushright}

	A dangerous Metamagic that allows an Initiate to set aside their tradition and take upon a dangerous new one. % TODO: Address the drawbacks mechanically.
	
	\textbf{General Shift: } The most common option, this simply lets an Initiate change to another tradition. This version costs no CP.
	
	\textbf{Insect Shaman:}	This metamagic is only able to be taught to an awakened by an Insect Spirit Queen and comes lined with fine print. Often, the Queen will sponsor the character with the CP for purchasing this Initiation - but with an equal amount of matching negatives representing the fine print. This metamagic replaces all of the Initiate's summons with Insect Spirits.
	
	\textbf{Toxic:} A shift towards corrupted energy, toxic magicians are a horrible corruption of their traditions with most of them going utterly insane. Often, they can offset the CP cost of this initiation with negative qualities representing their loss of sanity. This Metamagic turns all of the Initiate's summons into their Toxic versions.
	
	\textbf{Shedim: } A unique shift that does not shift the Initiate away from their original tradition. This Metamagic is only taught by Master Shedim, created after the closing of the DeeCee portal in order to bring more demons into the realm. This Metamagic allows the Initiate to summon Shedim.
	
	\textcolor{OliveGreen}{\textit{Statistics: Unusual Background (Outer Spirits) [5] }}
	
	\subsubsection{Unified Magical Theory}
	\begin{flushright}
		1 Points per Initiate Grade
	\end{flushright}
	
	One of the most amazing feats for Unified Magical Theory (UMT), achieved by a group of Shamans in Prague, this Metamagic allows an Initiate to summon spirits outside of their tradition.
	
	When picked, the Initiate must choose a type of spirit that they cannot currently summon, adding it to their list of available summons.
	
	\textcolor{OliveGreen}{\textit{Statistics: Unusual Background (Non-Standard Spirits) [1] }}
	
	\subsection{Spells}
	
	Spells are cast using \hyperref[spellcasting_skill]{the Spellcasting Skill.} They are priced according to your Magic, with higher level Magics allowing for more powerful spells.
	
	Each section of spells has their own class specific modifers, however one that you will see often is:	\textit{Requires (Spellcasting) Roll, -20\%}. This is made up of the Requires (10) Roll, -20\% and Requires (Spellcasting) Roll, -0\% limitations. For abilities that already have Attribute rolls, such as IQ or Will, it costs to difference between this and the respective Requires (Attribute) Roll limitations (-10\% and -15\% in those cases). 
	
	\subsubsection{Concentration versus Concentration}
	
	Many spells here have a small, but subtle distinction between two types of Concentration that they have. Spells that simply have the Terminal Condition, Loses Concentration, -20\% Limitation do not require constant Concentrate maneuvers. They are still subject to Will rolls to drop them, drop when the user goes unconscious, and so, as for normal abilities that requires constant Concentrate maneuvers.
	
	In contrast, some spells specify that they need constant Concentrate Maneuvers, which require the user to spend their maneuver manipulating or controlling the spell, as normally described in the Basic Set.
	
	\subsubsection{Combat Spells}
	
	Combat Spells focus on one primary goal: dealing damage. How they do so varies, with different ranges of effect, damage and damage types, etc.
	
	All Combat Spells have one of the following modifiers:
	
	\textcolor{OliveGreen}{\textit{Physical Spell, +15\%: (Increased Range, LOS, +40\%; Variable, +5\%; Magical, -10\%; Requires (Spellcasting) Roll, -20\%)}}
	
	\textcolor{OliveGreen}{\textit{Direct Touch, +40\%: (Malediction 1, +100\%; Variable, +5\%; Magical, -10\%; Melee Attack, C, No Parry, -35\%; Requires (Spellcasting) Roll, -20\%}}
	
	\textcolor{OliveGreen}{\textit{Direct Spell, +160\%: (Malediction 3, +200\%; Variable, +5\%; Magical, -10\%; Requires (Spellcasting) Roll, -20\%; Sense-Based, Reversed, Vision \& Touch, -15\%}}
	
	\paragraph{Acid Stream}
	
	A powerful corrosive spray covers the target, causing chemical burns. The attack has a range of Line-of-sight, accuracy of 3, and each Force does 1d-2 cor damage, with 1 Cycle after 10 seconds, evaporating soon after. Every 5 points of basic damage reduces the target's DR by 1.
	
	\begin{center}
		\begin{tabular}{|c|c|}
			\hline
			Magic & Base Cost \\
			\hline
			\hline
			Magic 1 & 10 \\
			Magic 2 & 20 \\
			Magic 3 & 29 \\
			Magic 4 & 37 \\
			Magic 5 & 44 \\
			Magic 6 & 50 \\
			Magic 7 & 54 \\
			Magic 8 & 57 \\
			\hline
		\end{tabular}
	\end{center}
	
	\textcolor{OliveGreen}{\textit{Statistics: Innate Attack, 1d-2 Cor (Cyclic 1, 10 sec, +50\%; Physical Spell, +15\%)}}
	
	\paragraph{Toxic Wave}
	
	Creates a powerful burst of corrosive chemicals, able to cause chemical burns. The attack has a range of Line-of-sight, accuracy of 3, and each Force does 1d-2 cor damage, with 1 Cycle after 10 seconds, evaporating soon after. Every 5 points of basic damage reduces the target's DR by 1. As well, the corrosive sprays out in an sphere, diving damage by 3\(\times\)The number of yards from the center of the spell's effect. 
	
	\begin{center}
		\begin{tabular}{|c|c|}
			\hline
			Magic & Base Cost \\
			\hline
			\hline
			Magic 1 & 13 \\
			Magic 2 & 26 \\
			Magic 3 & 38 \\
			Magic 4 & 49 \\
			Magic 5 & 59 \\
			Magic 6 & 68 \\
			Magic 7 & 75 \\
			Magic 8 & 81 \\
			\hline
		\end{tabular}
	\end{center}
	
	\textcolor{OliveGreen}{\textit{Statistics: Innate Attack, 1d-2 Cor (Cyclic 1, 10 sec, +50\%; Physical Spell, +15\%; Explosive 1, +50\%)}}
	
	\paragraph{Punch}
	
	Quite literally casting fist, this spell smacks the target with concussive force. You must succesfully touch the target to affect them, dealing 1d+1 cr damage for each Force, with basic damage double for the purposes of determining knockback.
	
	\begin{center}
		\begin{tabular}{|c|c|}
			\hline
			Magic & Base Cost \\
			\hline
			\hline
			Magic 1 & 13 \\
			Magic 2 & 25 \\
			Magic 3 & 36 \\
			Magic 4 & 45 \\
			Magic 5 & 52 \\
			Magic 6 & 57 \\
			Magic 7 & 60 \\
			Magic 8 & 64 \\
			\hline
		\end{tabular}
	\end{center}
	
	\textcolor{OliveGreen}{\textit{Statistics: Innate Attack, 1d+1 Cr (Double Knockback, +20\%; Physical Spell, +15\%; Melee Attack, C, No parry, -35\%)}}
	
	\paragraph{Clout}
	
	This spell smacks the target with psychokinetic force. The spell has and range of Line-of-sight, accuracy 3, and deals 1d cr damage for each Force, with basic damage double for the purposes of determining knockback.
	
	\begin{center}
		\begin{tabular}{|c|c|}
			\hline
			Magic & Base Cost \\
			\hline
			\hline
			Magic 1 & 14 \\
			Magic 2 & 27 \\
			Magic 3 & 38 \\
			Magic 4 & 49 \\
			Magic 5 & 58 \\
			Magic 6 & 65 \\
			Magic 7 & 69 \\
			Magic 8 & 72 \\
			\hline
		\end{tabular}
	\end{center}
	
	\paragraph{Blast}
	
	This spell smacks the target with psychokinetic force. The spell has and range of Line-of-sight, accuracy 3, and deals 1d-1 cr damage for each Force, with basic damage double for the purposes of determining knockback. As well, the blast explodes out in an sphere, diving damage by 3\(\times\)The number of yards from the center of the spell's effect. 
	
	\begin{center}
		\begin{tabular}{|c|c|}
			\hline
			Magic & Base Cost \\
			\hline
			\hline
			Magic 1 & 13 \\
			Magic 2 & 26 \\
			Magic 3 & 38 \\
			Magic 4 & 48 \\
			Magic 5 & 58 \\
			Magic 6 & 66 \\
			Magic 7 & 73 \\
			Magic 8 & 78 \\
			\hline
		\end{tabular}
	\end{center}
	
	\textcolor{OliveGreen}{\textit{Statistics: Innate Attack, 1d-1 cr (Double Knockback, +20\%; Physical Spell, +15\%; Explosive 1, +50\%)}}
	
	\paragraph{Death Touch}
	
	By channelling destructive magical power directly into the cells of an opponent, the spellcaster can cause effects reminiscent of radiation or necrosis. You must first successfully touch a target to affect them, then roll a Quick Contest of Spellcasting+Magic vs. HT. Success deals 1d tox damage to the opponent for each Force. This damage ignores DR.
	
	\begin{center}
		\begin{tabular}{|c|c|}
			\hline
			Magic & Base Cost \\
			\hline
			\hline
			Magic 1 & 12 \\
			Magic 2 & 22 \\
			Magic 3 & 32 \\
			Magic 4 & 41 \\
			Magic 5 & 48 \\
			Magic 6 & 54 \\
			Magic 7 & 58 \\
			Magic 8 & 61 \\
			\hline
		\end{tabular}
	\end{center}
	
	\textcolor{OliveGreen}{\textit{Statistics: Innate Attack, 1d tox (Direct Touch, +40\%)}}
	
	\paragraph{Manabolt}
	
	By channelling destructive magical power directly into the cells of an opponent, the spellcaster can cause effects reminiscent of radiation or necrosis. You may affect any target that you can see or touch, rolling a Quick Contest of Spellcasting+Magic vs. HT, with Long-Distance modifiers for range. Success deals 1d-1 tox damage to the opponent for each Force. This damage ignores DR.
	
	\begin{center}
		\begin{tabular}{|c|c|}
			\hline
			Magic & Base Cost \\
			\hline
			\hline
			Magic 1 & 15 \\
			Magic 2 & 29 \\
			Magic 3 & 43 \\
			Magic 4 & 56 \\
			Magic 5 & 68 \\
			Magic 6 & 78 \\
			Magic 7 & 88 \\
			Magic 8 & 96 \\
			\hline
		\end{tabular}
	\end{center}
	
	\textcolor{OliveGreen}{\textit{Statistics: Innate Attack, 1d-1 tox (Direct Spell, +160\%)}}
	
	\paragraph{Manaball}
	
	By channelling destructive magical power directly into the cells of an opponent, the spellcaster can cause effects reminiscent of radiation or necrosis. You may affect any target that you can see or touch, rolling a Quick Contest of Spellcasting+Magic vs. HT, with Long-Distance modifiers for range. Success deals 1d-2 tox damage to the opponent for each Force. As well, the energy suffuses into the surrounding area, dealing the same damage to those nearby, divided by 3\(\times\)The number of Yards from the center. This damage ignores DR. 
	
	\begin{center}
		\begin{tabular}{|c|c|}
			\hline
			Magic & Base Cost \\
			\hline
			\hline
			Magic 1 & 10 \\
			Magic 2 & 20 \\
			Magic 3 & 29 \\
			Magic 4 & 38 \\
			Magic 5 & 47 \\
			Magic 6 & 55 \\
			Magic 7 & 62 \\
			Magic 8 & 68 \\
			\hline
		\end{tabular}
	\end{center}
	
	\textcolor{OliveGreen}{\textit{Statistics: Innate Attack, 1d-2 tox (Direct Spell, +160\%; Explosion 1, +50\%)}}
	
	\paragraph{Flamethrower}
	
	This spell births a flash of explosion and flame into existent, burning the target with blisteringly heat. It deals 1d burn damage for each Force. Additionally, it treats all Flammability Classes (B433) as one lower!
	
	\begin{center}
		\begin{tabular}{|c|c|}
			\hline
			Magic & Base Cost \\
			\hline
			\hline
			Magic 1 & 13 \\
			Magic 2 & 25 \\
			Magic 3 & 35 \\
			Magic 4 & 45 \\
			Magic 5 & 53 \\
			Magic 6 & 59 \\
			Magic 7 & 62 \\
			Magic 8 & 65 \\
			\hline
		\end{tabular}
	\end{center}
	
	\textcolor{OliveGreen}{\textit{Statistics: Innate Attack, 1d burn (Incendiary 1, +10\%; Physical Spell, +15\%)}}
	
	\paragraph{Fireball}
	
	This spell births a flash of explosion and flame into existent, burning the target with blisteringly heat. It deals 1d-1 burn damage for each Force. Even more so, the fire explodes outwards in a sphere, dealing the same damage to those nearby, divided by 3\(\times\)The number of Yards from the center. Additionally, it treats all Flammability Classes (B433) as one lower! 
	
	\begin{center}
		\begin{tabular}{|c|c|}
			\hline
			Magic & Base Cost \\
			\hline
			\hline
			Magic 1 & 13 \\
			Magic 2 & 24 \\
			Magic 3 & 35 \\
			Magic 4 & 46 \\
			Magic 5 & 55 \\
			Magic 6 & 62 \\
			Magic 7 & 68 \\
			Magic 8 & 72 \\
			\hline
		\end{tabular}
	\end{center}
	
	\textcolor{OliveGreen}{\textit{Statistics: Innate Attack, 1d-1 burn (Explosion 1, +50\%; Incendiary 1, +10\%; Physical Spell, +15\%)}}
	
	\paragraph{Lightning Bolt}
	
	This spell creates and directs dangerous electricity. It deals 1d burn damage for each Force. While it cannot set things on fire, it does surge - which can cause electronics to short and die out.
	
	\begin{center}
		\begin{tabular}{|c|c|}
			\hline
			Magic & Base Cost \\
			\hline
			\hline
			Magic 1 & 13 \\
			Magic 2 & 25 \\
			Magic 3 & 35 \\
			Magic 4 & 45 \\
			Magic 5 & 53 \\
			Magic 6 & 59 \\
			Magic 7 & 62 \\
			Magic 8 & 65 \\
			\hline
		\end{tabular}
	\end{center}
	
	\textcolor{OliveGreen}{\textit{Statistics: Innate Attack, 1d burn (Physical Spell, +15\%; Surge, +20\%; No Incendiary, -10\%)}}
	
	\paragraph{Ball Lightning}
	
	This spell creates and directs dangerous electricity. It deals 1d burn damage for each Force. Additionally, the electricity explodes outwards in a sphere, dealing the same damage to those nearby, divided by 3\(\times\)The number of Yards from the center. While it cannot set things on fire, it does surge - which can cause electronics to short and die out.
	
	\begin{center}
		\begin{tabular}{|c|c|}
			\hline
			Magic & Base Cost \\
			\hline
			\hline
			Magic 1 & 13 \\
			Magic 2 & 24 \\
			Magic 3 & 35 \\
			Magic 4 & 46 \\
			Magic 5 & 55 \\
			Magic 6 & 62 \\
			Magic 7 & 68 \\
			Magic 8 & 72 \\
			\hline
		\end{tabular}
	\end{center}
	
	\textcolor{OliveGreen}{\textit{Statistics: Innate Attack, 1d burn (Explosion 1, +50\%; Physical Spell, +15\%; Surge, +20\%; No Incendiary, -10\%)}}
	
	\subsubsection{Detection Spells}
	
	Detection Spells specialize in data capture and analysis. This can range from looking at technology to looking at life signs, but their common trait is information.
	
	Detection Spells, unlike Combat Spells, are longer lasting and usually require an uninterrupted series of Concentrate Maneuvers when gaining any information from them (P155).
	
	When using a Detection Spell, it's important to remember that while you can usually only sustain one spell at a time, Detection spells are based on Afflictions, which means that you can apply \textit{the same spell} to different targets multiple times simultaneously, should you so wish, however you must still maintain concentration or lose them all!
	
	This means that you can give the entire crew Analyze Magic at the same time (if you don't pass out first!), but can still not mix spells among them without paying the extra cost!
	
	Most Detection Spells are grantable to other individuals, and as such need a special modifier to represent those capabilities:
	
	\textcolor{OliveGreen}{\textit{Detection Spell, +145\%: (Cancellable, +10\%; Extended Duration, \(\times\)500, +110\%\footnote{Gives around 8 hours per Margin of Success, which allows for 1 day timeframes.}; Malediction 1, +100\%; Variable, +5\%; Fixed Duration, +0\%; Magical, -10\%; Melee Attack, C, No Parry, -35\%; Requires Spellcasting Roll\footnote{Since Malediction already has a Will roll, this just switches it to a flat 10 roll + Requires Spellcasting, -20\%, priced for difference between them.}, -15\%}; Terminal Condition, Loses Concentration, -20\footnote{Taken from \textcolor{Blue}{\href{http://forums.sjgames.com/showpost.php?p=817197&postcount=7}{Kromm's Suggested value here for losing concentration}}, since Requires Concentration is not allowed on Afflictions.}\%)}
	
	\paragraph{Analyze Device}
	
	This spell allows the user to analyze the composition, makeup, and design of devices. 
	
	It can be cast on any individual by touch. To use it, make a Concentrate maneuver and you must win a Quick Contest of IQ+Magic (The Spellcaster's) vs. the device's HT. Success gives general information about the device based on Margin of Victory. Some examples of information, from easiest to hardest, are: general purpose, general design, identifying information (VIN, PID, etc), operation use, obscure information and usage, etc. Critical Success can yield advanced information as below.
	
	Success on the Spellcasting Roll by 5 or more, greatly improves the capabilities of it! The user \textit{automatically} succeeds on the IQ roll for general information, and can instead roll the same Quick Contest to gain advanced information. Some examples of advanced information, from easiest to hardest, are: chemical composition, advanced design (hardware design for instance), detailed forensic information, hidden information and functions, etc.
	
	The user must maintain line of sight or touch to the object throughout analysis, even if they know where it is via other senses.
	
	Highly technological devices can be much harder to analyze, providing a -3 penalty for devices such as basic electronics, composite materials, etc. and a -6 penalty for devices such as computers, cameras, highly advanced meta-materials, etc.
	
	Higher Force spells add a +1 per Force above 1 to all IQ rolls to analyze or use to ability well.
	
	\begin{center}
		\begin{tabular}{|c|c|}
			\hline
			Magic & Base Cost \\
			\hline
			\hline
			Magic 1 & 41 \\
			Magic 2 & 44 \\
			Magic 3 & 47 \\
			Magic 4 & 49 \\
			Magic 5 & 51 \\
			Magic 6 & 52 \\
			Magic 7 & 53 \\
			Magic 8 & 54 \\
			\hline
		\end{tabular}
	\end{center}
	
	\textcolor{OliveGreen}{\textit{ Statistics: Affliction (Analyze Device, +150\%; Detection Spell, +145\%) [39] further levels add +15\% to Analyze Device [1.5].}}
	
	\textcolor{OliveGreen}{\textit{Analyze Device, +155\%: (Detect, Devices (Very Common; Analyzing (Margin-Based, 5 MoS, -50\%), +50\%; Analysis Only, -50\%; Hard to Use 2 (Accessibility, Only High Tech Devices, -0\%), -10\%; Magical, -10\%; Resistable vs HT, -15\%; Sense-Based, Reversed, Sight \& Touch, -15\%) [15] further levels give Reliable [1.5]/+5\%}}
	
	\paragraph{Analyze Magic}
	
	This spell allows its user to analyze spells, powers, and other magical effects, even without being able to astrally perceive.
	
	It can be cast on any individual by touch. To use it, make a Concentrate maneuver and roll an IQ+Magic (The Spellcaster's). Success gives general information about the magic, exactly as if it were Assensed.
	
	Success on the Spellcasting Roll by 5 or more greatly improves its capabilities! The user \textit{automatically} succeeds on the IQ roll for general information, and can instead roll the same roll to gain advanced information. Some examples of advanced information, from easiest to hardest, are: Any information Empathy could provide, detailed information about the subjects emotional state (Provides +3 to the listed skills), hints about the awakened tradition, detailed analysis of the magical effect or astral signature, detailed health and diagnosis, detailed information about most cyberware, greater information about bioware, and hints against the style of technomancer if they are one.
	
	The user must maintain line of sight or touch to the object throughout analysis, even if they know where it is via other senses.
	
	Higher Force spells add a +1 per Force above 1 to all IQ rolls to analyze or use to ability well.
	
	\begin{center}
		\begin{tabular}{|c|c|}
			\hline
			Magic & Base Cost \\
			\hline
			\hline
			Magic 1 & 33 \\
			Magic 2 & 34 \\
			Magic 3 & 35 \\
			Magic 4 & 36 \\
			Magic 5 & 36 \\
			Magic 6 & 37 \\
			Magic 7 & 37 \\
			Magic 8 & 37 \\
			\hline
		\end{tabular}
	\end{center}
	
	\textcolor{OliveGreen}{\textit{ Statistics: Affliction (Analyze Magic, +75\^; Detection Spell, +145\%) [33] further levels add +5\% [0.5] to Analyze Magic}}
	
	\textcolor{OliveGreen}{\textit{Analyze Magic, +75\%: (Detect, Magic (Occasional; Analyzing (Margin-Based, 5 MoS, -50\%), +50\%; Analysis Only, -50\%; Magical, -10\%; Sense-Based, Reversed, Sight \& Touch, -15\%) [7.5] further levels give Reliable [0.5]/+5\%}}
	
	\paragraph{Analyze Truth}
	
	This spell allows its user to analyze statements to determine whether they are intentional falsehoods.
	
	It can be cast on any individual by touch. To use it, you must make a Concentrate maneuver and win an IQ+Magic (The Spellcaster's) vs the target's Will. 
	
	Success will give general information about the falsehood (As long as they believe it is false of course!), determined by the Margin of Victory. Some examples of information, from easiest to hardest are: Vaguely whether the statement is a lie or not, vague intention regarding the falsehood, which parts of the statement are lies, etc. These should generally provide, at a minimum, a +1 bonus to rolls that would benefit from knowing it is a lie, such as Law, Detect Lies, Psychology, etc.
	
	Success on the Spellcasting Roll by 5 ore more greatly improves the spell. The user \textit{automatically} succeeds on the Quick Contest for general information, and can instead roll the same Quick Contest for advanced information. Some examples of advanced information, from easiest to hardest, are: Definitively whether the statement is a lie, good ideas as to the intentions behind the lie (Vibe check), which parts of the statement are lies and in what ways, etc. These should generally provide greater bonuses to rolls that would benefit from knowing it is a lie, from +1 and above.
	
	The user must maintain line of sight or touch to the object throughout analysis, even if they know where it is via other senses.
	
	Higher Force spells add a +1 per Force above 1 to all IQ rolls to analyze or use to ability well.
	
	\begin{center}
		\begin{tabular}{|c|c|}
			\hline
			Magic & Base Cost \\
			\hline
			\hline
			Magic 1 & 28 \\
			Magic 2 & 29 \\
			Magic 3 & 29 \\
			Magic 4 & 30 \\
			Magic 5 & 30 \\
			Magic 6 & 30 \\
			Magic 7 & 30 \\
			Magic 8 & 30 \\
			\hline
		\end{tabular}
	\end{center}
	
	\textcolor{OliveGreen}{\textit{ Statistics: Affliction (Analyze Truth, +30\%; Detection Spell, +145\%) [28.5] further levels add +2.5\% [0.25] to Analyze Truth}}
	
	\textcolor{OliveGreen}{\textit{Analyze Truth, +30\%: (Detect, Intentional Falsehoods (Rare; Analyzing (Margin-Based, 5 MoS, -50\%), +50\%; Analysis Only, -50\%; Magical, -10\%; Resistable vs Will, -15\%; Sense-Based, Reversed, Sight \& Touch, -15\%) [3] further levels give Reliable [0.25]/+5\%}}
	
	\paragraph{Clairaudience}
		
	This spell gives the Awakened's target the ability to project their sense of hearing out to a distance, perceiving over barriers and walls with relative ease.
	
	To use it, the Awakened must touch their target and Concentrate for 1 second and roll a Spellcasting + Magic; success gives the target the ability to project their sense of hearing (natural or magical hearing only) to any viewpoint and facing within a radius equal to Force $\times$ 5 yards, so long as the Awakened does not lose their concentration. 
	
	Projecting their senses requires the user to Concentrate for \textit{2 seconds} and make a separate IQ + Magic (the spellcaster's) roll; success actually projects their sense of hearing to a point of their choice, as mentioned above. Viewpoints outside of the user's line of sight suffer a -5 penalty to this roll.
	
	The user can use this viewpoint for targeting spells and abilities, but calculates all range penalties from their \textit{body}, not the viewpoint. That being said, it is usually impossible to target such effects with hearing without the Precise enhancement to it (See Powers: Enhanced Senses).
	
	On a failure the spell simply fails, but a failure by 1 means the spell goes \textit{somewhere else}, as determined by the GM. Critical Failure cripples the spell for 1d hours (which affects all other spells, as mentioned in the \hyperref[magic]{Magic Section}.)
	
	The user can change the viewpoint, facing, location, or simply return their senses at any point. This requires 2 seconds of Concentration and a successful IQ + Magic (The spellcaster's). Failure works as normal, with critical failure preventing the target from using the ability for 1d hours.
	
	The viewpoint can be set relative to something it's inside, which allows the user to place it inside of a moving vehicle or simialr circumstance, with no additional penalty.
	
	Lastly, even if using the Clairvoyance and Clairaudience spell, the user can only have one viewpoint at a time, including all senses at once.
	
	\begin{center}
		\begin{tabular}{|c|c|}
			\hline
			Magic & Base Cost \\
			\hline
			\hline
			Magic 1 & 70 \\
			Magic 2 & 74 \\
			Magic 3 & 77 \\
			Magic 4 & 78 \\
			Magic 5 & 79 \\
			Magic 6 & 80 \\
			Magic 7 & 80 \\
			Magic 8 & 80 \\
			\hline
		\end{tabular}
	\end{center}
	
	\textcolor{OliveGreen}{\textit{ Statistics: Affliction (Clairaudience, +400\%; Detection Spell, +145\%) [64.5]}}
	
	\textcolor{OliveGreen}{\textit{Clairaudience, +400\%: Clairsentience (Reduced Time 5, 2 seconds, (Only for initial 1 minute\footnote{Although this could be set as an Accessibility for -20\%, I feel that since we're also including an opposite version for the switching of viewpoints that it's best set as a raw value.}, -50\%) +50\%; Accessibility, No Technological Senses, -10\%; Clairaudience, -30\%; Fixed Range, -5\%; Magical, -10\%; Reduced Range, 5 yards, -10\%; Takes Extra Time, 2 seconds (Shifting view only, -60\%), -5\%) [40] further levels add linear Increased Range of +5 yards.}}
	
	\paragraph{Clairvoyance}
		
	This spell gives the Awakened's target the ability to project their sense of vision out to a distance, perceiving over barriers and walls with relative ease.
	
	To use it, the Awakened must touch their target and Concentrate for 1 second and roll a Spellcasting + Magic; success gives the target the ability to project their sense of vision (natural or magical hearing only) to any viewpoint and facing within a radius equal to Force $\times$ 5 yards, so long as the Awakened does not lose their concentration. 
	
	Projecting their senses requires the user to Concentrate for \textit{2 seconds} and make a separate IQ + Magic (the spellcaster's) roll; success actually projects their sense of hearing to a point of their choice, as mentioned above. Viewpoints outside of the user's line of sight suffer a -5 penalty to this roll. The vision is exactly as the user's natural (or awakened) vision is, without any additional ability to see in darkness.
	
	The user can use this viewpoint for targeting spells and abilities, but calculates all range penalties from their \textit{body}, not the viewpoint.
	
	On a failure the spell simply fails, but a failure by 1 means the spell goes \textit{somewhere else}, as determined by the GM. Critical Failure cripples the spell for 1d hours (which affects all other spells, as mentioned in the \hyperref[magic]{Magic Section}.)
	
	The user can change the viewpoint, facing, location, or simply return their senses at any point. This requires 2 seconds of Concentration and a successful IQ + Magic (The spellcaster's). Failure works as normal, with critical failure preventing the target from using the ability for 1d hours.
	
	The viewpoint can be set relative to something it's inside, which allows the user to place it inside of a moving vehicle or simialr circumstance, with no additional penalty.
	
	Lastly, even if using the Clairvoyance and Clairaudience spell, the user can only have one viewpoint at a time, including all senses at once.
		
	\begin{center}
		\begin{tabular}{|c|c|}
			\hline
			Magic & Base Cost \\
			\hline
			\hline
			Magic 1 & 70 \\
			Magic 2 & 74 \\
			Magic 3 & 77 \\
			Magic 4 & 78 \\
			Magic 5 & 79 \\
			Magic 6 & 80 \\
			Magic 7 & 80 \\
			Magic 8 & 80 \\
			\hline
		\end{tabular}
	\end{center}
	
	\textcolor{OliveGreen}{\textit{ Statistics: Affliction (Clairvoyance, +400\%; Detection Spell, +145\%) [64.5]}}
	
	\textcolor{OliveGreen}{\textit{Clairvoyance, +400\%: Clairvoyance (Reduced Time 5, 2 seconds, (Only for initial 1 minute\footnote{Although this could be set as an Accessibility for -20\%, I feel that since we're also including an opposite version for the switching of viewpoints that it's best set as a raw value.}, -50\%) +50\%; Accessibility, No Technological Senses, -10\%; Clairvoyance, -10\%; Normal Sight, -20\%; Fixed Range, -5\%; Magical, -10\%; Reduced Range, 5 yards, -10\%; Takes Extra Time, 2 seconds (Shifting view only, -60\%), -5\%) [40] further levels add linear Increased Range of +5 yards.}}
	
	\paragraph{Combat Sense}
	
	The Awakened imbues their target with the intuition of a veteran soldier, detecting threats and defending better than ever before.
	
	To do so, they must touch their target and Concentrate for 1 second before rolling a Quick Contest of Spellcasting + Magic versus Will (Which can be waived). Winning provides the target with Combat Reflexes until the Awakened loses concentration.
	
	Additionally, a Margin of Victory of 5 or more also provides the target with Enhanced Dodge 1, making them even quicker!
	
	Higher Forces are more powerful and consistent, add a +1 bonus to the Spellcasting roll for each Force above 1.
	
	\begin{center}
		\begin{tabular}{|c|c|}
			\hline
			Magic & Base Cost \\
			\hline
			\hline
			Magic 1 & 43 \\
			Magic 2 & 44 \\
			Magic 3 & 45 \\
			Magic 4 & 46 \\
			Magic 5 & 47 \\
			Magic 6 & 47 \\
			Magic 7 & 47 \\
			Magic 8 & 48 \\
			\hline
		\end{tabular}
	\end{center} 	
	
	\textcolor{OliveGreen}{\textit{ Statistics: Affliction (Combat Reflexes, +150\%; Enhanced Dodge, Side-Effect, +30\%; Detection Spell, +145\%) higher forces add Reliable +1 }}
	
	\paragraph{Detect Enemies}
	
	The Awakened casts a spell that grants its target the ability to detect enemies in their close vicinity.
	
	To do so, they must touch their target and Concentrate for 1 second before rolling a Spellcasting + Magic roll; success grants their target the ability to detect individuals that hold hostile intentions to them, which can range from an assassin to security scoping them out as they pass by. 
	
	To detect someone, the user must Concentrate for 1 second and win a Quick Contest of Perception + Magic (The spellcaster's) versus Will, with range penalties (B550) for distance; successfully detecting enemies will alert the user to the direction and distance (in essence pinpointing them) to the nearest significant source (Which can often exclude sources that would not be immediately hostile to you) alongside a general idea of how many were detected nearby. Notably, this is not precise enough to allow for targeting of any abilities through the spell, however it does make it easier to target using your other senses.
	
	As a follow-up, the user can Concentrate for 1 second and roll an IQ + Magic (The spellcaster's); success allows them to distinguish things about what they detected; depending on Margin of Success the user can learn things such as the level of hostility (i.e. a guard is less hostile than an assassin), vague aspects or reasons of that hostility, vague intentions in regards to that hostility, etc. A critical success would provide detailed information by analyzing the form of the spell, this could tell things such as the reason for hostility, what immediate plans in regards to it are, so on. Where it would be beneficial, the GM can assign bonuses for knowing an individual's intentions to social skills, ranging from +1 and above.
	
	Higher Force spells can detect further and analyze more consistently, gaining a +1 to Per and IQ for each Force above 1.	
	
	\begin{center}
		\begin{tabular}{|c|c|}
			\hline
			Magic & Base Cost \\
			\hline
			\hline
			Magic 1 & 33 \\
			Magic 2 & 33 \\
			Magic 3 & 34 \\
			Magic 4 & 34 \\
			Magic 5 & 34 \\
			Magic 6 & 35 \\
			Magic 7 & 35 \\
			Magic 8 & 35 \\
			\hline
		\end{tabular}
	\end{center} 
	
	\textcolor{OliveGreen}{\textit{ Statistics: Affliction (Detect Enemies, +75\%; Detection Spell, +145\%) further levels add +2.5\% [0.25] to Detect Enemies}}
	
	\textcolor{OliveGreen}{\textit{Detect Enemies, +75\%: Detect, Directly Hostile Individuals (Rare; Precise, Nontargeting, +90\%; Accessibility, Living only, -15\%; Magical, -10\%; Resistable vs Will, -15\%) [7.5] further levels give Reliable +5\% [0.25]}}
	
	\paragraph{Detect Enemies, Extended}
	
	This spell works exactly like Detect Enemies above, except that it has superior range, allowing to to both pick up more targets - alongside more noise and useless targets!
	
	When rolling Per to detect, reduce all range penalties by -6 (effectively $\times$10 range), down to a minimum of 0.
	
	\begin{center}
		\begin{tabular}{|c|c|}
			\hline
			Magic & Base Cost \\
			\hline
			\hline
			Magic 1 & 34 \\
			Magic 2 & 35 \\
			Magic 3 & 35 \\
			Magic 4 & 35 \\
			Magic 5 & 36 \\
			Magic 6 & 36 \\
			Magic 7 & 36 \\
			Magic 8 & 36 \\
			\hline
		\end{tabular}
	\end{center} 
	
	\textcolor{OliveGreen}{\textit{ Statistics: Affliction (Detect Enemies, +88\%; Detection Spell, +145\%) further levels add +2.5\% [0.25] to Detect Enemies}}
	
	\textcolor{OliveGreen}{\textit{Detect Enemies, +88\%: Detect, Directly Hostile Individuals (Rare; Precise, Nontargeting, +90\%; Reliable 6 (Accessibility, Range Penalties Only\footnote{Firstly, this preclude Analyzing, which is around 40\% value}, -15\%), +26\%; Accessibility, Living only, -15\%; Magical, -10\%; Resistable vs Will, -15\%) [8.8] further levels give Reliable +5\% [0.25]}}
	
	\paragraph{Detect Individual}
		
	The Awakened casts a spell that grants its target the ability to detect a singular individual known to them - designated while casting - in their close vicinity.
	
	To do so, they must touch their target and Concentrate for 1 second before rolling a Spellcasting + Magic roll; success grants their target the ability to detect the specified individual while nearby.
	
	To detect the individual, the user must Concentrate for 1 second and win a Quick Contest of Perception + Magic (The spellcaster's) versus Will (Which can be waived), with range penalties (B550) for distance; successfully detecting them will alert the user to the direction and distance (in essence pinpointing them) to the nearest significant source alongside a general idea of how many were detected nearby (very often, this is simply the individual - unless they have clones or doppelgangers for some reason).  Notably, this is not precise enough to allow for targeting of any abilities through the spell, however it does make it easier to target using your other senses.
	
	As a follow-up, the user can Concentrate for 1 second and roll an IQ + Magic (The spellcaster's); success allows them to distinguish things about what they detected; depending on Margin of Success the user can learn things such as a vague state of health for the individual, vague qualities about them, etc. A critical success provides excessive information, such as their race, height, health, etc. Where it would be beneficial, the GM can assign bonuses for knowing such things about them to skills, ranging from +1 and above.
	
	Higher Force spells can detect further and analyze more consistently, gaining a +1 to all rolls for each Force above 1.		
	
	\begin{center}
		\begin{tabular}{|c|c|}
			\hline
			Magic & Base Cost \\
			\hline
			\hline
			Magic 1 & 42 \\
			Magic 2 & 43 \\
			Magic 3 & 44 \\
			Magic 4 & 45 \\
			Magic 5 & 45 \\
			Magic 6 & 46 \\
			Magic 7 & 46 \\
			Magic 8 & 46 \\
			\hline
		\end{tabular}
	\end{center} 
	
	\textcolor{OliveGreen}{\textit{ Statistics: Affliction (Detect Individual, +165\%; Detection Spell, +145\%) further levels add +5\% [0.5] to Detect Individual}}
	
	\textcolor{OliveGreen}{\textit{Detect Individual, +165\%: Detect, Known Individual (Occasional; Precise, Nontargeting, +90\%; Magical, -10\%; Resistable vs Will, -15\%) [16.5] further levels give Reliable +5\% [0.5]}}
	
	\paragraph{Detect Individual, Extended}
	
	This spell works exactly like Detect Individual above, except that it has superior range, allowing to to both pick up more targets - alongside more noise, at least as is possible with this spell!
	
	When rolling Per to detect, reduce all range penalties by -6 (effectively $\times$10 range), down to a minimum of 0.
	
	\begin{center}
		\begin{tabular}{|c|c|}
			\hline
			Magic & Base Cost \\
			\hline
			\hline
			Magic 1 & 45 \\
			Magic 2 & 46 \\
			Magic 3 & 46 \\
			Magic 4 & 47 \\
			Magic 5 & 48 \\
			Magic 6 & 48 \\
			Magic 7 & 49 \\
			Magic 8 & 49 \\
			\hline
		\end{tabular}
	\end{center} 
	
	\textcolor{OliveGreen}{\textit{ Statistics: Affliction (Detect Individual, +191\%; Detection Spell, +145\%) further levels add +5\% [0.5] to Detect Individual}}
	
	\textcolor{OliveGreen}{\textit{Detect Individual, +191\%: Detect, Known Individual (Occasional; Precise, Nontargeting, +90\%; Reliable 6 (Accessibility, Range Penalties Only\footnote{Firstly, this preclude Analyzing, which is around 40\% value}, -15\%), +26\%; Magical, -10\%; Resistable vs Will, -15\%) [19.1] further levels give Reliable +5\% [0.5]}}
	
	\paragraph{Detect Life}
		
	The Awakened casts a powerful spell that grants its target the ability to detect \textit{all forms of life}.
	
	To do so, they must touch their target and Concentrate for 1 second before rolling a Spellcasting + Magic roll; success grants their target the ability to detect all forms of life (Which notably, excludes spirits), within their vicinity.
	
	To detect anything, the user must Concentrate for 1 second and win a Quick Contest of Perception + Magic (The spellcaster's) versus Will, with range penalties (B550) for distance; successfully detecting will alert the user to the direction and distance (in essence pinpointing them) to the nearest significant source alongside a general idea of how many were detected nearby. Notably, this is not precise enough to allow for targeting of any abilities through the spell, however it does make it easier to target using your other senses.
	
	When deciding on what "the nearest significant source" and "how many were detected nearby" the GM is warned to explicitly avoid situations similar to one such as "There's a bacteria on your face and there's 30 billion sources nearby". This detection should automatically ignore anything insignificant to the user, usually most micro-organisms, small animals, etc. and, it should prioritize humans and larger animals as the \textit{most significant} sources, meaning the caster will usually be told the closest SM-1 and above creature nearby alongside the general number of them. This is not to say they won't be detected, but they won't overwhelm the spell and make it useless! Of course, sometimes such pedantry is useful, such as when in a quarantine zone searching trying to avoid a dangerous disease! Lastly, this spell never "overwhelms" the user with information in any way that causes stunning or similar negative effects - outside strange circumstances anyways - no matter how much life is detected.
	
	As a follow-up, the user can Concentrate for 1 second and roll an IQ + Magic (The spellcaster's); success allows them to distinguish things about what they detected; depending on Margin of Success the user can learn things such as the species or type of creature (e.g. human versus guard dog), metatype, quality of health, and so on. Criticall success can give detailed information, such as sex, sub-metatype, indentity, and so on. Notably, a simple success will provide the ability to filter through things detected, allowing one to sift through the hundreds of life forms this spell with always detect. Where it would be beneficial, the GM can assign bonuses for knowing such things to skills, ranging from +1 and above.
	
	Higher Force spells can detect further and analyze more consistently, gaining a +1 to all rolls for each Force above 1.	
	
	\begin{center}
		\begin{tabular}{|c|c|}
			\hline
			Magic & Base Cost \\
			\hline
			\hline
			Magic 1 & 76 \\
			Magic 2 & 79 \\
			Magic 3 & 82 \\
			Magic 4 & 84 \\
			Magic 5 & 86 \\
			Magic 6 & 87 \\
			Magic 7 & 88 \\
			Magic 8 & 89 \\
			\hline
		\end{tabular}
	\end{center} 	
	
	\textcolor{OliveGreen}{\textit{ Statistics: Affliction (Detect Life, +495\%; Detection Spell, +145\%) further levels add +15\% [1.5] to Detect Life}}
	
	\textcolor{OliveGreen}{\textit{Detect Life, +495\%: Detect, Life (Very Common; Precise, Nontargeting, +90\%; Magical, -10\%; Resistable vs Will, -15\%) [49.5] further levels give Reliable +5\% [1.5]}}
	
	\paragraph{Detect Life, Extended}
	
	This spell works exactly like Detect lIFE above, except that it has superior range, allowing to to both pick up more targets - alongside more noise and junk!
	
	When rolling Per to detect, reduce all range penalties by -6 (effectively $\times$10 range), down to a minimum of 0.
	
	\begin{center}
		\begin{tabular}{|c|c|}
			\hline
			Magic & Base Cost \\
			\hline
			\hline
			Magic 1 & 84 \\
			Magic 2 & 87 \\
			Magic 3 & 89 \\
			Magic 4 & 92 \\
			Magic 5 & 93 \\
			Magic 6 & 95 \\
			Magic 7 & 96 \\
			Magic 8 & 96 \\
			\hline
		\end{tabular}
	\end{center} 
	
	\textcolor{OliveGreen}{\textit{ Statistics: Affliction (Detect Life, +573\%; Detection Spell, +145\%) further levels add +15\% [1.5] to Detect Life}}
	
	\textcolor{OliveGreen}{\textit{Detect Life, +573\%: Detect, Life (Very Common; Precise, Nontargeting, +90\%; Reliable 6 (Accessibility, Range Penalties Only\footnote{Firstly, this preclude Analyzing, which is around 40\% value}, -15\%), +26\%; Magical, -10\%; Resistable vs Will, -15\%) [19.1] further levels give Reliable +5\% [57.3]}}
		
	\paragraph{Detect Magic}
	
	The Awakened casts a spell that grants its target the ability to detect a magic sources in their close vicinity.
	
	To do so, they must touch their target and Concentrate for 1 second before rolling a Spellcasting + Magic roll; success grants their target the ability to detect any magical source, such as spells, foci, lodges, spirits, adept powers, and so on.
	
	To detect the individual, the user must Concentrate for 1 second and win a Quick Contest of Perception + Magic (The spellcaster's) versus Will, with range penalties (B550) for distance; successfully detecting them will alert the user to the direction and distance (in essence pinpointing them) to the nearest significant source alongside a general idea of how many were detected nearby. Notably, this is not precise enough to allow for targeting of any abilities through the spell, however it does make it easier to target using your other senses.
	
	As a follow-up, the user can Concentrate for 1 second and roll an IQ + Magic (The spellcaster's); success allows them to distinguish things about what they detected; depending on Margin of Success the user can learn things such as the class of magic, type of source, some effects of the source or possibly even the specific effect, etc. Critical Success allows for very accurate information, such as the exact effect, specific type, tradition information, etc. Where it would be beneficial, the GM can assign bonuses for knowing such things about them to skills, ranging from +1 and above.
	
	Higher Force spells can detect further and analyze more consistently, gaining a +1 to all rolls for each Force above 1.	
	
	\begin{center}
		\begin{tabular}{|c|c|}
			\hline
			Magic & Base Cost \\
			\hline
			\hline
			Magic 1 & 42 \\
			Magic 2 & 43 \\
			Magic 3 & 44 \\
			Magic 4 & 45 \\
			Magic 5 & 45 \\
			Magic 6 & 46 \\
			Magic 7 & 46 \\
			Magic 8 & 46 \\
			\hline
		\end{tabular}
	\end{center} 
	
	\textcolor{OliveGreen}{\textit{ Statistics: Affliction (Detect Magic, +165\%; Detection Spell, +145\%) further levels add +5\% [0.5] to Detect Magic}}
	
	\textcolor{OliveGreen}{\textit{Detect Magic, +165\%: Detect, Magic (Occasional; Precise, Nontargeting, +90\%; Magical, -10\%; Resistable vs Will, -15\%) [16.5] further levels give Reliable +5\% [0.5]}}
	
	\paragraph{Detect Magic, Extended}
	
	This spell works exactly like Detect Magic above, except that it has superior range, allowing to to both pick up more targets - alongside more noise, at least as is possible with this spell!
	
	When rolling Per to detect, reduce all range penalties by -6 (effectively $\times$10 range), down to a minimum of 0.
	
	\begin{center}
		\begin{tabular}{|c|c|}
			\hline
			Magic & Base Cost \\
			\hline
			\hline
			Magic 1 & 45 \\
			Magic 2 & 46 \\
			Magic 3 & 46 \\
			Magic 4 & 47 \\
			Magic 5 & 48 \\
			Magic 6 & 48 \\
			Magic 7 & 49 \\
			Magic 8 & 49 \\
			\hline
		\end{tabular}
	\end{center} 
	
	\textcolor{OliveGreen}{\textit{ Statistics: Affliction (Detect Magic, +191\%; Detection Spell, +145\%) further levels add +5\% [0.5] to Detect Magic}}
	
	\textcolor{OliveGreen}{\textit{Detect Magic, +191\%: Detect, Magic (Occasional; Precise, Nontargeting, +90\%; Reliable 6 (Accessibility, Range Penalties Only\footnote{Firstly, this preclude Analyzing, which is around 40\% value}, -15\%), +26\%; Magical, -10\%; Resistable vs Will, -15\%) [19.1] further levels give Reliable +5\% [0.5]}}
		
	\paragraph{Mind Probe}
	
	The Awakened is able to delve into the mind of a target and question them for truthful answers - sometimes in ways even they do not know!
	
	To do so, the Awakened must Concentrate for 1 second and roll a Quick Contest of Spellcasting + Magic versus will; winning this allows the Awakened to break into the target's ego, making further questioning easier alongside allowing them to ask \textit{one} question that can be answered with a brief sentence and have it answered truthfully. If the Awakened fails, they can cast the spell again, but with a cumulative -2 penalty until 1 hour has passed. Critical failure leaves them immune to this spell for 24 hours.
	
	This answer is what the subject \textit{believes} to be true - if he doesn't know he will tell you; this can sometimes belie subconscious differences or beliefs as well.
	
	Once inside the target's ego, extracting questions becomes much easier; roll against an uncontested Spellcasting + Magic for further question (which still take 1 second of continuous Concentration), with success providing answers as detailed above. Failure provides a cumulative -2 penalty to ask the same (or very similar) question again until 1 hour has passed. Critical failure leaves them immune to this spell for 24 hours.
 	
 	Higher Forces make this even more powerful, adding +1 to all rolls for each Force above 1.
 	
	\begin{center}
		\begin{tabular}{|c|c|}
			\hline
			Magic & Base Cost \\
			\hline
			\hline
			Magic 1 & 45 \\
			Magic 2 & 46 \\
			Magic 3 & 46 \\
			Magic 4 & 47 \\
			Magic 5 & 48 \\
			Magic 6 & 48 \\
			Magic 7 & 49 \\
			Magic 8 & 49 \\
			\hline
		\end{tabular}
	\end{center} 
	
	\textcolor{OliveGreen}{\textit{ Statistics: Mind Probe (Invasive, +75\%; Variable, +5\%; Magical, -10\%; Requires (Spellcasting) Roll\footnote{Difference of Requires (IQ) Roll and Requires (10) Roll}, -10\%) higher Forces provide Reliable, +5\% [1] }}
		
	\subsubsection{Health Spells}
	
	Health Spells specialize in bodily control, buffing, and repair.
	
	Many Health spells provide many of the same enhancements which are captured below:
	
	\textcolor{OliveGreen}{\textit{Health Spell, +139\%: (Cancellable, +10\%; Extended Duration, \(\times\)500, +110\%\footnote{Gives around 8 hours per Margin of Success, which allows for 1 day timeframes.}; Malediction 1, +100\%; Variable, +5\%; Fixed Duration, +0\%; Hard to Use 2 (Accessibility, Low Essence Only, -40\%), -6\%; Magical, -10\%; Melee Attack, C, No Parry, -35\%; Requires Spellcasting Roll\footnote{Since Malediction already has a Will roll, this just switches it to a flat 10 roll + Requires Spellcasting, -20\%, priced for difference between them.}, -15\%}; Terminal Condition, Loses Concentration, -20\footnote{Taken from \textcolor{Blue}{\href{http://forums.sjgames.com/showpost.php?p=817197&postcount=7}{Kromm's Suggested value here for losing concentration}}, since Requires Concentration is not allowed on Afflictions.}\%)}	
	
	\paragraph{Antidote}
	
	The Awakened grants the target resistances to toxins and poisons, generally included any non-living, non-virus, metabolically hazardous chemical.
	
	To do so, the Awakened must Concentrate for 1 second and roll versus Spellcasting + Magic; success grants their target +3 to all HT rolls to resist Poisons and Toxins. Additionally, a margin of success of 5 or more increases that to +8.
	
	Individuals with lots of 'ware are more difficult to affect with this spell. The GM should assign a penalty up to -6 for individuals with impacted essence. A good heuristic is a -1 for every 10 points in 'ware caused disadvantages.
	
	Higher Forces are more reliable and improve the results; add +1 to the Spellcasting roll for each Force above 1.
	
	\begin{center}
		\begin{tabular}{|c|c|}
			\hline
			Magic & Base Cost \\
			\hline
			\hline
			Magic 1 & 29 \\
			Magic 2 & 30 \\
			Magic 3 & 31 \\
			Magic 4 & 32 \\
			Magic 5 & 32 \\
			Magic 6 & 33 \\
			Magic 7 & 33 \\
			Magic 8 & 33 \\
			\hline
		\end{tabular}
	\end{center}
	
	\textcolor{OliveGreen}{\textit{Statistics: Affliction (Antidote +3, +30\%; Antidote +8, Side Effect, +10\%; Health Spell, +139\%) [28.5] further levls add Reliable. }}
	
	\textcolor{OliveGreen}{\textit{Antidote is Resistance, Poisons +3/+8(Magical, -10\%) [3/5] }}
	
	\paragraph{Cure Disease}
	
	% TODO: This
	
	\paragraph{Decrease (Attribute))}
	
	The Awakened is able to decrease the attributes of a target, enfeebling them.
	
	To do so, roll a Quick Contest of Spellcasting + Magic versus the HT of the target; winning lowers the attribute of the target by 1 per margin of victory.	
	
	Individuals with lots of 'ware are more difficult to affect with this spell. The GM should assign a penalty up to -6 for individuals with impacted essence. A good heuristic is a -1 for every 10 points in 'ware caused disadvantages.
	
	Secondary or derived characteristics are largely unaffected, but all skill rolls are lowered. The exception is ST, which lowers Basic Lift and ST-based damage.
	
	Higher Forces make the spell even more powerful, adding a +1 bonus to the Spellcasting roll for each Force above 1.
	
	This power has different variations depending on the cost of the Attribute, each of which are listed below:
	
	\subparagraph{Decrease (DX)}
	
	\begin{center}
		\begin{tabular}{|c|c|}
			\hline
			Magic & Base Cost \\
			\hline
			\hline
			Magic 1 & 28 \\
			Magic 2 & 29 \\
			Magic 3 & 30 \\
			Magic 4 & 31 \\
			Magic 5 & 31 \\
			Magic 6 & 32 \\
			Magic 7 & 32 \\
			Magic 8 & 32 \\
			\hline
		\end{tabular}
	\end{center}	
	
	\textcolor{OliveGreen}{\textit{Statistics: Affliction (Reduced DX, Margin-Based, +30\footnote{Priced at 1/2 cost, as the original seems to be.}\%; Health Spell, +139\%) }}
	
	\subparagraph{Decrease (IQ or HT)}
	
	\begin{center}
		\begin{tabular}{|c|c|}
			\hline
			Magic & Base Cost \\
			\hline
			\hline
			Magic 1 & 27 \\
			Magic 2 & 28 \\
			Magic 3 & 29  \\
			Magic 4 & 29 \\
			Magic 5 & 30 \\
			Magic 6 & 31 \\
			Magic 7 & 31 \\
			Magic 8 & 31 \\
			\hline
		\end{tabular}
	\end{center}	
	
	\textcolor{OliveGreen}{\textit{Statistics: Affliction (Reduced IQ or HT, Margin-Based, +19.5\%; Health Spell, +139\%) }}
	
	\subparagraph{Decrease (Will)}
	
		\begin{center}
		\begin{tabular}{|c|c|}
			\hline
			Magic & Base Cost \\
			\hline
			\hline
			Magic 1 & 26 \\
			Magic 2 & 27 \\
			Magic 3 & 28 \\
			Magic 4 & 29 \\
			Magic 5 & 29 \\
			Magic 6 & 30 \\
			Magic 7 & 30 \\
			Magic 8 & 30 \\
			\hline
		\end{tabular}
	\end{center}	
	
	\textcolor{OliveGreen}{\textit{Statistics: Affliction (Reduced Will, Margin-Based, +10.5\%; Health Spell, +139\%) }}
		
	\subparagraph{Decrease (ST)}
	
	\begin{center}
		\begin{tabular}{|c|c|}
			\hline
			Magic & Base Cost \\
			\hline
			\hline
			Magic 1 & 26 \\
			Magic 2 & 27 \\
			Magic 3 & 28 \\
			Magic 4 & 28 \\
			Magic 5 & 29 \\
			Magic 6 & 29 \\
			Magic 7 & 30 \\
			Magic 8 & 30 \\
			\hline
		\end{tabular}
	\end{center}	
	
	\textcolor{OliveGreen}{\textit{Statistics: Affliction (Reduced ST, Margin-Based, +9\%; Health Spell, +139\%) }}	
	
	\subparagraph{Decrease (Per)}
	
	\begin{center}
		\begin{tabular}{|c|c|}
			\hline
			Magic & Base Cost \\
			\hline
			\hline
			Magic 1 & 26 \\
			Magic 2 & 27 \\
			Magic 3 & 27 \\
			Magic 4 & 28 \\
			Magic 5 & 29 \\
			Magic 6 & 29 \\
			Magic 7 & 30 \\
			Magic 8 & 30 \\
			\hline
		\end{tabular}
	\end{center}	
	
	\textcolor{OliveGreen}{\textit{Statistics: Affliction (Reduced Per, Margin-Based, +7.5\%; Health Spell, +139\%) }}
		
	\paragraph{Increase (Attribute)}
	
	This extremely powerful spell allows the Awakened to increase the attributes of a target, empowering their abilities and derived traits.
	
	To do so, roll a Quick Contest of Spellcasting + Magic versus the HT of the target; winning raises the attribute of the target by 1 \textit{for every 2} margin of victory, rounded up. Willing targets cannot waive this roll, as more resilient bodies are more difficult to affect.
	
	Individuals with lots of 'ware are more difficult to affect with this spell. The GM should assign a penalty up to -6 for individuals with impacted essence. A good heuristic is a -1 for every 10 points in 'ware caused disadvantages.
	
	Unlike when decreasing attribute, secondary and derived stats \textit{are} affected here; DX improves Basic Speed, HT gives FP, ST gives HP, etc. All of these are lost when the spell ends, lowering them by the same amount, which can do things like causes unconsciousness or death checks!
	
	Higher Forces make the spell even more powerful, adding a +1 bonus to the Spellcasting roll for each Force above 1.
	
	This power has different variations depending on the cost of the Attribute, each of which are listed below:
		
	\subparagraph{Increase (DX)}
	
	\begin{center}
		\begin{tabular}{|c|c|}
			\hline
			Magic & Base Cost \\
			\hline
			\hline
			Magic 1 & 55 \\
			Magic 2 & 56 \\
			Magic 3 & 57 \\
			Magic 4 & 58 \\
			Magic 5 & 58 \\
			Magic 6 & 59 \\
			Magic 7 & 59 \\
			Magic 8 & 59 \\
			\hline
		\end{tabular}
	\end{center}	
	
	\textcolor{OliveGreen}{\textit{Statistics: Affliction (Increased DX, Margin-Based, +600\footnote{Priced at 1/2 cost, as the original seems to be.}\%; Health Spell, +139\%) }}
	
	\subparagraph{Increase (IQ or HT)}
	
	\begin{center}
		\begin{tabular}{|c|c|}
			\hline
			Magic & Base Cost \\
			\hline
			\hline
			Magic 1 & 44 \\
			Magic 2 & 45 \\
			Magic 3 & 46  \\
			Magic 4 & 47 \\
			Magic 5 & 48 \\
			Magic 6 & 48 \\
			Magic 7 & 48 \\
			Magic 8 & 49 \\
			\hline
		\end{tabular}
	\end{center}	
	
	\textcolor{OliveGreen}{\textit{Statistics: Affliction (Increased IQ or HT, Margin-Based, +390\%; Health Spell, +139\%) }}
	
	\subparagraph{Increase (Will)}
	
	\begin{center}
		\begin{tabular}{|c|c|}
			\hline
			Magic & Base Cost \\
			\hline
			\hline
			Magic 1 & 35 \\
			Magic 2 & 36 \\
			Magic 3 & 37 \\
			Magic 4 & 38 \\
			Magic 5 & 39 \\
			Magic 6 & 39 \\
			Magic 7 & 39 \\
			Magic 8 & 40 \\
			\hline
		\end{tabular}
	\end{center}	
	
	\textcolor{OliveGreen}{\textit{Statistics: Affliction (Increased Will, Margin-Based, +210\%; Health Spell, +139\%) }}
	
	\subparagraph{Increase (ST)}
	
	\begin{center}
		\begin{tabular}{|c|c|}
			\hline
			Magic & Base Cost \\
			\hline
			\hline
			Magic 1 & 34 \\
			Magic 2 & 35 \\
			Magic 3 & 36 \\
			Magic 4 & 37 \\
			Magic 5 & 37 \\
			Magic 6 & 38 \\
			Magic 7 & 38 \\
			Magic 8 & 38 \\
			\hline
		\end{tabular}
	\end{center}	
	
	\textcolor{OliveGreen}{\textit{Statistics: Affliction (Increased ST, Margin-Based, +180\%; Health Spell, +139\%) }}
	
	
	\subparagraph{Increase (Per)}
	
	\begin{center}
		\begin{tabular}{|c|c|}
			\hline
			Magic & Base Cost \\
			\hline
			\hline
			Magic 1 & 32 \\
			Magic 2 & 33 \\
			Magic 3 & 34 \\
			Magic 4 & 35 \\
			Magic 5 & 36 \\
			Magic 6 & 36 \\
			Magic 7 & 36 \\
			Magic 8 & 37 \\
			\hline
		\end{tabular}
	\end{center}	
	
	\textcolor{OliveGreen}{\textit{Statistics: Affliction (Increased Per, Margin-Based, +150\%; Health Spell, +139\%) }}
	
	\paragraph{Increase Reflexes}
	
	This spell greatly enhances the target's reflexes, allowing them to react and dodge far better than before.
	
	To use it, roll a Quick Contest of Spellcasting + Magic versus the HT of the target; winning raises the target's Basic Speed by +1.0 for every \textit{for every 2} margin of victory, rounded up. Willing targets cannot waive this roll, as more resilient bodies are more difficult to affect.
	
	Individuals with lots of 'ware are more difficult to affect with this spell. The GM should assign a penalty up to -6 for individuals with impacted essence. A good heuristic is a -1 for every 10 points in 'ware caused disadvantages.
	
	This does not improve the target's Basic move at all, they simply react faster, improving their place in the turn order and their Dodge.
	
	Higher Forces make the spell even more powerful, adding a +1 bonus to the Spellcasting roll for each Force above 1.	
	
	\begin{center}
		\begin{tabular}{|c|c|}
			\hline
			Magic & Base Cost \\
			\hline
			\hline
			Magic 1 & 47 \\
			Magic 2 & 48 \\
			Magic 3 & 49  \\
			Magic 4 & 50 \\
			Magic 5 & 51 \\
			Magic 6 & 51 \\
			Magic 7 & 51 \\
			Magic 8 & 52 \\
			\hline
		\end{tabular}
	\end{center}
	
	\textcolor{OliveGreen}{\textit{Statistics: Affliction (+1.0 Basic Speed (No Basic Move, -5), Margin-Based, +150\%; Health Spell, +139\%) }}
	
	\paragraph{Resist Pain}
	
	The Awakened grants the target increased pain resistance, allowing them to overcome shock and fatigue.
	
	To do so, the Awakened must Concentrate for 1 second and roll versus Spellcasting + Magic; success grants their target High Pain Threshold. This provides a +3 bonus to HT rolls to avoid Knockdown and Stunning alongside to resist physical torture. Additionally, the target does not take Shock penalties, halves the penalties for any pain conditions, and can claim a +3 bonus to any Will roll to ignore pain in other situations.
	
	Individuals with lots of 'ware are more difficult to affect with this spell. The GM should assign a penalty up to -6 for individuals with impacted essence. A good heuristic is a -1 for every 10 points in 'ware caused disadvantages.
	
	Higher Forces are more reliable and improve the results; add +1 to the Spellcasting roll for each Force above 1.
	
	\begin{center}
		\begin{tabular}{|c|c|}
			\hline
			Magic & Base Cost \\
			\hline
			\hline
			Magic 1 & 35 \\
			Magic 2 & 36 \\
			Magic 3 & 37  \\
			Magic 4 & 38 \\
			Magic 5 & 38 \\
			Magic 6 & 39 \\
			Magic 7 & 39 \\
			Magic 8 & 39 \\
			\hline
		\end{tabular}
	\end{center}
		
	\textcolor{OliveGreen}{\textit{Statistics: Affliction (High Pain Threshold, +100\%; Health Spell, +139\%) [28.5] further levls add Reliable. }}
	
	\paragraph{Heal}
	
	This spell allows the caster to heal the physical injuries of individuals. It works somewhat differently than many spells.
	
	To use, concentrate for 1 second and roll a Spellcasting+Magic roll. Success lets you heal 2 HP for every Force of the spell, at the cost of having to resist FP Drain for every 2 HP healed. Even 1 HP of healing stops bleeding. If the Force is higher than you magic, this Drain is HP instead. Failure causes you to lose 1d FP or HP unresisted immediately, instead of normal Drain. Critical Failure \textit{causes} the subject to lose 1d HP.
	
	You can heal a crippled, but whole limb by making your Spellcasting at a -6 and spending an additional 2 FP or HP as necessary. If you completely heal the HP lost to the crippling injury, the limb will no longer be crippled. The magician only gets one attempt per crippled injury.
	
	If this spell is used on an individual multiple times, it has a cumulative -3 for each \textit{successful} use. This penalty lasts for 24 hours after the \textit{last successful attempt.} This means that multiple small healings can be extremely difficult, even if they do have easily managed drain!
	
	You can heal any creature that is carbon-based (No healing robots unfortunately!), however individuals with lots of 'ware are more difficult. The GM should assign a penalty up to -6 for individuals with impacted essence. A good heuristic is a -1 for every 10 points in 'ware caused disadvantages.
	
	Higher Force spells provide a +1 to all rolls for every Force above 1.
	
	\textcolor{NavyBlue}{\textit{Modifiers: -2 if you subject is unconscious, -3 per successful healing (Lasting for 24 hours since the latest healing), -6 to repair a crippled limb}}
		 
	\begin{center}
		\begin{tabular}{|c|c|}
			\hline
			Magic & Base Cost \\
			\hline
			\hline
			Magic 1 & 45 \\
			Magic 2 & 51 \\
			Magic 3 & 57 \\
			Magic 4 & 61 \\
			Magic 5 & 65 \\
			Magic 6 & 67 \\
			Magic 7 & 69 \\
			Magic 8 & 72 \\
			\hline
		\end{tabular}
	\end{center}
	
	\textcolor{OliveGreen}{\textit{Statistics: Healing (Affects Self, +50\%; Xenohealing, Carbon-Based Life, +60\%; Resistable Drain\footnote{Guesstimate at +5\% per Force.}; Capped\footnote{It is capped at 1 FP per Force, which is calculated as -30+2.5\% per FP cap.}; Hard to Use 2 (Accessibility, Low Essence Only, -40\%), -6\%; Injuries Only, -20\%; Injurious Magic (Accessibility, Force over Magic, -20\%)\footnote{This is based of Thaumatology p25's Injurious Magic. This makes FP Cost HAVE to be HP, limited only to FP over Magic.}, -24\%; Magical, -10\%; Requires Spellcasting Roll, -10\%\footnote{Priced as difference between Requires IQ Roll and Requires (10) Roll}) higher force adds Reliable +5\%}}
	
	\subsubsection{Illusion Spells}
	
	Illusions spells create fake stimuli that are used to deceive, debiliate, and control others - ranging from visual hallucinations to phantom pains, they are very powerful when used with a creative mind.
	
	Some illusions are afflictions, applying status to others, and often share these modifiers:
	
	\textcolor{OliveGreen}{\textit{Ranged Illusion Affliction, +175\%: (Cancellable, +10\%; Malediction 3, +200\%; Variable, +5\%; Magical, -10\%; Requires Spellcasting Roll, -15\%; Sense-Based, Reversed, Vision and Touch, -15\%) }}
	
	\textcolor{OliveGreen}{\textit{Touch Illusion Spell, +145\%: (Cancellable, +10\%; Extended Duration, \(\times\)500, +110\%\footnote{Gives around 8 hours per Margin of Success, which allows for 1 day timeframes.}; Malediction 1, +100\%; Variable, +5\%; Fixed Duration, +0\%; Magical, -10\%; Melee Attack, C, No Parry, -35\%; Requires Spellcasting Roll\footnote{Since Malediction already has a Will roll, this just switches it to a flat 10 roll + Requires Spellcasting, -20\%, priced for difference between them.}, -15\%}; Terminal Condition, Loses Concentration, -20\footnote{Taken from \textcolor{Blue}{\href{http://forums.sjgames.com/showpost.php?p=817197&postcount=7}{Kromm's Suggested value here for losing concentration}}, since Requires Concentration is not allowed on Afflictions.}\%)}
	
	\paragraph{Agony}
	
	The Awakened causes terrible illusory pains in their target. They can target an individual they can see or touch, with no penalties for range.
	
	Concentrate for 1 second and roll a Quick Contest of Spellcasting + Magic versus Will. Winning causes the target to feel great pain, providing a -1 penalty per Margin of Success to all DX, IQ, and self control rolls, double for those with Low Pain Tolerance and halved for those with High Pain Threshold. This lasts for a number of minute equal to their Margin of Success, or until the Awakened loses or drops concentration.
	
	Higher Forces make the pain even more terrible, providng a +1 bonus to the Spellcasting roll for each Force above 1.
	
	\begin{center}
		\begin{tabular}{|c|c|}
			\hline
			Magic & Base Cost \\
			\hline
			\hline
			Magic 1 & 31 \\
			Magic 2 & 32 \\
			Magic 3 & 33 \\
			Magic 4 & 34 \\
			Magic 5 & 35 \\
			Magic 6 & 35 \\
			Magic 7 & 35 \\
			Magic 8 & 36 \\
			\hline
		\end{tabular}
	\end{center}

	\textcolor{OliveGreen}{\textit{Statistics: Affliction (Based on Will, +20\%; Ranged Illusion Affliction, +175\%; Margin-Based, 1/2 Moderate Pain\footnote{Since Pain easily scales by +20\% per -2 penalty, this is simply 1/2 that for Margin-Based, giving -10\% per -1.}, +30\%; Terminal Condition, Loses Concentration, -20\%)  further levels add Reliable +5\%}}
	
	\paragraph{Mass Agony}
	
	This spell works exactly as Agony above, while also affecting a large area as opposed to a single target.
	
	When casting this spell, it affects all targets within a number of Yards of a point of the caster's choice equal to the Force of the spell. The power of the spell dissipates as it goes, making it easier to resist; add a +1 bonus to their resistance roll for every yard of distance a target is from the center.
	
		\begin{center}
		\begin{tabular}{|c|c|}
			\hline
			Magic & Base Cost \\
			\hline
			\hline
			Magic 1 & 31 \\
			Magic 2 & 37 \\
			Magic 3 & 41 \\
			Magic 4 & 43 \\
			Magic 5 & 45 \\
			Magic 6 & 46 \\
			Magic 7 & 46 \\
			Magic 8 & 47 \\
			\hline
		\end{tabular}
	\end{center}
	
	\textcolor{OliveGreen}{\textit{Statistics: Affliction ((AoE, 1 yard, +25\% Based on Will, +20\%; Ranged Illusion Affliction, +175\%; Margin-Based, 1/2 Moderate Pain\footnote{Since Pain easily scales by +20\% per -2 penalty, this is simply 1/2 that for Margin-Based, giving -10\% per -1.}, +30\%; Dissipation, -50\%; Terminal Condition, Loses Concentration, -20\%)  further levels add Reliable +5\% and a Linear AoE}}
	
	\paragraph{Bugs}
	
	The Awakened causes their target to feel terrible sensations of creatures and insects crawling over and inside the target, heavily distracting their mind and lowering their abilities to react to threats.
	
	Concentrate for 1 second and roll a Quick Contest of Spellcasting + Magic versus Will. Winning distracts the target, lowering their Basic Speed by 0.5 per Margin of Success. This lasts for a number of minute equal to their Margin of Success, or until the Awakened loses or drops concentration.
	
	Higher Forces make the sensation even more terrible, providng a +1 bonus to the Spellcasting roll for each Force above 1.
		
	\begin{center}
		\begin{tabular}{|c|c|}
			\hline
			Magic & Base Cost \\
			\hline
			\hline
			Magic 1 & 31 \\
			Magic 2 & 32 \\
			Magic 3 & 33 \\
			Magic 4 & 34 \\
			Magic 5 & 35 \\
			Magic 6 & 35 \\
			Magic 7 & 35 \\
			Magic 8 & 36 \\
			\hline
		\end{tabular}
	\end{center}
	
	\textcolor{OliveGreen}{\textit{Statistics: Affliction (Based on Will, +20\%; Ranged Illusion Affliction, +175\%; Margin-Based, -0.50 Basic Speed, +30\%; Terminal Condition, Loses Concentration, -20\%)  further levels add Reliable +5\%}}
	
	\paragraph{Swarm}
	
	This spell works exactly as Bugs above, while also affecting a large area as opposed to a single target.
	
	When casting this spell, it affects all targets within a number of Yards of a point of the caster's choice equal to the Force of the spell. The power of the spell dissipates as it goes, making it easier to resist; add a +1 bonus to their resistance roll for every yard of distance a target is from the center.
	
	\begin{center}
		\begin{tabular}{|c|c|}
			\hline
			Magic & Base Cost \\
			\hline
			\hline
			Magic 1 & 31 \\
			Magic 2 & 37 \\
			Magic 3 & 41 \\
			Magic 4 & 43 \\
			Magic 5 & 45 \\
			Magic 6 & 46 \\
			Magic 7 & 46 \\
			Magic 8 & 47 \\
			\hline
		\end{tabular}
	\end{center}
	
	\textcolor{OliveGreen}{\textit{Statistics: Affliction ((AoE, 1 yard, +25\% Based on Will, +20\%; Ranged Illusion Affliction, +175\%; Margin-Based, -0.5 Basic Speed, +30\%; Dissipation, -50\%; Terminal Condition, Loses Concentration, -20\%)  further levels add Reliable +5\% and a Linear AoE}}
	
	\paragraph{Confusion}
	
	The Awakened causes confusion in their target, making them uncertain and slow in their actions for most situations.
	
	Concentrate for 1 second and roll a Quick Contest of Spellcasting + Magic versus Will. Winning gives the target the Confusion disadvantage (B129) with a Self Control of 15. Each further Margin of Success lowers the Self Control by one category (e.g. MoS 3 is 15$\rightarrow$9), to a minimum of Automatica Failure. Individuals with Confusion are prone freezing up under strange places, during commotions, or whenever there's a need for deicisive action, which often preculdes any Tactics rolls or similar strategizing. The GM should also modify the Self Control roll by the situations, as described in the Basic Set. If the target fails their Self Control Roll, they freeze up, taking to Do Nothing maneuver or at least taking little of any effect. This does not prevent them from reacting, such as with Active Defenses, but they cannot act.
	
	Higher Forces make the sensation even more terrible, providng a +1 bonus to the Spellcasting roll for each Force above 1.
	
	\begin{center}
		\begin{tabular}{|c|c|}
			\hline
			Magic & Base Cost \\
			\hline
			\hline
			Magic 1 & 25 \\
			Magic 2 & 26 \\
			Magic 3 & 27 \\
			Magic 4 & 28 \\
			Magic 5 & 28 \\
			Magic 6 & 29 \\
			Magic 7 & 29 \\
			Magic 8 & 29 \\
			\hline
		\end{tabular}
	\end{center}

	\textcolor{OliveGreen}{\textit{Statistics: Affliction (Based on Will, +20\%; Ranged Illusion Affliction, +175\%; Margin-Based, Confused, +15\%; Terminal Condition, Loses Concentration, -20\%)  further levels add Reliable +5\%}}
	
	\paragraph{Mass Confusion}
	
	This spell works exactly as Confusion above, while also affecting a large area as opposed to a single target.
	
	When casting this spell, it affects all targets within a number of Yards of a point of the caster's choice equal to the Force of the spell. The power of the spell dissipates as it goes, making it easier to resist; add a +1 bonus to their resistance roll for every yard of distance a target is from the center.
	
	\begin{center}
		\begin{tabular}{|c|c|}
			\hline
			Magic & Base Cost \\
			\hline
			\hline
			Magic 1 & 30 \\
			Magic 2 & 36 \\
			Magic 3 & 39 \\
			Magic 4 & 42 \\
			Magic 5 & 43 \\
			Magic 6 & 44 \\
			Magic 7 & 45 \\
			Magic 8 & 46 \\
			\hline
		\end{tabular}
	\end{center}
	
	\textcolor{OliveGreen}{\textit{Statistics: Affliction ((AoE, 1 yard, +25\% Based on Will, +20\%; Ranged Illusion Affliction, +175\%; Margin-Based, Confusion, +15\%; Dissipation, -50\%; Terminal Condition, Loses Concentration, -20\%)  further levels add Reliable +5\% and a Linear AoE}}
	
	\paragraph{Chaos}
	
	This spell works exactly as Confusion above, but also affects Machines. Machines Will scores are usually equal to their Complexity $\times$ 2.
	
	Machines affected by Confused often are unable to perform their standard operations outside anything but the most mundane conditions. This can cause glitches or failure, not responding to inputs, nor responding.
	
	\begin{center}
		\begin{tabular}{|c|c|}
			\hline
			Magic & Base Cost \\
			\hline
			\hline
			Magic 1 & 30 \\
			Magic 2 & 31 \\
			Magic 3 & 32 \\
			Magic 4 & 33 \\
			Magic 5 & 33 \\
			Magic 6 & 34 \\
			Magic 7 & 34 \\
			Magic 8 & 34 \\
			\hline
		\end{tabular}
	\end{center}	
	
	\textcolor{OliveGreen}{\textit{Statistics: Affliction (Based on Will, +20\%; Cosmic, Affects Machines, +50\%; Ranged Illusion Affliction, +175\%; Margin-Based, Confused, +15\%; Terminal Condition, Loses Concentration, -20\%)  further levels add Reliable +5\%}}
	
	\paragraph{Chaotic World}
	
	This spell works exactly as Chaos above, while also affecting a large area as opposed to a single target.
	
	When casting this spell, it affects all targets within a number of Yards of a point of the caster's choice equal to the Force of the spell. The power of the spell dissipates as it goes, making it easier to resist; add a +1 bonus to their resistance roll for every yard of distance a target is from the center.
	
	\begin{center}
		\begin{tabular}{|c|c|}
			\hline
			Magic & Base Cost \\
			\hline
			\hline
			Magic 1 & 35 \\
			Magic 2 & 41 \\
			Magic 3 & 43 \\
			Magic 4 & 47 \\
			Magic 5 & 48 \\
			Magic 6 & 49 \\
			Magic 7 & 50 \\
			Magic 8 & 51 \\
			\hline
		\end{tabular}
	\end{center}
	
	\textcolor{OliveGreen}{\textit{Statistics: Affliction ((AoE, 1 yard, +25\% Based on Will, +20\%; Cosmic, Affects Machines, +50\%; Ranged Illusion Affliction, +175\%; Margin-Based, Confusion, +15\%; Dissipation, -50\%; Terminal Condition, Loses Concentration, -20\%)  further levels add Reliable +5\% and a Linear AoE}}
		
	\paragraph{Invisibility}
	
	The caster alters the perceptions of living beings near them, making themselves undetectable to them.
	
	To use it, the Awakened touches his target and Concentrates for 1 second and rolls Spellcasting + Magic\footnote{This roll is used for initial success and the contested Glamour roll to simplify play.}. Their margin of success is used for a Quick Contest versus the Will+4 of anyone who might visually perceive them. Whenever they win this Quick Contest, the subject and objects they are carrying are treated as invisible to that individual, while losing leaves the individual unaffected.
	
	When invisible, all electromagnetic spectrum based sight for living things will fail to see them, however other methods, such as sound, touch, smell, sonar, etc. all still work as normal. Notably, any machines are immune to this spell and will still perceive the subject as normal. Being invsibile grants the subject a +9 bonus to stealth in any situation where being seen would matter. Spells or abilities that would grant sight of the subject cannot see them, although they may still look at the area they are in. Of course, Astral Perception and other sight on the Astral Plane are entirely unaffected by this spell, as they can simply see the aura of the subject anyways.
	
	There is a limit to how much the subject may carry while under these effects, with object over their Light Encumbrance becoming visible while carried.
	
	At higher Forces, it becomes increasingly hard to resist this spell. At every even Force, the Will roll to resist the spell is lowered by 1, to a minium of Will-5; and at every odd level, the Awakened gains a +1 bonus to their Spellcasting roll.
		
	\begin{center}
		\begin{tabular}{|c|c|}
			\hline
			Magic & Base Cost \\
			\hline
			\hline
			Magic 1 & 56 \\
			Magic 2 & 59 \\
			Magic 3 & 62 \\
			Magic 4 & 64 \\
			Magic 5 & 66 \\
			Magic 6 & 67 \\
			Magic 7 & 68 \\
			Magic 8 & 69 \\
			\hline
		\end{tabular}
	\end{center}
	
	\textcolor{OliveGreen}{\textit{Statistics: Affliction (Invisibility, +300\%; Touch Illusion Spell ,+145\%) [54.5]  further forces increase Invisibility's level. }}
	
	\textcolor{OliveGreen}{\textit{Statistics: Invisibility (Can Carry Objects, Light, +20\%; Glamour, Will+4, Quick Contest\footnote{Limitation taken from \textcolor{Blue}{\href{http://forums.sjgames.com/showpost.php?p=669736&postcount=2}{PK \& Kromm's ruling on Quick Contests for traits like Resistable}} (and ergo, Glamour)}, -25\%; Magical, -10\%; Substantial Only, -10\%) [30] further levels add -1 to Glamour [1] and +1 Reliable [2] alternating.}}
	
	\paragraph{Realistic Invisibility}
	
	This spell works exactly like Invisibilty above, with some notable exceptions:
	
	The Quick Contest is a Spellcasting + Magic versus Per+4, with higher forces working in the same manner.
	
	The spell affect machines as well as living beings.
	
	\begin{center}
		\begin{tabular}{|c|c|}
			\hline
			Magic & Base Cost \\
			\hline
			\hline
			Magic 1 & 61\\
			Magic 2 & 64 \\
			Magic 3 & 67 \\
			Magic 4 & 69 \\
			Magic 5 & 71 \\
			Magic 6 & 72 \\
			Magic 7 & 73 \\
			Magic 8 & 74 \\
			\hline
		\end{tabular}
	\end{center}	
	
	\textcolor{OliveGreen}{\textit{Statistics: Affliction (Invisibility, +350\%; Tocuh Illusion Spell ,+145\%) [59.5]  further forces increase Invisibility's level. }}
	
	\textcolor{OliveGreen}{\textit{Statistics: Realistic Invisibility (Affects Machines, +50\%; Can Carry Objects, Light, +20\%; Glamour, Will+4, Quick Contest\footnote{Limitation taken from \textcolor{Blue}{\href{http://forums.sjgames.com/showpost.php?p=669736&postcount=2}{PK \& Kromm's ruling on Quick Contests for traits like Resistable}} (and ergo, Glamour)}, -25\%; Magical, -10\%; Substantial Only, -10\%) [30] further levels add -1 to Glamour [1] and +1 Reliable [2] alternating.}}
	
	\paragraph{Phantasm}
	
	The magician creates a convincing illusion of any object, creature, or scene they desire at an area of within line of sight.
	
	To do so, they must Concentrate continuously and succeed on a Spellcasting + Magic roll. This creates a highly detailed and animated three-dimensional illusion that fits within a one yard radius of the chosen point.
	
	These illusions lack any actual mass or way to directly affect the material plane, however their true value is in deception and distraction. When an individual views your phantasm, you must make a Quick Contest of Spellcasting + Magic versus their Will; if you win, the illusion seems real to that person, with the GM determining how they might react.
	
	\textcolor{NavyBlue}{\textit{Modifiers: Your victim gets +4 if someone who knows about the illusion warns him, or if you critically fail in a Quick Contest against someone else. He gets +10 if you create the illusion unsubtly and in plain sight, or if	he examines the illusion with a sense you can’t deceive – most often touch, but this can also be things like Sonar, Radar, and Infrahearing or Ultrahearing. }}
			
	\textcolor{NavyBlue}{\textit{At the GM’s option, inappropriate illusions (e.g., a pack of rabid wolves in a submarine) give a further +1 to +10, while believable ones (e.g., you pull out an illusionary gun) give from -1 to -5. If the final modifier is a net bonus, halve it if the victim is aware of magic, but not its specifics, as long as the effects of your illusion could pass as a reasonable magical effect (e.g. mind controlling a rabid animal as opposed to an illusion of it showing up).}}
	
	Some uses of Phantasm require a separate skill roll; in particular: creating an illusion scary enough to cause a fright check requires an Artist (Illusions) + Magic roll versus the higher of IQ or Will. To create a phantasm of someone the target knows, roll against the \textit{lower} of Acting and Artist (Illusion) versus the \textit{higher} of your target's IQ and Will.
	
	Additionally, roll a new Quick Contest whenever someone you have already fooled changes how they interact with the phantasm, such as interacting with a new sense (often touch, through trying to attack or interact with it). If you win, you simulate a believable response to what he just witnessed.
	
	While it's easy to create mass illusions such as a horde or crowd, individual people that interact realistically can be difficult, requiring more effort to prevent robotic or jarring responses. For each individual past the first, add a +4 bonus to resist the illusion.
	
	Phantasm creates illusions throughout the Infrared to Ultraviolet spectrums, but it is easier for those who already have the ability to see in one or both ofthose spectrums to do so; when purchasing this spell, use the costs based on whether you have Both (Hyperspectral Vision), are Missing 1 (Infravision or Ultravision), or are Missing 2 (Normal Vision). 
	
	Additionally, the caster can create illusions in the hearing spectrum that they can hear, which may include Infrasonic or Ultrahearing for certain people; the spell has no ability to create illusions in those hearing spectrums by default, and does not cost anything different for those who lack it.
	
	Higher Forces of this spell allow you to create larger and more convincing Illusions. For each Force past the first, add a +1 to your Spellcasting roll and all rolls in Quick Contests. Additionally, the area the Illusions must fit inside is increased, becoming a radius equal to the Force of the spell.
	
	\begin{center}
		\begin{adjustwidth}{-2.5mm}{}
		\begin{tabular}{|c|c|c|c|}
			\hline
			Magic & Both & Missing 1 & Missing 2 \\
			\hline
			\hline
			Magic 1 & 35 & 37 & 41 \\
			Magic 2 & 48 & 50 & 54 \\
			Magic 3 & 55 & 58 & 61 \\
			Magic 4 & 60 & 62 & 66 \\
			Magic 5 & 63 & 66 & 69 \\
			Magic 6 & 65 & 68 & 71 \\
			Magic 7 & 66 & 69 & 72 \\
			Magic 8 & 67 & 70 & 73 \\
			\hline
		\end{tabular}
		\end{adjustwidth}
	\end{center}	
	
	\textcolor{OliveGreen}{\textit{Statistics: Illusion (Based on Will, +20\%; Ranged, LoS, +80\%; Variable, +5\%; Accessibility (Not on Machines), -30\%; Magical, -10\%; Reduced AoE\footnote{Simply lowers the base AoE radius down to 1 yard.}, -25\%; Requires Spellcasting Roll\footnote{Difference between IQ and Requires (10) Roll}, -10\%; Requires (Spellcasting) Roll, -20\%) further levels add Reliable, +5\% [1.25] and a Linear AoE. For those without IR or UV, add Extended, +10\%; for those withouth both add Extended, +25\%}}
	
	\paragraph{Trid Phantasm}
	
	This spell works exactly like Phantasm above, except that it also affects machines instead of just living beings. Additionally, all Will resistance rolls are replaced with Per rolls.
	
	\begin{center}
		\begin{adjustwidth}{-2.5mm}{}
		\begin{tabular}{|c|c|c|c|}
			\hline
			Magic & Both & Missing 1 & Missing 2 \\
			\hline
			\hline
			Magic 1 & 37 & 40 & 43 \\
			Magic 2 & 50 & 53 & 56 \\
			Magic 3 & 58 & 60 & 64 \\
			Magic 4 & 62 & 65 & 69 \\
			Magic 5 & 66 & 68 & 72 \\
			Magic 6 & 68 & 70 & 74 \\
			Magic 7 & 69 & 71 & 75 \\
			Magic 8 & 70 & 72 & 76 \\
			\hline
		\end{tabular}
		\end{adjustwidth}
	\end{center}
	
	\textcolor{OliveGreen}{\textit{Statistics: Illusion (Ranged, LoS, +80\%; Variable, +5\%; Magical, -10\%; Reduced AoE\footnote{Simply lowers the base AoE radius down to 1 yard.}, -25\%; Requires Spellcasting Roll\footnote{Difference between IQ and Requires (10) Roll}, -10\%; Requires (Spellcasting) Roll, -20\%) further levels add Reliable, +5\% [1.25] and a Linear AoE. For those without IR or UV, add Extended, +10\%; for those withouth both add Extended, +25\%}}
	
	\paragraph{Silence}
	
	The Awakened creates an area that deafens sound passing through it, making it difficult to hear anything going on inside or behind it.
	
	To do so, Concentrate for 1 second and roll a Spellcasting + Magic; success creates the area at any point within the Awakened's line of sight the Awakened loses their concentration.
	
	The area provides a -1 penalty per Margin of Success to all Hearing or sound-based tests - including tests using Infrahearing or Ultrahearing - to hear something that occurs within the area, or if that sound would pass through the area to reach those listening.
	
	This penalty applies to active sound-based effects as well, such as Sonic weapons or Sonar. Attempts to target them with sound-based weapons suffers the penalty to hit rolls within or travelling through the area (similarly to how penalties to Vision apply to Guns rolls). If the weapon is an Affliction, the penalty is instead applies a bonus to resist.
	
	Sonar and similar detections systems apply the penalty to any rolls to actively detect targets; additionally, successful detections might not even be recognizeable as a human, instead showing up as a much smaller or further target than it actually is (due to the sound waves being absorbed). 
	
	This can still ellicit suspicion, whether it be read as sensor malfunction, interference or jamming, or simple oddities - it's up the GM what they are interpereted as, but it's recommended for such oddities to show up for margins of victory or loss of $\pm$1 or ties.
	
	\begin{center}
		\begin{tabular}{|c|c|}
			\hline
			Magic & Base Cost \\
			\hline
			\hline
			Magic 1 & 36 \\
			Magic 2 & 46 \\
			Magic 3 & 54 \\
			Magic 4 & 60 \\
			Magic 5 & 65 \\
			Magic 6 & 69 \\
			Magic 7 & 71 \\
			Magic 8 & 73 \\
			\hline
		\end{tabular}
	\end{center}
	
	\textcolor{OliveGreen}{\textit{Statistics: Stealth is Obscure (Margin-Based\footnote{Based on Psionic Power’s enhancement and PK’s design notes of it. In essence, a weighted sum of margins is about equal to ×3, so this is applied to the wall’s innate attack level cost as a final multiplier.}, $\times$3; Extended, Infrahearing and Ultrahearing, +40\%; Extended Duration, $\times$5000, +150\%; Increased Range, LOS, +40\%; Long-Range 2, +100\%; Ranged, +50\%; Stealthy, +100\%; Variable, +5\%; 1 yard AoE, -25\%; Magical, -10\%; Requires (Spellcasting) Roll, -20\%; Terminal Condition, Loses Concentration, -20\%) further levels increase AOE linearly.}}
	
	\paragraph{Stealth}
	
	The Awakened casts a spell that masks the sounds of a target and what they carry, making it more difficult to detect them.
	
	To use it, Concentrate for 1 second and roll a Spellcasting + Magic; success grants the target the effects until the Awakened loses their concentration.
	
	While under the effects, all sounds the target makes are muffled, providing a -1 penalty per Force to all Hearing or sound-based tests against them, including tests using Infrahearing or Ultrahearing. This affects anything on the target's person, but the GM is free to count large, bulky objects as unaffected (such as when carrying another person).
	
	This penalty applies to active sound-based effects as well, such as Sonic weapons or Sonar. Attempts to target them with sound-based weapons suffers the penalty to hit rolls (similarly to how penalties to Vision apply to Guns rolls). If the weapon is an Affliction, the penalty is instead applies a bonus to resist.
	
	Sonar and similar detections systems apply the penalty to any rolls to actively detect them; additionally, successful detections might not even be recognizeable as a human, instead showing up as a much smaller or further target than it actually is (due to the sound waves being absorbed). 
	
	This can still ellicit suspicion, whether it be read as sensor malfunction, interference or jamming, or simple oddities - it's up the GM what they are interpereted as, but it's recommended for such oddities to show up for margins of victory or loss of $\pm$1 or ties.
	
	\begin{center}
		\begin{tabular}{|c|c|}
			\hline
			Magic & Base Cost \\
			\hline
			\hline
			Magic 1 & 46 \\
			Magic 2 & 53 \\
			Magic 3 & 59 \\
			Magic 4 & 65 \\
			Magic 5 & 59 \\
			Magic 6 & 73 \\
			Magic 7 & 74 \\
			Magic 8 & 76 \\
			\hline
		\end{tabular}
	\end{center}
	
	\textcolor{OliveGreen}{\textit{Statistics: Affliction (Based on Will, +20\%; Extended Duration, $\times$500, +110\%; Fixed Duration, +0\%; Ranged Illusion Affliction, +175\%; Stealth, +36\%; Terminal Condition, Loses Concentration, -20\%) further levels increase Stealth's level}}
	
	\textcolor{OliveGreen}{\textit{Statistics: Stealth is Obscure (Defensive, +50\%; Extended, Infrahearing and Ultrahearing, +40\%; Stealthy, +100\%; Always On\footnote{It's a bit cheesy to include this on an Affliction Advantage, but without it the user could switch the power off, which can't happen normally.}, -50\%; No AOE, -50\%; Magical, -10\%) [3.6] }}
	
	\subsubsection{Manipulation Spells}
	
	\paragraph{Animate}
	
	The Awakened causes an object to spring to life, animated and under his direct control.
	
	To do so, Concentrate for 1 Second and roll versus Spellcasting + Magic; success allows the Awakened to animate objects of a number of HP dependant on the Force of the spell. After this initial concentration, your spell will continue until you lose your concentration, but does not requires constant Concentrate maneuvers.
	
	Each Force provide 2 ST to use for animation, which requires an amount of ST equal to an objects HP for unliving objects, or half its HP for homogenous objects.
	
	An animated object can grab, lift, strike, etc. with a ST equal to the amount needed to initially animate it; while its DX is equal to yours. It can move also long as it is not locked in place, with a Move equal to the ST of the spell minus the ST needed to Animate it. Additionally, objects with built-in methods of flight, such as RC helicopters, are able to fly using those methods. It is, of course, harder to animate more technologically advanced objects, adding an additional penalty up to -6 for such objects, as assessed by the GM. This spell \textit{can} animate multiple objects, but the ST is split amongst them all; see Mass Animate below for a better way to animate many things at once!
	
	While B558 will be an invaluable resource for determining what can be animated, a table for weight by HP is also included for HP animateable by up to Force 16.	
	
	\includegraphics*[width=6.1cm, height=11.7cm]{levitate.png}
	
	\begin{center}
		\scalebox{0.93}{
		\begin{tabular}{|c|c|c|}
			\hline
			HP & Unliving & Homogenous\\
			\hline
			\hline
			1 & 0.02 lb & 0.002 lb\\
			2 & 0.125 lb & 0.156 lb \\
			3 & 0.42 lb & 0.525 lb \\
			4 & 1 lb & 0.125 lb \\
			5 & 1.95 lb & 0.24 lb \\
			6 & 3.375 lb & 0.42 \\
			7 & 5.35 lb & 0.66 lb \\
			8 & 8 lb & 1 lb \\
			9 & 11.39 lb & 1.42 lb \\
			10 & 15.625 & 1.95 lb \\
			11 & 20.8 lb & 2.6 lb \\
			12 & 27 lb & 3.375 lb \\
			13 & 34.33 lb & 4.29 lb \\
			14 & 42.875 lb & 5.36 lb \\
			15 & 52.73 lb & 6.59 lb \\
			16 & 64 lb & 8 lb \\
			17 & 76.77 lb & 9.6 lb \\
			18 & 91.125 lb & 11.39 \\
			19 & 107.17 lb & 13.4 lb \\
			20 & 125 lb & 15.625 lb \\
			21 & 144.7 lb & 18.09 lb \\
			22 & 166.38 lb & 20.8 lb\\
			23 & 190.11 lb & 23.76 lb \\
			24 & 216 lb & 27 lb \\
			25 & 244.14 lb & 30.57 lb \\
			26 & 274.63 lb & 34.32 lb \\
			27 & 307.55 lb & 38.44 lb \\
			28 & 343 lb & 42.89 lb \\
			29 & 381.08 lb & 47.63 lb \\
			30 & 421.89 lb & 52.73 lb \\
			31 & 465.48 lb & 58.19 lb \\
			32 & 512 lb & 64 lb \\
			33 & — & 70.19 lb \\
			34 & — & 76.77 lb \\
			35 & — & 87.74 lb \\
			36 & — & 91.125 lb \\
			37 & — & 98.93 lb \\
			38 & — & 107.17 lb \\
			39 & — & 115.86 lb \\
			40 & — & 125 lb \\
			41 & — & 134.61 lb \\
			42 & — & 144.7 lb \\
			43 & — & 155.29 lb \\
			44 & — & 166.38 lb \\
			45 & — & 177.98 lb \\
			46 & — & 190.11 lb \\
			47 & — & 202.78 lb \\
			48 & — & 216 lb \\
			49 & — & 229.78 lb \\
			50 & — & 244.14 lb \\
			51 & — & 259.08 lb \\
			52 & — & 274.63 lb \\
			53 & — & 290.78 lb \\
			54 & — & 307.55 lb \\
			55 & — & 324.95 lb \\
			56 & — & 343 lb \\
			57 & — & 361.71 lb \\
			58 & — & 381.08 lb \\
			59 & — & 401.13 lb \\
			60 & — & 421.89 lb \\
			61 & — & 443.32 lb \\
			62 & — & 465.48 lb \\
			63 & — & 488.37 lb \\
			64 & — & 512 lb \\
			\hline
		\end{tabular}
		}
	\end{center}
	
	\begin{center}
		\begin{tabular}{|c|c|}
			\hline
			Magic & Base Cost \\
			\hline
			\hline
			Magic 1 & 13 \\
			Magic 2 & 24 \\
			Magic 3 & 33 \\
			Magic 4 & 39 \\
			Magic 5 & 44 \\
			Magic 6 & 49 \\
			Magic 7 & 54 \\
			Magic 8 & 60 \\
			\hline
		\end{tabular}
	\end{center}
	
	\textcolor{OliveGreen}{\textit{Statistics: TK 2 (Independent, +70\%; Animation, -20\%; Cannot Affect Self, -20\%; Cannot Punch, -10\%; Hard to Use 2 (Accessibility, Technology Only, -20\%), -8\%; Magical, -10\%; Requires (Spellcasting) Roll, -20\%; Terminal Condition, Losing Concentration, -20\%)) higher Forces are simply more levels.}}
	
	\paragraph{Mass Animate}
	
	This spell works similarly to Animate above, but has the ability to animate objects en-masse instead of splitting its capabilities over multiple.
	
	The spell affects any number of objects within an area with a radius equal to the Force of the spell, with a penalty to the Spellcasting roll equal to the amount of objects, sans one. 
	
	When being used in combat, the objects must take the same actions, which can greatly hinder them when combining differing objects! Meanwhile, out of combat, this spell can assist in any long-term task that is concluded in a single die roll, such as having a group of animate parts assemble themselves into a machine. This can be used to proved bonuses according to the Time Spent section (B346).
	
	Additionally, this spell, while much better at animating masses of smaller objects, struggles with larger ones! Instead of having 2 ST per Force, it only has 1 ST per Force; it's still capable of animating dozens of small objects, but will struggle for larger objects.
	
	As a note, it can animate Diffuse objects with a ST equal to 1/4 their HP, which can serve as a good estimate for cases such as a swarm of caltrops!	
	
	\begin{center}
		\begin{tabular}{|c|c|}
			\hline
			Magic & Base Cost \\
			\hline
			\hline
			Magic 1 & 10 \\
			Magic 2 & 25 \\
			Magic 3 & 42 \\
			Magic 4 & 59 \\
			Magic 5 & 78 \\
			Magic 6 & 94 \\
			Magic 7 & 110 \\
			Magic 8 & 124 \\
			\hline
		\end{tabular}
	\end{center}


	\textcolor{OliveGreen}{\textit{Statistics: TK 1 (AoE, 1 yard, +25\%; Independent, +70\%; Animation, -20\%; Cannot Affect Self, -20\%; Cannot Punch, -10\%; Hard to Use 2 (Accessibility, Technology Only, -20\%), -8\%; Magical, -10\%; Requires (Spellcasting) Roll, -20\%; Terminal Condition, Losing Concentration, -20\%)) higher Forces are simply more levels with higher linear AoE.}}
	
	\paragraph{Control Actions}
	
	The awakened can puppeteer individual's bodies, making them perform actions against their will.
	
	To do so, they must roll a Spellcasting + Magic roll; success lets the Awakened use the spell as long as they maintain continuous Concentration. The spell has 2 ST for every Force, which can be used to forcibly manipulate living beings.
	
	To control someone, the Awakened must Concentrate for 1 second and win a Quick Contest of the spell's ST versus the target's bodily ST. If they win, the target performs a maneuver of their choice, using their bodily ST and the caster's DX; additionally, the target's Move cannot exceed the Margin of Victory in the Quick Contest.
	
	Note: There's uncertainty whether the Awakened has to roll a Quick Contest every round to control their targets or simply has to succeed once. Given how expensive the ability is, a GM would not be outside their purvue empowering this to work similarly to Mind Control - where a single Quick Contest gives control and a second roll at any "moments of truth" can allow them to break out.
	
	\begin{center}
		\begin{tabular}{|c|c|}
			\hline
			Magic & Base Cost \\
			\hline
			\hline
			Magic 1 & 18 \\
			Magic 2 & 35 \\
			Magic 3 & 49 \\
			Magic 4 & 62 \\
			Magic 5 & 70 \\
			Magic 6 & 76 \\
			Magic 7 & 81 \\
			Magic 8 & 87 \\
			\hline
		\end{tabular}
	\end{center}
	
	\textcolor{OliveGreen}{\textit{Statistics: TK 2 (Animate Life-Forms, +100\%; Accessibility, Life-Forms Only\footnote{There should generally be many more useful inamate objects to control compared to life-forms.}, -30\%; Animation, -20\%; Cannot Affect Self, -20\%; Cannot Punch, -10\%; Magical, -10\%; Requires (Spellcasting) Roll, -20\%) [9] higher Forces are more levels. }}
	
	\paragraph{Mob Control}
	
	This spell works similarly to Control Thoughts above, except that it affects individuals in an area with a radius equal to the Force of the spell.
	
	Roll for each life-form in the area individually, with a penalty equal to the amount of life-forms sans one.
	This spell is very difficult to affect large beings, such as humans. It only have 1 ST per Force instead of 2 ST.
	
	\begin{center}
		\begin{tabular}{|c|c|}
			\hline
			Magic & Base Cost \\
			\hline
			\hline
			Magic 1 & 13 \\
			Magic 2 & 30 \\
			Magic 3 & 50 \\
			Magic 4 & 71 \\
			Magic 5 & 91 \\
			Magic 6 & 111 \\
			Magic 7 & 130 \\
			Magic 8 & 147 \\
			\hline
		\end{tabular}
	\end{center}
		
	\textcolor{OliveGreen}{\textit{Statistics: TK 1 (AoE, 1 yard, +25\%; Animate Life-Forms, +100\%; Accessibility, Life-Forms Only\footnote{There should generally be many more useful inamate objects to control compared to life-forms.}, -30\%; Animation, -20\%; Cannot Affect Self, -20\%; Cannot Punch, -10\%; Magical, -10\%; Requires (Spellcasting) Roll, -20\%) [9] higher Forces are more levels and a linear AoE. }} 

	\paragraph{Control Thoughts}
	
	This insidious spell allows the Awakened to control the literal thoughts of their target, making them think that whatever they are doing is their own ideas.
	
	To do so, Concentrate for 1 second and roll a Quick Contest of Spellcasting + Magic versus Will; success places the target under control for as long as the Awakened takes continuous Concentrate maneuvers, effectively giving them the Reprogrammable disadvantage. This last for a period after the Awakened drops their control, as the targert's ego has to reassert its normal mode of thinking. Notably, this spell is powerful enough that its range is not strictly line of sight, instead taking Long-Range penalties.
	
	\textcolor{NavyBlue}{\textit{Modifiers: Long-Range modifiers to the target; -1 per slave already under your control; +2 if you concentrate for a full minute before casting, or +4 if you concentrate for a full hour}}
	
	If the Awakened is incapacitated, stops concentrating, or forces the target to act against their morals, make a second Quick Contest at the "moment of truth", with success freeing them from control.
	
	If the Awakened loses any Quick Contest, they cannot use this spell on that target for 24 hours, and the target feels a sense of mental coercion coming from you - in addition to any general stimuli for spellcasting. This is usually enough to put people on guard, but anyone who lacks experience with similar spells or magic in general (A very rare thing!) will likely not recognize it. If the Awakened rolls a critical failure, they lose control of anyone else they are currently controlling - or have to roll against some mishap or crippling if they have no one under their control.
	
	After the spell wears off, the target does not have any supernatural inclination to believe or disbelieve that what he just did was genuine. If it was mundane (e.g. take a bathroom break), they would be very unlikely to notice, while ones that went against their principles (e.g. kill your grandmother), would be extremely easy to spot.
	
	Higher Forces are increasingly difficult to resist, gaining a +1 bonus to the Spellcasting roll for every Force above 1.
	
	\begin{center}
		\begin{tabular}{|c|c|}
			\hline
			Magic & Base Cost \\
			\hline
			\hline
			Magic 1 & 68 \\
			Magic 2 & 73 \\
			Magic 3 & 77 \\
			Magic 4 & 81 \\
			Magic 5 & 84 \\
			Magic 6 & 86 \\
			Magic 7 & 88 \\
			Magic 8 & 89 \\
			\hline
		\end{tabular}
	\end{center}	
	
	\textcolor{OliveGreen}{\textit{Statistics: Mind Control (Long-Range, +50\%; Magical, -10\%; Requires (Spellcasting) Roll, -10\%) further levels add Reliable}}
	
	\paragraph{Mob Mind}
	
	This particularly powerful spell works exactly like Control Thoughts above, except that it affects groups of individuals in an area with a radius equal to the Force of the spell.
	
	Roll the Quick Contest for every target in the area, with success and failure determining the result as per normal Control Thoughts. Distinct orders can be issued to individuals or in bulk to multiple individuals, but one can't mix different orders among different individuals.
	
	\begin{center}
		\begin{tabular}{|c|c|}
			\hline
			Magic & Base Cost \\
			\hline
			\hline
			Magic 1 & 93 \\
			Magic 2 & 122 \\
			Magic 3 & 139 \\
			Magic 4 & 151 \\
			Magic 5 & 159 \\
			Magic 6 & 165 \\
			Magic 7 & 168 \\
			Magic 8 & 171 \\
			\hline
		\end{tabular}
	\end{center}	

	\textcolor{OliveGreen}{\textit{Statistics: Mind Control (AoE, 1 yard, +25\%; Long-Range, +50\%; Magical, -10\%; Requires (Spellcasting) Roll, -10\%) further levels add Reliable and Linear AoE}}
	
	\paragraph{Influence}
	
	The Awakened implants a simple suggestion into his target's mind, making them act on it as if it were their own idea.
	
	To do so, Concentrate for 1 second and roll a Quick Contest of Spellcasting + Magic versus Will. Success implant the command into their subconscious, making them carry it out as if it were their own idea. The Awakened can issue mental commands for a number of seconds equal to their Margin of Victory, but after that the target will follow them to their own ability and inclinations. If the command goes against their morals, roll a second Quick Contest at the moment of truth.
	
	\textcolor{NavyBlue}{\textit{Modifiers: Long-Range modifiers to the target; -1 per slave already under your control; +2 if you concentrate for a full minute before casting, or +4 if you concentrate for a full hour}}
	
	If the Awakened loses any Quick Contest, they cannot use this spell on that target for 24 hours, and the target feels a sense of mental coercion coming from you - in addition to any general stimuli for spellcasting. This is usually enough to put people on guard, but anyone who lacks experience with similar spells or magic in general (A very rare thing!) will likely not recognize it. If the Awakened rolls a critical failure, they lose control of anyone else they are currently controlling - or have to roll against some mishap or crippling if they have no one under their control.
	
	What makes this spell so insidious is what happens after. While the target \textit{does} remember everything that they did, they will try tooth and nail to rationalize it however they can - or failing that simply try and forget about it! Let some street trash into the SKIF? Well, they didn't really look that bad and you're pretty sure they showed you valid ID? Confronted with camera footage that they did not? Well they must have shown it to you outside the camera's view. In general, it should take difficult coercion or expertise to get an individual to recognize that their thoughts were not their own.
	
	Higher Forces are increasingly difficult to resist, gaining a +1 bonus to the Spellcasting roll for every Force above 1.
	
	\begin{center}
		\begin{tabular}{|c|c|}
			\hline
			Magic & Base Cost \\
			\hline
			\hline
			Magic 1 & 75 \\
			Magic 2 & 80 \\
			Magic 3 & 85 \\
			Magic 4 & 89 \\
			Magic 5 & 92 \\
			Magic 6 & 94 \\
			Magic 7 & 95 \\
			Magic 8 & 96 \\
			\hline
		\end{tabular}
	\end{center}	
	
	\textcolor{OliveGreen}{\textit{Statistics: Mind Control (Independent, +70\%; Long-Range, +50\%; Rationalization, +20\%; Magical, -10\%; Reduced Duration, 1/60, -35\%; Requires (Spellcasting) Roll, -10\%; Suggestion, -40\%) further levels add Reliable}}
	
	\paragraph{Levitate}
	
	The Awakened is able to telekinetically lift people and objects, slowly levitating them through mid-air.
	
	To do so, they must continuously Concentrate and roll a Spellcasting + Magic; success allows them  to levitate their target. The spell has 2 ST for every Force, which affects its lift, encumbered move, how fast it lifts targets, and its score in Quick Contests against unwilling foes; below is a table for the maximum and unencumbered lift by Force, which represents the amount that the spell can lift in over 4 seconds and still move.
	
	The spell has a Move score equal to the Margin of Success on the Spellcasting roll, modified accordingly to its encumbrance for whatever it carries. This usually allows to to slowly move whatever it holds, but will require higher Force or great success to move quickly.
	
	The spell can grab unwilling people and things they hold, but the Awakened must attack the foe using DX or an unarmed combat skill (with the target defending against an invisible foe. B394), grappling the opponent on a success. Afterwards, they can perform normal grapple actiosn, including levitating them off the ground; often, they must win a Quick Contest of ST versus the target's bodily ST. If lifted off the ground, they are unable to perform any action that requires ground contact, such as running, but can otherwise try to break out, perform grappled actions, and so on.
	
	It's possible to slam objects into opponents, but not throw them. Slams (B371) requires an attack using DX or an unarmed combat skill to hit, dealing damage based on the HP of the target (B558) and the encumbered move of the spell.
		
	\begin{center}
		\begin{tabular}{|c|c|c|}
			\hline
			Force & Max Lift & Unencumbered\\
			\hline
			\hline
			Force 1 & 8 lb & 0.8 lb \\
			Force 2 & 32 lb & 3 \\
			Force 3 & 72 lb & 7 lb \\
			Force 4 & 130 lb & 13 lb \\
			Force 5 & 200 lb & 20 lb \\
			Force 6 & 290 lb & 29 lb \\
			Force 7 & 390 lb & 39 lb \\
			Force 8 & 510 lb & 51 lb \\
			Force 9 & 650 lb & 65 lb \\
			Force 10 & 800 lb & 80 lb \\
			Force 11 & 968 lb & 97 lb \\
			Force 12 & 1,152 lb & 115 lb \\
			Force 13 & 1,352 lb & 135 lb \\
			Force 14 & 1,568 lb & 157 lb \\
			Force 15 & 1,800 lb & 180 lb \\
			Force 16 & 2,048 lb & 205 lb \\
			\hline
		\end{tabular}
	\end{center}
	
	\begin{center}
		\begin{tabular}{|c|c|}
			\hline
			Magic & Base Cost \\
			\hline
			\hline
			Magic 1 & 27 \\
			Magic 2 & 42 \\
			Magic 3 & 56 \\
			Magic 4 & 67 \\
			Magic 5 & 75 \\
			Magic 6 & 80 \\
			Magic 7 & 85 \\
			Magic 8 & 91 \\
			\hline
		\end{tabular}
	\end{center}
	
	\textcolor{OliveGreen}{\textit{Statistics: Margin-Based portion is Telekinesis 1 (Margin-Based\footnote{Based on Psionic Power’s enhancement and PK’s design notes of it. In essence, a weighted sum of margins is about equal to ×3, so this is applied to the wall’s innate attack level cost as a final multiplier.}, $\times$3; Extended Range, LoS, +70\%; Accessibility, Levitating People/Objects Only\footnote{No remote operation, no real throwing, no striking, overall this is probably 1/6 utility}, -35\%; Cannot Punch, -10\%; Magical, -10\%; Move Only, -40\%; Requires (Spellcasting) Roll, -10\%) [9.75] }}
	
	\textcolor{OliveGreen}{\textit{Statistics: ST portion is Telekinesis 2 (Extended Range, LoS, +70\%; Accessibility, Levitating People/Objects Only, -35\%; Cannot Punch, -10\%; Magical, -10\%; Lift Only, -20\%; Requires (Spellcasting) Roll, -10\%) }}
	
	\paragraph{Mana Barrier}
	
	This spell creates an magical barrier on the astral plane, blocking spells and astral forms from passing through it.
	
	To use it, the awakened must Concentrate for 1 second and roll a Spellcasting + Magic roll; succes conjures the barrier, lasting until the caster loses Concentration.
	
	The wall block all astral forms and mana/astral based spells, which includes any spell that only affects living beings,  most critter powers, spirits, and any dual-natured creatures (including magician astrally perceiving or projecting). While it's not visible on the physical plane it is near opaque on the astral plane.
	
	The wall has 3 HP and 2 DR\footnote{Changed towards more HP than DR using the ideas \textcolor{Blue}{\href{http://forums.sjgames.com/showpost.php?p=2050064&postcount=3}{found here.}}} per Margin of Success (minimum 3 HP and 2 DR), is homogenous, and imposes a Vision penalty to see through using Astral Perception, Projection, spellcasting, etc. it equal to half its DR, (rounded up, maxmimum -10). Notably, this damage resistance applies to Maledictions applied through the mana barrier, using the Cover DR seen below, which is added to resistance rolls as normal for Afflictions, while providing that penalty to other spells such as Control Thoughts.
	
	The wall has one section per Force, each measuring 3$\times$1$\times$4 yards (length, width, height). These each of these sections can be arranged as the caster sees fit. Optionally, the GM may allow them to make smaller or larger segments in certain dimensions in return for lengthening or shortening others, such as halving width for double the length; if the width is altered, reduce the DR and HP proportionally (e.g. 1/2 a yard is 1.5 HP and 1 DR per MoS). If the GM does not allow this, it's still possible to layer sections back-to-back, just without adding their stats together.
	
	The barrier itself is a homogenous object, which can make it resistant to a small selection of mana spells, and it can be affected by damage following \textit{Damage to Objects} (B483): at 0 HP, the wall must succeed an HT 12 check every second or suffer failure - which usually means a section of it collapses. At every negative integer of HP, the wall must succeed against an HT roll or be destroyed and dissipate; a failure of 1 or 2 means that the spell remains, but is reduced to rubble and debris of the original force field. It dissipates automatically at -5$\times$HP.
	
	When attacking targets through the wall individuals take Vision penalties as described above, as if looking through a foggy substance. The wall itself provides Cover DR equal to its DR + HP/4, or 2.75 per MoS (rounded down).
	
	\begin{center}
		\begin{tabular}{|c|c|}
			\hline
			Magic & Base Cost \\
			\hline
			\hline
			Magic 1 & 68 \\
			Magic 2 & 70 \\
			Magic 3 & 71 \\
			Magic 4 & 72 \\
			Magic 5 & 73 \\
			Magic 6 & 73 \\
			Magic 7 & 73 \\
			Magic 8 & 73 \\
			\hline
		\end{tabular}
	\end{center}
	
	\textcolor{OliveGreen}{\textit{Statistics: Innate Attack, Cr (Margin Based\footnote{Based on Psionic Power's enhancement and PK's design notes of it. In essence, a weighted sum of margins is about equal to $\times$3, so this is applied to the wall's innate attack level cost as a final multiplier.}, $\times$3 to final; Area of Effect, 1 Yard, +25\%; Cosmic, No Die Roll (Accessibility, Innate Attack roll only, -20\%), +80\%; Extended Duration, $\times$5000, +150\%; Increased Range, LoS, +40\%; Long-Range 2, +100\%; Persistent, +40\%; Wall, Rigid, +60\%; Inaccurate 2, -10\%; Limited, AstralThreats, -40\%; Magical, -10\%; Malediction Only, +0\%; Requires (Spellcasting) Roll, -20\%; Terminal Condition, Losing Concentration, -20\%)}}
	
	\paragraph{Physical Barrier}
	
	This spell create a glowing, semi-transluencent force field over an area, blocking physical spell effects, creatures, and objects from passing through it.
	
	To use it, the awakened must Concentrate for 1 second and roll on a Spellcasting + Magic roll; success conjure the physical barrier. The wall lasts until the caster loses Concentration.
	
	The wall has 3 HP and 2 DR\footnote{Changed towards more HP than DR using the ideas \textcolor{Blue}{\href{http://forums.sjgames.com/showpost.php?p=2050064&postcount=3}{found here.}}} per Margin of Success (minimum 3 HP and 2 DR), is homogenous, and imposes a Vision penalty to see through it equal to half its DR\footnote{There's no consistent ruling on whether the rigid Wall enhancement block vision (e.g. a wall of stone versus Aluminum Oxide could both be rigid, but one is see through.), but Powers 42 implies it does. Since it could be a bane or boon I'm simply allowing it as a stacking penalty.}, (rounded up, maxmimum -10).
		
	The wall has one section per Force, each measuring 3$\times$1$\times$4 yards (length, width, height). These each of these sections can be arranged as the caster sees fit. Optionally, the GM may allow them to make smaller or larger segments in certain dimensions in return for lengthening or shortening others, such as halving width for double the length; if the width is altered, reduce the DR and HP proportionally (e.g. 1/2 a yard is 1.5 HP and 1 DR per MoS). If the GM does not allow this, it's still possible to layer sections back-to-back, just without adding their stats together.
	
	The barrier itself is a homogenous object, making it very resistant to most firearms, and it can be affected by damage following \textit{Damage to Objects} (B483): at 0 HP, the wall must succeed an HT 12 check every second or suffer failure - which usually means a section of it collapses. At every negative integer of HP, the wall must succeed against an HT roll or be destroyed and dissipate; a failure of 1 or 2 means that the spell remains, but is reduced to rubble and debris of the original force field. It dissipates automatically at -5$\times$HP.
	
	When attacking targets through the wall individuals take Vision penalties as described above, as if looking through a foggy substance. Certain damage types can pierce the wall to deal more damage to those behind it, as seen in \textit{Overpenetration} (B408); the wall itself provides Cover DR equal to its DR + HP/4, or 2.75 per MoS (rounded down).
	
	The wall itself can carry a certain amount of weight, measured in combined HP, under large amounts of stress, depending on its DR, as shown in the table below. This represents the total HP that can be present on the wall during a violent alteraction; if its during normal use, the GM can increase this HP by double or more. Gear HP can be found on B558, but a quick heuristic is to total the weight of gear for a given person and it will have a total HP of 1/5. GMs can feel free to convert this to a maximum weight instead of HP using the same table as well - although it will be technically inaccurate, due to collision rules using HP.
	
	\begin{center}
		\begin{tabular}{|c|c|}
			\hline
			DR & Max HP Held \\
			\hline
			\hline
			2 & 11 \\
			4 & 22 \\
			6 & 34 \\
			8 & 45 \\
			10 & 57 \\
			12 & 68 \\
			14 & 79 \\
			16 & 91 \\
			18 & 102 \\
			20 & 114 \\
			\hline
		\end{tabular}
	\end{center}	
		
	\begin{center}
		\begin{tabular}{|c|c|}
			\hline
			Magic & Base Cost \\
			\hline
			\hline
			Magic 1 & 71 \\
			Magic 2 & 74 \\
			Magic 3 & 75 \\
			Magic 4 & 76 \\
			Magic 5 & 77 \\
			Magic 6 & 77 \\
			Magic 7 & 77 \\
			Magic 8 & 77 \\
			\hline
		\end{tabular}
	\end{center}
	
	\textcolor{OliveGreen}{\textit{Statistics: Innate Attack, Cr (Margin Based\footnote{Based on Psionic Power's enhancement and PK's design notes of it. In essence, a weighted sum of margins is about equal to $\times$3, so this is applied to the wall's innate attack level cost as a final multiplier.}, $\times$3 to final; Area of Effect, 1 Yard, +25\%; Cosmic, No Die Roll (Accessibility, Innate Attack roll only, -20\%), +80\%; Extended Duration, $\times$5000, +150\%; Increased Range, LoS, +40\%; Long-Range 1\footnote{When you only need 3+ to auto succeed this is effectively the same as Long Range 2.}, +50\%; Persistent, +40\%; Wall, Rigid, +60\%; Inaccurate 3, -15\%; Limited, Nonmagical Threats plus Jet/Missile spells, -15\%; Magical, -10\%; Requires (Spellcasting) Roll, -20\%; Terminal Condition, Losing Concentration, -20\%)}}

	\subsection{Adept Powers}
	
	\paragraph{Adept Spell}
	\begin{flushright}
		1 + Spell Cost Points
	\end{flushright}
	
	This is an \textit{Adept Power} that allows the Adept to cast one spell innately without necessarily being a Magician, Mystic Adept, or Aspected Magician. Buy that particular spell (Alongside the Spellcasting skill to use it) as if you would any normal adept power; pay full price for the most expensive and 1/5 for everything else. If you buy multiple Adept Spells, you do not have to pay the additional 1 point for each spell - you must only pay it once. Adepts are limited to a maximum amount of Adept Spells equal to their Magic.
	
	\paragraph{Adrenaline Boost}
	\begin{flushright}
		3.75 Points, 4 Points for higher levels
	\end{flushright}
	
	As a free action, the adept can trigger this power, increasing their reaction time greatly. Their Basic Speed increases by 0.25 per level they have in the power, which improves all derived traits except Basic Move. 
	
	The adept can use the power for as long as they like, however after the power has finished they lose FP equal to their level in the ability.
	
	\textcolor{OliveGreen}{\textit{Statistics: Basic Speed +0.25 (Reduced Time 1, +20\%; Aftermath (Costs FP 1/sec), -10\%; Magical, -10\%, No Basic Move, -25\%) [3.75/level], further levels are Basic Speed by +0.25 (Reduced Time 1, +20\%; Aftermath(Costs FP +1/se)c, -5\%; Magical, -10\%) [4] }}
	
	\paragraph{Animal Empathy}
	\begin{flushright}
		4.5 Points, 4.5 Points for higher levels
	\end{flushright}
	
	\textcolor{OliveGreen}{\textit{Statistics: Animal Empathy (Magical, -10\%) [4.5] additional levels provide Animal Friend (Magical, -10\%) [4.5/level]}} 
	
	\paragraph{Astral Perception}
	\begin{flushright}
		13 Points
	\end{flushright}
	
	Grants the effect of the \hyperref[astral_perception]{Astral Perception} advantage, including an Unusual Background.
	
	\paragraph{Attribute Boost }
	\begin{flushright}
	Various Points
	\end{flushright}

	The adept can boost their physical attributes temporarily, granting immense and fleeting power. For this power, the adept takes a Ready maneuver and rolls against 10 + Magic, increasing their Attribute's level by their Margin of Success temporarily. 
	
	The adept can use the ability for as long as they wish, however after the ability is finished, they lose 2 FP for every second it was active.
	
	The cost of the advantage varies by attribute:
	
	\begin{itemize}
		\itemsep 0pt
		\item Attribute Boost ST (No HP) [17]
		\item Attribute Boost DX (No Basic Speed) [50]
		\item Attribute Boost HT (No Basic Speed, FP) [20]
		\item Attribute Boost Basic Speed (No Basic Move) [16]
		\item Attribute Boost Basic Move [20]
	\end{itemize}
	
	\textcolor{OliveGreen}{\textit{Statistics: Affliction ((Attribute), Margin-Based, +X\%; Accessibility, Self Only, -20\%\footnote{\GURPS Powers, The Weird.}; Aftermath (Costs FP 2/sec), -15\%; Requires Magic Roll, -20\%\footnote{Is Requires Attribute (10) Roll, -20\%}) }}
	
	\paragraph{Authoritative Tone}
	\begin{flushright}
		9 Points, 4.85 Points for higher levels
	\end{flushright}
			
	The power makes the adept speak in ways that makes people trust and believe in him. It gives +2 to Diplomacy, Fast-Talk, Mimicry, Performance, Politics, Public Speaking, Sex Appeal, and Singing, while also granting +2 on any reaction roll made by someone who can hear your voice.
	
	Further levels enhance improves other aspects, giving +1/level past the first to Fortune-Telling, Leadership, Panhandling, and Public Speaking skills alongside Reaction and Influence rolls with people you actively interact with \textit{and} can hear you.
			
	\textcolor{OliveGreen}{\textit{Statistics: Voice (Magical, -10\%) [9] further levels are Charisma +1 (Magical, -10\%; Nuisance Effect, Actively Interacting always requires hearing, -5\%\footnote{"Actively interacting" already includes hearing you, however this requires that any interact definitively includes hearing you (No sign language, charades, etc.)}) [4.85]}}
			
	
	\paragraph{Berserk}
	\begin{flushright}
		15 points
	\end{flushright}

	The adept can willingly enter a state of blind rage, greatly enhancing their physical traits at the cost of their mental faculties. As a Ready maneuver, the adept can activate the power, granting them all the effects of the Berserk (B124) disadvantage with Battle Rage:
	
	\begin{itemize}
		\itemsep 0pt
		\item Must All-Out-Attack a foe in range, or Move / Move and Attack into range.
		\item Can go guns blazing if over 20 yards, but can only reload weapons that take 1 second.
		\item Immune to stun, shock, and injurious penalty to Move. +4 to HT rolls to remain conscious or alive.
		\item Can roll SC after downing each foe, and once when there are no foes left - after which you target friends!
	\end{itemize}	

	You may attempt to resist or activate the disadvantage as normal - once when entering combat, when damaged over 1/4 your HP in a second, and when witnessing equivalent harm to allies. You do not need to be under the effects to gain the other benefits of the power.

	While the power is active, the adept's ST, DX, and HT are improved by +1. This also improves their Basic Speed and FP, but not HP. Whenever the power ends, remove these benefits and determine the effects (You may pass out from losing the FP, for instance).
	
	Additionally, your IQ, Per, and Will are all lowered by -1, alongside any derived traits, while the power is active, impeding your mental capabilities.
	
	After the power ends, the adept immediately loses 3 FP.
	
	\textcolor{OliveGreen}{\textit{Statistics: Berserk Attributes (ST +1 (No HP, -2) [4]; DX +1 [20]; HT +1 [13]) (Aftermath(Costs FP 1/sec), -10\%; Magical, -10\%; Temporary Disadvantage, Berserk, SC 12, Battle Rage, -15\%; Temporary Disadvantage, -1 IQ, -13\%; Temporary Disadvantage, -1 Per, -5\%; Temporary Disadvantage, -1 Will, -7\%) [14.8] }}
			
			
	\paragraph{Berserker's Rage}
	\begin{flushright}
		+21 points
	\end{flushright}
	\begin{flushright}
	\textit{Prerequisites: Berserk Adept Power}
	\end{flushright}
	
	This power works exactly like the Berserk power, with the following changes:
	
	ST is increased by +3, while DX and HT are increased by +2.
	
	Additionally, the adept loses 1 FP per \textit{minute} that the power was active, rounded up.
	
	\textcolor{OliveGreen}{\textit{Statistics: Increase attributes to ST +3 (No HP, -2) [12], DX +2 [40], and HT +2 [26]. Change Aftermath(Costs FP 1/sec), -10\%; to Aftermath(Costs FP 1/min), -5\%. [35.1 total] }}
	
	
	\paragraph{Cloak}
	\begin{flushright}
	 5.2 per Level
	\end{flushright}

	The adept's powers passively protect them from Detection Powers, such as Detection Spells like Detect Life, Detection Powers like Search, and so on. This affects the adept alongside anything in their direct vicinity (notably including their carried equipment).
	
	Anything trying to use such an effect (even for helpful purposes) takes a penalty equal to the adept's levels in this power. This penalty does not apply to any of the adept's Detection Powers, nor is the effect especially detectable itself (Beyond the normal rules for Adept Powers).

	\textcolor{OliveGreen}{\textit{Statistics: Obscure, Detect Abilities (Defensive, +50\%; Extended 2, +40\%\footnote{Detection Abilities covers a good amount of spells and critter powers, so 2 should "cover" a wide enough range to justify them all.}; Stealthy, +100\%; Limited (Magical), -20\%; Magical, -10\%) [5.2]}}
	
	\paragraph{Combat Sense}
	\begin{flushright}
		14 Points + Variable Points at higher levels
	\end{flushright}

	The adept gains an instinctive sense for threats in their vicinity, improving their ability to defend against them. 
	
	The power provides them with a +1 to all active defenses, +1 to the Fast-Draw skill, and a +2 to Fright checks. Additionally, they never "freeze" in surprise situations and gain a +6 on all IQ rolls to wake up and recover from surprise or mental stun.
	
	In surprise situations, the adept provides a +1 to initiative rolls, or +2 if they are the leader.
	
	Higher levels improve this even more! The adept can improve one of four defenses, each with their own costs: Unarmed Parries [4.5], All Parries [9], Dodges [13.5], and/or Blocks [4.5]. 
	
	The adept can only take levels up to half of their magic, rounded down. As well, the GM is recommended to require \textit{all options} be taken before any can be improved to the next level. Do note that All Parries does include Unarmed Parries.

	\textcolor{OliveGreen}{\textit{Statistics: Combat Reflexes (Magical, -10\%) [14] further levels add one of Enhanced Defense Parry, Dodge, and/or Block (Magical, -10\%) [4.5/9], [13.5], [4.5]}} 
	
	\paragraph{Critical Strike}
	\begin{flushright}
		4 Points
	\end{flushright}
	
	The adept's powers drive their weapon more powerfully and deadlier than otherwise. 
	
	Select a skill when taking this power. The adept increases their Striking ST by 4 when using weapons with that skill.
	
	\textcolor{OliveGreen}{\textit{Statistics: Striking ST 4 (One Skill Only, -40\%; Magical, -10\%) [4] }}
	
	\paragraph{Danger Sense}
	\begin{flushright}
		14 Points
	\end{flushright}

	The adept can detect impending danger even in the most unlikely situations. 
	
	In any situation involving ambush, impending disaster, or similar hazards the GM rolls against the adept's Perception + Magic. Success gives enough warning to take action, while critical success gives details as the the nature of the threat as well.

	\textcolor{OliveGreen}{\textit{Statistics: Danger Sense (Magical, -10\%) [14] }}
	
	
	\paragraph{Demara}
	\begin{flushright}
		7 Points
	\end{flushright}

	This power allows the adept to take in experience and technique extremely quickly when presented with suitable reference material and enough time.
	
	By spending an hour watching or reading training, explanation, or demonstration media of a skill (in any format, including in person and recordings), the adept can temporarily add 1 point to that skill. This stacks with any current points, as long as the material is sufficiently advanced for higher levels (e.g. adding 1 point to 15 points of Physics requires Phd level media).
	
	This point lasts for up to 6 hours, or until the adept uses this ability to learn another skill.
	
	The power can be purchased multiple times. Each time, it allows the adept to store a separate skill, allowing him to "learn" multiple skills, or to stack multiple points into one skill.

	\textcolor{OliveGreen}{\textit{Statistics: Modular Ability 6 per slot, 3 per point (Slow and External) (Magical, -10\%; Maximum Duration 6 hours, -5\%; Trait-Limited, Skills Only, -10\%) [4.5 + 2.25] }}
	
	
	\paragraph{Eidetic Sense Memory}
	\begin{flushright}
		7 Points
	\end{flushright}

	The adept is able to perfectly recall any stimuli or detailed sensory input that they experienced. This allows them to perform feats like flip through a book and read it later or shift through conversations after the nightclub is empty. Some of these feats may still require separate skill rolls for analysis still.
	
	This affects recall, not comprehension, so gives no benefit to skills themselves (Other than Speed-Reading B222). However, for any IQ roll for learning you may add a +10 bonus. Some examples of this include memorizing paydata or information to slip through security and earning familiarities with equipment or accents. If the GM is using Quick-Learning Under Pressure (B292) or Maintaining Skill (B294), this bonus applies to them too!

	\textcolor{OliveGreen}{\textit{Statistics: Photographic Memory (Magical, -10\%) [9] }}
	
	\paragraph{Elemental Body}
	\begin{flushright}
		Variable Points
	\end{flushright}
	\begin{flushright}
		\textit{Prerequisites: Elemental Strike}
	\end{flushright}
	
	
	\begin{center} 
	\begin{adjustwidth}{-3mm}{}
		\scalebox{0.93}{
			\begin{tabular}{|c|c|c|c|c|}
			\hline
			Magic & Acid & Cold & Fire & Lightning \\
			\hline
			\hline
			Magic 1 & 6 & 5 & 5 & 5 \\
			Magic 2 & 12 & 10 & 10 & 10 \\
			Magic 3 & 17 & 14 & 15 & 14 \\
			Magic 4 & 22 & 18 & 20 & 18 \\
			Magic 5 & 28 & 23 & 24 & 23 \\
			Magic 6 & 33 & 27 & 29 & 27 \\
			Magic 7 & 38 & 32 & 33 & 32 \\
			Magic 8 & 42 & 34 & 37 & 34 \\
			\hline
			\end{tabular}
		}
		\end{adjustwidth}
	\end{center}
	
	\textbf{Acid:}
	
	\textcolor{OliveGreen}{\textit{Statistics: Innate Attack, Burn Xd-2 (Aura, +80\%; Backlash(Drain FP)\footnote{See \hyperref[drain_mods]{the Drain FP limitation}.}; Magical, -10\%; Melee Attack, -30\%) }}
	
	\textbf{Cold:}
	
	\textcolor{OliveGreen}{\textit{Statistics: Innate Attack, Burn Xd-1 (Aura, +80\%; Backlash(Drain FP); Magical, -10\%; Melee Attack, -30\%; No Incendiary, -10\%) }}
	
	\textbf{Fire:}
	
	\textcolor{OliveGreen}{\textit{Statistics: Innate Attack, Burn Xd-1 (Aura, +80\%; Backlash(Drain FP); Magical, -10\%; Melee Attack, -30\%) }}
	
	\textbf{Lightning:}
	
	\textcolor{OliveGreen}{\textit{Statistics: Innate Attack, Burn Xd-1 (Aura, +80\%; Surge, +20\%; Backlash(Drain FP); Magical, -10\%; Melee Attack, -30\%; No Incendiary, -10\%) }}

	\paragraph{Elemental Strike}
	\begin{flushright}
		Variable Points
	\end{flushright}

	This power adds elemental enhancements to an adept's unarmed attacks, allowing him to electrocute, burn, or more with a single blow. Notably, these effects are not visible on the physical plane - although follow-on effects such as setting something on fire or shocking something are!
	
	The cost for this is dependent on the adept's Thrusting Damage, ergo his ST\footnote{Please note that most of this power is made using Modifying ST-Based Damage (P146), and as such any unorthodox Striking damage will NOT work with it. Cold, Fire, and Lightning are built as an upgraded attack minus the actual attack itself. Acid is simply a Follow-up attack.}, and on the element chosen.
	
	These effects are not applied to any strength-based weapons used by the adepts, such as swords or brass knuckles, although they do not necessarily need skin contact to work (gloves are acceptable).
	
	Additionally, they do not apply to any non-human-standard unarmed attacks, such as horns, claws, teeth, etc. To apply to them, those attacks \textit{must also include} the respective elemental types found below. For Fire, Lightning, and Cold simply add their enhancements. For Acid, build another Follow-Up attack in the same fashion and use the Alternative Abilities (P11) rules for it.
	
	\begin{center} 
		\begin{adjustwidth}{-1mm}{}
			\scalebox{0.87}{
				\begin{tabular}{|c|c|c|c|c|}
				\hline
				Striking ST & Acid & Cold & Fire & Lightning \\
				\hline
				\hline
				1-2 & 0 & 0 & 0 & 0 \\
				3-4 & 1 & 1 & 1 & 1 \\
				5-6 & 1 & 1 & 1 & 1 \\
				7-8 & 1 & 1 & 1 & 1 \\
				9-10 & 2 & 1 & 1 & 1 \\
				11-12 & 3 & 2 & 1 & 1 \\
				13-14 & 4 & 3 & 1 & 1 \\
				15-16 & 6 & 3 & 1 & 1 \\
				17-18 & 7 & 4 & 1 & 2 \\
				19-20 & 7 & 4 & 1 & 2 \\
				21-22 & 8 & 5 & 1 & 2 \\
				23-24 & 10 & 6 & 1 & 2 \\
				25-26 & 11 & 6 & 2 & 2 \\
				27-28 & 11 & 7 & 2 & 2 \\
				29-30 & 12 & 7 & 2 & 2 \\
				\hline
			\end{tabular}
		}
		\end{adjustwidth}
	\end{center}


	\textbf{Acid:}
	
	The adept's fists channel acidic forces, decaying the armour of anything that his fists strike. Whenever the adept strikes something with his unarmed attacks, the target's DR is reduced by 1 for every 5 basic damage rolled.
	
	\textcolor{OliveGreen}{\textit{Statistics: Innate Attack, Cor X\footnote{Rolling dice for this follow-up is waived - it's built to be the same as the Striking ST, so we just use that roll!} (Follow-up, Unarmed Attacks, +0\%; No Wounding, -50\%; Magical, -10\%) }}

	\textbf{Cold\footnote{There's no "snap armour" equivalent in \GURPS, so AD is the best alternative.}:}
	
	The adept can channel frostbite cold in a sharp instant, weakening armour for an instant. This provides your unarmed attacks with an armour divisor of 2.
	
	\textcolor{OliveGreen}{\textit{Statistics: Innate Attack, Cr X (Armour Divisor 2, +45\% (Magical, -10\%))- NOT Innate Attack, Cr X }}

	\textbf{\\Fire:}
	
	The adept's unarmed attacks can ignite fires and cause incendiary effects. The adept's base striking damage when attacking with fists is used for Making Things Burn (B433).

	\textcolor{OliveGreen}{\textit{Statistics: Innate Attack, Cr X (Incendiary, +9\% (Magical, -10\%))- NOT Innate Attack, Cr X }}
	
	\textbf{Lightning:}
	
	The adept's unarmed attacks are charged with electricity, enough to fry electronics. Any critical hit will disable electronics (or those with the Electric disadvantage) and any electronic hit by an attack dealing over 1/3 HP requires an HT to avoid shorting out for seconds equal to Margin of Failure, or until repaired on a critical failure.
	
	\textcolor{OliveGreen}{\textit{Statistics: Innate Attack, Cr X (Surge, +18\% (Magical, -10\%))- NOT Innate Attack, Cr X }}
	
	\paragraph{Empathic Healing}
	\begin{flushright}
		17 Points
	\end{flushright}

	The adept is able to magically take on the wounds of others in order to heal them.
	
	To do so, concentrate for 4 seconds and then make an IQ + Magic test. Success allows them to transfer HP loss from another creature to themselves, which ignores any immunities or reductions. Even 1 HP of healing stops bleeding (The adept does not begin bleeding either). Failure transfers nothing and costs an immediate 1d HP, with critical failure costing the target 1d HP as well.
	
	Crippled (but still whole) limbs can be healed as well, by making the roll at a -6 and spending an additional 2 HP. If the adept completeley heals the HP lost to the crippling injury, the limb will no longer be crippled. The adept only gets on attempt per crippled injury.
	
	If the power is used on an individual multiple times, it has a cumulative -3 for each \textit{successful} use. The penalty lasts for 24 hours after the \textit{last successful attempt}. This makes multiple small healings difficult!
	
	You can heal any creature that is carbon-based, however individuals with lots of 'ware are more difficult. The GM should assign a penalty up to -6 for individuals with impacted essence. A good heuristic is a -1 for every 10 points in 'ware-caused disadvantages.

	\textcolor{OliveGreen}{\textit{Statistics: Healing (Xeno-healing, Carbon-Based lifeforms, +60\%; Empathic, -50\%; Hard to Use 2 (Accessibility, Only on low Essence Individuals, -40\%), -6\%; Injuries Only, -20\%; Magical, -10\%; Takes Extra Time 2 (4 Seconds), -20\%) [17] }}
	
	\paragraph{Enhanced Perception}
	\begin{flushright}
		4.5 Points per Level
	\end{flushright}

	The adept's powers sharpen their senses, allowing them to surveil their surroundings with extreme clarity. This power adds +1 Perception per level. Adepts are limited to levels equal to half their magic, rounded down.

	\textcolor{OliveGreen}{\textit{Statistics: Perception (Magical, -10\%) [4.5] }}
	
	\paragraph{Flexibility}
	\begin{flushright}
		5/14 Points
	\end{flushright}

	Adepts with this power can bend and flex their bodies past metahuman norms. This becomes very useful for tasks that requires bending and maneuvering in odd ways, providing a +3 bonus to Climbing, Escape rolls to get free of restraints, the Erotic Art skill, and on attempts to break free in close combat (B391). Additionally, they may ignore up to -3 for working in close quarters (Such as an Explosive check to disarm a bomb in a vent or small casing; or a Mechanics roll to work on an engine inside a car).
	
	At the second level, these bonuses increase further, allowing the adept to maneuver their body in any way that is not outright abnormal! The adept adds +5 to the same rolls and ignore up to -5 in close quarter penalties.

	\textcolor{OliveGreen}{\textit{Statistics: Flexible (Magical, -10\%) [5] or Double-Jointed (Magical, -10\%) [14] }}

	\paragraph{Focused Archery}
	\begin{flushright}
		2 Points per Level
	\end{flushright}

	The adept is able to focus their qi when drawing a bow or crossbow, enabling them to pull loads higher than their actual strength. This power increases the adept's Striking ST when loading Bows or Crossbows, letting them use stronger bows. The adept can have a maximum of 3 levels.

	\textcolor{OliveGreen}{\textit{Statistics: Striking ST 2 (Crossbow/Bow, -40\%; Magical, -10\%) [2] }}

	\paragraph{Freefall}
	\begin{flushright}
		9 Points
	\end{flushright}

	The power helps the adept to absorb the kinetic impact from falls, cushion his landings. Subtract 5 yards from a fall automatically (as if succeeding on an Acrobatics check). Additionally, a successful DX roll halves damage from a fall.

	\textcolor{OliveGreen}{\textit{Statistics: Catfall (Magical, -10\%) [9]}}

	\paragraph{Hanging}
	\begin{flushright}
		4 Points
	\end{flushright}

	The adept gains the ability to adhere to surfaces, such as walls, for short periods of time. They cannot move while attached, but otherwise need not make any climbing checks. They can maintain this power for up to 5 minutes, afterwards they must reattach elsewhere.
	
	If falling, the adept can attempt to adhere to a nearby surface. On a successful DX roll, they can touch the surface and may follow up with a ST roll at a -1 for every 5 yards already fallen. Success allows the adept to arrest their fall, however a failure still lets them subtract 5 yards from their distance fallen due to slowing down during the failed attempt.
	
	Clothing and armour does not impact the power, however their power does not work when at Heavy encumbrance or above.

	\textcolor{OliveGreen}{\textit{Statistics: Clinging (Accessibility, No movement, -40\%\footnote{Based on the assumption that this would reduce the useful situations for clinging by around 1/20.}; Accessibility, No Heavy Encumbrance, -20\%\footnote{Priced as a reverse to the Can Carry Objects Enhancement from Power-Ups 4.}; Magical, -10\%; Nuisance Effect, Max 5 Minutes, -10\%\footnote{Valued at 10\%, because -5\% simply seemed to small to limit any long term clinging. Optionally treat as an Accessibility}) [4] }}

	\paragraph{Improved Physical Attribute}
	\begin{flushright}
		Variable Points
	\end{flushright}
	`
	Just buy the Attribute with Magical, -10\% my dude.
	
	\paragraph{Improved Reflexes}
	\begin{flushright}
		14/27/41 Points
	\end{flushright}

	The adept's reflexes are vastly improved, allowing them to react with lightning speed. For each level, increase the Adept's Basic Speed (and all derived attributes from it, except Basic Move) by +1. The adept may qualify for certain advantages given a \hyperref[high_basic_speed]{high Basic Speed}.

	\textcolor{OliveGreen}{\textit{Statistics: Improved Basic Speed +1.0 (No Basic Move, -5; Magical, -10\%) [13.5] further levels are +2.0 and +3.0. }}

	\paragraph{Improved Sense}
	\begin{flushright}
		Various Points
	\end{flushright}

	These powers grants sensory capabilities not normal possible for metahumans. These can be picked from the list below:
	
	\subparagraph{3D Spatial Sense [9 Points]}
	
		The Adept gains intuitive knowledge of directions in three dimensions. They always know where north is, can retrace any path within a month, and gain a +3 bonus to Body Sense, Navigation (Air, Land, or Sea), a +2 bonus to Aerobatics, Free Fall, and Navigation (Hyperspace or Space), and a +1 bonus to Piloting.
		
		\textcolor{OliveGreen}{\textit{Statistics: 3D Spatial Sense (Magical, -10\%) [9]}}
			
	\subparagraph{Sensitive Touch [9 Points]}
	
		The adept's sense of touch gains as much resolution as human vision. This allows them to recognize very fine details about objects they touch as readily as a normal human could by vision, allowing distinguishing of fine relief, small variations in heat, tremors as people approach, and so on. 
		
		This gives +4 on Touch rolls and on skill rolls that are dependant on touch (such as a Forensic to distinguish between two fabrics). They disregard penalties for working by touch so long as a task doesn't \textit{require} visual information, and even then that might be possible (e.g. distinguishing colours based on how well they retain heat in the sun); unfamiliarity penalties still apply in such a case. 
		
		This doesn't let the adept target someone with an attack based on vibrations, but you could notice a person's general presence. For that, take Vibration Sense below.
			
		\textcolor{OliveGreen}{\textit{Statistics: Sensitive Touch (Magical, -10\%) [9] }}
			
	\subparagraph{Acute Sense [2/4/6/8/9 Points]}
	
		One of the Adept's senses has its clarity increases greatly, providing a +1 bonus per level to its respective Sense rolls. The maximum levels are half the adept's Magic, rounded \textit{up.} This can be applied to more than the normal five senses, including options such as Magical Sense SURGE Power, the Search Critter Power, and so on.
		
		\textcolor{OliveGreen}{\textit{Statistics: Acute Sense 1-6 (Magical, -10\%) [2-9]}}
				
	\subparagraph{Night Vision [1/2/3/4/5/6/7/8/9 Points]}
	
		The adept's low-light vision increases, allowing them to see and act more clearly in the dark. The lower darkness penalties by 1 per level in this power, as long as there is \textit{some light}, meaning -9 darkness or better.
		
		\textcolor{OliveGreen}{\textit{Statistics: Night Vision 1-9 (Magical, -10\%) [1-9]}}
		
	\subparagraph{Infravision [9 Points]}
	
		The adept can see through the \textcolor{NavyBlue}{\href{https://en.wikipedia.org/wiki/Infrared}{entire infrared spectrum}}, which has a variety of capabilities; it allows them to see the radiated thermal infrared emitted from most objects, they are able to see through certain objects that infrared passed through (many plastics and some fabrics), allows them to spot hot components such as sensors in walls, and to see \textit{active} millimeter-wave and terahertz detectors (such as often are used in portal scanners), and see IR lasers (which are commonly used for both communication and lethal purposes). See Powers: Enhanced Senses p8 for more, including limiting vision to Near or Thermal Infrared. Remember that modern IR cameras \textit{do not} see in the whole spectrum, often only the Near-Infrared spectrum.
		
		The adept can act in absolute darkness as long as their surroundings emit heat (largely all objects). Spotting targets that emit heat grants a +2 to all Vision rolls. As well, following heat trails provides a +3 bonus to Tracking if the trail is no longer than an hour old (varied based on environment). It can also act as quality equipment for uses that would benefit from infravision, including a +2 to Naturalist and Prospecting, +2 to Artist (Pottery or Sculpting) or Metallurgy for heat-based tasks, +2 to vision-based Forensics, Observation, or Search, and a +4 to Vision or Observation vs Camouflage or Disguise.
		
		Color is not discernible through the infrared spectrum, although certain colors emit and absorb heat better and may be discernible with a roll at \textit{at least -4}. A Vision at -4 can also be used to read via reflected heat. Distinguishing between objects of similar sizes and shape is difficult, requiring a roll at -4. 
		
		Bright flashes of infrared heat can blind you exactly as very bright lights, often ocurring from bursts of fire, explosions, high powered lasers, and often.. very bright lights (especially cheap ones).
	
		\textcolor{OliveGreen}{\textit{Statistics: Infravision (Magical, -10\%) [9] }}
			
	\subparagraph{Ultravision [9 Points]}
	
		The adept can see through the \textcolor{NavyBlue}{\href{https://en.wikipedia.org/wiki/Ultraviolet}{entire ultraviolet spectrum}}, which has some useful capabilities; it allows them to see better in low light conditions (ultraviolet is often present from sunlight and artificial lights), see \textit{active} ultraviolet scanners (such as blacklights), and see UV lasers (uncommon options for communication and lethal purposes). UV notably does not pass through windows or normal visors, although it is possible to specially make ones that do.
		
		The adept can see better in low-light conditions where UV is present (e.g. sunlight, even under cloud cover and artificial lights), letting them ignore -2 in darkness penalties, although you still cannot see with no light. It penetrates water well, halving Vision penalties underwater. Wherever UV is present, the wider spectrum provides a +2 to Vision rolls, as well as vision-based Forensics, Observation, and Search. The power can also act as quality equipment for uses that would benefit from seeing UV, including a +2 bonus to Naturalist or Prospecting and a +4 to Vision or Observation vs Camouflage or Disguise.
	
		\textcolor{OliveGreen}{\textit{Statistics: Ultravision (Magical, -10\%) [9] }}
			
	\subparagraph{Hyperspectral Vision [23 Points]}
	
		This combines the benefits of Infravision and Ultravision. The adept gains a +3 bonus to Vision, and vision-based Forensics, Observation, Search, and Tracking. It also acts as quality gear, as described in the respective traits. In complete darkness, the power acts as Infravision.
	
		\textcolor{OliveGreen}{\textit{Statistics: Hyperspectral Vision (Magical, -10\%) [23] }}
	
	\subparagraph{Telescopic Vision [5/10/15/19] Points}
	
		This power allows the adept to magnify their vision in similar ways to using binoculars or a scope. Each level lets them always ignore -1 in range penalties to Vision rolls, or -2 per level if they take an Aim maneuver. Alternatively, the adept may use the power similarly to a variable-power scope, giving +1 Accuracy for each second Aiming, to a maximum equal to their level. The benefits of this \textit{are} cumulative with technological aids (lenses \textit{do} stack magnification), as long as they are \textit{purely optical, not digital}; as well the GM may assign darkness penalties equal to the lower bonus of the power and the equipment (Which may be overcome with powers or other traits.
		
		There are two alternative powers as well; one makes it so that the adept \textit{must} Aim for a number of seconds equal to their levels to gain any Accuracy bonus, and costs [5/9/14/19]. Alternatively, they many give up their ability to gain Accuracy bonuses at all, costing [2/3/4/6].
	
		\textcolor{OliveGreen}{\textit{Statistics: Telescopic Vision (Variable, +5\%; Magical, -10\%) }}
	
	\subparagraph{Discriminatory Hearing [14 Points]}
	
		The resolution for the adept's power increases immensely, improving their ability to notice, distinguish, and differentiate between sounds of any kind. This allows them to always distinguish between similar sounding, but unique objects, about as easily as a human can do by voice, via a single Hearing roll; they can differentiate between two car engines, or firearms, or so on. The adept is even adept at remembering such sounds too, requiring a minute and a successful IQ roll to commit to memory, requiring 24 hours of wait on a failiure.
		
		This increase in resolution provides a +4 bonus to all Hearing rolls (effectively improving effective hearing to 8 times as far) and a +4 bonus to Shadowing noisy targets. It can also stand in for quality gear in certain circumstances.
	
		\textcolor{OliveGreen}{\textit{Statistics: Discriminatory Hearing (Magical, -10\%) [14] }}
	
	\subparagraph{Subsonic Hearing [5 Points]}
	
		The adept gains the ability to hear very low frequency sound waves (Below 40 Hz). These sounds are present in many large or heavy circumstances, including large weather events (tornadoes, hurricanes, etc), large ground events (earthquakes, volcanoes, etc), large bodies of water moving (waves, waterfalls, etc.), large movements of creatures (from either large animals or a large number of them), many animals calls (whales, elephants, hippos, tigers, cats, etc.), sonic booms and explosions, and some machines (especially larger and mechanical ones). 
		
		Wherever subsonic sounds are available, the adept can make Sense rolls to notice and identify them as they would with normal hearing. The sounds themselves can travel around twice as far as normal sounds and penetrate through buildings, the ground, and water readily. This may provide bonuses to normal hearing, counting the source as being louder or closer than it would normally with standard hearing - or simply allow the check at a longer range. The power always provides a +1 bonus to Tracking if the quarry is moving over the ground. 
		
		The power is able to stand in for fine-quality equipment in many cases, allowing the adept to claim bonuses even without the equipment. It provides +2 to Geology for detecting earthquakes, +2 to Meteorology, and +2 to Survival to locate large animals.
	
		\textcolor{OliveGreen}{\textit{Statistics: Subsonic Hearing (Magical, -10\%) [5] }}
	
	\subparagraph{Ultrasonic Hearing [5 Points]}
	
		The adept gains the ability to hear very high frequency sound waves (Above 20 kHz). This allows them to hear things from dog whistles to sonar. 
		
		Sonar itself comes in many forms; bats use it for echolocation (although many other animals make sounds in the range too, such as dogs, birds, and crickets), some motion detectors use it (as opposed to IR), sonar rangefinders are an alternative to lasers, it is used for general sensors as well (ultrasounds, some anti-collision sensors, and so on). Like all passive detectors, the adept may detect Active Sonar out to twice its effective range at no penalty.
	
		\textcolor{OliveGreen}{\textit{Statistics: Ultrahearing (Magical, -10\%) [5] }}
	
	\subparagraph{Parabolic Hearing [4/8/11/15/ 19/22] Points}
	
		This power helps to adept to collect and filter out sound waves, allowing him to hear sounds from further away and through more background noise. Each level \textit{doubles} the distance at which a sound can be heard with no penalty (B358, P:ES21), effectively negating -1 in range modifiers per level. It also screens out background noises (HT158), either ignoring one source per level \textit{or} ignoring -2 in penalties per level (The GM should choose one and/or the other, as these are implied, but unofficial effects).
	
		\textcolor{OliveGreen}{\textit{Statistics: Parabolic Hearing (Magical, -10\%) [3.6 per level] }}
		
	\subparagraph{Discriminatory Smell [14 or 21] Points}
	
		The adept acquires immense resolution with his sense of smell, improving their ability to notice, discern, and differentiate between smells. This allows them to distinguish between the smells of people, objects, and even places. The adept may even memorize smells, requiring a minute and a successful IQ roll, requiring 24 hours of wait for a failure.
		
		This increase in resolution provides a +4 bonus to all Smell rolls and on any tasks that relies on smell, which always includes the Tracking skill.
		
		Some adepts have the ability to determine emotional state via hormones and physiological responses. This functions like Empathy (B51, all of Social Engineering) while you are able to smell them well (usually within 2 yards). This power costs [21] Points.
	
		\textcolor{OliveGreen}{\textit{Statistics: Discriminatory Smell (Magical, -10\%) [14]. Optionally adds Emotion Sense, +50\% [21] }}	
		
	\subparagraph{Discriminatory Taste [9] Points}
	
		This power functions very similarly to Discriminatory Smell due to their linked systems. However, the adept is required to ingest a substance to use it, usually being bodily fluids. They can also make an IQ roll to recognize the substance, identify whether it is safe, etc. It is also possible to perform analysis using other skills, such as Cooking, Chemistry, Pharmacy, Poisons, etc.
		
		It provides a +4 bonus to Taste rolls, as well as any task that relies on Taste, such as most Cooking rolls.
		
	\subparagraph{Vibration Sense [9] Points}
	
		The adept is able to detect the locations, speed, and size of objects based on their vibrations. This power only works in Air \textit{or} Water, although a version is available for both at [14] Points.
		
		The power is not precise enough to act as a replacement for sight; the adept can locate moving objects in the dark, but cannot tell most anything about them (e.g. whether they are armed, what they look like, etc). In a perfectly still area, it is still possible to have a hazy understanding of large objects and openings purely on the flow of the air or water, which is enough to avoid barriers before running into them.
		
		To make a Sense roll, consult the Size and Speed Table (B550) and apply \textit{separate bonuses} for size and speed alongside a penalty for range. Wind or current will interfere with your sense similarly to fog or smoke; look up the air speed on the table and apply it as a penalty.
		
		A successful Sense roll reveals the rough size, speed, location, and direction - which allows you to target it with an attack - but nothing about shape, colour, etc. The modifiers to your sense roll apply to any attacks (exactly as with Vision), but can never provide a \textit{bonus}.  
	
		\textcolor{OliveGreen}{\textit{Statistics: Vibration Sense (Magical, -10\%) [9] }}
	
	\paragraph{Indomitable Will}
	\begin{flushright}
		2/4/6/8/9/11/13/14 Points
	\end{flushright}

		The adept's will is hardened by this power, reducing his susceptibility to fright. Each level in this power provides a +1 bonus to Fright Checks or resistance rolls to the Intimidation skill, while also acting as a penalty to the opponents Intimidation skill. 
		
		At level 8, the adept becomes \textit{immune} to fear in all forms. They are exempt from Fright checks and most reaction modifiers alongside Intimidation failing (unless the opponent has the Empathy advantage).

		\textcolor{OliveGreen}{\textit{Statistics: Fearlessness (Magical, -10\%) [1.8 per level] last level is Unfazeable (Magical, -10\%) [14] }}

	\paragraph{Inertia Strike}
	\begin{flushright}
		Points
	\end{flushright}

		This power allows the adept to channel additional momentum into their target after a hit, making it easier to knock them around. Each level of this power adds 1d cr dbkb to their unarmed or melee attacks, purely for the purposes of determining Knockback (B378).

		\textcolor{OliveGreen}{\textit{Statistics: Innate Attack, Cr (Double Knockback, +20\%; Follow-Up, Universal (Accessibility, Unarmed and Melee Only, -20\%), +40\%; Magical, -10\%; No Wounding, -50\%) [5] }}
	
	\paragraph{Iron Gut}
	\begin{flushright}
		4/5/6/7 Points
	\end{flushright}

		The adept becomes able to eat just about anything, reducing both his need for life-support, food, and water, while also improving his ability to resist ingested toxins. 
		
		This lowers the quality of such things to 2/3, 1/3, 1/20, and 1/100 for each level. This should generally lower cost of living by 4\%, 8\%, 10\%, and 12\% respectively for lowered food costs (it's cheap to eat out of the trash!). Additionally, each level provides a +1 bonus to resist food-borne toxins and diseases.
		
		Finally, the power provides a +3 bonus to resist Ingested Toxins (Which includes food-borne toxins, but covers some more things).

		\textcolor{OliveGreen}{\textit{Statistics: Reduced Consumption (Cast-Iron Stomach, -50\%; Magical, -10\%) [0.8 per level] and Resistant, Ingested Toxins (+3, x1/3; Magical, -10\%) [3]}}
	
	\paragraph{Iron Will}
	\begin{flushright}
		4/8/11/15/18/22 Points
	\end{flushright}

		This power helps protect and alert the adept to mental intrusions, such as from mind altering spells, adept powers, and critter powers, even while unconscious.
		
		Each level adds a +1 bonus to resist mental attacks, including but not limited to the Control Thoughts spell, Influence Critter Power, and Commanding Voice Adept Power. It also resists attempts to locate the adept's mind; the opponent must win a Quick Contest against the adept's Will + Magic + Mind Shield to find them. As well, the adept is alerted by any \textit{failed} attack on their mind, but does not necessarily know anything more than that they were attacked.
		
		The adept can lower this as a free action at the start of their turns, allowing friendly magicians to read their minds or for similar activities.

		\textcolor{OliveGreen}{\textit{Statistics: Mind Shield (Magical, -10\%) [3.6 per level] }}
	
	\paragraph{Kiai}
	\begin{flushright}
		27/36/44/51/58/64/69/72 Points
	\end{flushright}

		The adept can release a bloodcurdling cry that can terrify \textit{anyone} who hears it - even allies.
		
		When the adept uses a free action\footnote{It's a bit difficult to determine whether it's a Free Action or Ready Maneuver to activate Terror, but because it's so expensive we're erring on the side of caution.} to turn on this power, anyone that is able to hear their cry must immediately make a Fright Check (B360), with normal Fright Check modifiers as applicable. Failure means the target must roll on the Fright Check Table as normal. 
		
		The adept can keep this power on for as long as they can howl, which may cause repeat fright checks against those who begin the hear them, failed to resist the last time, or so on.
		
		If the target succeeds on their result, they are immune to the adept’s power for 1 hour; targets also gain a +1 bonus to resist for every time the power has affected them in the past 24 hours. 
		
		After activating the power, the adept must resist a number of drain equal to the levels they have in the power.
		
		At higher levels, their battle cry becomes extremely terrifying. Impose a -1 penalty to their Fright Check for each level beyond the first.

		\begin{center} 
			\begin{adjustwidth}{-0mm}{}
				\scalebox{1.0}{
					\begin{tabular}{|c|c|c|}
						\hline
						Level & Penalty & Cost\\
						\hline
						\hline
						1 & +0 & 27 \\
						2 & -1 & 36 \\
						3 & -2 & 44 \\
						4 & -3 & 51 \\
						5 & -4 & 58 \\
						6 & -5 & 64 \\
						7 & -6 & 69 \\
						8 & -7 & 72 \\
						\hline
					\end{tabular}
				}
			\end{adjustwidth}
		\end{center}
	
		\textcolor{OliveGreen}{\textit{Statistics: Terror, Hearing 1 (Drain FP; Magical, -10\%) [27]}}
	
	\paragraph{Killing Hands}
	\begin{flushright}
		7/14 Points
	\end{flushright}

		The adept's hands become finely tuned killing machines, granting them a number of surreal benefits to a normal martial artist.
		
		First, they add +1 crushing damage per die to the adept's unarmed punch damage. Additionally, the adepts limbs \textit{are treated as if they were weapons}. This allows them to parry weapons with no unarmed parry penalty (B377) - regardless of skill. This also imposes the same penalty to opponents attempting to parry the adept unarmed - as if he were wielding a melee weapon such as a sword. 
		
		The adept is able to use techniques or styles that might require a weapon as opposed to to unarmed attacks, although the GM should confirm that each of these at least makes physical sense (Although lean on the side of: If a Wuxia movie would allow it, it can probably be done here too!).
		
		At the second level, the adept can also include his legs in this power, applying these benefits to them as well. This allows them the same benefits, but also notably allows the adept to parry with their legs - although the GM is free to impose rolls to keep balance, avoid falling, or so on, in a similar way to the DX roll for missed kicks.

		\textcolor{OliveGreen}{\textit{Statistics: Two CR Strikers (Limb, -20\%; Magical, -10\%) [3.5 each] further levels double it for legs. }}
	
	\paragraph{Light Body}
	\begin{flushright}
		5/9/14/18/23/27 Points
	\end{flushright}

		By temporarily decreasing his body's effective weight, the adept can achieve amazing feats of jumping.
		
		Taking this power multiplies the adept's jumping distance and height (B352) by $\times$1.5. The adept's Move while jumping their the greater of his normal Move and 1/5 their long jump distance. This also allows for the adept to slam into a foe by jumping, using his maximum jump as his move. Lastly, falling a distance equal to or less than the adept's maximum high jump deals \textit{no damage}, and can be increased by Catfall or the Acrobatics skill as normal.
		
		Higher levels increase this power exponentially. At the next level increase the multiplier to $\times$2, and continue multiplying as for the previous levels for any higher levels (e.g. Level 5 is $\times$2$\times$2$\times$1.5 ($\times$6 total)).

		\textcolor{OliveGreen}{\textit{Statistics: Super Jump 1/2\footnote{We're taking half levels here (Similarly to Enhanced Move), to allow for more resolution in levels.} (Magical, -10\%) [4.5 per level]}}

	\paragraph{Linguistics}
	\begin{flushright}
		4/5 Points
	\end{flushright}

		The adept gains the ability to temporarily pick up languages. After being exposed to a language for 1 hour, the adept can temporarily gain Broken in the language, allowing them decent communication abilities. Should they acquire another language, they lose their already memorized language from this power.
		
		At higher levels, the adept gains Accented instead of Broken, allowing much better communication abilities.
		
		The adept can also use this to improve their already existing language skills, raising by 1 or 2 categories (e.g. Broken$\rightarrow$Accented or Broken$\rightarrow$Native) respectively.
		
		This power can be purchased multiple times, allowing the adept to remember as many languages as were purchased alongside swapping out whichever languages they please when memorizing new ones.

		\textcolor{OliveGreen}{\textit{Statistics: Modular Ability, 4 Points per slot, 3 Points per point (Slow, External) (Magical, -10\%; Requires IQ Roll, -10\%; Trait-Limited, Languages Only, -30\%) [4] higher levels add 1 Point as [5]}}
	
	\paragraph{Magic Sense}
	\begin{flushright}
		Points
	\end{flushright}

		\textcolor{OliveGreen}{\textit{Statistics: Detect, Magic (Magical, -10\%) }}
	
	\paragraph{Master Archer}
	\begin{flushright}
		23 Points
	\end{flushright}

		This power allows the adept to perform truly heroic feats when using the Bow skill, in ways that would seem straight out of a Trid. Due to it's cinematic nature, the GM should decide whether this power is allowed - especially if they are intending to run a Black Trenchcoat style of game.
		
		When performing an Attack or All-Out-Attack, the adept adds the bow's Acc \textit{without taking an Aim maneuver}. If they do aim, they gain the normal benefits for aiming multiple seconds.
		
		When instead performing a Move-and-Attack, the adept instead ignores the Bow's bulk penalty, alongside any penalty for Flying Attacks (MA107) or Acrobatic Attacks (MA107). They may also ignore Bulk in Close Combat, however they do not gain any Acc bonus in such a case.
		
		Adepts with this power are particularly adept at rapid firing arrows, and \textit{halve} the penalties for Quick Shooting Bows (MA119), \textit{regardless of the maneuver}. When performing such a feat with an Attack or All-Out-Attack maneuver, add Acc to skill as normal, but not to the skill roll to ready the bow hastily. 
		
		In an overall similar fashion, half the penalties for Dual-Weapon Attack (MA83) when shooting two arrows at once.
		
		Finally, halve all Fast-Draw penalties, rounded down.

		\textcolor{OliveGreen}{\textit{Statistics: Unusual Background\footnote{Heroic Archer is a \textit{very powerful} ability, even in a setting where guns exist. While arrows might not be as effective as a bullet, the ability to place them extremely accurately in rapid fashion, combined with technological bows, all calls for an Unusual Background Tax.} [5] and Heroic Archer (Magical, -10\%) [18] }}
	
	\paragraph{Metabolic Control}
	\begin{flushright}
		5/9/14/18/23/27 Points
	\end{flushright}

		The adept is able to control their metabolic functions to a certain degree, allowing them to aid natural processes and even enter a deathlike trance for long periods.
		
		Due to the adept's increased control over pulse, blood flow, digestion, and respiration, they gain a +1 bonus to HT rolls for tasks that would benefit from these. This is often context dependant, such as lowering pulse rate to stop the spread of sepsis, and in complex situations may require a skill roll (Physiology, Physician, etc.) to determine the best thing to do. However, it should always apply to Bleeding rolls and rolls to recover (not resist) from Toxins and Disease.
		
		The adept can also enter into a form of hibernation. In such a trance, it is hard to discern whether the adept is still alive, requiring a successful Quick Contest of Diagnosis vs HT + Magic + Metabolic Control. In such a state, each level reduces the adept's oxygen needs by 10\% and \textit{doubles} the amount of time they can go without food and water.	The adept is unaware of their surroundings in such a state, but will still awaken if injured. 

		A mental "alarm clock" can be set to awaken them after a certain amount of time has passed.
		
		The GM should consider whether the allow 10 levels of this ability or not, due to the \textit{extreme} conditions it allows hibernation in (for one, the adept would no longer need oxygen!), although these are not unprecedented levels of control even in the real world! 

		\textcolor{OliveGreen}{\textit{Statistics: Metabolism Control (Magical, -10\%) [4.5 per level] }}

	\subsection{Spirits}
	
	Spirits are mysterious beings originating from the Metaplanes, coming in an as wide a variety of shapes and minds as humans and animals, if not more so. They are categorized into a number of wide categories, however the specifics of their form, mentality, and sometimes powers vary widely by the summoner's tradition, such as angels for Christian Theurgists. As well, in the metaplanes themselves, it can vary \textit{even more}, including mythical beasts, dinosaurs, dead humans (supposedly), or worse beings such as Shedim and Invae.
	
	For game purposes, spirits are summoned using the \hyperref[summoning]{Summoning} and \hyperref[binding_spirits]{Binding} advantages. Their abilities down below are used to determine their cost for pricing as Allies, all of which must have the Appears Constantly, Special Abilities, Summonable, and Favor modifiers - with expections for certain cases such as unwilling summons, free spirits, wild spirits, and so on.
	
	Because there is no set starting points for a campaign, the cost are as follows (with some additional interpolation), with an additional Force Guide for 200 and 100 points:
	\begin{center} \label{spirit_ally_cost}
		\begin{tabular}{|c|c|c|c|}
			\hline
			\% Points & Cost & 200 Pts & 100 Pts\\
			\hline
			\hline
			5\% & 1 & Force 1 & Force 1 \\
			10\% & 1 & - & - \\
			15\% & 2 & Force 2 & - \\
			20\% & 2 & - & - \\
			25\% & 2 & Force 3 & Force 2 \\
			40\% & 3 & Force 4 & - \\
			50\% & 4 & Force 5 & Force 3 \\
			65\& & 5 & Force 6 & - \\
			75\% & 6 & Force 7 & Force 4 \\
			90\% & 8 & Force 8 & - \\
			100\% & 10 & Force 9 & Force 5 \\
			115\% & 13 & Force 10 & - \\
			125\% & 15 & Force 11 & Force 6 \\
			140\% & 18 & Force 12 & - \\
			150\% & 20 & Force 13 & Force 7 \\
			165\% & 23 & Force 14 & - \\
			175\% & 25 & Force 15 & Force 8 \\
			190\% & 28 & Force 16 & - \\
			200\% & 30 & - & Force 9 \\
			225\% & 35 & - & Force 10 \\
			250\% & 40 & - & Force 11 \\
			275\% & 45 & - & Force 12 \\
			300\% & 50 & - & Force 13 \\
			325\% & 55 & - & Force 14 \\
			\hline
		\end{tabular}
	\end{center}
	
	\subsubsection{Spirit Meta-Trait}\label{spirit_meta_trait}
	
	All Spirits share some fundamental qualities that define them as spirits, although even this can be hard to fundamentally nail down due to the tumultuous nature of magic. The GM should feel free to both edit this to their liking, but also to edit it even within the realm of lore, as all attempts to nail down what makes a spirit a spirit have so far, failed.   
	
	\begin{flushright}
		69 Points
	\end{flushright}
	\textbf{Advantages:} 
	\hyperref[astral_perception]{Astral Perception [12]}; Doesn't Breath [20]; Doesn't Eat or Drink [10]; Doesn't Sleep [10]; Flight (Planetary, -5\%; Magical, -10\%; Slow, Basic Move, -25\%) [24]; Injury Tolerance (Unbreakable Bones) [10]; Immunity, Metabolic Hazards [30]; \hyperref[spark]{Spark [4]}; Telecommunication, Telesend (Full Communion, +20\%; Accessibility, Summoner Only, -80\%; Magical, -10\%) [9]; Unaging [15]; Unusual Background (Spirit)\footnote{This is a small Unusual Background meant to represent the spirits ability to take Critter powers, Injury Tolerance, etc.} [5]
	\\\textbf{Disadvantages:} 
	Dependency, Mana (Very Common, -5; Constantly, \(\times\)5) [-25]; Fragile (Unnatural) [-50]; Unusual Biochemistry [-5]
	
	\subsubsection{Spirit Force}\label{spirit_force}
	\begin{flushright}
		9.5 Points per Force
	\end{flushright}
	
	More powerful spirits are highly magical and increasingly hard to damage. This trait provides +1 DR when manifesting against physical attacks and increases the spirit's Magic.
	
	As a spirit's force increases, they have a number of traits that increase alongside it. This trait is automatically applied for higher force spirits and does not count towards any CP bonuses per Force.
	
	\textbf{Advantages:}
	Damage Resistance 1 (Force Field, +20\%; Accessibility, Only when Manifested/Possessing, -20\%; Limited Defenses, Physical Attacks, -20\%; Can't Wear Armour, -40\%) [2]; \hyperref[magic]{Magic [7.5]}
	
	\subsubsection{Spirit Morphology Traits}
	
	For spirits that take their own form, which can vary among many of traditions and summoners. There are some constant themes that most spirits tend to follow though:
	
	\subsubsection*{Elemental}\label{elemental}
	\begin{flushright}
		7 Points
	\end{flushright}
	
	Elemental covers most spirits that are fundamental forces, such as Fire, Air, Earth, etc. For most traditions, these take the form of representation of those forces, which means that they lack almost any form of convential morphology.
	
	These are almost universally given the Injury Tolerance (Homogenous) trait, but the GM is technically within their right to use Injury Tolerance (Diffuse). This is, however, discouraged; while there is not much lore on exactly \textit{how} a spirit manifestation works, it is likely similar in many ways to how magic forms mana into a physical form for spellcasting. This could imply that they are more "mana" than element, meaning that they are more amorphous blob of mana than trying to hurt "fire". This is additionally beneficial, because diffuse is \textit{much more powerful} than any lore would support.
	
	Notably, while the spirit has No Head, to No Brain trait is left to the \hyperref[materialization]{Materialization Section}.
	
	\textbf{Advantages:} 
	High Pain Threshold [10]; Injury Tolerance (No Blood, No Eyes, No Neck, No Head\footnote{Does not include no Brain, see Materialization section.}) [17]; Peripheral Vision [15]
	
	\textbf{Disadvantages:}
	No Fine Manipulators [-30]; No Sense of Smell/Taste [-5]
	
	\textbf{Traits:}
	Non-Standard Morphology\footnote{Any rules that rely on physiology modifiers are heavily penalized as normal.} [0]
	
	\subsubsection*{Bodily}\label{bodily}
	\begin{flushright}
		5 or 0 Points
	\end{flushright}
	
	These spirits have actual bodily morphology, often manifesting in forms like humans or animals. While they still lack anything that might be considered a biology, they do have a number of morphological traits that are valid targets nevertheless.
	
	Some spirits with this style of morphology have inferior forms (e.g. Quadraped), which are often compensated with superior senses.
	
	\textbf{Advantages:} 
	Injury Tolerance (No Blood) [5];
	
	\textbf{Disadvantages:}
	\textit{\\Choose where applicable:\\}
	Biped [0] or\\
	Quadraped [-35]; Discriminatory Hearing [15]; Discriminatory Smell [15]\\
	
	
	\subsubsection{Spirit Type Traits}
	
	Different traditions have differing spirit types, which affect how they interact with the material plane.
	
	\subsubsection*{Inhabitation}
	
	Inhabitation spirits are, at a glance, very similar to Possession spirits, however these are much more insidious. While a Possession spirit is like someone squatting in your house, an Inhabitation spirit is a home invasion, murdering the owner and posing as them.
	
	Due to this nature, they are very often limited to outer spirits, such as Invae and Shedim.
	
	% TODO: This. Include dependency on outer planes w/ it stopping while.. possessing probably.
	
	\subsubsection*{Materialization}\label{materialization}
	\begin{flushright}
		62 Points
	\end{flushright}
	
	Spirits who materialize create a physical body from mana, actually creating a physical body on the material plane. 
	
	Their bodies are not anything that would approach normal biology, which makes them particularly hard to damage without magical effects; they are treated as Homogenous to any physical attack (Although, do note that their No Brain and No Vitals remains versus magical attacks regardless).
	
	The spirit can jump between the two planes. The requires 10 seconds of concentration, 1 FP, and an IQ roll. Success jumps to the corresponding time on the astral plane. The rules for the plane can be found under the \hyperref[astral_projection]{Astral Projection advantage above}. 
	
	This is a more difficult task than projection, due to the fact that the spirit is actually moving between the planes themselves, which is why it has a different modifiers. 
	
	\textit{\textcolor{NavyBlue}{Modifiers: +1 per Level of Magic, -1 per second of less concentration.}}
	
	As well, the spirit's manifestion does not innately allow for tag-alongs or following. While manifesting, the spirit is Dual-Natured.
	
	\textbf{Advantages:} 
	Injury Tolerance (No Brain, No Vitals) [10]; Injury Tolerance (Homogenous) (Limited Defense, Physical Attacks, -20\%)\footnote{This Injury Tolerance (Homogenous) lacks No Brain and No Vitals, because those aspects are NOT ignored by magical attacks.} [32]; Jumper, Astral (Improved, +10\%; Cannot Escort, -10\%; Cannot Follow, -20\%; Naked, -30\%; Limited Access, Astral Plane \& Home Plane, -15\%; Magical, -10\%; Nuisance Effect, Dual-Natured, -5\%) [20]
	
	\subsubsection*{Possession}
	\begin{flushright}
		62 Points
	\end{flushright}
	
	Possession spirits are unable to interact with the material plane on their own. They must possess people or object in order to do so.
	
	The spirit can only posses something from that astral plane (no switching from body to body). To do so, they must All-Out Concentrate for 1 minute, touch the subject, and roll IQ vs the subject's Will, if living, or HT if unliving (The GM may make exceptions to use Will for objects guided by intelligences, such as AI or even Pilot programs). If the spirit fails, the subject is immune to their possession attempts for 24 hours.
	
	The spirit takes a -3 penalty if trying to possess someone with a good amount of cyberware or a highly manufactured item, or a -6 penalty if trying to possess someone with a lot of cyberware or a highly technological or electrical item.
	
	While possessing, the spirit uses the subject ST, DX, and HT (or their own DX for inanimate objects), as well as their physical advantages and disadvantages. They keep their mental attributes, advantages, and their skills, although physical skills must use their new attributes.
	
	For the purposes of possessing an unliving object, the spirit needs a level of TK equal to the HP of an unliving object's HP, or half the HP of a homogeneous object. The object itself can grab, lift, strike, etc. with a ST equal to the TK required to lift it, move and jump if not fixed (with a move equal to the spirit's TK level minus the minimum level), etc. Objects that have special modes of transport may use them (such as flying a RC helicopter).
	
	The spirit does not have any particular access to a living beings or electronic system's memories, nor do they have utter control over any electric system that they possess (Although they do have control of its physical body).
	
	While possessing, the spirit may still be affected by anything that can normally affect a manifested spirit. Attacks that penetrate or ignore the subject’s DR can injure them, but the subject's HP act as extra DR for this purpose. 
	
	As well, an awakened individual with the \hyperref[banishing]{Banishing skill} can force the spirit to leave by winning a Quick Contest of Banishing vs. the Spirit's Will.
	
	Do note that the as the spirit can still be targeted normally, they do count as being Dual-Natured.
	
	\textbf{Advantages:} Possession (Decreased Immunity, +50\%; Link, TK, +10\%; All-Out Concentrate, -25\%; Environmental, Astral Only, -20\%; Hard to Use 2 (Accessibility, Technology Only, -20\%), -8\% Immediate Preparation Required, 1 Minute, -30\%; Spiritual, -20\%; Magical, -10\%; No Memory Access, -10\%) [37]; TK 25 (Link, Possession, +10\%; Accessibility, Only things being possessed, -40\%\footnote{Accessibility based on the estimate of less than 10,000 useful items available at any given moment, but more than 17 (6\%)}; Animation, -20\%; Cannot Affect Self, -20\%; Cannot Punch, -10\%; Magical, -10\%) [25]\\
	
	While Basic Set p558 will likely be an invaluable resource for determining HP based of weight, some times the GM may wish to know the weight of an object from its HP. A table below is provided for up to 25 HP.
	
	\begin{center}
		\begin{tabular}{|c|c|c|}
			\hline
			HP & Unliving & Homogenous\\
			\hline
			\hline
			1 & 0.02 lb & 0.002 lb\\
			2 & 0.125 lb & 0.156 lb \\
			3 & 0.42 lb & 0.525 lb \\
			4 & 1 lb & 0.125 lb \\
			5 & 1.95 lb & 0.24 lb \\
			6 & 3.375 lb & 0.42 \\
			7 & 5.35 lb & 0.66 lb \\
			8 & 8 lb & 1 lb \\
			9 & 11.39 lb & 1.42 lb \\
			10 & 15.625 & 1.95 lb \\
			11 & 20.8 lb & 2.6 lb \\
			12 & 27 lb & 3.375 lb \\
			13 & 34.33 lb & 4.29 lb \\
			14 & 42.875 lb & 5.36 lb \\
			15 & 52.73 lb & 6.59 lb \\
			16 & 64 lb & 8 lb \\
			17 & 76.77 lb & 9.6 lb \\
			18 & 91.125 lb & 11.39 \\
			19 & 107.17 lb & 13.4 lb \\
			20 & 125 lb & 15.625 lb \\
			21 & 144.7 lb & 18.09 lb \\
			22 & 166.38 lb & 20.8 lb\\
			23 & 190.11 lb & 23.76 lb \\
			24 & 216 lb & 27 lb \\
			25 & 244.14 lb & 30.57 lb \\
			26 & — & 34.32 lb \\
			27 & — & 38.44 lb \\
			28 & — & 42.89 lb \\
			29 & — & 47.63 lb \\
			30 & — & 52.73 lb \\
			31 & — & 58.19 lb \\
			32 & — & 64 lb \\
			33 & — & 70.19 lb \\
			34 & — & 76.77 lb \\
			35 & — & 87.74 lb \\
			36 & — & 91.125 lb \\
			37 & — & 98.93 lb \\
			38 & — & 107.17 lb \\
			39 & — & 115.86 lb \\
			40 & — & 125 lb \\
			41 & — & 134.61 lb \\
			42 & — & 144.7 lb \\
			43 & — & 155.29 lb \\
			44 & — & 166.38 lb \\
			45 & — & 177.98 lb \\
			46 & — & 190.11 lb \\
			47 & — & 202.78 lb \\
			48 & — & 216 lb \\
			49 & — & 229.78 lb \\
			50 & — & 244.14 lb \\
			\hline
		\end{tabular}
	\end{center}
	
	\subsubsection*{Bound Spirit}
	\begin{flushright}
		-5 Points
	\end{flushright}
	
	\textbf{Disadvantages:} Sense of Duty (Summoner) [-5]
	
	
	\subsection{List of Spirits}
	
	Here are a list of templates for known spirits. A conjurer does not necessarily have access to all of them, and when creating a character should get with their GM to determine which spirits are available to their tradition.
	
	Every Spirit has a Force rating when summoned. The base template represents \textit{Force 0}, and underneath are a list of improvements that can be added when summoning Force 1 and above (See Spirit Powers below), alongside limitations to those options for each type. Each increase of Force provides an amount of CP that is determined by the spirit Type, which can be chosen from the list, or any other trait that the GM approves; spirits gain a substantial amount of free points at Force 1 that can be spent on any of these improvements as normal, which include any normal CP bonuses for higher forces and \textit{does not} include Spirit Force for Force 1. Some powers are marked with a '+' sign, indicating that they have higher costs for Magic levels above 1.
	
	One important departure from Shadowrun is that spirits \textit{do not} come with a plethora of powers out of the gate. They must instead be purchased using either the starting points or using points for higher Forces. Not all of the points need to be spent. Notably, all Spirits are able to improve their base attributes with any points available. Each attribute can only be increased by an amount equal to Force (e.q Force 3 lets one increase each attribute by up to 3). Exceptions to this are Basic Speed (Up to +1.0 total), Basic Move (Up to +2), and HP/FP (Up to +30\% of (Their Base HP + Force)) - for all of these round down.	
	
	Some spirit types might have certain traits that are automatically improved at higher levels of Force if taken, in which case they will be noted there. While the GM technically decides and is the final arbitrator on what traits a spirit is summoned with, it is recommended to allow the summoner to choose what they want, as long as it seems reasonable.
	
	Additionally, all increases of Force automatically increase the level of the \hyperref[spirit_force]{Spirit Force trait} from above, which does not count against the spirit's CP for each Force.
	
	Some spirits are given multiple templates to choose from (Such as a Spirit of Water's liquid or frozen forms). These must be determined when summoning and should generally not change without GM discretion (i.e. a frozen Spirit of Water won't generally melt into a liquid form).
		
	\subsubsection{Spirit of Fire}
	\begin{flushright}
		-69 Points Force 0
	\end{flushright}
	
	A Spirit of Fire takes, unsurprisingly, the form of something heavily related to fire. Most often, especially for tradition close to hermetics, this takes the form of some sort of fire elemental, force, etc. For shamanic ones, this could be a bit less elemental (Although many still are), taking the form of fire associated things or creatures, such as a Phoenix. In such a case, it is acceptable to switch \textit{Elemental [7]} with \textit{Bodily [5/0]}, and take either +1 HP/+1 ST respectively.
	
	Spirits of fire are apt combat spirits, with above average Perception and Dexterity, while on the flip side having some of the more accessible countermeasures. They are naturally resistant to sources of heat or fire, but there is a limit as extraordinarily dangerous sources can still affect their materialized form.
	
	They have the capability to induce accidents and confusion in opponents, or for less subtle effects can fling fireballs, burn people at the touch, or even completely engulf them in the flames of their body.
	
	Unlike many spirits, they do require oxygen to materialize well, losing FP as per suffocation otherwise. As well, their form is more draining in the presence of water (or any general fire-retardant chemical), causing 1d HP damage per minute of exposure.
	
	\textbf{Attributes:}
	ST 4 [-36]; DX 8 [-40]; IQ 6 [-52]; HT 7 [-39]; HP 9 [10]; Basic Speed 4.75 [20]; Basic Move 5 [5]; Per 7 [-15]; Will 6 [-28]; FP 7 [0]
	
	\textbf{Advantages:}
	Damage Resistance 9 (Limited, Heat/Fire, -40\%) [27]; \hyperref[elemental]{Elemental [7]}; Manifestation/Possession [62]; \hyperref[spirit_meta_trait]{Spirit Meta-Trait [69]};
	
	\textbf{Disadvantages:}
	Doesn't Breath with Oxygen Combustion\footnote{This updates the Doesn't Breath in the Spirit Meta-Trait to add Oyxgen Combustion, -50\% to it.} [-10]; Innumerate [-5]; Invertebrate [-20]; Weakness, Water (1d/minute) [-40]
	
	\textbf{Skills:}
	Aerobatics (H) DX [4]; Assensing (H) Per [4]; Brawling (E) DX+2 [4]; Innate Attack (E) DX+2 [4]
	
	\subsubsection*{Spirit of Fire Improvements}
	
	\textbf{\textit{Force 1 Points: [69]}}
	
	\textbf{\textit{Automatic effects at each Force:\\}}
	Damage Resistance 1 (Limited, Heat/Fire, -40\%) [3]	
	\hyperref[spirit_force]{Spirit Force [10.5+]}\\\\
	
	\textbf{\textit{Optional effects at Higher Forces, +12 CP per Force above 1:\\}}
	
	\hyperref[accident]{Accident [38]}\\
	\hyperref[confusion]{Confusion [42+]}\\
	\hyperref[elemental_attack]{Elemental Attack (Fire) [5 per Magic]}\\
	\hyperref[energy_aura]{Energy Aura [5.5 per Magic]}\\
	\hyperref[engulf]{Engulf [29+];}\\	
	\hyperref[fear]{Fear [42+]}\\
	\hyperref[guard]{Guard [12+]}\\
	\hyperref[noxious_breath]{Noxious Breath [23+]}\\
	\hyperref[search]{Search [29+]}\\	
	
	\subsubsection{Spirit of Air}
	\begin{flushright}
		 -87 Points Force 0
	\end{flushright}
	
	A Spirit of Air takes, once again unsurprisingly, the form of something heavily related to air. For hermetics, this is often a traditional air elemental, also heavily associated with lightning. For other traditions, this might also take the form of great birds or other things that represent or are associated with the air. In such a case, it is acceptable to switch \textit{Elemental [7]} with \textit{Bodily [5/0]}, and take either +1 HP/+1 ST respectively. Other possible forms include: flocks of crows or butterflies, a woman made of clouds, etc.
	
	Spirits of Air are heavily power focused, with relatively weak defense and abilities, made up for by their bulky power lists. They tend to have abysmal ST and HT, but have great DX and improved Basic Speed (Although their low HT counteracts this largely). These spirits are also extremely mobile, with a max speed of \(\times\)1.5!
	
	Their powers are highly varied and capable. They are able to cause accidents or confusion in opponents, while also able to obscure things around them from notice and also mystically search for anything that they are familiar with. Less subtly, they are able to greatly speed up or slow down almost anything flying they can see - including themselves, their summoner, or the rigger's rotodrone!
	
	Unlike other spirits, their form of flight is impacted by air currents. Wind can push the spirit in its direction by 1 yard per second for every 5 mph of wind. This can sometimes be beneficial, providing a backdraft for the spirit! However, it is more likely to affect their ability to stay exactly where they want to be at a given moment. They are also particularly vulnerable to high and low pressures. Beyond the normal effects, whenever they are in a \textit{Thin or Dense} pressure (B429), their bodies are unable to properly maintain themselves, causing 1d unresisted HP damage per minute of exposure. The GM is within their rights to assign small bonus damage to high pressure attacks, such as crushing explosives.
	
	\textbf{Attributes:}
	ST 3 [-42]; DX 9 [-20]; IQ 6 [-52]; HT 4 [-78]; HP 6 [6]; Basic Speed 4.50 [25]; Basic Move 6 [10]; Per 6 [-20]; Will 6 [-28]; FP 4 [0]
	
	\textbf{Advantages:}
	\hyperref[elemental]{Elemental [7]}; Enhanced Move, Air, \(\times\)1.5 (All-Out, -20\%; Magical, -10\%) [7]; Manifestation/Possession [62]; \hyperref[spirit_meta_trait]{Spirit Meta-Trait [69]}; 
	
	\textbf{Disadvantages:}
	Flight with Lighter than Air\footnote{This updates the Flight in the Spirit Meta-Trait to add Lighter than Air, -10\% to it.} [-4]; Innumerate [-5]; Invertebrate [-20]; Weakness, Low/High Pressures (1d/minute) [-20]
	
	\textbf{Skills:}
	Aerobatics (H) DX [4]; Assensing (H) Per [4]; Brawling (E) DX+2 [4]; Innate Attack (E) DX+2 [4]
	
	\subsubsection*{Spirit of Air Improvements}
	
	\textbf{\textit{Force 1 Points: [87]}}
	
	\textbf{\textit{Automatic effects at each Force:\\}}
	\hyperref[spirit_force]{Spirit Force [10.5 per Magic]}\\\\
	
	\textbf{\textit{Optional effects at Higher Forces, +15 CP per Force above 1:\\}}
	\hyperref[accident]{Accident [38]}\\
	\hyperref[concealment]{Concealment [12+]}\\
	\hyperref[confusion]{Confusion [42+]}\\
	\hyperref[engulf]{Engulf [29+];}\\
	\hyperref[movement]{Movement (Air) [54]}\\
	\hyperref[search]{Search [29+]}\\
	\hyperref[elemental_attack]{Elemental Attack (Lightning) [5 per Force]}\\
	\hyperref[energy_aura]{Energy Aura [5.5 per Magic}\\
	\hyperref[fear]{Fear [42+]}\\
	\hyperref[guard]{Guard [12+]}\\
	\hyperref[noxious_breath]{Noxious Breath [23+]}\\
	\hyperref[psychokinesis]{Psychokinesis}\\	
	
	\subsubsection{Spirit of Earth}
	\begin{flushright}
		-62 Points Force 0
	\end{flushright}
	
	Spirits of Earth, yet again, take the form or things heavily related to earth and nature. For hermetic styled traditions they have a wider range than normal, taking the form of dirt, sand, or sometimes even wood, stone, or metal (As detailed in the templates below). For more shamanic traditions, this can also take the form of burrowing creatures, subterranean creatures, etc. In such a case, it is acceptable to switch \textit{Elemental [7]} with \textit{Bodily [5/0]}, and take either +1 HP/+1 ST respectively.
	
	Spirits of Earth are extremely hard and strong, making them viable for many laborious tasks alongside the task of bodyguarding. They have great ST and HT, at the price of very low DX and low IQ and Basic Speed (Although their high HT counteracts this marginally). They also have additional DR, past the normal for everyday spirits. It is semi-ablative (B47), but will often provide that extra kick of protection to make them much tankier.
	
	Their powers are limited, but powerful. They lack much in the way of subtlty, only able to prevent accident and catastrophes for those nearby them. Overtly however, they are able to bind opponents by controlling the ground, greatly speed up or slow down almost anything ground based, and mystically search for anything familiar to them.
	
	Spirits of Earth are also extremely versatile in the conditions that they can go into. They are immune to the effects of pressure and vacuum entirely. They are, however, not sealed (You can still water and Blight them)!
	
	\textbf{Attributes:}
	ST 10 [0]; DX 5 [-100]; IQ 5 [-65]; HT 10 [0]; HP 9 [-2]; Basic Speed 3.50 [-5]; Basic Move 3 [0]; Per 6 [-20]; Will 6 [-28]; FP 10 [0]
	
	\textbf{Advantages:}
	Damage Resistance 2 (Semi-Ablative, -20\%) [8]; \hyperref[elemental]{Elemental [7]}; Manifestation/Possession [62]; Pressure Support 3 [15]; \hyperref[spirit_meta_trait]{Spirit Meta-Trait [69]}; Vacuum Support [5]
	
	\textbf{Disadvantages:}
	Innumerate [-5]; Invertebrate [-20]
	
	\textbf{Skills:}
	Assensing (H) Per [4]; Brawling (E) DX+2 [4]; Innate Attack (E) DX+2 [4]
	
	\subsubsection*{Spirit of Earth Improvements}
	
	\textbf{\textit{Force 1 Points: [62]}}
	
	\textbf{\textit{Automatic effects at each Force:\\}}
	\hyperref[spirit_force]{Spirit Force [10.5 per Magic]}\\\\
	Damage Resistance +1 (Semi-Ablative, -20\%) [4]
	
	\textbf{\textit{Optional effects at Higher Forces, +11 CP per Force above 1:\\}}	
	\hyperref[binding]{Binding [3.3 per Will]}\\
	\hyperref[guard]{Guard [21+]}\\
	\hyperref[movement]{Movement (Ground) [54]}\\
	\hyperref[search]{Search [29+]}	
	\hyperref[concealment]{Concealment [12+]}\\
	\hyperref[confusion]{Confusion [42+]}\\
	\hyperref[engulf]{Engulf [29+];}\\
	\hyperref[elemental_attack]{Elemental Attack (Earth) [5.5 per Force]}\\
	\hyperref[fear]{Fear [42+]}\\
	
	\subsubsection*{Spirit of Wood}
	\begin{flushright}
		-61 Points Force 0
	\end{flushright}

	Spirits of Wood differ notably from Spirits of Plant in their state. Plants represent living flora in their shape and design, while Spirits of Wood are lumber, building materials, driftwood, and so on.
	
	\textbf{Attributes:}
	ST 10 [0]; DX 6 [-80]; IQ 5 [-65]; HT 10 [0]; HP 11 [2]; Basic Speed 3.25 [-5]; Basic Move 4 [5]; Per 6 [-20]; Will 6 [-28]; FP 9 [-3]
	
	\textbf{Advantages:}
	Damage Resistance 2 (Semi-Ablative, -20\%) [8]; \hyperref[elemental]{Elemental [7]}; Manifestation/Possession [62]; \hyperref[spirit_meta_trait]{Spirit Meta-Trait [69]}; Vacuum Support [5]
	
	\textbf{Disadvantages:}
	Fragile (Combustible) [-5]; Innumerate [-5]; Numb [-20]
	
	\textbf{Skills:}
	Assensing (H) Per [4]; Brawling (E) DX+2 [4]; Innate Attack (E) DX+2 [4]
	
	\subsubsection*{Spirit of Metal}
	\begin{flushright}
		-61 Points Force 0
	\end{flushright}

	\textbf{Attributes:}
	ST 10 [0]; DX 4 [-120]; IQ 5 [-65]; HT 10 [0]; HP 10 [0]; Basic Speed 3.0 [-10]; Basic Move 2 [-5]; Per 6 [-20]; Will 6 [-28]; FP 9 [-3]
	
	\textbf{Advantages:}
	Damage Resistance 5 [25]; \hyperref[elemental]{Elemental [7]}; Manifestation/Possession [62]; Pressure Support 3 [15]; \hyperref[spirit_meta_trait]{Spirit Meta-Trait [69]}; Vacuum Support [5]
	
	\textbf{Disadvantages:}
	Innumerate [-5]
	
	\textbf{Skills:}
	Assensing (H) Per [4]; Brawling (E) DX+2 [4]; Innate Attack (E) DX+2 [4]

	
	\subsubsection{Spirit of Beasts}
	\begin{flushright}
		-70 Points Force 0
	\end{flushright}
	
	Spirits of Beast are a strange and extremely diverse class of beings. They are known to take the form of any non-sapient animal - mythical or not - although they do sometimes stay away from more meta-physical depictions that are used for other elements, such as a Stormbird for a Spirit of Air. They aren't limited to singular animals, sometimes manifesting as an entire flock (Although this does not change their traits unless the GM \textit{really }wants to alter the templates!).
	
	The variety of forms can sometimes mean a required change in their Bodily Advantage, such as an Emu being Bodily (Bipedal) or lacking Discriminatory Smell. The spirit can also trade out the Discriminatory Smell and/or Hearing [15/15], usually gaining some mix of things like +2 HT [26], +1 DX \& +0.50 Basic Speed [30], or +5 ST [30]. GMs and players are free to mix these up as they wish, as long as the end up within \(\pm5\) points and represent the beast better. Taking Bodily (Bipedal) does require offseting its [35] point cost.
	
	The Spirits themselves are solid all-rounders, with good ST and HT alongside average everything else. Their perception is aided by their enhanced senses, allowing for extremely precise smelling, tasting, and hearing alongside their good night vision.
	
	They lack a large amount of powers, but are unique in their Animal Control power, which allows them to command large amounts of non-sapient animals. Otherwise, they can instill fear into a subject or greatly increase or decrease the speed of a target. 
	
	Their movement power is somewhat less intuitive than other spirits, because it is related to the form that their animal takes; an eagle may have it's home territory as an area that eagles normally live (e.g. wilderness of Alaska) or it may simply have the air as its home territory. Alternatively, a cat may have the alleys of any urban area (akin to a feral cat), or alternatively just the ground as its territory. The GM should strive to make them relatively equal in frequency given the campaign and its setting (e.g. a campaign permanently in Alaska should not allow the former eagle example, while it should also not require "the ocean" for an aquatic animal in a campaign that will never see water).
	
	Spirits of Beast are additionally strange in their animistic mindset. They are bestial, meaning that they generally lack many "civilized" concepts such as property. They \textit{are not unintelligent,} they simply react like an animal would. Think similar to how a chimpanzee act and reacts. They are also unable to read text or abstract images, making them unable to even use many aspect of metahuman society!
	
	\textbf{Attributes:}
	ST 8 [-12]; DX 7 [-60]; IQ 6 [-52]; HT 8 [-26]; HP 10 [4]; Basic Speed 4.0 [5]; Basic Move 5 [5]; Per 7 [-15]; Will 6 [-28]; FP 8 [0]
	
	\textbf{Advantages:}
	\hyperref[elemental]{Bodily (Quadraped) [0]}; Manifestation/Possession [62]; Night Vision 5 [5]; \hyperref[spirit_meta_trait]{Spirit Meta-Trait [69]}
	
	\textbf{Disadvantages:} 
	Bestial [-10]; Dyslexia [-10]; Innumerate [-5]; Non Iconographic [-10]
	
	\textbf{Skills:} 
	Assensing (H) Per [4]; Brawling (E) DX+2 [4]
	
	\subsubsection*{Spirit of Beast Improvements}
	
	\textbf{\textit{Force 1 Points: [70]}}
	
	\textbf{\textit{Automatic effects at each Force:\\}}
	\hyperref[spirit_force]{Spirit Force [10.5 per Magic]}\\\\
	
	\textbf{\textit{Optional effects at Higher Forces, +15 CP per Force above 1:\\}}
	\hyperref[animal_control]{Animal Control [25+]}\\
	Discriminatory Hearing [15] (If removed from base trait)\\
	Discriminatory Smell [15] (If removed from base trait)\\
	\hyperref[fear]{Fear [42+]}\\
	\hyperref[movement]{Movement (Various) [54]}\\
	\hyperref[concealment]{Concealment [12+]}\\
	\hyperref[confusion]{Confusion [42+]}\\
	\hyperref[guard]{Guard [21+]}\\
	\hyperref[natural_weapon]{Natural Weapon [Var]}\\
	\hyperref[noxious_breath]{Noxious Breath [23+]}\\
	\hyperref[search]{Search [29+]}\\
	\hyperref[venom]{Venom}\\
	
	\subsubsection{Spirit of Water}
	\begin{flushright}
		-87 Points Force 0
	\end{flushright}
	
	Spirits of Water most often take forms associated with the actual molecule itself, as a water elemental or some other representation. Sometimes (more often for shamanic traditions) they also take the forms of things like mermaids, sea serpents, or anything water related as well! As usual, such forms can justify a switch from \textit{Elemental [7]} to \textit{Bodily [5/0]}, and take either +1 HP/+1 ST respectively. 
	
	Some Spirits of Water also manifest as other forms of water, such as ice - which is detailed in other templates down below.
	
	\textbf{Attributes:}
	ST 6 [-24]; DX 7 [-60]; IQ 6 [-52]; HT 6 [-52]; HP 8 [4]; Basic Speed 4.0 [15]; Basic Move 6 [10]; Per 6 [-20]; Will 6 [-28]; FP 6 [0]
	
	\textbf{Advantages:}
	Amphibious [10]; Chameleon 1 [5]; \hyperref[elemental]{Elemental [7]}; Manifestation/Possession [62]; Pressure Support 3 [15]; Slippery 5 [10]; \hyperref[spirit_meta_trait]{Spirit Meta-Trait [69]}
	
	\textbf{Disadvantages:}
	Innumerate [-5]; Invertebrate [-20]; Vulnerability, Heat/Fire \& other Dehydrating Attacks (Common, \(\times3\)) [-45]
	
	\textbf{Skills:}
	Assensing (H) Per [4]; Brawling (E) DX+2 [4]; Innate Attack (E) DX+2 [4];
	
	\subsubsection*{Spirit of Water Improvements}
	
	\textbf{\textit{Force 1 Points: [87]}}
	
	\textbf{\textit{Automatic effects at each Force:\\}}
	\hyperref[spirit_force]{Spirit Force [10.5 per Magic]}\\\\
	
	\textbf{\textit{Optional effects at Higher Forces, +15 CP per Force above 1:\\}}
	\hyperref[concealment]{Concealment [12+]}\\
	\hyperref[confusion]{Confusion [42+]}\\
	\hyperref[engulf]{Engulf [29+]}\\
	\hyperref[movement]{Movement (Sea) [54]}\\
	\hyperref[search]{Search [29+]}\\
	\hyperref[accident]{Aciddent [39]}\\
	\hyperref[binding]{Binding [3.3 per Will]}\\
	\hyperref[elemental_attack]{Elemental Attack (Water) [5.5 per Magic]}\\
	\hyperref[energy_aura]{Energy Aura (Ice) [4.5 per Magic]}\\
	\hyperref[guard]{Guard [21+]}\\
	\hyperref[weather_control]{Weather Control}\\
	
	
	\subsubsection*{Spirit of Ice}
	\begin{flushright}
		-87 Points Force 0
	\end{flushright}
	
	\textbf{Attributes:}
	ST 6 [-24]; DX 7 [-60]; IQ 6 [-52]; HT 7 [-39]; HP 8 [4]; Basic Speed 4.0 [15]; Basic Move 4 [0]; Per 6 [-20]; Will 6 [-28]; FP 7 [0]
	
	\textbf{Advantages:}
	Damage Resistance 3 (Semi-Ablative, -20\%) [12]; \hyperref[elemental]{Elemental [7]}; Manifestation/Possession [62]; Pressure Support 3 [15]; \hyperref[spirit_meta_trait]{Spirit Meta-Trait [69]}; Vacuum Support [5]
	
	\textbf{Disadvantages:}
	Fragile (Brittle) [-15]; Innumerate [-5]; Vulnerability, Heat/Fire \& other Dehydrating Attacks (Common, \(\times3\)) [-45]
	
	\textbf{Skills:}
	Assensing (H) Per [4]; Brawling (E) DX+2 [4]; Innate Attack (E) DX+2 [4];
	
	\subsubsection{Spirit of Man}
	\begin{flushright}
		-100 Points Force 0
	\end{flushright}
	
	Spirits of Man are extremely interesting concepts to metahumanity, least of all for their apparent ability to mimic the identities of dead people - or that they might just \textit{be} dead people. The most common form by far are those that are metahuman in shape - but they are by not means limited to this! They can take the form of anything heavily related to metahumanity, such as infrastructure like street signs and trash cans, human-like animals such as monkeys and coyotes, or human-associated animals such as dogs! In such cases, they may switch their \textit{Bodily (Bipdeal) [5]} trait, or may even take the \textit{Elemental [7]} trait!
	
	\textbf{Attributes:}
	ST 4 [-36]; DX 6 [-80]; IQ 7 [-39]; HT 7 [-39]; HP 4 [0]; Basic Speed 3.50 [5]; Basic Move 4 [5]; Per 7 [-15]; Will 6 [-28]; FP 6 [0]
	
	\textbf{Advantages:}
	\hyperref[bodily]{Bodily (Biped) [5]}; Infravision [10]; Manifestation/Possession [62]; Night Vision 5 [5];  \hyperref[spirit_meta_trait]{Spirit Meta-Trait [69]}
	
	\textbf{Disadvantages:}
	\textit{Choose -40 points in disadvantages (Preferably mental) to represent the spirit's mentality. Some niche suggestions:\\}
	Compulsive Behaviour (Ghostly Repetition)\footnote{From Horror p23} [-1 to -15]\\
	Invertebrate [-20] \& Innumerate [-5] (For non-humanoid spirits)\\
	Delusion (They are the spirit of a dead person) [-10]\\
	
	\textbf{Skills:}
	Assensing (H) Per [4]; Brawling (E) DX+2 [4]; Spellcasting (H) 10 [4]; Innate Attack (E) DX+2 [4];
	
	\subsubsection*{Spirit of Man Improvements}
	
	\textbf{\textit{Force 1 Points: [100]}}
	
	\textbf{\textit{Automatic effects at each Force:\\}}
	\hyperref[spirit_force]{Spirit Force [10.5 per Magic]}\\\\
	
	\textbf{\textit{Optional effects at Higher Forces, +15 CP per Force above 1:\\}}
	\hyperref[accident]{Accident [39]}\\
	\hyperref[concealment]{Concealment [12+]}\\
	\hyperref[confusion]{Confusion [42+]}\\
	\hyperref[guard]{Guard [21+]}\\
	\hyperref[influence]{Influence [58]}\\
	\hyperref[search]{Search [29+]}\\
	\hyperref[fear]{Fear [42+]}\\
	Spell (Any spell that the summoner knows)\footnote{Spirits rarely cast at a Force above their magic, although they can.} [var]\\
	\hyperref[movement]{Movement (Various) [54]}\\
	\hyperref[psychokinesis]{Psychokinesis}\\
	
	
	\subsubsection{Spirit of Guidance}
	\begin{flushright}
		-58 Points Force 0
	\end{flushright}
	
	Spirits of Guidance are there to provide direction to summoners, often taking the form of things closely associated with that, such as explorers or navigators, ancient ancestor, or as mundane as a seeing eye dog. They can take less obvious forms as well, such as a street sign, although these can be rare depending on the tradition and summoner; in such cases, the GM can switch out the Bodily (Biped) trait for Elemental instead.
	
	For the purposes of most tasks, Spirits of Guidance are not well suited due to their less impressive selection of powers and attributes. However, they truly shine in providing information - which can be an invaluable tool for any summoner. Specifically, each spirit has its own way of providing divinations to the user - usually influenced heavily by the summoner's tradition. This can take the form of cryptic riddles and hints from an explorer to reading the bones for a shaman - it can even be strange symbols on a street sign!
	
	\textbf{Attributes:}
	ST 7 [-18]; DX 5 [-100]; IQ 6 [-52]; HT 9 [-13]; HP 7 [0]; Basic Speed 4.25 [15]; Basic Move 4 [0]; Per 6 [-20]; Will 6 [-28]; FP 9 [0]
	
	\textbf{Advantages:}
	\hyperref[bodily]{Bodily (Biped) [5]}; \hyperref[divination]{Divination [22]}; Manifestation/Possession [62]; \hyperref[spirit_meta_trait]{Spirit Meta-Trait [69]}
	
	\textbf{Disadvantages:}
	Curious (SC 12) [-5]; Disciplines of Faith (Ritualism) [-5]; Odious Personal Habit 2 (Cryptic Responses) [-10]
	
	\textbf{Skills:}
	Assensing (H) Per [4]; Brawling (E) DX+2 [4]; Counterspelling (H) 10 [4]; Thaumatology (VH) IQ+0 [8]
	
	\subsubsection*{Spirit of Guidance Improvements}
	
	\textbf{\textit{Force 1 Points: [58]}}
	
	\textbf{\textit{Automatic effects at each Force:\\}}
	\hyperref[spirit_force]{Spirit Force [10.5 per Magic]}\\\\
	
	\textbf{\textit{Optional effects at Higher Forces, +15 CP per Force above 1:\\}}
	\hyperref[confusion]{Confusion [42+]}\\
	\hyperref[engulf]{Engulf [29+]}\\
	\hyperref[fear]{Fear [42+]}\\
	\hyperref[guard]{Guard [21+]}\\
	Discriminatory Hearing [15] \\
	Discriminatory Smell [15] \\
	\hyperref[influence]{Influence [58]}\\
	Infravision [10] \\
	Night Vision 5 [5] \\
	\hyperref[magical_guard]{Magical Guard [3.3 per Magic]}\\
	\hyperref[search]{Search [29+]}\\
	\hyperref[shadow_cloak]{Shadow Cloak}\\
	
	\subsubsection{Spirit of Plant}
	\begin{flushright}
		-70 Points Force 0
	\end{flushright} 
	
	Spirits of Plant can be interesting in contrast to wooden Earth Spirits. Plant Spirits tend to focus more on the flora themselves than the idea of wood or the ground supporting it, meaning that they often take the forms of auspicious trees or plants as opposed to literal bodies of wood. Many traditions deliniate between cultivated and wild plants as well.
	
	\textbf{Attributes:}
	ST 7 [-18]; DX 6 [-80]; IQ 5 [-65]; HT 9 [-13]; HP 10 [6]; Basic Speed 4.0 [5]; Basic Move 5 [5]; Per 6 [-20]; Will 6 [-28]; FP 9 [0]
	
	\textbf{Advantages:}
	Damage Resistance 2 (Semi-Ablative, -20\%) [8]; \hyperref[elemental]{Elemental [7]}; Manifestation/Possession [62]; \hyperref[spirit_meta_trait]{Spirit Meta-Trait [69]}
	
	\textbf{Disadvantages:}
	Inumerate [-5]; Numb [-20]
	
	\textbf{Skills:}
	Assensing (H) Per [4]; Brawling (E) DX+2 [4]; Counterspelling (H) 10 [4]
	
	\subsubsection*{Spirit of Plant Improvements}
	
	\textbf{\textit{Force 1 Points: [70]}}
	
	\textbf{\textit{Automatic effects at each Force:\\}}
	\hyperref[spirit_force]{Spirit Force [10.5 per Magic]}\\\\
	
	\textbf{\textit{Optional effects at Higher Forces, +15 CP per Force above 1:\\}}
	\hyperref[accident]{Accident [39]}\\
	\hyperref[concealment]{Concealment [12+]}\\
	\hyperref[confusion]{Confusion [42+]}\\
	\hyperref[engulf]{Engulf [29+]}\\
	\hyperref[fear]{Fear [42+]}\\
	\hyperref[guard]{Guard [21+]}\\
	\hyperref[magical_guard]{Magical Guard [3.3 per Magic]}\\
	\hyperref[movement]{Movement (Land or Area) [54]}\\
	\hyperref[noxious_breath]{Noxious Breath [23+]}\\
	\hyperref[search]{Search [29+]}\\
	\hyperref[silence]{Silence}\\
	
	\subsubsection{Guardian Spirit}
	\begin{flushright}
		-60 Points Force 0
	\end{flushright}
	
	Taking a much more metaphysical approach than many spirits, Guardian Spirits often form as string warriors, avenging angels, guardian ancestors or spirits, and so on. Their forms are often dual-purposed to frighten opponents and reassure their allies, and as such often are impressive and mythological in their style, such as Norse Valkyries or Islamic Ifreets. While this most often means they have the Bodily (Biped) advantage, it's not impossible for them to take different or even abstract forms entirely, in which case this trait should be switched out appropriately.	
	
	Guardian Spirits themselves are very generalist, with the only noteworthy focus being their emphasis on physical combat. Where they truly shine is their ability to learn \textit{any} combat skill - effectively turning them into a summoner's swiss army knife in terms of combat versatility. Need someone to fire the LAW you just picked up? Need someone with Gunner (Machine Gun) to use the stolen RPK? Need a grappler on demand to grab the fussy VIP? All of these and more make Guardian spirits versatile beyond their initial offerings.
	
	\textbf{Attributes:}
	ST 8 [-12]; DX 8 [-40]; IQ 6 [-52]; HT 7 [-39]; HP 8 [0]; Basic Speed 4.5 [15]; Basic Move 4 [0]; Per 6 [-20]; Will 6 [-28]; FP 7 [0]
	
	\textbf{Advantages:}
	\hyperref[bodily]{Bodily (Biped) [5]}; Manifestation/Possession [62]; \hyperref[spirit_meta_trait]{Spirit Meta-Trait [69]}
	
	\textbf{Disadvantages:}
	\textit{Choose -40 points in disadvantages (Preferably mental) to represent the spirit's mentality. Some niche suggestions:\\}
	Code of Honor [-5 to -15]; Pacifism (Cannot Harm Innocents) [-10]; Sense of Duty (Summoner, a people, etc) [-5]
	
	\textbf{Skills:}
	Assensing (H) Per [4]; Brawling (E) DX+2 [4]; Melee Weapon Skill (Chose one) (A) DX [4]; Innate Attack (E) DX+2 [4]; Counterspelling (H) 10 [4]
	
	\subsubsection*{Guardian Spirit Improvements}
	
	\textbf{\textit{Force 1 Points: [60]}}
	
	\textbf{\textit{Automatic effects at each Force:\\}}
	\hyperref[spirit_force]{Spirit Force [10.5 per Magic]}\\\\
	
	\textbf{\textit{Optional effects at Higher Forces, +15 CP per Force above 1:\\}}
	\hyperref[animal_control]{Animal Control [25+]}\\
	\hyperref[concealment]{Concealment [12+]}\\
	\hyperref[elemental_attack]{Elemental Attack (Any) [Var]}\\
	\hyperref[guard]{Guard [21+]}\\
	\hyperref[fear]{Fear [42+]}\\
	\hyperref[influence]{Influence [58]}\\
	\hyperref[magical_guard]{Magical Guard [3.3 per Magic]}\\
	\hyperref[movement]{Movement (Various) [54]}\\
	\hyperref[natural_weapon]{Natural Weapon [Var]}\\
	Skill (Choose any Combat skill, the spirit gains [4] points in it) (Can be taken Force times, can be stacked)\\
	\hyperref[psychokinesis]{Psychokinesis}\\
	
	\subsubsection{Task Spirit}
	\begin{flushright}
		-80 Points Force 0
	\end{flushright}
	
	Yet another metaphysical style of spirit, Task Spirits take forms related to workers - often physical, although that is changing somewhat as traditions blend with the modern world. These can be carpenters, smiths, or even hackers, but can also be non-human, such as Golems. Their style can range greatly, from regal consultant to slave, and overlap heavily with other spirits, especially Spirits of Man.
	
	These Spirits excel at any form of mundane work, ranging from physical labor to typing data in excel. The effect of the Sixth World is obvious on them, with many Task Spirits able to perform even Technological Tasks (although they are unable to interface with many modern pieces of technology and require older peripherals to do so). This variety in effect makes them absurdly versatile and allow summoners to find a solution for almost any mundane task with even the most simple spirit and a bit of time. Like most spirits, they struggle in truly difficult situations unless they are of a higher force, but the Task Spirit shines when given a goal and a month to complete it, not when defusing a ticking timebomb.
	
	\textbf{Attributes:}
	ST 8 [-12]; DX 7 [-60]; IQ 7 [-39]; HT 6 [-52]; HP 7 [-2]; Basic Speed 3.75 [15]; Basic Move 4 [5]; Per 7 [-15]; Will 6 [-28]; FP 6 [0]
	
	\textbf{Advantages:}
	\hyperref[bodily]{Bodily (Biped) [5]}; Manifestation/Possession [62]; \hyperref[spirit_meta_trait]{Spirit Meta-Trait [69]}
	
	\textbf{Disadvantages:}
	\textit{Choose -40 points in disadvantages (Preferably mental) to represent the spirit's mentality. Some niche suggestions:\\}
	Slave Mentality [-40]; Hidebound [-5]; Ham-Fisted [-5/-10]; Hunchback [-10]
	
	\textbf{Skills:}
	Assensing (H) Per [4]; Brawling (E) DX+2 [4]; Any Artisan, Crafting, Technical, or similar skill (E/A/H) Att+2/Att+1/Att+0 [4];
	
	\subsubsection*{Task Spirit Improvements}
	
	\textbf{\textit{Force 1 Points: [80]}}
	
	\textbf{\textit{Automatic effects at each Force:\\}}
	\hyperref[spirit_force]{Spirit Force [10.5 per Magic]}\\\\
	
	\textbf{\textit{Optional effects at Higher Forces, +15 CP per Force above 1:\\}}
	\hyperref[accident]{Accident [39]}\\
	\hyperref[binding]{Binding [3.3 per Will]}\\
	\hyperref[concealment]{Concealment [12+]}\\
	Discriminatory Hearing [15] \\
	Discriminatory Smell [15] \\
	\hyperref[influence]{Influence [58]}\\
	Infravision [10] \\
	\hyperref[movement]{Movement (Various) [54]}\\
	Night Vision 5 [5] \\
	\hyperref[search]{Search [29+]}\\
	Skill (Choose any Technological or Athletic skill, the spirit gains [4] points in it) (Can be taken Force times, can be stacked)\\
	\hyperref[psychokinesis]{Psychokinesis}\\
	
\end{multicols}