\section{Matrix}

\begin{multicols}{3}
	
	\subsection{Books}
	
	The Matrix rules make use of GURPS Pyramid \#3/21 Cyberpunk article Console Cowboys and Cyberspace. It is a selection of rules for running a semi-complex minigame of hacking, which serves purpose for the Matrix well.
	
	Cyberpunk works through three parts: it creates an additional hardware modifier for computers to create cyberdecks, it creates rules for interface modes while hacking, and most importantly it creates a large swath of programs that are used to accomplish specific tasks in cyberspace.
	
	Cyberpunk also has a ruleset for determining opposition using the GURPS Action BAD system. However, these rules systematically misrepresent Shadowrun's security design philosophy, in that when facing powerful organizations, you will more often be facing many layers of weak opponents and security, and when facing a less-powerful individual, you will often face one strong opponent. BAD simply assigns a difficulty based on how powerful the entire opponent is, which tends to overvalue organizations and undervalue individual deckers. As such, it is not used for the setting.
	
	Another useful book is GURPS High Tech - Electricity and Electronics. It provides in depth explanations for many hardware and software related topics. Notably, we will be making use of the Dedicated modifier, which prevents programs from being changed, but applies x0.5¥ and x0.2lbs. 
	
	\subsection{Shadowrun Specifics}
	
	Shadowrun has a small selection of decking tools or systems that are unrepresented in Cyberpunk:
	
	\subsubsection{Cold \& Hot Sim}\label{sim_mode}
	
	When entering into VR, any character with 1 or more points in Computer Hacking (or at the GMs discretion) can disable their Sim-filter Module, turning on Hot-Sim. 
	
	Hot-Sim adds +2 to all Computer Hacking, Computer Operation, Computer Programming, or Expert (Computer Security) rolls to use Cyberpunk's programs, however it provides a -2 to resistance rolls against the Damage program or any other effect that the Sim-Filter Module is usually protecting against (Such as traumatic Simsense recordings) alongside rolls to resist Link-Locking, due to relaxed security. Explicitly, this bonus never applies to Threading rolls.
	
	\subsubsection{Link-Lock Program}
	
	\paragraph{Link-Lock Program (Continuous)}
	
	\textit{\textcolor{NavyBlue}{\\Default : Computer Hacking-3\\
			Base Complexity : 5}}
	
	Sends continuous keep-alive signals to the target’s VR module, forcing them to stay connected to the Matrix even if they try to disconnect. 
	
	This program can only be effectively used on individuals that are in VR, only causing meaningless errors for individuals using terminals. While this program is running, successfully using the Control or Damage program against an individual causes link-locking. 
	
	Link-locking prevents a user from disabling or leaving VR through normal methods by overwriting any signals. This can be potentially lethal, as the user is unable to retreat from any dangerous situation without outside help.
	
	If a user wishes to jack out there are a couple routes, none that are easy. They can attempt to use the Control program to temporarily gain control over their cyberdeck's keep-alive systems, in which case they must win a Quick Contest versus the Link-Lock program with their Control program with a -3 modifier.
	
	Alternatively, they can attempt to simulate the Link-Lock's incoming signals in order to spoof their commands to the cyberdeck. They must also win a Quick Contest versus the Link-Lock program with their Spoof program with a -2 modifier.
	
	Next, they can attempt to jam the incoming keep-alive signals from the opposing decker. This does not stop the keep-alive signals, but it would - in theory - prevent them from reaching the user's systems and preventing jacking out. However, most competently designed Link Lock programs embed themselves in breached systems to prevent this exact flaw (Which is why they require a successful Control or Damage program use), which makes jamming the opposing decker have little effect (and may in fact, prove negative should the opposing decker disable their program and have no way to signal that to the worm in your system!).
	
	In the case that you already have access to the opposing decker's systems, you can attempt to simply disable this program through any conventional means to control programs: Alter to change the program, Control to signal for the program to end or change, etc.
	
	Lastly, you can take the painful road by hard jacking out. This is where a friendly companion of your physically either unplugs your connection to your cyberdeck, turns it off, or just breaks it. This immediately requires the user to roll HT as if resisting the effects of the Damage program, with a -4 penalty. In cases where there is not a Link-Lock program running and the user is hard jacked out, this roll is made at no penalty.
	
	\subsubsection{Home Rules}
	
	\paragraph{Below Minimum Complexity Programs}\label{low_complexity}
	
	One of the shortcomings of the Pyramid \#3/21 article is the lack of low complexity programs, which was done in order to make high complexity options more valuable and restrict cyberdeck usage to certain levels. However, it makes it seem weird that potato computers, TL8 computers, and similar situations are utterly unable to run certain programs at a snail's pace - especially when some of those are relatively trivial applications! The idea that an embedded device or a dinosaur phone (Both Complexity 3) are unable to use the Alter and Control Program or that a dinosaur piece of embedded hardware (Complexity 2) can't even run a firewall both seem very odd.
	
	As a middleground, the GM should offer a heavily penalized option for running lower than minimum complexity programs. This comes in two flavors: firstly, if the decker is having to make the functionality on computers that lack the software entirely, this should use the \textit{Improvised Equipment} penalty for more mundane programs (e.g. Alter, Control, Listen, etc) or the \textit{No Equipment} penalty for less mundane ones (e.g. Stealth, Damage) (B345). If they have actual software, but it is either created to work on a slower system or is simply slowed down by the hardware itself, apply a -2 penalty to all rolls using software for each Complexity below the minimum.
	
	\subsection{Cyberdecks}
	
	The GM is heavily advised to help a player design their own Cyberdecks, which are as much a tool as they are a form of expression by the Decker. They can provide a wide array of specializations and opportunities for hackers, whether that be portability, program limitations, high complexity, improved defenses, terminal styles, or more!
	
	\subsubsection*{Erika MCD-1}
	
	\textit{\textcolor{NavyBlue}{Cost: 412.5¥, Weight: 5.5 lbs. LC 4}}
	
	A medium-sized, bottom tier cyberdeck about the size of a laptop, with built-in Basic VR, a Portable Terminal (-1 to skill), and built in Cable Jack and Tiny Radio Communicator.
	
	Equipment: Cable Jack, Portable Terminal, Small Radio Communicator (10 mi range).
	
	\textit{\textcolor{OliveGreen}{Statistics: Personal Computer, Complexity 4 (Slow, \(\times\)1/20¥; Cyberdeck, \(\times\)1.25¥), 62.5¥, 5 lbs; Portable Terminal, 50¥, 0.5 lbs; Basic VR, Complexity 4, 300¥}}
	
	\subsubsection*{Microdeck Summit}
	
	\textit{\textcolor{NavyBlue}{Cost: 435¥, Weight: 0.55 lbs. LC 4}}
	
	An alternative to the MCD-1, this small cyberdeck is about the size of a phone, with built-in Basic VR, a Datapad (-2 to skill), and built in Cable Jack and Tiny Radio Communicator. It trades off a bit of cost in return for portability.
	
	Equipment: Cable Jack, Portable Terminal, Tiny Radio Communicator (1 mi range).
	
	\textit{\textcolor{OliveGreen}{Statistics: Small Computer, Complexity 4 (Cyberdeck, \(\times\)1.25¥), 125¥, 0.5 lbs; Datapad, 10¥, 0.05 lbs; Basic VR, Complexity 4, 300¥}}
	
	\subsubsection*{Fairlight Excalibur}
	
	\textit{\textcolor{NavyBlue}{Cost: 1,878,550¥, Weight: 10.5 lbs., LC 2}}
	
	The holy grail of decks, the Fairlight Excalibur has practically every bell and whistle that one could want. It keeps its hefty size - a large laptop or small workstation - largely as a sign of its status, while providing the computing power of an entire business and then some.
	
	The Excalibur is Complexity 7, Hardened, and can run 50\% more programs that normal. It also has Total VR built-in for the system, a portable terminal (Completely detachable for most decker's purposes), and a Small Radio Communicator that also comes with a built in Secure Encryption Chip, which makes communications basically uncrackable to anything but Quantum Computers at the expensive of a 1 second delay.
	
	Equipment: Cable Jack, Portable Terminal, Secure Encryption Chip, 500¥; Small Radio Communicator (1 mi range).
	
	\textit{\textcolor{OliveGreen}{Statistics: Personal Computer, Complexity 7 (Genius, \(\times\)500¥; Hardened, \(\times\)2¥, \(\times\)2lb; High-Capacity, \(\times\)1.5¥; Cyberdeck, \(\times\)1.25¥), 1,875,000¥, 10 lbs; Portable Terminal, 50¥, 0.05 lbs; Basic VR, Complexity 6, 3,000¥}}
	
	\subsection{Decryption Computers}
	
	Due to the advent of superior computing power and quantum computing, encryption is not a given for security. Here are some devices designed to be used alongside decks for decrypting files or communications.
	
	These usually do not come with any terminals, instead designed to be linked up with driving computers that send them information for decryption, without slowing down the original computer in the meantime.
	
	\subsubsection*{Standard Decryption Computer}
	
	\textit{\textcolor{NavyBlue}{Cost: 5,500¥, Weight: 2 lbs. LC 3}}
	
	This computer is a workhorse for decryption purposes. It has Complexity 10 for the purposes of decryption, meaning it takes 36 seconds per attempt for Basic Encryption and 1 hour per attempt for Secure Encryption. As well, it comes with a library of hacks, flaws, and exploits for known Encryptions while provides a +1 (Quality) bonus to your Cryptography rolls.
	
	However, while the computer can perform basic tasks, it cannot switch out programs, leaving it dedicated to the task of decryption.
	
	Equipment: Cable Jack, Small Radio Communicator (10 mi range).
	
	\textit{\textcolor{OliveGreen}{Statistics: Personal Computer, Complexity 5 (Dedicated, \(\times\)0.5¥ \(\times\)0.2lb; Quantum, \(\times\)10¥, \(\times\)2lb), 5,000¥, 2 lbs; Decryption Program, 500¥}}
	
	\subsubsection*{Small Decryption Computer}
	
	\textit{\textcolor{NavyBlue}{Cost: 1,000¥, Weight: 1 lbs. LC 3}}
	
	A smaller computer for hackers on a budget or on the go. It has Complexity 9 for the purposes of decryption, meaning it takes 6 minutes per attempt for Basic Encryption and 3 hours per attempt for Secure Encryption. As well, it comes with a library of hacks, flaws, and exploits for known Encryptions while provides a +1 (Quality) bonus to your Cryptography rolls.
	
	However, while the computer can perform basic tasks, it cannot switch out programs, leaving it dedicated to the task of decryption.
	
	Equipment: Cable Jack, Small Radio Communicator (10 mi range).
	
	\textit{\textcolor{OliveGreen}{Statistics: Small Computer, Complexity 4 (Dedicated, \(\times\)0.5¥ \(\times\)0.2lb; Quantum, \(\times\)10¥, \(\times\)2lb), 500¥, 0.2 lbs; Decryption Program, 500¥}}
	
	
	\subsubsection*{Beefy Decryption Computer}
	
	\textit{\textcolor{NavyBlue}{Cost: 50,500¥, Weight: 16 lbs. LC 3}}
	
	An extremely beefy computer for hackers who want near real-time decryption of almost any system. It has Complexity 11 for the purposes of decrpytion, meaning it takes 3 seconds per attempt for Basic Encryption and 6 minutes per attempt for Secure Encryption. As well, it comes with a library of hacks, flaws, and exploits for known Encryptions while provides a +1 (Quality) bonus to your Cryptography rolls.
	
	However, while the computer can perform basic tasks, it cannot switch out programs, leaving it dedicated to the task of decryption.
	
	\textit{\textcolor{OliveGreen}{Statistics: Microframe Computer, Complexity 6 (Dedicated, \(\times\)0.5¥ \(\times\)0.2lb; Quantum, \(\times\)10¥, \(\times\)2lb), 50,000¥, 16 lbs; Decryption Program, 500¥}}
	
	\subsection{G.O.D. and Overwatch}
	
	The Grid Overwatch Divison (G.O.D.) serves as the corporate court enforced moderators for the wider Matrix. Their main responsibility is to investigate and punish instances of hacking, fraud, and so on, with heavy preference to those that affect corporations and the Corporate Court. Generally, they are a threat to be contended with in every server, as any Decker that spends to long making noise on the matrix will inevitably draw the watchful eye of GOD. 
	
	Whenever the Decker performs illegal actions, they generally leave traces and artefacts that can be analayzed and detected to purport malicious behaviour. This tends to happen on a timescale longer than most runs take place, however it's entirely possible for things to go wrong at any point, at which point some of GOD's numerous scrollers, bots, demiGODs, and more have flagged the suspicious behaviour and called for immediate investigation. If that happens, GOD will send a demiGOD (or in serious cases a GOD themselves) over to investigate the situation.
	
	What happens next depends on the behaviour in question. If the DM decides the activity is obviously (Usually in the case of brute force attacks, flagged behaviour, and so on), then the demiGOD will immediately begin attempting to prevent further harm by the decker, trace their physical location, and boot them from the Matrix. All of this, is usually pretty easy for them given that most hosts and grids give them admin access combined with their impressive skills and hardware; in general, a demiGOD can be rated as a middle of the road Rating 12 Spyder (effective skill ~23), with a GOD being rated as a high rating 12 Spyder (Effective skill ~25) \textit{or higher}. Most often, they will be performing multiple program invocations in a single turn to analyze (to find any traps or contingencies), brute force break down the Decker's ICE, trace their connection, and either Damage and/or Control to dumpshock them - and most often they will win! If things aren't going particularly well too.. they'll just call in another GOD, so don't stick around.
	
	Of course, from the Decker's perspective this is really \textit{really} bad. The number two best option in this case, is to survive the onslaught long enough to sever the connection yourself before you get traced and dumped - if you're feeling lucky you might be able to finish one last task before you go. The \textit{number one} best option is to not be found in the first place. GOD does not investigate anything and everything - there's not enough skilled hackers and hardware in the world to justify that; instead, they prioritize showing up in three situations of decreasing priority: 
	
	Firstly, a very high profile megacorp server is involved. This is the case where you hit an MCT Zero Zone, NeoNet's Matrix Servers, and so on. In these cases, while they do have well trained Spyders and security, the potential value of these assets can often times warrant GOD being the second responders to any intrusion case. Of course, don't forget that Rule Zero (Shadowrunners exist) applies, so immediatelly bringing the nuclear option should be reserved for levels of play so high that the runners are making deals with dragons on the regular.
	
	Secondly, if suspicious activity has left enough large scale traces that it's been flagged as statistically significant and warranting investigation. This is the most common way that a Decker will meet GOD and it can happen regardless of whether the owners of the host they are in know about the Decker or not. Because of GOD's ubiquitous admin rights to the vast majority of the Matrix as a root of trust and security alongside their massive compute advantage over everyone else (Complexity 9 hardware is 1,000 to 10,000 times more powerful to most hosts), they are often able to pick up the tiniest details for their detection and analysis. Usually, this is abstracted into an Overwatch Score, which ranges from 0 to 40. Whenever you perform an illegal action, the DM will increase your Overwatch score by the Margin of \textit{Success} (not Victory) for your opponent, divided by an amount pertaining to how bad the action was - usually 2 for most illegal actions, 3 for \textit{very} stealthy illegal actions, and 1 for \textit{very} loud illegal actions. This is kept hidden from the Decker unless they use the Analyze program on themselves to determine their score. Additionally, as time goes on GOD will eventually be able to process the data further, causing an accrual of Overwatch Score over time. Consult the \textit{Speed/Range Table} (B550), reading Linear Measurement as the amount of minutes that have passed since your initial illegal action, and the adding the total size bonus to your Overwatch score. When the score reaches 40, GOD has picked up their scent an convergence is imminent.
	
	Alternatively, the GM may opt for a more realistic option that has less bookkeeping. Every minute, roll a Quick Contest of the Decker's Stealth versus GOD's Skill, as indicated below. If GOD \textit{wins} the Quick Contest, they have noticed something is amiss and will respond according to their margin of Victory, with victory by 5 more usually meaning they respond with immediate hostilities.
	
	\textcolor{NavyBlue}{\textit{Modifiers: -20 for the Size of the Matrix\footnote{Derived using the Speed/Range Table, assuming that there are ~3 billion devices/hosts/websites (Around double the amount in 2020s), which follow Zipf's law meaning that 1/5 make up the majority of any interest. Then, those are divided among 9.4 million cybersecurity professionals (Double the global number in 2020s), of which 1/10 work directly or indirectly with GOD, 1/10 work as active demiGODs, GODs, or analysts for them.}, Bonus for time since initial illegal action as above, +1 to +4 for engaging in "loud" hacking such as Breach or Damage programs, shutting down important systems, stealing or deleting heavily controlled data, and/or causing substantial real world effects, -1 to -4 for "quiet" hacking such as making devices only perform standard behaviour or accessing things through commonplace channels (Note that using just Spoof, just Analayze, or similar actions are not enough here, those are +0 instances!), +2 for hacking an important AA megacorp host or unimportant AAA host, +4 for hacking an important AAA megacorp host.}}
	
	Lastly, if the Host's spyders are being destroyed by the Decker, GOD will eventually show up whether or not convergence has happened. Usually, GOD wants to work similarly to the FBI or other agencies - on their own terms, on their own cases, and when they want. They won't answer calls for help simply because a host is being pummeled by a skilled decker, instead they will step in if they feel the Decker is becoming a threat to the Matrix itself or so on. Often, this means that they will still only arrive when Convergence happens, however it is not impossible for them to decide to step in before then if the Decker is being to disruptive.
	
	Don't forget that not all hosts allow GOD admin rights or even into their hosts at all. The obvious examples are hosts that are engaged in (unsanctioned and unignorable) illegal activity, are paranoid, or simply want to control things entirely in house. However, there are also airgapped hosts that can't reach the wider Matrix for security reasons and out of date hosts (Often those that survived the Crash 2.0) which can't conform with necessary standards for GOD. In such situations, GOD will generally ignore anything done in them, often because they get no information, but for instances where GOD are simply barred from entering but the host still reports data, allows detection and analysis, or simply alerts GOD if things go bad, then a demiGOD will be waiting outside the host for whenever the Decker decides to leave.
	
	\subsection{Guidance}
	
	When designing matrix opposition, it can be pretty daunting to individuals who don't understand networking or computer science. While going off gut intuition and Hollywood style hacking films can lead to some pretty acceptable results, putting in the extra level of realism can simultaneously make the security feel more fleshed out and realistic, while also making it more difficult without resorting to simply increasing Skill Levels.
	
	\subsubsection{Networks}
	
	Networks are connections of computers that are connected to each other via ethernet (physical wires) or remotely (wifi). In Shadowrun, the vast majority of Networks are composed of almost entirely - if not entirely - remote connections. 
	
	This is done for a variety of reasons: Physical technology has lagged behind remote technology, GOD serves as a police force for remote networks, and the ever classic convenience and cost - it's just more cost and time efficient in the grand scheme to coordinate with the greater remote Matrix than to be an eclectic company.
	
	That being said, there are some situations where physicals networks are still used. The most obvious is the situation where devices are still plugged via cable to a computer, such as an MRI being physically plugged into the computer that runs it. The less obvious ones are situations where taking the difficult road is worth it, such as protecting sensitive information (Trade Secrets, Research), paranoid individuals' home servers, or old systems that would cost more to upgrade than maintain.
	
	When designing a network for a run, it helps to know three broad types of networks found in the Matrix: The Matrix itself, an intranet, and an extranet. 
	
	The Matrix is the remote connections of all the computers available to metahumanity - plus something.. else, if the stories were to be believed. It's made up of a very large amount of networks, ranging from home servers to AAA office hosts, that are all connected to their Matrix provider. 
	
	A Matrix provider is a company such as NeoNet (Rest in peace) that serves Matrix traffic from network to network, provides maintenance and/or customer support for networks and hosts, and connects grids on the Matrix. A Matrix provider connects networks through special services; they rent out connections that are difficult to maintain (Such as cross-continental cables), maintain licenses for special services that keep track of internet traffic and route it to locations, and maintain hardware to perform those services. In essence, they maintain licenses and hardware that allows them to connect networks and grids to other networks and grids.
	
	The nice thing about Matrix providers, is that as long as you have access to a computer or service that uses them (Such as a home subscription, public wifi, or a cracked grid), you can access any computer that the Matrix provider's software has interacted with before (provided it is still online, available, and unchanged) simply by routing your traffic to other computers that can eventually route your traffic to your destination. This allows you to overcome many limitations of range when it comes to the Matrix, but does restrict you to accessing those destinations through their expected routes - as opposed to say, hacking into the building by taking apart a camera or stealing a commlink. \textcolor{Blue}{\href{https://blog.thelifeofkenneth.com/2017/11/creating-autonomous-system-for-fun-and.html}{For further reading by interested readers.}}
	
	An Intranet is a private network that is set up to only allow access to individuals from a company internally. Usually this is done by setting up ICE that block all incoming traffic to company devices, instead only allowing traffic to be routed through dedicated servers set up to handle outside data; this means that (without hacking in), a user would not be able to communicate with a commlink on the Intranet directly, but might be able to access it by sending data through a more secure central server. Sometimes, this is done via physical cables only (mostly in sensitive locations only), where company commlinks have their wireless capabilities removed and can only connect to the internet by sending and receiving data through that central server - if at all!
	
	An Extranet, like an Intranet, is a private network for a company. However, Extranets meant to serve users from outside a normal company rather than inside. Most often, this takes the place of things like public websites, user login and services, etc, basically anything that someone not part of the company can connect to. These can have varying degrees of connection to the company's Intranet as well; the company may set up a VPN that allows outside computers using it to access the Intranet, certain non-sensitive parts of the Intranet may be accessible upon authorization, etc. The lines can get quite blurry.
	
	On any network, a given account is provided permission to do a variety of tasks, while prevented from performing others. You can generally break permission sets into three groups: User, Security/Dev, and Admin. 
	
	User account permissions are the vast majority, being made up of anyone who does not need specialized access to the network itself. They are usually limited in altering anything in the network and on their computer \textit{at all}, and if they can alter anything it is usually limited to things directly related to their job (e.g. research files stored remotely for a scientist). Some users have very expansive lists of things related to their job, such as software developers, which can lead to them having much more permissions than others. 
	
	Security accounts are used to provide general IT and cybersecurity services to the network, meaning that their permissions are often much more relaxed. They are usually able to access wide swathes of the network, remotely control company computers (sometimes without notification!), access security features such as ICE and programs, and more. They are however, limited in that they are still restricted to the role of providing IT and cybersecurity services - not playing GOD; this means that they can't often disable security entirely, allow illicit data, or delete digital records, but this can vary based on the needs of the company - and often times can be largely circumvented.
	
	Admin accounts are entirely in control of the network. If it is possible to do something on the network, admins can do it. Often times, this does not include people like CEOs and is instead limited to leadership of IT, Cybersecurity, etc.
	
	When designing a network, it's generally important to know a small list of things: 
	
	\begin{itemize}
		\itemsep 0pt
		\item Who provides the network its connection to the Matrix (if at all)? 
		\item How many program slots are taken up by security programs? How much by company use / free-floating?
		\item Does the company have an Intranet?
		\begin{itemize}
			\itemsep 0pt
			\item What traffic might it allow directly to devices on the intranet, if any at all?
			\item What broad categories of accounts are on the network? Users (What jobs for each?), Security, and Admin.
			\item What are the stats of the servers hosting the intranet? Especially, the programs used to provide ICE and Security?
			\item Does the company allow remote access via VPN?
		\end{itemize}
		\item Does the company have an Extranet?
		\begin{itemize}
			\itemsep 0pt
			\item What services is hosted on the Extranet? A public website, login portal, VPN?
			\item Is any part of the Intranet accessible to the Extranet?
		\end{itemize}
		\item Is any part of the network airgapped (Cut off from the internet)?
		\item Is any part of the network physical instead of remote?
		\item How is data/paydata stored? Is it kept on group servers in shared folders so everyone can access it? Is it carried between computers on physical media for security? Is it kept on individual computers?
		\item What is the networks response to intruders?
		\begin{itemize}
			\itemsep 0pt
			\item Does it use lethal measures like Black IC?
			\item How does it respond to accidents/false flags? Does it care if a worker gets Black IC'ed?
			\item What defenses (Programs, hardware, etc) are prepared? How do they launch if set off unsupervised?
		\end{itemize}
	\end{itemize} 
	
	\subsubsection{Hosts}
	
	Hosts are the colloquial term for networks on the Matrix. In Shadowrun lore, there is some discrepancy on whether they have any physical presence at all, or are entirely run on some mix of computing power stolen from the wider Matrix and/or mystic Matrix techno-magic from dead Technomancers. 
	
	For the purposes of this book, we will take the stance that there are still hardware components to hosts, but their software aspect is still grown and cultivated by Matrix providers to be run on said hardware, providing the Complexity boost from TL 8 to TL 9 (Both to provide greater reason for physical infiltration and to make it logically consistent). 
	
	If the GM takes issue with this, simply change any hardware running a host (Which is generally the server portions of Intranets and Extranets) into a purely software format - and also try not to think very hard how it works.
	
	Hosts in base Shadowrun are generally given ratings from 1-12, ranging from home LANs to AAA Zero Zone equivalents, which determine how difficult they are to hack. In GURPS, the skill of a network's ICE program is a modifier to the Spyder's (or whomever set up/maintains the host) skill, based on Complexity; as such, the rating of a host is dependant on the Spyder's base skill and the Complexity of the host.
	
	\begin{center}
		\begin{tabularx}{0.47\textwidth}{|c|c|c|X|}
			\hline
			Rating\footnote{See Section \ref{Host Ratings BTS}} & Spyder Skill\footnote{All SL are determined for Hard Skills. Lower by one for Very Hard Skills. } & Complexity & Effective Skill \\
			\hline
			\hline
			1 & 4-7 & 4 & 4-8 \\
			2 & 7-8 & 4 & 7-9 \\
			3 & 7-8 & 5 & 8-10 \\
			4 & 10-11 & 5 & 11-12 \\
			5 & 10-11 & 6 & 12-14 \\
			6 & 10-12 & 6 & 12-15 \\
			7 & 10-14 & 6 & 12-17 \\
			8 & 11-14 & 7 & 14-18 \\
			9 & 13-15 & 7 & 16-19 \\
			10 & 14-16 & 8 & 18-21 \\
			11 & 15-17 & 8 & 18-22 \\
			12 & 16-19 & 8-9 & 20-25\\
			\hline
		\end{tabularx}
	\end{center}

	Below is a table that provides a range of example Spyder Skills, Complexities, and total Effective Skill for Hosts, given by a Rating system similar to Shadowrun's original style.
		
	Another important consideration when designing hosts is software. It can be tempting to see Complexity 7 Hosts as an opportunity to slap Complexity 7 ICE on them, but hosts have further purpose than to be literal brick walls!
	
	Hosts have limited program slots/processing power, which has to be used providing their services. If an admin takes their Complexity 7 Host and runs a Complexity 7 ICE and Listen on it at all times, there would be no program slots left for any non-trivial commercial use!
	
	How much of a host's program slots are left to commercial use is a fine art that is best determine by the GM based upon the host's needs. A decent range is around 1/5 - 1/3 of the program slots reserved for commercial use. However, some considerations are: 
	
	\begin{itemize}
		\itemsep 0pt
		\item How many employees and customers access the host regularly? 
		\item Is the work done on the host software dependant (Skills with TL/9)?
		\item Does the work require lots of large services (Network routing, communications, traffic systems, etc)?
	\end{itemize}
	
	There is one caveat to this advice: some companies will have dedicated servers that are set up to handle their firewall and security. 
	
	In such a case, \textit{all of the program slots} can be used for security purposes! However, these are often much less powerful than the host as a whole, so the effect is not that different.
	
	\begin{wrapfigure}[14]{l}{70pt}
		
	\end{wrapfigure}

	The math surrounding programs slots can sometimes be tedious. An easier way to go about it is to remember that all programs take up 1/10 the complexity of the size above them. If you start out with the 2 slots allowed for a computer at base, you can multiply by 10, subtract out any used slots at that level (Which is Complexity-1), then multiply by 10 again and continue for each complexity down to 1. You can simply reverse this by dividing by 10 repeatedly, after which you have a decimal number for which each number left to right indicates your remaining slots for each complexity from highest to lowest. As a quick example, with a Complexity 6 Computer running 1 C6, 4 C5, 2 C4, 0 C3, and 1 C2, you perform the following steps: \[(((((2 - 1) \times 10 - 4) \times 10  - 2) \times 10 - 0) \times 10 - 1)\]
	
	This gives a total of 5,799, which can be divided by 10 four times (Going back from C2 scale to C6 scale) for a value of 0.5799 - indicating you have 0 C6, 5 C5, 7 C4, 9 C3, and 9 C2 program slots remaining.
	
	
	\subsubsection{Distributed Processing}
	
	Sometimes, it can be beneficial to run a large number of computers simultaneously that act as one larger computer. This is often used in the case of supercomputers and botnets, leveraging the easier accessibility of smaller computers to create a more powerful group.
	
	Distributed systems determine the total power of the distributed systems as if they were one. Because each step of complexity is a tenfold increase in power, 10 lower complexity devices should be treated as 1 Complexity higher, and so on. 
	
	For singular tasks that only affect parts of the network, treat them as either a smaller distributed network or a singular computer as necessary. If any part of the network is disabled, the overall statistics need to be re-calculated; it is adviseable to keep some buffer space then, in case an opponent disables some devices.
	
	For software purposes, each computer needs to run software for communicating and coordinating with the network; usually this is a base Complexity 2 Listen program, but in the cases where there are orders of magnitudes of devices, the GM may increase the required Complexity.
	
	A quick calculation is: if a piece of software needs to be run by every computer individually, increase its effective Complexity alongside the network's Complexity. Singular programs split across the network can be run using the combined complexity.
	
	When working with a network of varying Complexities, it is recommended to convert all of them to one level of Complexity (e.g. Convert 1 Complexity 5 computer to 100 Complexity 3 computers for calculations).
	
	\textit{\textcolor{OliveGreen}{Example: Mark, master decker, is running a distributed network of 10,274 Small Computers, made up of commlinks, smart-home devices, and other small interfaces. The overall Complexity of the Network is 8, which the GM has ruled needs a Complexity 2 Listen program on each device, which could normally run 2,000,000 Complexity 2 programs, however because it is technically running 10,274 programs, the Listen program is treated as Complexity 6 on the Network.}}
	
	Acting in a distributed manner has a number of benefits and drawbacks. Most importantly, it allows a decker to make use of multiple computer's processing power, which can help overcome low Complexity limitations. 
	
	As well, it provides multiple vectors for attack, while limiting reprisals, since the decker can attack through any device (or all of them!), while the enemy can only attack devices that they know of and must trace the network's connection back to the master computer if they wish to deal any real damage. 
	
	If any device goes down too, it does not immediately take down the decker, instead limiting the effectiveness of the network somewhat.
	
	However, there are some downsides too. Achieving higher complexities requires an extremely large amount of devices (Complexity 7 Requires 1,000 Small Computers weighing 500 lbs), which can be logistically hard to store and often weigh more than a single computer. 
	
	As well, each computer needs to be running software for communicating with the rest of the network, which can reduce the amount of programs available, and can be prohibitively expensive if they have to purchase all of the software - although it should be possible for most deckers to write the software themselves. 
	
	When working in very large groups, a network can be very difficult to hide from observers - especially if the computers in the network were not legally acquired. Many thousands of un-noteworthy computers all turning on a single target is something that will attract more attention than if one computer had originally, especially to organizations like GOD, which have a big picture view of the Matrix as a whole.
	
	\textit{\textcolor{OliveGreen}{Mark's network weighs 5,137 lbs and costs 1,027,400¥ in hardware and 308,220¥ in software (If the GM does not let him duplicate code!); compared to a Macroframe weighing 4,000 lbs and costing 1,000,000¥ in hardware, it will certainly cost more and take up more space, but it is also much harder to defeat, requiring an opponent to either take down all computers, or successfully trace the Listen programs back to Mark's master computer.}}
		
	\end{multicols}
	\begin{multicols}{2}
	
	\subsubsection{Software Packages}\label{software_packages}
	
	Creating Software, especially when those pieces interact with each other like with Firewalls, can be a daunting task for some. As such, I've assembled some example setups of Software that can be used for inspiration, quick use, or anything of the sort.
	
	All of these packages come in a Level, starting at 0. Levels determine what Complexity the software is run at, which in turn determines skill defaults as covered in Cyberpunk. As such, for more powerful hardware, simply increase the Level for a more powerful piece of software.
	
	The total Complexity of the Package is listed in a format like: $\times$2 Complexity 3; $\times$3 Complexity 3 + Level, which indicates the packages has 2 Complexity 3 programs that don't increase with Level, and 3 Complexity 3 Programs that increase with Level. In this case, a Level 3 Package would have 2 Complexity 3 Programs and 3 Complexity 6 Programs, and may be appropriate for a Host or Device of Complexity 7 or more.
	
	\paragraph{Firewalls}\label{firewall}
	
	Firewalls are programs that are, at their core, meant to control what traffic comes in or out of a computer with the purpose of protecting it from malicious activity. 
	
	Certain superior models will also perform actions upon suspicious behaviour, such as alerting admins, tracing the connection, or in extreme cases attacking the connection.
	
	They aren't necessarily limited to that however! Firewalls may shutdown the computer if triggered, spawn IC to respond, delete files when alerted, trigger whole suites (such as a Data Bomb!), or more. There's plenty of possibility, so get creative.
	
	\textbf{Inferior Firewall} 
	
	\textbf{$\times$2 Complexity 3+L}
	
	For this simple Firewall, the ICE prevents access to the device, while the Listen allows authorized users to bypass the ICE.
	
	\vbox{
		\begin{itemize}
			\itemsep 0pt
			\item ICE :: Complexity 3 + L
			\item Listen :: Complexity (2+1) + L
		\end{itemize}
	}
	
	\textbf{Standard Firewall} 
	
	\textbf{$\times$1 Complexity 3 and $\times$3 Complexity 3+L}
	
	A standard Firewall: the ICE protects the device, the Listen allows authorized users access, the Analyze monitors the security of the ICE and Listen, and the Trigger acts when anomalies are detected by the Analyze - usually it calls the admins.
	
	\begin{itemize}
		\itemsep 0pt
		\item ICE :: Complexity 3 + L
		\item Listen :: Complexity (2+1) + L
		\item Analyze :: Complexity 3 + L
		\item Trigger :: Complexity 3
	\end{itemize}
	
	\textbf{Superior Firewall} 
	
	\textbf{$\times$1 Complexity 3 and $\times$5 Complexity 3+L}
	
	Closer to an IPS (Intrusion Protection System), this Firewall works the same as above, except instead of the Trigger just calling the admins, it also attempts to find the malicious connection and Analyze it (For things like IP, Device ID, etc).
	
	\begin{itemize}
		\itemsep 0pt
		\item ICE :: Complexity 3 + L
		\item Listen :: Complexity (2+1) + L
		\item Analyze :: Complexity 3 + L
		\item Trigger :: Complexity 3
		\item Search :: Complexity 3 + L
		\item Analyze :: Complexity 3 + L
	\end{itemize}
	
	\paragraph{Matrix Search / Perception}
	
	Used to find and analyze information on a network. It may require breaching or spoofing ICE to gain all of the information.
	
	\begin{itemize}
		\itemsep 0pt
		\item x2 C(3+L)
		\item Search :: Complexity 3 + L
		\item Analyze :: Complexity 3 + L
	\end{itemize}
	
	\paragraph{Hack on the Fly}
	
	This is the method for stealthily infiltrating a Host or network. While it's simple in design, it is almost never in execution. The decker should thoroughly analyze the security before attempting, as many firewalls are made up of multiple constituent parts that must be unravelled in turn. 
	
	At its simplest, this means spoofing an Analyze watching ICE to give an OK, then spoofing the Listen to gain access. Further complicated systems may require Spoofing multiple programs (or even devices!) at once in order to prevent any from noticing.
	
	\begin{itemize}
		\itemsep 0pt
		\item x1 C(4+L)
		\item Spoof :: Complexity 4 + L
	\end{itemize}
	
	\paragraph{Sleaze}
	
	This is used to prevent the decker from going noticed while in the system. It can prevent them from being detected, analyzed, searched, eavesdropped, etc. 
	
	However, if a Spyder already has cause for concern, inconclusive results could give them more reason to continue investigating.
	
	\begin{itemize}
		\itemsep 0pt
		\item x1 C(4+L)
		\item Stealth :: Complexity 4 + L
	\end{itemize}
	
	\paragraph{Brute Force}
	
	Usually the quickest way to enter the system, this method penetrates and disables an ICE in one go. 
	
	Many security setups have Analyze programs set up to detect this sort of entry, multiple layers of ICE to slow them down, or simply human eyes that check on the program, all of which can make this form of entry go loud. Preventing that can require combining this with other tactics, such as Spoof or Control.
	
	\begin{itemize}
		\itemsep 0pt
		\item x1 C(3+L)
		\item Breach :: Complexity 3 + L
	\end{itemize}
	
	
	\subsubsection{Spyders}
	
	Spyders serve as the network IT and cybersecurity for corporations. Usually, the term Spyder implies an  individual of competence, but in reality almost anyone can be put in charge of such a position, regardless of credentials - which is often the case in smaller businesses.
	
	\subsubsection*{Standard Spyder}
	
	\textbf{Attributes:}
	ST 9 [-6]; DX 10 [0]; IQ 11(12) [13]; HT 10 [0]
	
	\textbf{Secondary Attributes:} HP 7 [0]; Per 11 [5]; Will 11 [7]; FP 10; Basic Speed 5.0; Basic Move 5
	
	\textbf{'Ware:} 
	Wireless Datajack (Base Grade) [6, 12,000¥]; Cerebral Booster 1 (Cultured Bioware, Base Grade) [4, 27,000¥]
	
	\textbf{Primary Skills:} 
	Computer Operation (E) IQ+2 [4]-14; Computer Programming (H) IQ+1 [8]-13; Computer Hacking (VH) IQ; Electronics Operation/TL9 (Security) (A) IQ+1 [4]-13; Electronics Operation/TL9 (Surveillance) (A) IQ [2]-12; Electronics Repair/TL9 (Computers) (A) IQ [2]-12; Expert Skill (Computer Security) (H) IQ+1 [8]-13; Research (A) IQ+1 [4]-13
	
	\textbf{Secondary Skills:} 
	Area Knowledge (Cyberspace) (A) IQ-1 [1]-11; Current Affairs/TL9 (Cyberspace) (E) IQ+1 [2]-13; Mathematics/TL9 (Applied) (H) IQ-2 [1]-10; Mathematics/TL9 (Computer Science) (H) IQ [4]-12
	
	\textbf{Perks:}
	Console Monkey [1]
				
	\subsection{IC}
	
	Intrusion Countermeasures (Not to be confused with the program ICE), are collections of programs that are designed to automatically respond to certain threat vectors inside of a host. These can provide a staggering range of capabilities, from Patrol IC constantly scanning credentials, to the notorious Black IC trying to flatline unauthorized users.
	
	When building a host, it's a good idea to decide what IC are running on the server at all times, alongside what their automated response schedule looks like. Most Hosts that are of decent size will be running Patrol IC at all times, focused on tasks such as scanning all users, watching important files, or scanning for malicious activity. Some may run additional IC as well to prevent loading times.
	
	It's important to take into consideration the fact that IC are \textit{dumb.} They are, at best, competitive with Pilot programs, and as such should not be trusted to actively deal with intrusion unsupervised. Most importantly, this implies that a host \textit{should never} leave lethal (Or even less-than-lethal) IC running 24/7, unless they are running a draconian ship! 
	
	It only takes one researcher accidentally using the wrong password, one intern curiously searching the directory, one person accidentally opening a file, and so on in order to end up with a hurt or killed person on your network! The only worse thing than getting your valuable research stolen, is killing those who would make it (or re-make it) in the first place!
	
	The threat of IC is counterbalanced by the fact that IC can take quite some time to load, or alternatively must take up significant resources, which causes penalties to rolls. IC still follow the Invoking Programs rules (Pyramid \#3/21 p11), meaning that each subprogram must be loaded once per turn, or alternatively multiple subprograms may be loaded at once at the cost of -1 to all rolls for each subprogram past the first. Deckers should take advantage of the spin up times to either disable the IC and/or its subprograms or complete their objective and jack out.
	
	\subsubsection{Black IC}
	
	\textbf{$\times$3 Complexity 4+L and $\times$2 Complexity 5+L}
	
	Black IC is the real nasty stuff watching over hosts that \textit{must absolutely not have intruders}. This stuff is only used when it's acceptable to risk your employees and workers - after all any piece of software can glitch when you're performing routine maintenance. It's intended purpose is to frag whomever the host tells it too - and it's usually quite good at that.
	
	It's usually not running on a host at all times for safety reasons, but place like MCT Zero Zones will often ignore these trivialities. This can make it relatively vulnerable to spin up on the host, due to the relative complexity of the IC itself, so programs are usually spun up only after the prior one has completed its task - the exception of course being Link-Lock, which will try to spin up alongside Damage.
	
	These are often accompanied by firewalls in order to prevent their subprograms from simply being shut off.
	
	\begin{itemize}
		\itemsep 0pt
		\item Search :: Complexity (3+1) + Level
		\begin{itemize}
			\itemsep 0pt
			\item Used to trace the connection of the Decker, gain his physical location and allow for counterhacking.
		\end{itemize}
		\item Analyze :: Complexity (3+1) + Level
		\begin{itemize}
			\itemsep 0pt
			\item Analyzes the security of the Decker's IC, in order to find the quickest route to break in.
		\end{itemize}
		\item Breach :: Complexity (3+1) + Level
		\begin{itemize}
			\itemsep 0pt
			\item Used to quickly break down the IC of the Decker.
		\end{itemize}
		\item Link-Lock :: Complexity 5 + Level
		\begin{itemize}
			\itemsep 0pt
			\item Locks the decker down so that the Black IC can properly track their location and - most importantly - kill them.
		\end{itemize}
		\item Damage :: Complexity 5 + Level
		\begin{itemize}
			\itemsep 0pt
			\item Frag the Decker.
		\end{itemize}				
	\end{itemize}
	
	\subsubsection{Patrol IC}
	
	\textbf{$\times$1 Complexity 3 and $\times$1 Complexity 3+L}
	
	A staple of any professional host, Patrol IC are a form of IDS (Intrusion Detection Systems) with the purposes of scanning the host for anomalous activities. These programs are designed to regularly Search for anomalous activity, usually looking at security logs, analyzing personnel activity and matching to known locations/activity/personalities, scanning files for malicious code, and so on. This is, of course, represented in the Quick Contest between the Patrol IC's Search versus the Decker's Stealth.
	
	If the Patrol IC notices anything obviously anomalous, it will Trigger and report this to the administrators. Otherwise, it will Analyze the anomaly in order to better determine the course of action. Generally, the GM can arbitrate when this difference occurs whenever the Decker gets spotted; usually it comes down to the actions taken by the Decker and whether they are obviously malicious, but it could also occur on ties or low Margins of Victory for the IC if thematically appropriate.
	
	More powerful versions will also be running an Analyze program on themselves (adjusted by Level accordingly) in order to prevent tampering of the Patrol IC, which are often found in any enterprise level system.
	
	\begin{itemize}
		\itemsep 0pt
		\item Search (Continuous) :: Complexity 3 + Level
		\begin{itemize}
			\itemsep 0pt
			\item Used to search for unknown or anomalous users, devices, and activity for further analysis. Generally higher complexity implies superior ways to detect, match, and analyze for those anomalies.
		\end{itemize}
		\item Trigger :: Complexity 3
		\begin{itemize}
			\itemsep 0pt
			\item Reports any likely anomalies to the admins.
		\end{itemize}
	\end{itemize}

	The above is usually followed up with the following programs:

	\begin{itemize}
		\itemsep 0pt
		\item Analyze (Continuous) :: Complexity 3 + Level
		\begin{itemize}
			\itemsep 0pt
			\item Used to scan files, users, and devices in order to determine any suspicious activity. Generally higher complexity implies superior ways to detect, match, and analyze for those anomalies.
		\end{itemize}
		\item Trigger :: Complexity 3
		\begin{itemize}
			\itemsep 0pt
			\item Reports any likely anomalies to the admins.
		\end{itemize}
	\end{itemize}

	\subsubsection{Bloodhoud IC}
	
	\textbf{$\times$1 Complexity 3 and $\times$3 Complexity 3+L}
	
	An uncommon alternative to Patrol IC that combines the capabilities of Trace IC, Bloodhound IC take a proactive approach to detecting and dealing with malicious actors that allow for a quicker response than normal. This has all of the obvious benefits, but is usually kept to the simpler tasks of tracking physical locations and gathering data for authorities, as opposed to engaging in all-out matrix combat without the administrator's approval.
	
	Their design varies wildly, depending on the server admin's goals, but the two most common are: Adding in an Analyze program (adjusted by Level accordingly) to discover any tricks or vulnerabilities in a decker's ICE before attempting to breach and to enhance reliability and security by running an Analyze program on themselves (adjusted by Level accordingly) to prevent tampering.
	
	\begin{itemize}
		\itemsep 0pt
		\item Search :: Complexity 3 + Level
		\begin{itemize}
			\itemsep 0pt
			\item Used to look for any anomalous users, similarly to Patrol IC. Can be left off if fed information from elsewhere.
		\end{itemize}
		\item Trigger :: Complexity 3
		\begin{itemize}
			\itemsep 0pt
			\item Used to report anomalous activity to the administrators and to trigger the latter programs.
		\end{itemize}
		\item Breach :: Complexity 3 + Level
		\begin{itemize}
			\itemsep 0pt
			\item Breaches the ICE for the decker, allowing for Analyze to track their location. Sometimes includes another Analyze program invoked to analyze the ICE before breaching.
		\end{itemize}
		\item Analyze :: Complexity 3 + Level
		\begin{itemize}
			\itemsep 0pt
			\item Attempts to track the matrix and physical location of the user, alongside generally scraping whatever information it can.
		\end{itemize}
	\end{itemize}
	
	\subsubsection{Flicker IC}
	
	\textbf{$\times$3 Complexity 5+L}
	
	A modern invention, the Flicker IC attempts to exploit some of the inherent speed of the Matrix to buy time for other information gathering processes and IC by disconnecting or interrupting the VR capabilities of the cyberdeck. These are most often found in hosts that care a lot about discovering their intruder's identities instead of just geeking them - whether that be for blackmail, tracking them down, or whatnot. Removing anyone suddenly from VR can be dangerous in general, leading some hosts to make Flicker IC jack the user out using the Control program before disabling VR capabiltiies - although that is by no means required or common.
	
	Disabling the VR module presents a number of immediate bonuses for the host compared to the decker. The Matrix generally runs faster than meatspace, meaning the GM has reasonable leeway regarding turn orders here. However, the main benefit is the penalty applied to the decker for having a worse interface - which often comes out to a -6 for using a terminal (if they even have one!) against Total VR. And of course, being dumped from Total VR also causes Dumpshock, which can incapacitate a Decker further. 
	
	If they're still conscious, most deckers opt to power off their deck (Which usually must be done by removing power supply or using hardware shutoffs, since the IC will likely win any Quick Contest to shut off using software).
	
	\begin{itemize}
		\itemsep 0pt
		\item Breach :: Complexity 3 + 2 + Level
		\begin{itemize}
			\itemsep 0pt
			\item Breaches the ICE for the decker, allowing for Control to manipulate their device.
		\end{itemize}
		\item Control :: Complexity 4 + 1 + Level
		\begin{itemize}
			\itemsep 0pt
			\item Disconnects your VR module without turning off the device, allowing for more unresistant access to the decker's device.
		\end{itemize}
		\item Link-Lock :: Complexity 5 + Level
		\begin{itemize}
			\itemsep 0pt
			\item Used to maintain the connection if the Decker attempts to sever it in the meantime.
		\end{itemize}
	\end{itemize}
	
	\subsubsection{Marker / Jammer / Acid IC}
	
	\textbf{$\times$2 Complexity 4+L}
	
	Marker, Jammer, and Acid IC all fall under a core idea of disabling programs and capabiltiies that give deckers the edge over their competition. Most often, cowboy jockeys have superior skills and equipment than a corporation is able to bare on them in short notice, for which these IC are intended to level the playing field by preventing certain programs from running in order to guarantee the goals of other IC or Spyders. None of these IC are necessarily exclusive - the Alter program can disable any other program as necessary - however, they usual focus on synergizing with other IC instead of trying to break down everything.
	
	Marker IC focuses on removing the Stealth program, Jammer IC does so for the Damage program, and Acid IC removes the ICE program.
	
	\begin{itemize}
		\itemsep 0pt
		\item Breach :: Complexity 3 + 1 + Level
		\begin{itemize}
			\itemsep 0pt
			\item Breaches the ICE for the decker, allowing for Alter to manipulate the programs on the device.
		\end{itemize}
		\item Alter :: Complexity 4 + Level
		\begin{itemize}
			\itemsep 0pt
			\item Attempts to alter or disable corresponding programs, such as Stealth, ICE, or Damage.
		\end{itemize}
	\end{itemize}
	
	\subsubsection{Corporate Control Software}
	
	\textbf{$\times$1 Complexity 3 and $\times$1 Complexity 4}
	
	A common piece of software for any professional enterprise, this software allows Spyders to watch over wageslave activity and prevent any unwanted behaviour such as going on the certain parts of the Matrix, releasing controlled information, and so on. Due to its capabilities for controlling large swathes of computers, it's not uncommon for this program to also have additional security such as Analyze programs attached to it.
	
	\begin{itemize}
		\itemsep 0pt
		\item Analyze :: Complexity 3
		\item Control :: Complexity 4
	\end{itemize}
	
	\subsubsection{Data Bomb}
	
	\textbf{$\times$1 Complexity 3 and $\times$4 Complexity 4+L and $\times$1 Complexity 5+L}
	
	Data Bombs are a curious kind of "smart trap" in the matrix that are sometimes used to maliciously protect highly sensitive files. They differ from the usual stupidity of normal traps, in that they have highly capable programming that can allow for more accurate discrimination of targets - alongside the wide array of payloads one can bring in the matrix. Of course, they can and do still harm unsuspecting users (or sometimes people who put that password in wrong too many times), which needs to be taken account for by the spyders setting them up.
	
	Data Bombs themselves are payloads attached to some piece of software - usually important files but it can be programs or users and so on. Their purpose is to automatically attack against unauthorized intrusions, which can range from mundane warnings to attempting to kill the intruder - the most common of which is to fry the hardware of the intruder. Preventing this is usually done by successfully verification when accessing the item, most often unlocking the file, but it can be stranger mechanisms (Which are usually covered under the Quick Contests against the Analyze program, or at least can be discovered in that way).
	
	Regardless, traps are still one thing: indiscriminate. Any self-respecting spyder will tune the lethality and sensitivity of a data bomb to safeguard against casual intrusion, because bored employees will always go poking around things. This is firstly covered in simply choosing a payload beside the Damage program, such as Alter or Control, that won't cause as many issues in case of accident. It can also include a higher threshold for anomalous behaviour, represented in needing the Analyze program needing a higher Margin of Success or Victory to set off the Trigger program - or perhaps requiring follow-up scans.
	
	\begin{itemize}
		\itemsep 0pt
		\item Stealth :: Complexity 4 + Level
		\begin{itemize}
			\itemsep 0pt
			\item Used to keep itself hidden on the file, program, or other location.
		\end{itemize}
		\item Analyze :: Complexity 3 + 1 + Level
		\begin{itemize}
			\itemsep 0pt
			\item Watches the file, its ICE, or some other program in order to detect anomalous activity.
		\end{itemize}
		\item Trigger :: Complexity 3
		\begin{itemize}
			\itemsep 0pt
			\item Will trigger the following programs upon detecting anomalous enough activity.
		\end{itemize}
		\item Search :: Complexity 3 + 1 + Level
		\begin{itemize}
			\itemsep 0pt
			\item Used to trace back the activity to the anomalous connection point.
		\end{itemize}
		\item Breach :: Complexity 3 + 1 + Level
		\begin{itemize}
			\itemsep 0pt
			\item Breaches the ICE of the decker, allowing for followup response
		\end{itemize}
		\item Damage :: Complexity 5 + Level
		\begin{itemize}
			\itemsep 0pt
			\item Either frags the decker, or fries the computer. Can be switched out with any payload program as necessary.
		\end{itemize}
	\end{itemize}

	
	\subsection{Sample Hosts}
	
	Provided are some sample host writeups based on the official Shadowrun hosts found in Data Trails/Kill Code.
	
	\subsubsection{The Seattle Metroplex Administration Host}
	
	\hspace{\parindent}\textbf{Host Computer:} Mainframe, Complexity 7
	
	\textbf{Programs:}
	
	\begin{itemize}
		\itemsep 0pt
		\item $\sim$10,000 Complexity 2 Programs providing tools, searching, payment, sign-in, system controls, etc. Some of these are Complexity 3-4 for better security or technological reasons, such as payment or system controls.
		\item $\times$1 Complexity 6 ICE (+3), providing security for the host
		\item $\times$1 Complexity 6 Listen (+4), letting through employees and citizens with SINs
		\item $\times$1 Complexity 6 Analyze (+3), watches the ICE and Listen programs to detect any signs of hacking.
		\item $\times$1 Complexity 3 Trigger, alerts system admins if there is any sign of hacking via Analyze.
	\end{itemize}

	\textbf{IC:}
	
	The host is meant to be used regularly by the public, so all IC should focus on detection and prevention. Dangerous IC are right out and even privacy breaching IC meant to track down offenders might cause legal issues (Largely relating to extraterritoriality and corporations) and will be used with caution.
	
	\begin{itemize}
		\itemsep 0pt
		\item 1 Level 3 Patrol IC with anti-tamper ($\times$2 Complexity 6 and $\times$1 Complexity 3)
	\end{itemize}
	
	
	\textit{Leftover Space:}
	\begin{itemize}
		\itemsep 0pt
		\item $\times$1 Complexity 7 Program
		\item $\times$3 Complexity 6 Programs
		\item $\times$9 Complexity 5 Programs
		\item $\times$9 Complexity 4 Programs
		\item $\times$8 Complexity 3 Programs
	\end{itemize}
	
	\textbf{Spyder:}
	2 Standard Spyders on rotating 12h shifts.
	
	
	\subsection*{Computer Templates}
	
	For templates on computers, see \hyperref[ai_computer]{the AI section on computer bodies.}
	
\end{multicols}