\section{Critters}

\begin{multicols*}{3}
	
	\subsection{What is a Critter}
	
	Critter is a broad term in Shadowrun, that technically means any non-human fauna, but is more often used to refer to Awakened fauna and sometimes Emerged fauna.
	
	\subsection{Awakened Critter Powers}
	
	Critter powers act in many ways like spells, however they majorly differ in the fact that they lack drain. Critter powers are usually able to be used as much as the user wishes, which makes them quite powerful and expensive.
	
	They are usually unavailable to most characters, being limited to magical creatures or individuals afflicted with special magical effects or diseases, such as the Bestow power or HMHVV.
	
	\subsubsection{Common Traits}\label{cp_ct}
	
	Many critter powers have some common setups that can be listed under a group of modifiers:
	
	\begin{itemize}
		\itemsep 0pt
		\item Direct Power, +245\%: \textit{Malediction, LoS, +250\%; No Signature, +20\%; Sense-Based, Reversed, Vision \& Touch, -15\%; Substantial Only, -10\%}
	\end{itemize}
	
	\subsubsection{Accident}\label{accident}
	\begin{flushright}
		39 Points
	\end{flushright}
	
	The critter causes seemingly normal accidents to occur around the the target. The critter rolls Will+Magic vs. the target's Per; success gives the target the Unluckiness disadvantage for minutes equal to margin of victory. If the margin of victory is 5 ore more, or the target rolls a critical failure, they are also Cursed for minutes equal to margin of victory.
	
	Unluckiness provides a single, not directly lethal way that the target will be screwed over with during the timeframe. This can be anything that is remotely plausible, from arbitrarily missing a vital die roll, to weapons breaking, to enemies showing up at the worst time, etc. If unsure, the GM can always treat this as the reverse of Lucky, making the target roll thrice and take the worst for an important roll - although they are entirely within their right to consider the roll an automatic failure instead. As well, whenever anything wrong could happen to someone on the target's team, it will happen to them.
	
	Cursed is much more lethal. Like its cousin Unluckiness, whenever something wrong happens, it \textit{always} affects the target. Whenever something good happens it \textit{always} misses the target. For the timeframe, the GM should be pernicious if not outright hostile to the target. If something remotely plausible could hose them, it should. Guns should refuse to work, everyday accidents should happen repeatedly if they can fit inside the timeframe, the GM should call for re-rolls or outright failures to rolls, or anything else, or a combination of them all (A very likely occurrence!). This is truly a \textit{-75 points disadvantage in a minute long timeframe}, and the target should \textit{earn those points with prejudice.}
	
	\textcolor{OliveGreen}{\textit{Statistics: Affliction (Based On (Per), +20\%; \hyperref[cp_ct]{Direct Power, +245\%}; Disadvantage, Unluckiness, +10\%; Disadvantage, Cursed, Side-Effect, +15\%; Magical, -10\%) [38]}}
	
	
	\subsubsection{Animal Control}\label{animal_control}
	\begin{flushright}
		25/50/75 Points
	\end{flushright}
	
	The critter has the ability to influence the behaviour of other non-sapient animals. This power affects all animals withing a 2 yard radius of a point, allowing them to control entire swarms of smaller animals. To do so, roll a Quick Contest of the critter's IQ+Magic vs the highest Will of the group.
	
	\textcolor{NavyBlue}{\textit{Modifiers: Long-Distance Modifiers to the furthest subject, -1 per slave already under control, +2 for concentrating a full minute, or +4 for concentrating a full hour.}}
	
	If the critter wins, the animals will obey the every command of it for as long as it concentrates \textit{or the animals leave the critter's line of sight}, and for minutes afterwards equal to margin of victory. The critter can only give them commands that are relatively normal behaviour for the animals (such as a flock of birds following or attacking something, not using a pistol or stealing a motorcycle), and the animals themselves perform the task as they best see fit - which can often have unpredictable results due to their low IQ.
	
	If the critter is incapacited or forces them to do something against their "principles" (e.g. making a flock of birds fight each other, or a rat avoid a free meal), roll another Quick Contest. If the animal wins, they break free.
	
	If the critter loses any Quick Contest, they are unable to affect these animals for 24 hours, and they also feel a mental coercion coming from the critter, which can make certain territorial animals attack!
	
	Critters with higher Magic can affect extremely large areas, allowing for truly \textit{massive} swarms of animals to be under their control. \textit{Fear the Force 12 Rat Spirit and its army of vermin!}
	
	\begin{center}
		\begin{tabular}{|c|c|c|}
			\hline
			Magic & Radius & Point Cost\\
			\hline
			\hline
			Magic 1-6 & 2 yd & 25 \\
			Magic 7-12 & 4 yd & 50 \\
			Magic 13-18 & 8 yd & 75 \\
			\hline
		\end{tabular}
	\end{center}
	
	\textcolor{OliveGreen}{\textit{Statistics: Mind Control (Area of Effect, 2 yards, +50\%; Long Range, +50\%; Accessibility, Only commands that are natural to the animal, -30\%\footnote{Priced on the fact that making animals perform any number of more complex tasks could be very useful, but you are limited to semi-natural behaviour}; Accessibility, Non Sapient Animal, -50\%\footnote{Priced on the fact that you'll most see tons more humans, which are also more valuable targets. Same with many critters.}; Suggestion, -40\%; Terminal Condition, Out of sight, -20\%; Magical, -10\%) [25] further levels increase Area of Effect's level [25]}}
	
	
	\subsubsection{Binding}\label{binding}
	\begin{flushright}
		3.3/3.9 Points per Level
	\end{flushright}
	
	This power is somewhat different than the original Shadowrun one, which covered a \textit{very wide} array of any ability that could bind something to something else (Whether it be sticky grapple, binding attack, or sticky climbing). This power is the ability to grapple and opponent with some sort of substance either shot by the critter (sticky webbing) or found in the environment (grappling with the earth).
	
	The power is an attack with Range 100, Acc 3, RoF 1, Rcl 1. On a successful hit, the target is grappled, meaning they cannot Move or Change Posture and are at -4 DX. 
	
	The binding itself has a ST equal to the levels of the power. It can also be continuously applied, adding +1 ST per additional layer. If an opponenet wants to escape, they must win a Quick Contest of ST or Escape against the binding's ST with a bonus equal to the critter's Magic. If they fail, they lose 1 FP but may try again. 
	
	The binding can alternatively be destroyed. Innate attacks hit automatically, but other attacks are at the usual -4. External attacks risk hitting the victim (B392). The binding itself has DR equal to 1/3 its level (round down). Eeach point of damage to the binding reduces its ST by 1, destroying it at ST 0.
	
	Certain attacks rely on environmental conditions to work, the most obvious case being an Earth Spirit requiring ground of some sort nearby that it can bind the target with.
	
	Critters usually have a number of levels in this ability equal to their Willpower.
	
	\textcolor{OliveGreen}{\textit{Statistics: Binding (Increased Range, LoS, +40\%; Environmental, -20\%; Magical, -10\%) [3.3 per level] for abilities without Environmental, it is [3.9 per level]}}
	
	
	\subsubsection{Concealment}\label{concealment}
	\begin{flushright}
		12/21/32/34/60/77/88/99 Points
	\end{flushright}
	
	The critter has the ability to mystically hide themselves alongside people and things nearby from perception. When active, anything \textit{of the critter's choice} within 2 yards of it becomes harder to view, taking penalties to all Vision based rolls (Including those using Ultravision, Infravision, and LADAR) and on any rolls that rely on Vision (Such as shooting rolls).
	
	There is no bonus to notice this concealment, as onlookers and cameras simply fail to process or record them - however, if anyone \textit{does} manage to break past the concealment and notice them, the power is immediately terminated and cannot be used again for 5 minutes\footnote{Timeframe taken from Maximum Duration -0\% limitation.}. The penalty also does not affect anyone that the critter conceals, allowing for perfect vision out of the area.
	
	The penalty is determined by the critter's Magic, which can also increase the radius that it can affect things, all detailed below.
	
	\begin{center}
		\begin{adjustwidth}{-2mm}{}
			\scalebox{0.95}{
				\begin{tabular}{|c|c|c|c|}
					\hline
					Magic & Penalty & Radius & Points\\
					\hline
					\hline
					Magic 1-2 & -2 & 2 yd & 12 \\
					Magic 3-4 & -3 & 4 yd & 21 \\
					Magic 5-6 & -4 & 8 yd & 32 \\
					Magic 7-8 & -5 & 16 yd & 45 \\
					Magic 9-10 & -6 & 32 yd & 60 \\
					Magic 11-12 & -7 & 64 yd & 77 \\
					Magic 13-14 & -8 & 64 yd & 88 \\
					Magic 15+ & -9 & 64 yd & 99 \\
					\hline
				\end{tabular}
			}
		\end{adjustwidth}
	\end{center}
	
	\textcolor{OliveGreen}{\textit{Statistics: Obscure, Vision (Defensive, +50\%; Extended, Ultra, Infra, \& LADAR, +60\%; Selective Area, +20\%; Stealthy, +100\%; Magical, -10\%; Terminal Condition, Being spotted, -20\%) [6 per level] further levels add Area of Effect, +50\% [1 per level]}}
	
	\subsubsection{Confusion}\label{confusion}
	\begin{flushright}
		42/56/70/84/98 Points
	\end{flushright}
	
	The critter has the ability to instill Confusion in a target in can see that can also see \textit{or} hear the critter (Choose only one). 
	
	Roll a Quick Contest between the Critter's Will+Magic vs the Target's Will, modified by normal Fright Check Modifiers (B360) as applicable. If the Critter wins, the target must immediately roll 3d on the Confusion Table (P85), adding their Margin of Failure to the result.
	
	If the target succeeds on their result, they are immune to the Critter's power for 1 hour, they also gain a +1 bonus to resist for every time the Critter has targeted them in the past 24 hours.
	
	Higher Magic critters are doubly more effective at this, imposing an additional -1 penalty every 3 Magic as detailed below.
	
	\begin{center}
		\begin{tabular}{|c|c|c|}
			\hline
			Magic & Penalty & Point Cost\\
			\hline
			\hline
			Magic 1-3 & -0 & 42 \\
			Magic 4-6 & -1 & 56 \\
			Magic 7-9 & -2 & 70 \\
			Magic 10-12 & -2 & 84 \\
			Magic 13-15 & -2 & 98 \\
			\hline
		\end{tabular}
	\end{center}
	
	\textcolor{OliveGreen}{\textit{Statistics: Terror, Confusion (Active, +0\%; Increased Range LoS, +70\%; Sense Based, Vision, Reversed, -20\%; Magical, -10\%) [42] further levels add -1 to resist [14]
	}}
	
	\subsubsection{Elemental Attack}\label{elemental_attack}
	\begin{flushright}
		Variable Points
	\end{flushright}
	
	The critter can attack using an elemental force hurling at their foe. Roll to hit using Innate Attack (Projectile). The cost and effects vary depending on the element of choice, and are covered below. See Innate Attack (B60) for more detailed rules on the statistics of the attacks.
	
	This power is generally bought at a level equal to the critter's Magic, although the GM can make some exception for more or less powerful attacks.
	
	Not every elemental attack is created equal as well. Modifiers can be added or subtracted from these in order to better represent a specific use-case (e.g. Changing Ice to Burn damage with No Incendiary, -10\% to represent severe frostbite as opposed to a literal chunk of ice). While doing so, one should generally aim for a window of 3-7 points per level.
	
	\begin{center} 
		\begin{adjustwidth}{-4mm}{}
			\scalebox{0.82}{
				\begin{tabular}{|c|c|c|c|c|c|}
					\hline
					Weapon & Damage & Acc & Range & Points & Notes \\
					\hline
					\hline
					Fire & 1d burn & 3 & 25/50 & 5 & [1] \\
					Lightning & 1d burn & 3 & 25/50 & 5 & [2,4]\\
					Water & 1d cr & 3 & 25/50 & 5.5 & [3]\\
					Ice & 1d cr [1d] & 3 & 10/20 & 4.75 & \\
					Earth & 1d cr & 3 & 25/50 & 5.5 & [5]\\
					Metal & 1d(2) pi- & 3 & 25/50 & 4.2 & \\
					\hline
				\end{tabular}
			}
		\end{adjustwidth}
	\end{center}
	
	[1] - Treat all flammability classes (B433) as 1 level lower.
	
	[2] - Surge. Critical hits disable electronics. Damage over HP/3 must make HT roll to avoid shorting out for seconds equal to margin of failure; critical failure disables until repaired.
	
	[3] - Double all basic damage for the purposes of calculating knockback.
	
	[4] - No incendiary.
	
	[5] - Deals double blunt trauma.
	
	\textcolor{OliveGreen}{\textit{\textbf{Fire:} Innate Attack, Burning (Incendiary 1\footnote{Power-Ups 4, p19 covers increasing flammability class (B433) for incendiary,}, +10\%; Increased 1/2D Range, \(\times\)5, +10\%; Magical, -10\%; Reduced Range, \(\times1/2\), -10\%) [5 per level].}}
	
	\textcolor{OliveGreen}{\textit{\textbf{Lightning:} Innate Attack, Burning (Increased 1/2D Range, \(\times\)5, +10\%; Surge, +20\%; Magical, -10\%; No Incendiary, -10\%; Reduced Range, \(\times1/2\), -10\%) [5 per level].}}
	
	\textcolor{OliveGreen}{\textit{\textbf{Water:} Innate Attack, Crushing (Double Knockback, +20\%; Increased 1/2D Range, \(\times\)5, +10\%; Magical, -10\%; Reduced Range, \(\times1/2\), -10\%) [5.5 per level] }}
	
	\textcolor{OliveGreen}{\textit{\textbf{Ice:} Innate Attack, Crushing (Fragmentation 1, +15\%; Increased 1/2D Range, \(\times\)5, +10\%; Magical, -10\%; Reduced Range, \(\times1/5\), -20\%) [4.75 per level].)}}
	
	\textcolor{OliveGreen}{\textit{\textbf{Earth:} Innate Attack, Crushing (Double Blunt Trauma, +20\%; Increased 1/2D Range, \(\times\)5, +10\%; Magical, -10\%; Reduced Range, \(\times1/2\), -10\%) [5.5 per level] ) }}
	
	\textcolor{OliveGreen}{\textit{\textbf{Metal:} Innate Attack, Small Piercing (Armor Divisor (2), +50\%; Increased 1/2D Range, \(\times\)5, +10\%; Magical, -10\%; Reduced Range, \(\times1/2\), -10\%) [4.2 per level].) }}
	
	
	\subsubsection{Energy Aura}\label{energy_aura}
	\begin{flushright}
		Various Points
	\end{flushright}
	
	\begin{center} 
		\begin{adjustwidth}{-0mm}{}
			\scalebox{1.0}{
				\begin{tabular}{|c|c|c|c|}
					\hline
					Weapon & Damage & Points & Notes \\
					\hline
					\hline
					Fire & 1d burn  & 5.5 & [1] \\
					Lightning & 1d burn  & 5.5 & [2,3]\\
					Ice & 1d burn  & 4.5 & [3]\\
					\hline
				\end{tabular}
			}
		\end{adjustwidth}
	\end{center}
	
	[1] - Treat all flammability classes (B433) as 1 level lower.
	
	[2] - Surge. Critical hits disable electronics. Damage over HP/3 must make HT roll to avoid shorting out for seconds equal to margin of failure; critical failure disables until repaired.
	
	[3] - No incendiary.
	
	\textcolor{OliveGreen}{\textit{\textbf{Fire:} Innate Attack, Burning (Always On, -40\%; Aura, +80\%; Incendiary 1\footnote{Power-Ups 4, p19 covers increasing flammability class (B433) for incendiary,}, +10\%; Magical, -10\%; Melee Attack, C, -30\%; ) [5.5 per level].}}
	
	\textcolor{OliveGreen}{\textit{\textbf{Lightning:} Innate Attack, Burning (Always On, -40\%; Aura, +80\%; Surge, +20\%; Magical, -10\%; Melee Attack, C, -30\%; No Incendiary, -10\%) [5.5 per level].}}
	
	\textcolor{OliveGreen}{\textit{\textbf{Lightning:} Innate Attack, Burning (Always On, -40\%; Aura, +80\%; Magical, -10\%; Melee Attack, C, -30\%; No Incendiary, -10\%) [4.5 per level].}}
	
	
	\subsubsection{Engulf}\label{engulf}
	\begin{flushright}
		29/36/44 Points
	\end{flushright}
	
	The critter physically engulfs their target. To use the ability, the critter must first grapple their target, whose SM cannot exceed the critter's. On their next turn, and on each successive turn, roll a Quick Contest: 10+Magic vs. the victim's higher of ST or HT. If the critter wins, the target takes damage equal to the margin of victory; otherwise no damage. Usually, this is large-area injury (B400), unless the GM decides otherwise. As well, any Aura attack will be able to continually hit the victim.
	
	If the victim fails to break out between the time that he is grappled and when the critter makes Quick Contests, he's pinned (B370). This type of pin is completely hand free (usually providing the +3 for more free hands). However, a victim with abilities such as Innate Attack or Spines hit the critter automatically while engulfed.
	
	Higher levels of Magic make the attack \textit{even more potent.} At certain levels, add an armor divisor to the attack.
	
	\begin{center}
		\begin{tabular}{|c|c|c|}
			\hline
			Magic & AD & Point Cost\\
			\hline
			\hline
			Magic 1-3 & (1) & 29 \\
			Magic 4-6 & (2) & 36 \\
			Magic 7-9 & (3) & 44 \\
			Magic 10-12 & (5) & 51 \\
			Magic 13+ & (10) & 66 \\
			\hline
		\end{tabular}
	\end{center}
	
	\textcolor{OliveGreen}{\textit{Statistics: Constriction Attack (Based on (Magic), +20\%; Engulf, +80\%; Magical, -10\%) [29] further levels adds Armor Divisor 2, +50\% [36]; 3, +100\% [44]; 5, +150\% [51]; 10, +200\% [66]}}
	
	
	\subsubsection{Fear}\label{fear}
	\begin{flushright}
		42/56/70/84/98 Points
	\end{flushright}
	
	The critter has the ability to instill Fear in a target in can see that can also see \textit{or} hear the critter (Choose only one). 
	
	Roll a Quick Contest between the Critter's Will+Magic vs the Target's Will, modified by normal Fright Check Modifiers (B360) as applicable. If the Critter wins, the target must immediately roll 3d on the Fright Check Table (B360), adding their Margin of Failure to the result.
	
	If the target succeeds on their result, they are immune to the Critter's power for 1 hour, they also gain a +1 bonus to resist for every time the Critter has targeted them in the past 24 hours.
	
	Higher Magic critters are doubly more effective at this, imposing an additional -1 penalty every 3 Magic as detailed below.
	
	\begin{center}
		\begin{tabular}{|c|c|c|}
			\hline
			Magic & Penalty & Point Cost\\
			\hline
			\hline
			Magic 1-3 & -0 & 42 \\
			Magic 4-6 & -1 & 56 \\
			Magic 7-9 & -2 & 70 \\
			Magic 10-12 & -2 & 84 \\
			Magic 13-15 & -2 & 98 \\
			\hline
		\end{tabular}
	\end{center}
	
	\textcolor{OliveGreen}{\textit{Statistics: Terror, Fright (Active, +0\%; Increased Range LoS, +70\%; Sense Based, Vision, Reversed, -20\%; Magical, -10\%) [42] further levels add -1 to resist [14]
	}}
	
	
	\subsubsection{Guard}\label{guard}
	\begin{flushright}
		21/42/84 Points
	\end{flushright}
	
	The critter has the ability to prevent dangerous mishaps from occurring in its vicinity. Whenever an event that would be extremely dangerous - usually one that can cause a major wound, fright check, incapacitation in a dangerous place, or similar levels of threat - causes an Active Defense roll, resistance roll, HT, is a critical hit on an attack roll, or something similar, the ability can activate.
	
	For allies, the critter makes the target either roll two more times and take the best; while for opponents they must roll two more times and take the worst. This can only occur once every 24 hours.
	
	The ability can only activate for \textit{true emergencies.} Combat is not by default an emergency - especially for combat characters - it has to be dangerous beyond that. This is why it is usually a major wound, critical hit on an attack roll, or fright check.
	
	Higher Magic critters can Guard more times per day, as indicated in the table below:
	
	\begin{center}
		\begin{tabular}{|c|c|c|}
			\hline
			Magic & Timeframe & Point Cost\\
			\hline
			\hline
			Magic 1-6 & 1/day & 21 \\
			Magic 7-12 & 2/day & 42 \\
			Magic 13+ & 6/day & 84 \\
			\hline
		\end{tabular}
	\end{center}
	
	\textcolor{OliveGreen}{\textit{Statistics: Luck (Wishing, +100\%; Emergencies Only, -30\%; Defensive, -20\%; Magical, -10\%) [21] further levels are Extraordinary Luck [42] and Ridiculous Luck [84]}}
	
	
	\subsubsection{Influence}\label{influence}
	\begin{flushright}
		58 Points
	\end{flushright}
	
	Mind control with suggestion
	TODO DESC
	
	
	\textcolor{OliveGreen}{\textit{Statistics: Mind Control (Independent, +70\%; Long Range, +50\%; Rationalization, +20\%; Accessibility, only on sapient creatures\footnote{Based on the flip limitation seen on Animal Control}, -10\%; Magical, -10\%; Reduced Duration, \(\times1/60\), -65\%; Suggestion, -40\%) [58]}}
	
	
	\subsubsection{Movement}\label{movement}
	\begin{flushright}
		54 Points
	\end{flushright}
	
	This power allows a critter to greatly speed up or slow down certain objects. It only works on things that are already predisposed to motion, such as vehicles or creatures - no slowing down a door to block your opponents! 
	
	As well, the creature is only able to affect things within its own domain! For many mundane creatures, this means their home turf and likely the area surrounding it. For spirits, this tends to be associated with whatever their types is, with Air spirits affecting things that fly and Earth spirits affecting things touching the ground. In cases where these are not present, it doesn't necessairly prevent activating this power, it just makes the results \textit{at best} unpredictable.
	
	When attempting to speed something up, the creature rolls against their Will+Magic. The target gains Enhanced Move in half levels equal to the Margin of Succes (e.g. Margin of Success 3 gives Enhanced Move 1.5 \(\times\)3 maximum speed). Due to the uncoordinated use of this movement, all DX and Handling rolls are made with a -2 penalty and the user must focus completely on directing themselves; count this as an All-Out maneuver!
	
	When applying this to vehicles and creatures with Enhanced Move already, \textit{do not multiply their Enhanced Move.} Instead, add the Margin of Success to their Enhanced Move in half levels (e.g. A horse with Enhanced Move 1, giving \(\times\)2 max speed, increases their Enhanced Move to 2.5 with Margin of Success 3, giving \(\times\)6 max speed!)
	
	When attempting to slow something down, the creatures rolls a Quick Contest of Will+Magic vs the target's Will or HT for objects. If the critter wins, the target has their Basic Move reduced by 20\% per Margin of Victory, to a maximum of 0 move at MoV 5.
	
	The GM is entirely within their rights to call for a Control Roll whenever this ability is applied, especially when applied without any prior warning!
	
	It is all around harder to affect technological targets, providing a -6 to all rolls against them, whether to speed up or slow down.
	
	You can only have one type of these effects active at a time, no slowing down \textit{and} speeding up - although you can apply the same affect multiple times (Although to different targets of course).
	
	\textcolor{OliveGreen}{\textit{Statistics: Affliction (\hyperref[cp_ct]{Direct Power, +245\%}; Movement, Margin-Based, +180\%; Accessibility, Things predisposed to motion, -20\%\footnote{This accessibility is somewhat hard to classify, but is based on the assumption that it could be used with some objects such as falling items, doors, etc. usefully.}; Environmental, -20\%; Hard-To Use 2 (Accessibility, Technology Only, -20\%), -8\%; -Magical, -10\%) [47]}}
	
	\textcolor{OliveGreen}{\textit{Movement: Enhanced Move, \(\times\)1.5 (All-Out, -20\%; Handling Penalty 2, -10\%; Magical, -10\%) [6]}}
	
	\textcolor{OliveGreen}{\textit{Statistics: Affliction (\hyperref[cp_ct]{Direct Power, +245\%}; Slower Move, Margin-Based, +30\%; Accessibility, Things predisposed to motion, -20\%; Environmental, -20\%; Hard-To Use 2 (Accessibility, Technology Only, -20\%), -8\%; Magical, -10\%) [32] with Alternative Abilities [7]}}
	
	
	\subsubsection{Search}\label{search}
	\begin{flushright}
		29 Points +2 Points per Level
	\end{flushright}
	
	The critter has the magical ability to find objects or individuals that are familiar to them. To do so, the critter concentrates for 10 minutes and then makes a Per+Magic roll, modified by long-distance range modifiers to the target. On a success, they learn the direction of the thing that most matches their search, or whatever qualifies as the most significant source (e.g. if they are looking for their summoner, they might get their twin brother if not in range). As well, they will generally know how many familiar objects or individuals are present within range. 
	
	To narrow down the search, the critter must analyze their results. Roll against IQ+Magic; success lets them determine more precise details, with better margin of success providing better results. Some example results (In order of low to high difficulty) are: Search through all of the results they got for one specific one, discern between false positives, determine basic qualities about the target (such as emotional state, metatype, etc), determine more advanced qualities (such as health, vague knowledge of effects on them, etc).
	
	The critter can search for any item that they are normally familiar with, but if provided images or drawings can also search for ones that they are passingly familiar with, at a penalty. If the critter is able to get a good mental image from their summoner, alongside it being an object or creature that is easy to discern (e.g. A specific car with a license plate, as opposed to a car's model), make the roll at a -3. If instead, they can't get a mental image and instead rely on a drawn one \textit{or} if the object or creature is not easy to discern, make the roll at -6. If they have to rely on only a drawing \textit{and} the object or creature is hard to discern, make the roll at -10.
	
	Higher Magic creatures are doubly effective at tracking down objects, gaining a +1 bonus for every 2 Magic past 2. This is noted in the table below.
	
	This ability does not immediately let the spirit know exactly where the target is (Especially not enough to target them with attacks or effects), even though it provides direction. This is enough for them to generally track them down regardless, although they may need to detect multiple times to track down for longer searches.
	
	Individuals or objects that are behind mana barriers or otherwise protected from divination will be harder to detect. The effects of this will usually be covered in their sections. 
	
	\begin{center}
		\begin{tabular}{|c|c|c|}
			\hline
			Magic & Bonus & Point Cost\\
			\hline
			\hline
			Magic 1-2 & +0 & 29 \\
			Magic 3-4 & +1 & 31 \\
			Magic 5-6 & +2 & 33 \\
			Magic 7-8 & +3 & 35 \\
			Magic 9-10 & +4 & 37 \\
			Magic 11-12 & +5 & 39 \\
			Magic 13-14 & +6 & 41 \\
			Magic 15-16 & +7 & 42 \\
			\hline
		\end{tabular}
	\end{center}
	
	\textcolor{OliveGreen}{\textit{Statistics: Detect, Known Objects and Beings (Very Common; Long Ranged, +50\%; Immediate Preparation Required, 10 Minutes, -45\%; Magical, -10\%) [29] further levels add Acute Sense (Search) [2].}}
	
	\subsection{Infected}
	
	\subsection*{HMHVV Strain I}
	
	Infection Power
	
	\subsubsection{Banshee}
	
	\subsubsection{Dzoo-Nou-Qua}

	\subsubsection{Goblin}
	
	\subsubsection{Noseferatu}
	
	\subsubsection{Vampire}
	Body, Reaction, Strength, Willpower, Intuition, Charisma
	
	\subsubsection{Wendigo}
	
	\subsection*{HMHVV Strain II}
	
	Scratches or bodily fluids
	
	\subsubsection{Bandersnatch}
	
	\subsubsection{Fomóraig}
	
	\subsubsection{Gnawer}
	
	\subsubsection{Grendel}
	
	\subsubsection{Harvester}
	
	\subsubsection{Loup-Garou}
	
	\subsection*{HMHVV Strain III}
	
	Scratches and bodily fluids
	
	\subsubsection{Ghoul}
	
	\subsection{Emerged Critter Powers}
	
	Many of these powers use the Critter Resonance, -5\% power modifier. This works exactly like the Resonance Complex Form power modifier, being made up of the Supernatural Counters, -5\% component.
	
	\subsubsection{Camouflage}\label{camouflage}
	\begin{flushright}
		33 Points
	\end{flushright}

	The critter has the ability to hide files from matrix searches, making them unable to be detected without specifically searching for that specific file.
	
	To do so, the critter rolls Will + Resonance, or a Quick Contest of Will + Resonance versus the file's Will (Usually Complexity $\times$ 2) for files it doesn't own. The owner can waive resistance rolls for the file. Success makes the file undetectable to any matrix search that isn't looking for that specific file. General searches, searches for files of its class, closely related files, or even descriptors that closely match it all fail.
	
	This lasts for 10 minutes per margin of success/victory, and can be renewed by the critter as normal. Using this power as well, it entirely undetectable to mundane means.

	\textcolor{OliveGreen}{\textit{Statistics: Affliction (Camouflage, +100\%; Based on Will, +20\%; Extended Duration, $\times$10, +40\%; Malediction 1, +100\%; No Signature, +20\%; Cybernetic Only, -50\%; Critter Resonance, -5\%) [33]}}
	
	\textcolor{OliveGreen}{\textit{Statistics: Camouflage is Invisibility, Matrix Searches (Can Carry Objects, No Encumbrance\footnote{Equivalent of saying the file can carry a good amount of data or whatnot.}, +10\%; Accessibility, Not for searches specifically for this file, -0\%\footnote{I figure most searches would be for vague overall tags and ideas, instead of a specific file, meaning it ends up with 6\% or less, which is a 0\% accessibility.}; Accessibility, Files Only, -30\%\footnote{No people, or devices, etc. which make up at least 4/5 of good targets.}; Critter Resonance, -5\%; Machines Only, -50\%) [10] }}
	 
	\subsubsection{Cookie}\label{cookie}
	\begin{flushright}
		TBD Points
	\end{flushright}

	I'm gonna be real, you can do this with the normal matrix system exactly the same. So unless there's a reason or alternative to be implemented here, I'm going to leave this out for now.

	\subsubsection{Diagnostics}\label{diagnostics}
	\begin{flushright}
		22 Points
	\end{flushright}

	The critter is able to perform rapid repairs, diagnostics, and optimizations on the hardware of an item, allowing to be heavily optimized for a very specific task.
	
	To use this power, the critter must All-Out-Concentrate and roll a suitable Electronics Repair, Mechanic, Electrician, or so on Skill with a +1 bonus for All-Out Concentrate alongside a bonus equal to its Resonance. 
	
	The critter must describe explicitly and in detail what they are optimizing the hardware for (e.g. I'm optimizing the smartgun's targeting platform in order to compensate for the complex fluid dynamics of the heavy wind in order to let the firearm's user shoot the target with a three rounds burst to the face hit location, before the opponent is able to draw their firearm or move). 
	
	Success on the skill can give a bonus based on how closely your optimization matched reality. If it matched \textit{almost exactly}, the user of the hardware gains a bonus to their skill to use it equal to the margin of success. 
	
	If the situation is not quite the same, the bonus is halved, rounded down to a minimum of +1 (e.g. the opponent doesn't intend to draw his weapon, the wind temporarily dies down, or another person enter's the smartgun's view). 
	
	If the situation is clearly different, the bonus is divded by 3, and rounded down to a minimum of +0 (e.g. the smartgun was out of ammo, the user first full-auto instead of a burst, someone tackles the target before the shots go off).
	
	The GM should add a bonus of up to +2 or a penalty of \textit{any size} for particularly good or bad descriptions. Especially, if the player tries to make their description overly vague in order to guarantee their optimization comes true, they should be taking \textit{at least} a -2 penalty, if not \textit{much more}.

	\textcolor{OliveGreen}{\textit{Statistics: Visualization (Blessing, +50\%; Reduced Time, 6, +120\%; All-Out Concentrate, -25\%; Aspected, Hardware, -20\%; Critter Resonance, -5\%; Requires (Hardware Skill) Roll, +0\%) [22] }}

	\subsubsection{Electron Storm}\label{electron_storm}
	\begin{flushright}
		13/18/23/28/33/38 Points
	\end{flushright}

	The critter surrounds its target in a hail of dangerous and corrupting datastreams, causing severe damage over time.
	
	The critter must Concentrate for 1 second and make a Quick Contest of Computer Hacking + Resonance versus HT (+3 for Hardened systems). Success deals 1d burn damage directly to the target device. Additionally, the next second, it deals an \textit{additional} 1d burn damage; this cycle can be halted with a successful Computer Hacking or Expert Skill (Computer Security), either destorying the datastreams as they come in, or programming safeguards against their continued attacks. At the GM's option though, this power is quick and deadly enough to warrant a Fright Check! There's nothing subtle about this of course, even if the target doesn't know what Resonance is.
	
	At higher Resonances, the critter can sustain the power for even longer. For every 2 Resonance above 1, the power lasts for 1 \textit{more} second, causing cyclic damage as described above.
	
	 	\begin{center}
	 	\begin{tabular}{|c|c|c|}
	 		\hline
	 		Resonance & Cycles & Points \\
	 		\hline
	 		\hline
	 		Resonance 1-2 & 1+1 & 13 \\
	 		Resonance 3-4 & 1+2 & 18 \\
	 		Resonance 5-6 & 1+3 & 23 \\
	 		Resonance 7-8 & 1+4 & 28 \\
	 		Resonance 9-10 & 1+5 & 33 \\
	 		Resonance 11-12 & 1+6 & 38 \\
	 		\hline
	 	\end{tabular}
	 \end{center}

	\textcolor{OliveGreen}{\textit{Statistics: Innate Attack, Burn 1d (Based on IQ, +20\%; Based on HT, +20\%; Cyclic 1, +100\%; Malediction 1, +100\%; Accessibility, Only on Matrix, -40\%; Critter Resonance, -5\%; Cybernetic Only, -50\%; Requires (Computer Hacking) Roll, +0\%) [13] further levels add more Cyclic.}}

	\subsubsection{Gremlins}\label{gremlins}
	\begin{flushright}
		21 Points
	\end{flushright}

	The critter has the ability to cause a device to malfunction for seemingly no reason at all. To do so, the critter concentrates for 1 second and rolls a Quick Contest of Will + Resonance versus HT. Success gives the device the Unluckiness disadvantage for minutes equal to margin of victory. If the margin of victory is 5 or more, or the target rolls a critical failure, they are also Cursed for minutes equal to margin of victory. These disadvantages apply to all atempts to use the device!
	
	Unluckiness provides a single, not directly lethal way that the device and its users will be screwed over with during the timeframe. This can be anything that is remotely plausible, from arbitrarily missing a vital die roll, to weapons breaking, to computers receiving forced updates, etc. If unsure, the GM can always treat this as the reverse of Lucky, making the target roll thrice and take the worst for an important roll - although they are entirely within their right to consider the roll an automatic failure instead. As well, whenever anything wrong could happen to any number of related devices (such as all those carried on a user's person, or all those inside a building), it will happen to this one.
	
	Cursed is much more lethal. Like its cousin Unluckiness, whenever something wrong happens, it \textit{always} affects the device and its users. Whenever something good happens it \textit{always} misses them. For the timeframe, the GM should be pernicious if not outright hostile to them. If something remotely plausible could hose them, it should. Guns should refuse to work, everyday accidents should happen repeatedly if they can fit inside the timeframe, the GM should call for re-rolls or outright failures to rolls, or anything else, or a combination of them all (A very likely occurrence!). This is truly a \textit{-75 points disadvantage in a minute long timeframe}, and the device and any and all users should \textit{earn those points with prejudice.}

	\textcolor{OliveGreen}{\textit{Statistics: Affliction (Gremlins 1, +8\%; Gremlins 2, Side Effect, +12\%; Based on HT, +20\%; Malediction 1, +100\%; No Signature, +20\%; Cybernetic Only, -50\%; Critter Resonance, -5\%) [21]}}
	
	\textcolor{OliveGreen}{\textit{Statistics: Gremlins 1 is Unlucky (Aspected, Hardware, -20\%; Critter Resonance, -5\%) [8] and Gremlins 2 is Cursed (Aspected, Hardware, -20\%; Critter Resonance, -5\%) [57]}}

	\subsubsection{Hash}\label{hash}
	\begin{flushright}
		20/21/22/23 Points
	\end{flushright}

	By protecting a file with Resonance algorithmic encryption, a critter can prevent \textit{any} attempts to access or decrypt a file.
	
	To do so, the critter must currently have the file on its icon/device, Concentrate for 1 second, and roll a Will + Resonance. If the file is not owned by the critter, then this is a Quick Contest of Will + Resonance versus Will, which can be waived by the file's owner. Success makes the file illegible to anyone for 1 minute per margin of success.
	
	The critter must keep the file on hand the whole time, as leaving or transferring it elsewhere causes the encryption to immediately unravel.
	
	At higher Resonances, the critter gains greater bonuses to its rolls, adding +1 for each Resonance past 1.
	
	\begin{center}
		\begin{tabular}{|c|c|c|}
			\hline
			Resonance & Bonus & Points \\
			\hline
			\hline
			Resonance 1 & +0 & 20 \\
			Resonance 2 & +1 & 20 \\
			Resonance 3 & +2 & 21 \\
			Resonance 4 & +3 & 21 \\
			Resonance 5 & +4 & 22 \\
			Resonance 6 & +5 & 22 \\
			Resonance 7 & +6 & 23 \\
			Resonance 8 & +7 & 23 \\
			\hline
		\end{tabular}
	\end{center}

	\textcolor{OliveGreen}{\textit{Statistics: Affliction (Hash, +100\%; Based on Will, +20\%; Malediction 1, +100\%; Accessibility, Must maintain contact, -20\%\footnote{Based on two idea: Firstly is that while you'll be able to play around it, you'll lose around half of your capabilities if you need to keep the file personally on you at all times. Secondly, the Terminal Condition limitation has -20\% for common conditions, which I would consider removing from their possession as being.}; Accessibility, Only on Files, -20\%\footnote{No communications, no firewalls or keys, nothing but a normal file}; Critter Resonance, -5\%; Cybernetic Only, -50\%; Melee Attack, C, -30\%) [20] further levels add Reliable, +5\%}}

	\textcolor{OliveGreen}{\textit{Statistics: Hash is Resistant, Immunity to Decryption (Occaisonal; Critter Resonance, -5\%) [10]}}

	\subsubsection{Stability}\label{stability}
	\begin{flushright}
		17 Points
	\end{flushright}

	The critter has the ability to turn even the most dire hardware failures into astounding successes.
	
	Whenever a piece of hardware that the critter has complete access to and is being observed by the critter is directly involved with a Critical Failure - or if the hardware is owned by an opponent and involved with a Critical Success - the critter can activate this power.
	
	Doing so requires a free action and a successful IQ + Resonance roll. This allows the critter to roll an additional two times and take the best result out of all three rolls. This power can only be used once every hour of \textit{out of character time}, or alternatively once every day in game.

	\textcolor{OliveGreen}{\textit{Statistics: Luck (Wishing, +100\%; Accessibility, Only on Critical Failures, -35\%\footnote{Critical Failures usually only make up around 1-5\% of possibilities, however their effects are much worse than that, so I've used the category above that.}; Accessibility, Only on Hardware the critter has full access to, -20\%\footnote{Practically all of the hardware the critter would use this on it will have access to. In contrast, this restricts almost all use of preventing opponent's hardware from working, which is about 50\% reductions overall.}; Aspected, Hardware, -20\%; Critter Resonance, -5\%; Requires IQ Roll, -10\%) [17] }}

	\subsubsection{Suppression}\label{suppression}
	\begin{flushright}
		52 Points
	\end{flushright}

	The critter has the ability to worm their resonance into the systems on the network to delay alarms and warnings from hosts, giving the Emerged and their allies extremely valuable to complete their objectives.
	
	Using this power requires a Expert Skill (Computer Security) + Resonance roll every minute for it to be on. If alarms or warnings are activated on the network while this power is on, they are stopped for 30 seconds; afterwards this power shuts down and cannot be activated again for 5 minutes.
	
	This does not stop anyone or anything that was attempting to raise the alarm from acting, simply that they are unable to activate alarms through the network. They can still jack out and run around screaming, attack the critter, and so on.

	\textcolor{OliveGreen}{\textit{Statistics: Obscure, Matrix Alarms 10 (Area of Effect\footnote{This is simply a justification for it covering allies and all, can also be treated as Cosmic, Covers Network, +50\%.}, +50\%; Defensive, +50\%; Extended, Silent Alarms\footnote{Should cover loud and silent equivalent alarms.}, +20\%; Stealthy, +100\%; Critter Resonance, -5\%; Maximimum Duration, 30 seconds, (Accessibility, only when alarms are on, -40\%\footnote{Priced as such because, most of the value would come from being able to use it when alarms are not on yet, and additionally, alarms should not be on very much.}), -45\%; Requires (Expert Skill (Computer Security) Roll, -10\%)) [52] }}

	\subsubsection{Watermark}\label{watermark}
	\begin{flushright}
		26 Points
	\end{flushright}

	The critter leaves a hidden message on a matrix icon it can touch, up to about a minute long, in the matrix that can be read by other Resonance-entities or an Resonance individual that it knows. 
	
	To do so, the critter Concentrates for 1 minute and rolls Computer Programming + Resonance. If successful, look up the margin of success on the Size column of the \textit{Size and Speed/Range Table (B550)}; the corresponding linear measurement is the \textit{maximum} radius that the critter can set the trigger up for.
	
	If a valid target enters the area, they immediately hear the message once (and only them).
	
	\textcolor{OliveGreen}{\textit{Statistics: Telesend (Delay, Triggered, Resonance-entity enters area, +50\%; Critter Resonance, -5\%; Melee Attack, C, -30\%; Immediate Preparation Requires, 1 minute, -30\%; Requires (Computer Programming) Roll, +0\%) [26] }}
	
	\subsection{Example Critters}
	
\end{multicols*}